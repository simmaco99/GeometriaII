\documentclass[a4paper,12pt]{article}
\usepackage[a4paper, top=2cm,bottom=2cm,right=2cm,left=2cm]{geometry}

\usepackage{bm,xcolor,mathdots,latexsym,amsfonts,amsthm,amsmath,
					mathrsfs,graphicx,cancel,tikz-cd,hyperref,booktabs,caption,amssymb,amssymb,wasysym}
\hypersetup{colorlinks=true,linkcolor=blue}
\usepackage[italian]{babel}
\usepackage[T1]{fontenc}
\usepackage[utf8]{inputenc}
\newcommand{\s}[1]{\left\{ #1 \right\}}
\newcommand{\sbarra}{\backslash} %% \ 
\newcommand{\ds}{\displaystyle} 
\newcommand{\alla}{^}  
\newcommand{\implica}{\Rightarrow}
\newcommand{\iimplica}{\Leftarrow}
\newcommand{\ses}{\Leftrightarrow} %se e solo se
\newcommand{\tc}{\quad \text{ t. c .} \quad } % tale che 
\newcommand{\spazio}{\vspace{0.5 cm}}
\newcommand{\bbianco}{\textcolor{white}{,}}
\newcommand{\bianco}{\textcolor{white}{,} \\}% per andare a capo dopo 																					definizioni teoremi ...


% campi 
\newcommand{\N}{\mathbb{N}} 
\newcommand{\R}{\mathbb{R}}
\newcommand{\Q}{\mathbb{Q}}
\newcommand{\Z}{\mathbb{Z}}
\newcommand{\K}{\mathbb{K}} 
\newcommand{\C}{\mathbb{C}}
\newcommand{\F}{\mathbb{F}}
\newcommand{\p}{\mathbb{P}}

%GEOMETRIA
\newcommand{\B}{\mathfrak{B}} %Base B
\newcommand{\D}{\mathfrak{D}}%Base D
\newcommand{\RR}{\mathfrak{R}}%Base R 
\newcommand{\Can}{\mathfrak{C}}%Base canonica
\newcommand{\Rif}{\mathfrak{R}}%Riferimento affine
\newcommand{\AB}{M_\D ^\B }% matrice applicazione rispetto alla base B e D 
\newcommand{\vett}{\overrightarrow}
\newcommand{\sd}{\sim_{SD}}%relazione sx dx
\newcommand{\nvett}{v_1, \, \dots , \, v_n} % v1 ... vn
\newcommand{\ncomb}{a_1 v_1 + \dots + a_n v_n} %a1 v1 + ... +an vn
\newcommand{\nrif}{P_1, \cdots , P_n} 
\newcommand{\bidu}{\left( V^\star \right)^\star}

\newcommand{\udis}{\amalg}
\newcommand{\ric}{\mathfrak{U}}
\newcommand{\inclu}{\hookrightarrow }
%ALGEBRA

\newcommand{\semidir}{\rtimes}%semidiretto
\newcommand{\W}{\Omega}
\newcommand{\norma}{\vert \vert }
\newcommand{\bignormal}{\left\vert \left\vert}
\newcommand{\bignormar}{\right\vert \right\vert}
\newcommand{\normale}{\triangleleft}
\newcommand{\nnorma}{\vert \vert \, \cdot \, \vert \vert}
\newcommand{\dt}{\, \mathrm{d}t}
\newcommand{\dz}{\, \mathrm{d}z}
\newcommand{\dx}{\, \mathrm{d}x}
\newcommand{\dy}{\, \mathrm{d}y}
\newcommand{\amma}{\gamma}
\newcommand{\inv}[1]{#1^{-1}}
\newcommand{\az}{\centerdot}
\newcommand{\ammasol}[1]{\tilde{\gamma}_{\tilde{#1}}}
\newcommand{\pror}[1]{\mathbb{P}^#1 (\R)}
\newcommand{\proc}[1]{\mathbb{P}^#1(\C)}
\newcommand{\sol}[2]{\widetilde{#1}_{\widetilde{#2}}}
\newcommand{\bsol}[3]{\left(\widetilde{#1}\right)_{\widetilde{#2}_{#3}}}
\newcommand{\norm}[1]{\left\vert\left\vert #1 \right\vert \right\vert}
\newcommand{\abs}[1]{\left\vert #1 \right\vert }
\newcommand{\ris}[2]{#1_{\vert #2}}
\newcommand{\vp}{\varphi}
\newcommand{\vt}{\vartheta}
\newcommand{\wt}[1]{\widetilde{#1}}
\newcommand{\pr}[2]{\frac{\partial \, #1}{\partial\, #2}}%derivata parziale
%per creare teoremi, dimostrazioni ... 
\theoremstyle{plain}
\newtheorem{thm}{Teorema}[section] 
\newtheorem{ese}[thm]{Esempio} 
\newtheorem{ex}[thm]{Esercizio} 
\newtheorem{fatti}[thm]{Fatti}
\newtheorem{fatto}[thm]{Fatto}

\newtheorem{cor}[thm]{Corollario} 
\newtheorem{lem}[thm]{Lemma} 
\newtheorem{al}[thm]{Algoritmo}
\newtheorem{prop}[thm]{Proposizione} 
\theoremstyle{definition} 
\newtheorem{defn}{Definizione}[section] 
\newcommand{\intt}[2]{int_{#1}^{#2}}
\theoremstyle{remark} 
\newtheorem{oss}{Osservazione} 
\newcommand{\di }{\, \mathrm{d}}
\newcommand{\tonde}[1]{\left( #1 \right)}
\newcommand{\quadre}[1]{\left[ #1 \right]}
\newcommand{\w}{\omega}

% diagrammi commutativi tikzcd
% per leggere la documentazione texdoc
	

\begin{document}
\textbf{Lezione del 24 Febbraio del Prof. Frigerio}
\begin{defn}[Omeomorfismo locale]\bianco
$f:\, X \to Y$ \`e un omeomorfismo locale se 
$$ \forall x \in  \quad \exists U \ni p \text{ aperto in } X \text{ e } V \ni f(p) \text{ aperto in } Y $$
tale che 
$$ f(U)=V \text{ e } f_{\vert U} : U \to V \text{ omeomorfismo}$$
\end{defn}
\begin{fatti}\bbianco
\begin{enumerate}
\item Un omeomorfismo locale \`e una mappa aperta\\
Sia $\Omega \subset X $ aperto, $\forall p\in X$ seleziono un aperto $U_p \ni p $ come nella definizione di sopra
$$ f( \Omega) = f\left( \bigcup_{p\in X} ( \Omega\cap U_p) \right)= \bigcup_{p \in X} f(\Omega \cap U_p)$$
Ora poich\`e $U_p$ \`e omeomorfo mediante $f$ ad un aperto di $Y$ ne segue che $f(\Omega \cap U_p)$ \`e un aperto di un aperto di $Y$ dunque aperto.\\
Ora la tesi segue in quanto unione di aperti \`e aperto
\item $f$ omeomorfismo locale $\implica$ $f^{-1}(y)$ discreto in $Y$\\
Dato $x \in f^{-1}(y)$ per definizione di omeomorfismo locale, $\exists U \ni x $ aperto in $X$ e tale che $f_{\vert U} $ \`e iniettiva.\\
Ora $f^{-1}(x) \cap U=\{x \}$ da cui $\{ x \} $ \`e aperto
\end{enumerate}
\end{fatti}
\spazio
\begin{defn}[Rivestimento]\bianco
Una mappa $p:\, E \to X$ \`e un rivestimento se 
\begin{itemize}
\item $X$ \`e connesso per archi
\item 
$$ \forall x \in X \quad \exists U \ni x \text{ aperto in } X \quad p^{-1}(U) = \amalg_{i\in I} V_i$$
tali che $I \neq \emptyset$ e 
$$ V_i \text{ aperto in } Y \text{ e } p_{\vert V_i}:\, V_i \to U \text{ omeomorfismo } \forall i \in I$$

\end{itemize}
\end{defn}
\begin{defn}Un intorno come nella definizione si dice ben rivestito
\end{defn}
\begin{fatti}\bbianco
\begin{enumerate}
\item Un rivestimento \`e un omeomorfismo locale
\item Un rivestimento \`e surgettivo
\end{enumerate}
\end{fatti}

\begin{ese}Sia $p:\, \R \to S^1 $ con $p(t)=( \cos 2\pi t, \sin 2 \pi t)$, $p$ \`e un rivestimento.
\proof Dato $x_0 \in S^1$, allora $x_0 = p(t_0)$ per qualche $t_0$.\\
Scelgo $U= S^1\sbarra \{ -x_0\}$, ora $$p^{-1}(U) = \amalg_{k \in \Z} \left( t_0 - \frac{1}{2}+k, t_0+\frac{1}{2}+k \right)$$
Ora $\forall k$ la restrizione di $p$ su $\left( t_0 - \frac{1}{2}+k, t_0+\frac{1}{2}+k \right)$ \`e un omeomorfismo
\end{ese}
\begin{ese} Sia $p:(-1,1) \to S^1 $ con $p(t) = ( \cos 2 \pi t , \sin 2\pi t)$.\\
Tale mappa \`e un omeomorfismo locale surgettivo, ma non \`e un rivestimento in quanto $(1,0) \in S^1 $ non ha un intorno ben rivestito
\end{ese}
\begin{ex}[Retta con 2 origini]\bianco Sia $\sim $ la relazione su $\R \times \{-1,1 \}$ dato da 

$$ (x,\varepsilon) \sim (y , \varepsilon') \quad \ses \quad (x,\varepsilon)=(y,\varepsilon') \text{ o }( x = y \text{ e } x \neq 0 )$$
Allora la mappa $\pi: \frac{\R \times \{ -1 , 1\} }{\sim} \to \R$ \`e un omomorfismo locale surgettivo \\($0\in \R$ non ha intorni ben rivestiti)
\end{ex}
\begin{ex}$\pi:\, S^n \to \mathcal{P}^n \left( \R \right)$ \`e un rivestimento \\
La dimostrazione verr\`a fatta avanti con strumenti diversi e non a mano.
\end{ex}
\spazio
\begin{prop}[Unicit\`a del sollevamento]\bianco
Sia $p:\, E \to X $ un rivestimento, $Y$ connesso per archi e $f:\, Y \to X$.\\
Siano $\tilde{f}, \tilde{g}:\, Y \to E$ con $f=p\circ \tilde{f}=p \circ \tilde{g}$.\\
Se $\exists y_0 \in Y$ tale che $\tilde{f}(x_0)=\tilde{g}(x_0)$ allora $\tilde{f}=\tilde{g}$
\proof Poich\`e $Y$ \`e connesso per archi basta mostrare che $\Omega=\{ y \in Y \, \vert \, \tilde{f}(y)=\tilde{g}(y)\}$ \`e sia aperto che \`e chiuso ( $\Omega$ non \`e vuoto, dunque \`e tutto $Y$).\\
Mostriamo che $\Omega$ \`e aperto.\\
Sia $y\in \Omega$ allora $\tilde{x_0}=\tilde{f}(y)=\tilde{g}(y)$ e $x_0=p(\tilde{x_0})$\\
Per definizione di rivestimento $\exists U\in I(x_0)$ ben rivestito dunque $\pi^{-1}(U)=\amalg V_i$.\\
Sia $i_0$ tale che $\tilde{x_0}\in V_{i_0}$.\\
Essendo $\tilde{f}$ e $ \tilde{g}$ continue, $\exists W \in I(y)$ tale che $\tilde{f}(W)\subset V_{i_0}$ e $\tilde{g}(W)\subset V_{i_0}$\\
Ora $p_{\vert V_{i_0}}$ \`e iniettiva dunque da $p\circ \tilde{f}=p \circ \tilde{g}$ deduco che $\tilde{f}_{\vert W}= \tilde{g}_{\vert W}$\\
Mostriamo $\Omega$ \`e chiuso, mostrando che $Y \sbarra \Omega$ \`e aperto.\\
Sia $y\in Y \sbarra \Omega$ per cui $\tilde{f}(y) \neq \tilde{g}(y)$, tuttavia $p\circ \tilde{f}(y)=p \circ \tilde{g}(y)=f(y)=x_0$.\\
Se $U$ \`e un intorno ben rivestito di $x_0$ si ha $p^{-1}(U) \amalg V_i$.\\
Ora essenndo $\tilde{f}(y) \neq \tilde{g}(y)$ deduco che $\tilde{f}(y) \in V_{i_0}$ e $\tilde{g}(t) \in V_{i_1}$ con $i_0 \neq i_1$.\\
Dalla continuit\`a delle funzioni $\exists W \in I(y)$ con $\tilde{f}(W)\subset V_{i_0}$ e $\tilde{g}(W)\subset V_{i_1}$.\\
Ora $V_{i_0}$ e $V_{i_1}$ sono disgiunti, da cui $W \in Y \sbarra \Omega$
\end{prop}
\spazio
\begin{prop}[Esistenza del sollevamento di cammini]\bianco
Siano $p:\, E \to X$ e $\gamma :\, [0,1] \to X$ continua .\\
Sia $x_0 =\gamma(0)$ e $\tilde{x_0} = p^{-1}(x_0)$, allora $\exists ! \tilde{\gamma}:\, [0,1]\to E $ continua con $\tilde{\gamma}(0)=\tilde{x_0}$ e $\gamma=p \circ \tilde{\gamma}$
\proof L'unicit\`a segue dalla proposizione precedente essendo $[0,1]$ connesso per archi e avendo fissato $\tilde{x_0}$.\\
Ricopriamo $X$ con aperti ben rivestiti $\ds \{ U_i\}_{i \in I}$ e sia $\varepsilon>0$ un numero di Lebesgue per il ricoprimento .\\
Se $\frac{1}{n}< \varepsilon$ $\forall k =0, \dots, n-1$ si ha $\gamma \left( \left[ \frac{k}{n},\frac{k+1}{n} \right] \right)\subset U_k$ ben rivestito.\\
Definisco induttivamente $\tilde{\gamma}$ su $\left[ \frac{k}{n},\frac{k+1}{n} \right]$ come segue.\\
Per definizione di rivestimento $\exists V_0$ aperto di $E$ tale che $\tilde{x_0}\in V_0$ e $p_0=p_{\vert V_0}$ \`e un omeomorfismo.\\
$\forall t\in \left[ 0, \frac{1}{n} \right]$ pongo $\tilde{\gamma}(t)= p_0^{-1} (\gamma(t))$.\\
Una volta definito $\tilde{\gamma}$ continua su $ \left[ 0 ,\frac{k}{n}\right]$, trovo $V_k \subset E$ tale che $\tilde{\gamma}\left( \frac{k}{n} \right) \in V_k$ e $p_k= p_{\vert V_k}$ omeomorfismo.\\
Pongo $\forall  t \in \left[ \frac{k}{n},\frac{k+1}{n} \right]$ $\tilde{\gamma}=p_k^{-1}(\gamma(t))$.\\
\endproof
\end{prop}
\end{document}