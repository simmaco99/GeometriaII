\documentclass[a4paper,12pt]{article}
\usepackage[a4paper, top=2cm,bottom=2cm,right=2cm,left=2cm]{geometry}

\usepackage{bm,xcolor,mathdots,latexsym,amsfonts,amsthm,amsmath,
					mathrsfs,graphicx,cancel,tikz-cd,hyperref,booktabs,caption,amssymb,amssymb,wasysym}
\hypersetup{colorlinks=true,linkcolor=blue}
\usepackage[italian]{babel}
\usepackage[T1]{fontenc}
\usepackage[utf8]{inputenc}
\newcommand{\s}[1]{\left\{ #1 \right\}}
\newcommand{\sbarra}{\backslash} %% \ 
\newcommand{\ds}{\displaystyle} 
\newcommand{\alla}{^}  
\newcommand{\implica}{\Rightarrow}
\newcommand{\iimplica}{\Leftarrow}
\newcommand{\ses}{\Leftrightarrow} %se e solo se
\newcommand{\tc}{\quad \text{ t. c .} \quad } % tale che 
\newcommand{\spazio}{\vspace{0.5 cm}}
\newcommand{\bbianco}{\textcolor{white}{,}}
\newcommand{\bianco}{\textcolor{white}{,} \\}% per andare a capo dopo 																					definizioni teoremi ...


% campi 
\newcommand{\N}{\mathbb{N}} 
\newcommand{\R}{\mathbb{R}}
\newcommand{\Q}{\mathbb{Q}}
\newcommand{\Z}{\mathbb{Z}}
\newcommand{\K}{\mathbb{K}} 
\newcommand{\C}{\mathbb{C}}
\newcommand{\F}{\mathbb{F}}
\newcommand{\p}{\mathbb{P}}

%GEOMETRIA
\newcommand{\B}{\mathfrak{B}} %Base B
\newcommand{\D}{\mathfrak{D}}%Base D
\newcommand{\RR}{\mathfrak{R}}%Base R 
\newcommand{\Can}{\mathfrak{C}}%Base canonica
\newcommand{\Rif}{\mathfrak{R}}%Riferimento affine
\newcommand{\AB}{M_\D ^\B }% matrice applicazione rispetto alla base B e D 
\newcommand{\vett}{\overrightarrow}
\newcommand{\sd}{\sim_{SD}}%relazione sx dx
\newcommand{\nvett}{v_1, \, \dots , \, v_n} % v1 ... vn
\newcommand{\ncomb}{a_1 v_1 + \dots + a_n v_n} %a1 v1 + ... +an vn
\newcommand{\nrif}{P_1, \cdots , P_n} 
\newcommand{\bidu}{\left( V^\star \right)^\star}

\newcommand{\udis}{\amalg}
\newcommand{\ric}{\mathfrak{U}}
\newcommand{\inclu}{\hookrightarrow }
%ALGEBRA

\newcommand{\semidir}{\rtimes}%semidiretto
\newcommand{\W}{\Omega}
\newcommand{\norma}{\vert \vert }
\newcommand{\bignormal}{\left\vert \left\vert}
\newcommand{\bignormar}{\right\vert \right\vert}
\newcommand{\normale}{\triangleleft}
\newcommand{\nnorma}{\vert \vert \, \cdot \, \vert \vert}
\newcommand{\dt}{\, \mathrm{d}t}
\newcommand{\dz}{\, \mathrm{d}z}
\newcommand{\dx}{\, \mathrm{d}x}
\newcommand{\dy}{\, \mathrm{d}y}
\newcommand{\amma}{\gamma}
\newcommand{\inv}[1]{#1^{-1}}
\newcommand{\az}{\centerdot}
\newcommand{\ammasol}[1]{\tilde{\gamma}_{\tilde{#1}}}
\newcommand{\pror}[1]{\mathbb{P}^#1 (\R)}
\newcommand{\proc}[1]{\mathbb{P}^#1(\C)}
\newcommand{\sol}[2]{\widetilde{#1}_{\widetilde{#2}}}
\newcommand{\bsol}[3]{\left(\widetilde{#1}\right)_{\widetilde{#2}_{#3}}}
\newcommand{\norm}[1]{\left\vert\left\vert #1 \right\vert \right\vert}
\newcommand{\abs}[1]{\left\vert #1 \right\vert }
\newcommand{\ris}[2]{#1_{\vert #2}}
\newcommand{\vp}{\varphi}
\newcommand{\vt}{\vartheta}
\newcommand{\wt}[1]{\widetilde{#1}}
\newcommand{\pr}[2]{\frac{\partial \, #1}{\partial\, #2}}%derivata parziale
%per creare teoremi, dimostrazioni ... 
\theoremstyle{plain}
\newtheorem{thm}{Teorema}[section] 
\newtheorem{ese}[thm]{Esempio} 
\newtheorem{ex}[thm]{Esercizio} 
\newtheorem{fatti}[thm]{Fatti}
\newtheorem{fatto}[thm]{Fatto}

\newtheorem{cor}[thm]{Corollario} 
\newtheorem{lem}[thm]{Lemma} 
\newtheorem{al}[thm]{Algoritmo}
\newtheorem{prop}[thm]{Proposizione} 
\theoremstyle{definition} 
\newtheorem{defn}{Definizione}[section] 
\newcommand{\intt}[2]{int_{#1}^{#2}}
\theoremstyle{remark} 
\newtheorem{oss}{Osservazione} 
\newcommand{\di }{\, \mathrm{d}}
\newcommand{\tonde}[1]{\left( #1 \right)}
\newcommand{\quadre}[1]{\left[ #1 \right]}
\newcommand{\w}{\omega}

% diagrammi commutativi tikzcd
% per leggere la documentazione texdoc

\begin{document}
\textbf{Lezione del 5 Dicembre del Prof. Frigerio}
\begin{oss}Quando non altro specificato, supponiamo tutte le funzioni continue
\end{oss}
\begin{defn}Siano $f,g:\, X \to Y$ continue.\\
Una \textbf{ omotopia } tra $f$ e $g$ \`e una mappa continua 
$$ H:\, X \times [0,1]\to Y \quad H(x,0)=f(x) \quad  H(x,1)=g(x)\quad \forall x\in X $$
Nel seguito indicheremo l'intervallo $[0,1]$ con $I$
\end{defn}
\begin{oss}$\forall t\in [0,1]$ la mappa $H_t(x)=H(x,t)$ \`e continua per cui $H$ descrive un'interpolazione continua tra $f$ e $g$: deforma $f$ in $g$
\end{oss}
\begin{defn}$f$ si dice \textbf{ omotopa } a $g$  e si indica con $f \sim g$ se esiste un  omotopia tra $f$ e $g$
 \end{defn}
 \begin{prop}Essere omotopi \`e una relazione di equivalenza sull'insieme $C(X,Y)$ delle funzioni continue da $X$ a $Y$.\\
 L'insieme delle classi di omotopia di tali funzioni si denota con $[X,Y]$
 \proof \bbianco
 \begin{itemize}
 \item Riflessiva: $f \sim f$ infatti basta prendere $H(x,t)=f(x)$ $\forall x \in X$ e $\forall t\in I$ 
 \item Transitiva. Sia $H$ l'omotopia tra $f$ e $g$ allora $K(x,t)=H(x,1-t)$ \`e un omotopia tra $g$ e $f$
 \item Transitiva. Sia $H$ \`e l'omotopia tra $f$ e $g$ e $K$ \`e l'omotopia tra $g$ e $h$.\\
 Costruiamo un omotopia tra $f$ e $h$: $J:\, X \times [0,1] \to Y$ cos\`i definita 
 $$ J(x,t)= \begin{cases} H(x,2t) \text{ se } t \in \left[ 0, \frac{1}{2} \right] \\
 K(x,2t-1) \text{ se } t \in \left[ \frac{1}{2},1 \right]
 \end{cases}$$
 $J$ \`e continua in quanto \`e ben definita ed inoltre \`e continua la restrizione sui chiusi $X\times \left[ 0, \frac{1}{2} \right]$ e $ X\times \left[ \frac{1}{2},1 \right]$ (sono ricoprimento fondamentale)
 \end{itemize}
\endproof
 \end{prop}
 \begin{ese}Se $Y$ \`e convesso di $\R^n$ (e.g $\R^n$ stesso), allora $\vert [ X , Y] \vert =1$ cio\`e tutte le mappe $f:\, X \to Y$ sono omotope tra loro.\\
 Date $f,g:\, X \to Y$ la funzione $H(x,t)= t f(x) + (1-t) g(x)$ \`e ben definita essendo $Y$ convesso ed inoltre \`e continua, dunque \`e l'omotopia cercata
 \begin{oss} In realt\`a basta $Y$ stellato rispetto a $p\in Y$.\\
 $H(x,t)=tf(x)+ (1-t)p$ dunque $H$ \`e un omotopia tra $f$ e la costante $p$, da cui la tesi per transitivit\`a di $\sim $ 
 \end{oss}
 \end{ese}
 \spazio
 \begin{defn}Sia $X$ uno spazio topologico, denotiamo con $\pi_0(X)$ l'insieme delle componenti connesse per archi di $X$
 \end{defn}
 \begin{defn}Sia $f:\, X \to Y$ allora tale funzione induce una ben definita funzione
 $$ f_\star : \, \pi_0(X) \to \pi_0(Y)$$ 
 definita in modo che $f(C) \subseteq f_\star(C)$ $\forall C \in \pi_0(X)$ 
 \begin{oss}$f_\star$ manda una componente connessa di $X$ nell'unica componente connessa di $Y$ che contiene $f(C)$ (le componenti connesse sono disgiunte)
 \end{oss}
 \end{defn}
 \begin{lem}Se $f\sim g$ allora $f_\star=g_\star$
 \proof Sia $C \in \pi_0(X)$ e sia $x_0 \in C$.\\
 Se $H$ \`e un omotopia tra $f$ e $g$, la mappa $$\gamma:\, [0,1]\to Y \quad \gamma(t) = H(x_0,t)$$
 \`e un cammino continuo in $Y$ che congiunge $f(x_0)$ e $g(x_0)$.\\
 $f(x_0)$ e $g(x_0)$ giacciono nella stessa componente connessa per archi di $Y$ che \`e sia $f_\star(C)$ ( contiene $f(x_0)$) sia $g_\star(C)$ (contiene $g(x_0)$.\\
 Dato che le componenti connesse sono disgiunte si ottiene $f_\star(C)=g_\star(C)$
 \endproof
 \end{lem}
 \begin{cor}Se $X \subseteq \R^n$ \`e stellato rispetto a $p$, allora c'\`e una biezione tra $[X,Y]$ e $\pi_0(Y)$ 
 \proof Essendo $X$ stellato, \`e connesso per archi ovvero $ \vert \pi_0(X)\vert=1$.\\
 Definiamo 
 $$ \psi:\, C (X,Y) \to \pi_0(Y) \quad \psi(f)= f_\star(X)$$
 Per il lemma $\psi$ induce una ben definita funzione $\varphi;\, [X,Y ]\to \pi_0(Y)$, mostriamo che $\varphi$ \`e biettiva
 \begin{itemize}
 \item Suriettiva. Dato $C\in \pi_0(Y)$ scelgo $y \in C$ e pongo $f(x)=y$ allora $f$ \`e continua e $\psi(f)=C$ da cui $\varphi([f])=C$ 
 \item Iniettiva. Data $f\in C (X,Y)$ allora $f$ \`e omotopa alla costante $f(p)$ in quanto $H(x,t)= f( tx+(1-t)p)$  \`e ben definita in quanto $X$ stellato.\\
 Dato $f,g$ con $\psi(f)=\psi(g)$ abbiamo $f(p)$ e $g(p)$ vivono nella stessa componente connessa per archi di $Y$ e dunque le costanti $f(p)$ e $g(p)$ sono omotope tramite $H(x,t)=\delta$ dove $\delta$ \`e un arco che congiunge $f(p)$ e $g(p)$.\\
 $ f \sim f(p) \sim g(p) \sim g$ dunque $[f]=[g]$\endproof
 \end{itemize}
 \end{cor}
 \spazio
 \begin{defn}$f:\, X \to Y$ \`e un' \textbf{ equivalenza omotopica } se ammette un inversa omotopica cio\`e 
 $$ g:\, Y \to X $$ 
 tali che $f \circ g \sim Id_Y$ e $ g \circ f \sim Id_X$.\\
 Due spazi si dicono \textbf{ omotopicamente equivalenti} (o omotopici) se esiste un'equivalenza omotopica tra di loro
 \end{defn}
 \begin{prop}Essere omotopici \`e una relazione di equivalenza, la transitivit\`a si mostra usando il seguente 
 \begin{lem}Siano $f_0,f_1:\, X \to Y$ e $g_0,g_1:\, Y \to Z $ continue $$ f_0 \sim f_1 \text{ e } g_0 \sim g_1 \quad \implica \quad g_0 \circ f_0 \sim g_1 \circ f_1$$
 \proof Sia $H$ \`e l'omotopia tra $f_0$ e $f_1$ e $K$ l'omotopia tra $g_0$ e $g_1$.\\
 La mappa $ (x,t) \to K(H(x,t))$ \`e un'omotopia tra $g_0 \circ f_0 $ e $g_1 \circ f_1$
 \end{lem}
 \end{prop}
 \spazio
 \begin{defn}$X$ \`e  \textbf{ contraibile } se \`e omotopicamente equivalente ad un punto
 \end{defn}
\begin{prop}$X \subseteq \R^n$ stellato $\implica$ $X$ contraibile
\proof Sia $Y=\{ q\}$, definiamo allora le funzioni
$$ f:\, X \to Y \quad f(x)=q $$ 
$$ g:\, Y \to X \quad g(q)=x_0 \text{ a caso } $$
ora $f$ e $g$ sono continue inoltre $f \circ g = Id_Y \sim Id_Y$ mentre $g \circ f $ \`e omotopa a $Id_X$ poich\`e $X$ stellato per cui tutte le funzioni sono omotope tra loro
\endproof
\end{prop}
\begin{prop}Se $f:\, X \to Y$  \`e equivalenza omotopica allora $f_\star:\, \pi_0(X) \to \pi_0(Y)$ \`e una bigezione
\proof Segue dal fatto che le mappe omotope inducono la stessa mappa sui $\pi_0$ ed inoltre $( f\circ g)_\star = f_\star \circ g_\star$
\end{prop}
\begin{oss}$X$ contrattile $\implica$ $X$ connesso per archi.\\
Essendo contrattile esiste una bigezione tra $\pi_0(X)$ e le componente connesse per archi del punto, Ora l'insieme fatto da un solo punto ha una sola componente connessa, dunque anche $X$ ha una sola componente connessa per archi
\end{oss}
\begin{defn}Sia $X$ topologico. $C \subseteq X$ di dica
\begin{itemize}
\item \textbf{Retratto } se $\exists r:\, X \to C$ (retrazione) continua tale che $r(x)=x$ $\forall x\in C$
\item \textbf{Retratto di deformazione} se esiste $r$ come sopra.\\
Inoltre esiste un omotopia $H$ tra $Id_X$ e $i \circ r$ tale che  $H(x,t)=x$ $\forall c \in C $ e $\forall t \in [0,1]$.\\
Dove $i:\, C \to X$ \`e l'inclusione
\end{itemize}
\end{defn}
\begin{ese}Se $p\in X$ allora $\{ p\}$ \`e un retratto di $X$
\end{ese}
\begin{ese}$S^n$ \`e un retratto di deformazione di $\R^{n+1}\sbarra \{0\}$.\\
La retrazione \`e data da $f(x)= \frac{x}{\norma x \norma}$.\\
L'omotopia tra $i \circ r $ e l'identit\`a \`e dato da 
$H(x,t)=(1-t) x + t \frac{x}{\norma x \norma}$ 
\end{ese}
\end{document}