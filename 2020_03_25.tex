\documentclass[a4paper,12pt]{article}
\usepackage[a4paper, top=2cm,bottom=2cm,right=2cm,left=2cm]{geometry}

\usepackage{bm,xcolor,mathdots,latexsym,amsfonts,amsthm,amsmath,
					mathrsfs,graphicx,cancel,tikz-cd,hyperref,booktabs,caption,amssymb,amssymb,wasysym}
\hypersetup{colorlinks=true,linkcolor=blue}
\usepackage[italian]{babel}
\usepackage[T1]{fontenc}
\usepackage[utf8]{inputenc}
\newcommand{\s}[1]{\left\{ #1 \right\}}
\newcommand{\sbarra}{\backslash} %% \ 
\newcommand{\ds}{\displaystyle} 
\newcommand{\alla}{^}  
\newcommand{\implica}{\Rightarrow}
\newcommand{\iimplica}{\Leftarrow}
\newcommand{\ses}{\Leftrightarrow} %se e solo se
\newcommand{\tc}{\quad \text{ t. c .} \quad } % tale che 
\newcommand{\spazio}{\vspace{0.5 cm}}
\newcommand{\bbianco}{\textcolor{white}{,}}
\newcommand{\bianco}{\textcolor{white}{,} \\}% per andare a capo dopo 																					definizioni teoremi ...


% campi 
\newcommand{\N}{\mathbb{N}} 
\newcommand{\R}{\mathbb{R}}
\newcommand{\Q}{\mathbb{Q}}
\newcommand{\Z}{\mathbb{Z}}
\newcommand{\K}{\mathbb{K}} 
\newcommand{\C}{\mathbb{C}}
\newcommand{\F}{\mathbb{F}}
\newcommand{\p}{\mathbb{P}}

%GEOMETRIA
\newcommand{\B}{\mathfrak{B}} %Base B
\newcommand{\D}{\mathfrak{D}}%Base D
\newcommand{\RR}{\mathfrak{R}}%Base R 
\newcommand{\Can}{\mathfrak{C}}%Base canonica
\newcommand{\Rif}{\mathfrak{R}}%Riferimento affine
\newcommand{\AB}{M_\D ^\B }% matrice applicazione rispetto alla base B e D 
\newcommand{\vett}{\overrightarrow}
\newcommand{\sd}{\sim_{SD}}%relazione sx dx
\newcommand{\nvett}{v_1, \, \dots , \, v_n} % v1 ... vn
\newcommand{\ncomb}{a_1 v_1 + \dots + a_n v_n} %a1 v1 + ... +an vn
\newcommand{\nrif}{P_1, \cdots , P_n} 
\newcommand{\bidu}{\left( V^\star \right)^\star}

\newcommand{\udis}{\amalg}
\newcommand{\ric}{\mathfrak{U}}
\newcommand{\inclu}{\hookrightarrow }
%ALGEBRA

\newcommand{\semidir}{\rtimes}%semidiretto
\newcommand{\W}{\Omega}
\newcommand{\norma}{\vert \vert }
\newcommand{\bignormal}{\left\vert \left\vert}
\newcommand{\bignormar}{\right\vert \right\vert}
\newcommand{\normale}{\triangleleft}
\newcommand{\nnorma}{\vert \vert \, \cdot \, \vert \vert}
\newcommand{\dt}{\, \mathrm{d}t}
\newcommand{\dz}{\, \mathrm{d}z}
\newcommand{\dx}{\, \mathrm{d}x}
\newcommand{\dy}{\, \mathrm{d}y}
\newcommand{\amma}{\gamma}
\newcommand{\inv}[1]{#1^{-1}}
\newcommand{\az}{\centerdot}
\newcommand{\ammasol}[1]{\tilde{\gamma}_{\tilde{#1}}}
\newcommand{\pror}[1]{\mathbb{P}^#1 (\R)}
\newcommand{\proc}[1]{\mathbb{P}^#1(\C)}
\newcommand{\sol}[2]{\widetilde{#1}_{\widetilde{#2}}}
\newcommand{\bsol}[3]{\left(\widetilde{#1}\right)_{\widetilde{#2}_{#3}}}
\newcommand{\norm}[1]{\left\vert\left\vert #1 \right\vert \right\vert}
\newcommand{\abs}[1]{\left\vert #1 \right\vert }
\newcommand{\ris}[2]{#1_{\vert #2}}
\newcommand{\vp}{\varphi}
\newcommand{\vt}{\vartheta}
\newcommand{\wt}[1]{\widetilde{#1}}
\newcommand{\pr}[2]{\frac{\partial \, #1}{\partial\, #2}}%derivata parziale
%per creare teoremi, dimostrazioni ... 
\theoremstyle{plain}
\newtheorem{thm}{Teorema}[section] 
\newtheorem{ese}[thm]{Esempio} 
\newtheorem{ex}[thm]{Esercizio} 
\newtheorem{fatti}[thm]{Fatti}
\newtheorem{fatto}[thm]{Fatto}

\newtheorem{cor}[thm]{Corollario} 
\newtheorem{lem}[thm]{Lemma} 
\newtheorem{al}[thm]{Algoritmo}
\newtheorem{prop}[thm]{Proposizione} 
\theoremstyle{definition} 
\newtheorem{defn}{Definizione}[section] 
\newcommand{\intt}[2]{int_{#1}^{#2}}
\theoremstyle{remark} 
\newtheorem{oss}{Osservazione} 
\newcommand{\di }{\, \mathrm{d}}
\newcommand{\tonde}[1]{\left( #1 \right)}
\newcommand{\quadre}[1]{\left[ #1 \right]}
\newcommand{\w}{\omega}

% diagrammi commutativi tikzcd
% per leggere la documentazione texdoc

\begin{document}
\section{Ancora sul logaritmo complesso}
\begin{prop}Sia $\log:\, D \to \C$ una branca del logaritmo ($D$ aperto connesso di $\C$) allora $\log$ \`e olomorfa con $log'(z)= \frac{1}{z}$ ($0\not \in D$)
\proof $\log(exp(x))=z+c$ dunque $\log'(exp(z))exp'(z) = 1$ da cui essendo $exp'(z) = exp(z)$ si ha $\log'(exp(z))= \frac{1}{exp(z)}$\\
(Detto bene: fissato $z_0\in D$, esiste $y_0\in\C$ con $z_0=exp(y_0)$.\\
Per continuit\`a di $exp$, dato un intorno $V$ di $z_0$ in $D$, esiste un intorno $W$ di $y_0$ in $\C$  con $exp(W)\subseteq V$.\\
Le uguaglianze di sopra sono verificate per $y\in W$ e $z\in Y$)
\end{prop}
\begin{oss}Nella proposizione abbiamo usato il seguente fatto:\\
Sia $f:\, D \to D'$ \`e olomorfa e bigettiva e $g:\, D'\to D$ \`e l'inversa di $f$.\\
Se $f'(z_0)\neq 0 $ allora $g$ \`e olomorfa in $f(z_0)$ e vale 
$$g'(f(z_0))=\frac{1}{f'(z_0)}$$
Infatti da $g\circ f = Id $ e dal teorema della funzione inversa (Analisi II) poich\`e $f'(z_0) = a+ ib \neq 0 $, 
$d_{f_{z_0}}=\begin{pmatrix}
a & - b \\   b & a
\end{pmatrix}$ \`e invertibile.\\
Per cui $g$ \`e differenziabile in $f(z_0)$ e $d_{g_{f(z_0)}}= (d_{f_{z_0}})^{-1}$.\\
Infine si verifica facilmente che se $A=\begin{pmatrix}
 a & -b \\ b  & a 
\end{pmatrix}$ con $a+ib\neq 0 $ allora ha come inversa $\begin{pmatrix}
c & -d \\ d & c 
\end{pmatrix}$ con $c+id = \frac{1}{a+ib}$
\end{oss}
\spazio
\begin{prop}Sia $D=B(0,1)$ e sia $\log:\{ Re(z)>0\} \to \C$ la branca principale.\\
Allora $\forall z\in D$ vale $\log(1+z) =\sum (-1)^{n+1} \frac{z^n}{n}$
\proof Siano $f,g:\, D \to C$ con $f(z) = \log(1+z)$ e $g(z) =\sum (-1)^{n+1} \frac{z^n}{n}$.\\
Poich\`e il raggio di convergenza della serie data \`e $1$, $g$ \`e ben definita e analitica, dunque olomorfa e 
$$g'(z) = \sum_{n\geq 1}  (-1)^{n+1} n \frac{z^{n-1}}{n} = \sum_{n\geq 1} (-1)^{n+1} z^{n-1} =\sum (-1)^n z^n = \sum (-z)^n = \frac{1}{1-(-z)}=\frac{1}{1+z}$$
Perci\`o se $h=f-g$ si ha $h'(z) = f'(z) - g'(z) = 0$ per $z\in D$.\\
Dunque $h'\equiv 0$ in $D$, dunque $h$ \`e costante in $D$, $h(z)=c$ ma $h(0)=0$ da cui $f=g$
\endproof
\end{prop}
\newpage
\begin{prop}Sia $D$ aperto connesso di $\C$ e $f:\, D\to \C$ analitica.\\
I seguenti fatti sono equivalenti
\begin{itemize}
\item[(i)]$\exists z_0\in D$ con $f^{(n)} (z_0)=0$ per ogni $n \in \N$
\item[(ii)] $\exists U\subseteq D$ aperto con $\ris  f U \equiv 0$
\item[(iii)] $f\equiv 0 $
\end{itemize}
\proof \bbianco
\begin{itemize}
\item $(iii) \implica (i)$ ovvio
\item $(i) \implica (ii)$ Per analicit\`a $f(z)= \sum a_n (z-z_0) $ per $z\in B(z_0,R)$ per qualche $R>0$.\\
Ora $a_n = \frac{f^{(n)}(z_0)}{n!}$ (vedi alla fine della dimostrazione).\\
Dunque se $f^{(n)}(z_0)=0$ allora $a_n=0$ perci\`o $f\equiv 0 $ su $B(z_0,R)$
\item $(ii)\implica (iii)$ Sia $\Omega\subseteq D $ 
$$\Omega=\{ z\in D \, \vert \, \exists U \ni z \text{ aperto con }\ris f  U\equiv 0\}$$
Essendo $D$ connesso, basta provare che \`e aperto e chiuso.\\
$\Omega$ \`e aperto per definizione, proviamo che \`e chiuso .\\
Sia $z\in D$ con $z=\lim z_n$ dove $z_n \in \W$\\
Ora $f^{(k)}(z_n)=0$ per ogni $n,k\in \N$ (in quanto se $z_n \in W$ allora $f\equiv 0$ in un intorno di $z_0$ dunque $f^{(k)}(z_n)=0$ per ogni $k\in \N$).\\
Ma $f^{(k)}$ \`e continua, dunque $f^{(k)}(z) = \lim_n f^{(k)}(z_n)$.\\
Ora tutte le derivate di $f$ si annullano in $z_0$ dunque $z\in \W$ (ripercorrere la dimostrazione $(i)\implica (ii)$)
\end{itemize}
\end{prop}
\begin{oss}
Nella dimostrazione abbiamo usato che se $f$ \`e analitica anche $f'$ lo \`e e se $f(z) =\sum a_n z^n $ allora $f'(z) =\sum_{n\geq 1 }  a_n n z^{n-1}$.\\
Da cui $f'$ \`e analitica, dunque derivabile.\\
Iterando otteniamo che $f$ \`e derivabile infinite volte e 
$f^{(n)}(z_0)=a_n n!$
\end{oss}
\begin{oss}L'enunciato della proposizione \`e falso se si suppone f $C^\infty$.\\
Consideriamo la funzione 
$$f:\, \C\to \C \quad f(a+ib)=
 \begin{cases} e^{-\frac{1}{a}} \text  { se } a >0 \\  0 \text{ se } a\leq 0 \end{cases}$$
Ora $f$ \`e $C^\infty$ si annulla in $\{ Re(z) \leq 0 \}$ dunque su un aperto di $\C$ ma non \`e nulla su $\C$ (l'insieme $\W$ sopra definito non \`e chiuso)
\end{oss}
\spazio
\begin{cor}
Sia $D$ aperto connesso di $\C$ e siano $f,g:\, D \to \C$ analitiche.
\begin{itemize}
\item Se $f=g$ su un aperto $U\subseteq D$ allora $f=g$ su $D$ 
\item Se $\exists z_0\in D$ con $f^{(n)}(z_0)=g^{(n)}(z_0)$ per ogni $n\in \Z$ allora $f=g$ su $D$
\end{itemize}
\proof Basta applicare il teorema precedente a $h=f-g$
\end{cor}
\begin{cor}L'anello delle funzioni analitiche su $D$ (aperto connesso di $\C$) \`e un dominio di integrit\`a
\proof Se $f,g:\, D \to \C$ sono tali che $fg\equiv 0 $ allora siano:
$$A=\{ z \, \vert \, f(z)=0\} \quad B=\{ z \, vert \, g(z)=0\}$$
Allora $D=A\cup B$, essendo $A,B$ chiusi allora umo di essi ha parte interna non vuota (ex), supponiamo $A^\circ\neq 0 $ da cui $f\equiv 0$ su un aperto e dunque su $D$
\end{cor}
\newpage
\section{Zeri di funzioni analitiche}
\begin{defn}Sia $f:\, D \to \C$ analitica con $f\not \equiv 0 $.\\
$\forall z_0\in D $ definiamo $$ord_{z_0}(f)=\min\{ n\in \N \, \vert \, f^{(n)}(z_0)\neq 0\}$$
\begin{oss}Poich\`e la funziono non \`e identicamente nulla, l'insieme di cui cerchiamo il minimo \`e non vuoto
\end{oss}
\begin{oss}$f(z_0)=0 \, \implica ord_{z_0}(f)\geq 1$
\end{oss}
\begin{defn}Uno zero $z_0$ di $f$  si dice semplice se $ord_{z_0}(f) =1$\\
Uno zero $z_0$ di $f$  si dice semplice se $ord_{z_0}(f) >1$\\
\end{defn}
\end{defn}
\begin{oss}Se $f(z)=\sum a_n(z-z_0)$, poich\`e $f^{(n)}(z_0)=n!a_n$ allora $$ord_{z_0}(f)=\min\{ n \in \N\, \vert \, a_n \neq 0 \}$$
\end{oss}
\begin{prop} $f:\, D \to C$ con $D$ aperto connesso di $\C$.\\
Allora si ha
$$f(z) = (z-z_0)^{ord_{z_0}(f)}g(z)$$
con $g:\, D \to \C$ analitico e con $g(z)\neq 0 $ in un intorno di $z_0$
\proof Sia $k=ord_{z_0}(f)$.\\
In un intorno di $z_0$ si ha
$$f(z)= \sum a_n(z-z_0)^n =\sum_{n\geq k} a_n (z-z_0)^n =(z-z_0)^k \sum_{n \geq k} a_n (z-z_0)^{n-k}=(z-z_0)^k \sum a_{n+k} (z-z_0)^n$$
Sia $g(z)=\sum a_{n+k} (z-z_0)^n$, la serie di potenze che definisce $g$ ha lo stesso raggio di convergenza di $f$\\
In  particolare $\exists U$ intorno di $z_0$ tale che $f(z) = (z-z_0)^k g(z)$.\\
Ora $g(z_0)=a_k\neq 0$, dunque esiste un intorno di $z_0$ con $g$ non identicamente nulla nell'intorno
\end{prop}
\begin{cor}Sia $f:\, D \to \C$ ($D$ aperto connesso di $\C$) e $f\not\equiv 0$\\
Allora $C=\{ z \in D \, \vert \, f(z)=0\}$ \`e discreto e chiuso in $D$
\proof Sia $z_0\in C$ con $k=ord_{z_0}(f)$ allora in un intorno $U$ di $z_0$ si ha $$f(z)=(x-x_0)^k g(z)$$ e tale che $g(z)\neq 0 $ per $z\in U$\\
Se $z\in U \setminus\{ z_0\}$ allora $C\cap U =\{ z_0\}$ dunque \`e discreto.\\
$C$ \`e chiuso essendo pre-immagine di $\{0\}$ mediante una funzione continua
\end{cor}
\newpage
\section{1-forme differenziali complesse}
$\C$ pu\`o essere visto come un $\R$-spazio vettoriale di dimensione $2$ con base $\{ 1 , i\}$.\\
Fissando questa base si ha $End_\R(\C)\cong M(2,2,\R)$ (con $End_\R(\C)$ indichiamo l'insieme degli endomorfismi di $\C$ che sono $\R$-lineari).\\
L'isomorfismo \`e dato da 
$$\psi \to \begin{pmatrix}
Re (\psi(1)) & Re (\psi(i)) \\
Im(\psi(1)) & Im(\psi(i))
\end{pmatrix}$$
Notiamo che 
$$\di x:\, \C\to \C \quad \di x(a+ib)=a$$
$$\mathrm{d}y:\, \C\to \C \quad \di y(a+ib)=b$$
\`e una base di $End_\R(\C)$ inteso come $\C$-spazio vettoriale
\begin{defn}Sia $D$ un aperto di $\C$ una 1-forma differenziale complessa su $D$ \`e una funzione 
$\w:\,D \to End_\R(\C)$ continua (rispetto alla topologia di $M(2,2,\R)\cong \R^4$)
\end{defn}
\begin{oss}Pi\`u concretamente una 1-forma differenziale complessa $\w$ su $D$ corrisponde a $2$ funzioni $P,Q:\, D \to \C$ tali che 
$$\w(z)=P(z)\di x + Q(z) \di y = P\di x + Q\di y $$
\end{oss}
\begin{oss}
La continuit\`a di $\w$ deriva da quella di $P$ e $Q$ \\
Infatti se $$P(z) = a(z) + i b(z) \text{ e }  Q(z) = c(z) + i d(z)$$
$$\w(z)(1) = P(z) \di x (1) + Q(z)\di y(1)  = P(z) = a(z)= i b(z)$$
$$\w(z)(i) = P(z) \di x (i) + Q(z)\di y(i)  = Q(z) = c(z)= i d(z)$$ 
dunque 
$$\w(z) =\begin{pmatrix}
a(z) & c(z) \\ b(z) & d(z) 
\end{pmatrix}$$ 
da cui $\w$ \`e continua se e solo se $a,b,c,d$ lo sono, se e solo se $P,Q$ lo sono 
\end{oss}
\begin{ese}Se $f:\, D \to \C$ \`e $C^1$ allora $\di f $ \`e una 1-forma differenziale complessa, data da 
$$\di f =\begin{pmatrix}
 \pr {Re(f} u & \pr {Re(f)} v \\
 \pr {Im(f)} u & \pr {Im(f)} v 
\end{pmatrix}$$
\end{ese} 
\begin{oss}Un'altra base utile di $End_\R(\C)$ \`e data da 
$$\di z = \di x + i \di y $$
$$\di \overline{z}=\di x - i \di y$$
$\tonde{ \text{\`e una base in quanto } \di x =\frac{\di z + \di \overline{z} }{2} \text{ e } \di y = \frac{\di z - \di \overline{z}}{2i}= -i \frac{\di z - \di \overline{z}}{2}}$
Ora se $f$ \`e differenziabile allora 
$$ \di f = \pr f x \di x + \pr f y \di y = $$ $$=\pr f x \tonde{\frac{\di z + \di \overline{z}}{2}}+\pr f y \tonde{ -\frac{i}{2}(\di z -\di \overline{z})} =$$
$$=\frac{1}{2}\tonde{\pr f x - \pr f y }\di z +\frac{1}{2}\tonde{\pr f x + i \pr f y }\di \overline{z} $$
\end{oss}
\begin{defn}Sia $f$ differenziabile 
$$\pr f z =\frac{1}{2}\tonde{\pr f x - \pr f y }$$
$$\pr f {\overline{z}} =\frac{1}{2}\tonde{\pr f x + i \pr f y }$$
Perci\`o risulta per costruzione
$$\di f = \pr f z \di z + \pr f {\overline{z}} \di \overline{z}$$
\end{defn}
\begin{oss}$f$ \`e olomorfa $\ses$ $\di f$ \`e $\C$-lineare dunque 
$$ f\text{ olomorfa} \quad \ses \quad \pr f y = \di f (i) = i \di f(1) =   i \pr f x  \quad \ses \quad \pr f x = - i \pr f y \quad \ses \quad \pr f z =0 $$
\end{oss}
\begin{oss}Assumendo $f$ olomorfa 
$$\pr f x (z) = \di f_z(1)= f'(z)$$
$$\pr f y (z) = \di f_z(i) = i \di f_z(1) = i f'(z)$$
Perci\`o 
$$\pr f z = f'(z)$$
\end{oss}
\begin{cor}Se $f$ \`e olomorfa $\di f = f' \di z$
\proof
$\di f = \pr f z \di z + \pr f {\overline{z}} \di \overline{z}= f' \di z + 0 $
\end{cor}
\end{document}