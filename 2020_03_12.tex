\documentclass[a4paper,12pt]{article}
\usepackage[a4paper, top=2cm,bottom=2cm,right=2cm,left=2cm]{geometry}

\usepackage{bm,xcolor,mathdots,latexsym,amsfonts,amsthm,amsmath,
					mathrsfs,graphicx,cancel,tikz-cd,hyperref,booktabs,caption,amssymb,amssymb,wasysym}
\hypersetup{colorlinks=true,linkcolor=blue}
\usepackage[italian]{babel}
\usepackage[T1]{fontenc}
\usepackage[utf8]{inputenc}
\newcommand{\s}[1]{\left\{ #1 \right\}}
\newcommand{\sbarra}{\backslash} %% \ 
\newcommand{\ds}{\displaystyle} 
\newcommand{\alla}{^}  
\newcommand{\implica}{\Rightarrow}
\newcommand{\iimplica}{\Leftarrow}
\newcommand{\ses}{\Leftrightarrow} %se e solo se
\newcommand{\tc}{\quad \text{ t. c .} \quad } % tale che 
\newcommand{\spazio}{\vspace{0.5 cm}}
\newcommand{\bbianco}{\textcolor{white}{,}}
\newcommand{\bianco}{\textcolor{white}{,} \\}% per andare a capo dopo 																					definizioni teoremi ...


% campi 
\newcommand{\N}{\mathbb{N}} 
\newcommand{\R}{\mathbb{R}}
\newcommand{\Q}{\mathbb{Q}}
\newcommand{\Z}{\mathbb{Z}}
\newcommand{\K}{\mathbb{K}} 
\newcommand{\C}{\mathbb{C}}
\newcommand{\F}{\mathbb{F}}
\newcommand{\p}{\mathbb{P}}

%GEOMETRIA
\newcommand{\B}{\mathfrak{B}} %Base B
\newcommand{\D}{\mathfrak{D}}%Base D
\newcommand{\RR}{\mathfrak{R}}%Base R 
\newcommand{\Can}{\mathfrak{C}}%Base canonica
\newcommand{\Rif}{\mathfrak{R}}%Riferimento affine
\newcommand{\AB}{M_\D ^\B }% matrice applicazione rispetto alla base B e D 
\newcommand{\vett}{\overrightarrow}
\newcommand{\sd}{\sim_{SD}}%relazione sx dx
\newcommand{\nvett}{v_1, \, \dots , \, v_n} % v1 ... vn
\newcommand{\ncomb}{a_1 v_1 + \dots + a_n v_n} %a1 v1 + ... +an vn
\newcommand{\nrif}{P_1, \cdots , P_n} 
\newcommand{\bidu}{\left( V^\star \right)^\star}

\newcommand{\udis}{\amalg}
\newcommand{\ric}{\mathfrak{U}}
\newcommand{\inclu}{\hookrightarrow }
%ALGEBRA

\newcommand{\semidir}{\rtimes}%semidiretto
\newcommand{\W}{\Omega}
\newcommand{\norma}{\vert \vert }
\newcommand{\bignormal}{\left\vert \left\vert}
\newcommand{\bignormar}{\right\vert \right\vert}
\newcommand{\normale}{\triangleleft}
\newcommand{\nnorma}{\vert \vert \, \cdot \, \vert \vert}
\newcommand{\dt}{\, \mathrm{d}t}
\newcommand{\dz}{\, \mathrm{d}z}
\newcommand{\dx}{\, \mathrm{d}x}
\newcommand{\dy}{\, \mathrm{d}y}
\newcommand{\amma}{\gamma}
\newcommand{\inv}[1]{#1^{-1}}
\newcommand{\az}{\centerdot}
\newcommand{\ammasol}[1]{\tilde{\gamma}_{\tilde{#1}}}
\newcommand{\pror}[1]{\mathbb{P}^#1 (\R)}
\newcommand{\proc}[1]{\mathbb{P}^#1(\C)}
\newcommand{\sol}[2]{\widetilde{#1}_{\widetilde{#2}}}
\newcommand{\bsol}[3]{\left(\widetilde{#1}\right)_{\widetilde{#2}_{#3}}}
\newcommand{\norm}[1]{\left\vert\left\vert #1 \right\vert \right\vert}
\newcommand{\abs}[1]{\left\vert #1 \right\vert }
\newcommand{\ris}[2]{#1_{\vert #2}}
\newcommand{\vp}{\varphi}
\newcommand{\vt}{\vartheta}
\newcommand{\wt}[1]{\widetilde{#1}}
\newcommand{\pr}[2]{\frac{\partial \, #1}{\partial\, #2}}%derivata parziale
%per creare teoremi, dimostrazioni ... 
\theoremstyle{plain}
\newtheorem{thm}{Teorema}[section] 
\newtheorem{ese}[thm]{Esempio} 
\newtheorem{ex}[thm]{Esercizio} 
\newtheorem{fatti}[thm]{Fatti}
\newtheorem{fatto}[thm]{Fatto}

\newtheorem{cor}[thm]{Corollario} 
\newtheorem{lem}[thm]{Lemma} 
\newtheorem{al}[thm]{Algoritmo}
\newtheorem{prop}[thm]{Proposizione} 
\theoremstyle{definition} 
\newtheorem{defn}{Definizione}[section] 
\newcommand{\intt}[2]{int_{#1}^{#2}}
\theoremstyle{remark} 
\newtheorem{oss}{Osservazione} 
\newcommand{\di }{\, \mathrm{d}}
\newcommand{\tonde}[1]{\left( #1 \right)}
\newcommand{\quadre}[1]{\left[ #1 \right]}
\newcommand{\w}{\omega}

% diagrammi commutativi tikzcd
% per leggere la documentazione texdoc

\begin{document}
\textbf{Seconda parte della lezione del 12  Marzo}
\section{Morfismi di rivestimento}
\begin{defn}Dati $p_1:\, E_1 \to X$ e $p_2:\, E_2 \to X$ rivestimenti.\\
Un morfismo tra $p_1$ e $p_2$ \`e una mappa $\varphi:\, E_1 \to E_2$ tale che $p_2 \circ \varphi=p_1$ cio\`e questo diagramma commuta
$$\begin{tikzcd} E_1 \arrow{rd}{p_1} \arrow{r}{\varphi} & E_2 \arrow{d}{p_2}\\
& X
\end{tikzcd}$$

\end{defn}
\begin{defn}Un morfismo $\varphi$ come sopra  \`e un isomorfismo se $\exists \psi:\, E_2 \to E_1$ con $\psi$ inversa di $\varphi$
\end{defn}
\begin{oss}Composizione di morfismi \`e un morfismo. $Id_E$ \`e un morfismo di $p:\, E \to X$.\\
Dunque
$$ Aut(E)=Aut(E,p)=\{ \varphi:\, E \to E\, \vert \, \text{ isomorfismo} \} $$
tale insieme dotato delle composizione \`e un gruppo
\end{oss}
\spazio 
Fissato un rivestimento connesso $p:\, E \to X$ 
\begin{thm} Valgono i seguenti fatti 
\begin{itemize}
\item[(i)]$Aut(E)$ agisce su $E$ in maniera propriamente discontinua
\item[(ii)] $Aut(E)$ agisce sulle fibre di $E$
\item[(iii)] Se $F$ \`e una fibra  di $E$, $\tilde{x}_0, \tilde{x}_1\in F$ allora 
$$ \exists\varphi\in Aut(E) \text{ con } \varphi(\tilde{x}_0)= \tilde{x}_1 \quad \ses \quad p_\star ( \pi_1(E, \tilde{x}_0)=p_\star ( \pi_1(E,\tilde{x}_1)$$
\end{itemize}
\proof \bbianco 
\begin{itemize}
\item[(i)]Dato $\tilde{x}\in E$, sia $U$ un intorno ben rivestito e connesso per archi di $p(\tilde{x})\in X$, si ha dunque che 
$\inv p = \prod V_i$ e sia $i_0$ tale che $\tilde{x}\in V_{i_0}$.\\
Basta vedere che se $\varphi\in Aut(E)$ e $\varphi(V_{i_0})\cap V_{i_0}\neq \emptyset$ allora $\varphi=Id$.\\
Se $\varphi(V_{i_0})\cap V_{i_0} \neq \emptyset$ allora sia $z\in V_{i_0}$ tale che $\varphi(z)\in V_{i_0}$.\\
Poich\`e $p\circ \varphi=p$ allora $p(\varphi(z))=p(z)$ ma per definizione di intorno ben rivestito $\ris{p}{V_{i_0}}$ \`e un omeomorfismo, dunque iniettiva da cui $\varphi(z)=z$.\\
Ora sia $\varphi$ sia $Id$ sono sollevamenti di $p$ (a partire da $E$) che coincidono in un punto, segue per unicit\`a del sollevamento $\varphi=Id$
\item[(ii)] Se $F=\inv p (x_0)$ e $\tilde{x}\in F$ allora poich\`e $p \circ \varphi = p $ allora $p(\varphi(\tilde{x}))= p(\tilde{x})$ dunque $f(\tilde{x})\in F$
\item[(iii)]$\implica$ Se $\varphi\in Aut(E)$ con $\vp(\tilde{x}_0)=\tilde{x}_1$ poich\`e $p\circ \vp = p $ allora $p_\star \circ \vp_\star = p_\star$ come mappe $\pi_1(E,\tilde{x}_0)\to \pi_1(X, x_0)$.\\
In particolare $p_\star(\vp_\star(\pi_1(E, \tilde{x}_0)) =p_\star ( \pi_1(E,\tilde{x}_0))$.\\
Ora $\vp_\star (\pi_1(E, \tilde{x}_0)=\pi_1(E, \tilde{x}_1)$ in quanto essendo $\vp$ omeomorfismo  $\vp_\star $ \`e isomorfismo.\\
$\iimplica$ Se $p_\star(\pi_1(E,\tilde{x}_0))= p_\star(\pi_1(E, \tilde{x}_1))$ applicando il teorema dell'esistenza di sollevamenti a 
$$\begin{tikzcd}
& (E, \tilde{x}_1) \arrow[d,"p"]\\
(E, \tilde{x}_0) \arrow[r,"p"] & (X, x_0)
\end{tikzcd}$$
otteniamo $\vp:\, E \to E$ con $\vp(\tilde{x}_0)=\tilde{x}_1$ e tale che $p \circ vp = p $.\\
Analogamente si ottiene $\psi:\, E \to E$ con $\psi(\tilde{x}_1)=\tilde{x}_0$\\
Ora $\varphi\circ \psi $ e $\psi \circ \vp $ sono sollevamenti dell'identit\`a che coincidono in un punto in quanto si ha 
$\vp(\psi(\tilde{x}_1))=\tilde{x}_1$ e $\psi(\vp(\tilde{x}_0))= \tilde{x}_0$.\\
Per unicit\` del sollevamento si ha $\psi \circ \vp = \vp \circ \psi = Id_E$
\end{itemize}
\endproof
\end{thm}
\begin{thm}[Le azioni di monodromia e di $Aut(E)$ commutano]\bianco 
Sia $F=\inv p (x_0)$, sia $\forall \vp \in Aut(E),\, \, \alpha\in \pi_1(X, x_0), \, \, \tilde{x}\in F$ si ha
$$ \vp(\tilde{x}\az \alpha) = \vp(\tilde{x}) \az \alpha$$
\proof Sia $\alpha=[\gamma]$.\\
$\vp \circ \sol{\gamma}{x}$ \`e un sollevamento di $\gamma$ (in quanto $p\circ \vp \circ \sol{\gamma}{x} = p\sol \gamma x = \gamma$) e ha come punto iniziale $\vp\left( \sol \gamma x (0)\right)= \vp (\tilde{x})$, dunque per unicit\`a del sollevamento si ha $\vp \circ \sol \gamma x =  \widetilde{\gamma}_{\vp(\tilde{x})}$ da cui 
$$\vp(\tilde{x}\az \alpha)=\wt \gamma_{f(\tilde{x})}(1)=\vp \sol\gamma x (1)=\varphi\left( \sol \gamma x (1)\right) = \vp(\tilde{x}\az \alpha)$$
\endproof
\end{thm}
\begin{defn}$p:\, E \to X$ rivestimento connesso si dice \textbf{regolare} se $\forall F$ fibra di $E$, l'azione di $Aut(E)$ su $F$ \`e transitiva
\end{defn}

\begin{thm}I seguenti fatti sono equivalenti 
\begin{itemize}
\item[(i)]$p$ \`e rivestimento regolare
\item[(ii)]$\exists F$ fibra di $E$ tale che l'azione di $Aut(E)$ su $F$ sia transitiva
\item[(iii)]$\forall \tilde{x}\in E$ si ha $p_\star(\pi_1(E,\tilde{x})\triangleleft \pi_1(X, p(\tilde{x}))$
\item[(iv)]$\exists \tilde{x}\in E$ si ha $p_\star(\pi_1(E,\tilde{x})\triangleleft \pi_1(X, p(\tilde{x}))$
\end{itemize}
\proof
\begin{itemize}
\item $(iii)\implica (iv)$ e $(i) \implica (ii)$ in modo ovvio
\item Mostriamo che $(iv)\implica (iii)$.\\
Supponiamo che la condizione valga per un fissato $\tilde{x}$ e sia $\tilde{y}\in E$ generico.\\
Sia $\tilde{\gamma} \in \Omega(\tilde{x},\tilde{y})$ e poniamo $\gamma=p\circ \tilde{\gamma}$.\\
Se $x=p(\tilde{x})$ e $y=p(\tilde{y})$ allora $\amma\in \Omega(x,y)$.\\
Si vede facilmente che il seguente diagramma commuta ($\tilde{\amma}_\sharp$ e $\gamma_\sharp$ sono isomorfismi)

$$\begin{tikzcd}
\pi_1(E, \tilde{x}) \arrow{d}{p_\star} \arrow{r}{\tilde{\gamma}_\sharp } &\pi_1(E, \tilde{y}) \arrow{d}{p_\star}\\
\pi_1(X,x) \arrow{r}{\gamma_\sharp} & \pi_1(X,y)
\end{tikzcd}$$
Dunque $$p_\star (\pi_1(E, \tilde{x}))\normale \pi_1(X,x) \quad \ses \quad p_\star (\pi_1(E, \tilde{y}))\normale \pi_1(X,y)$$
\item $(ii) \ses (iv)$ \\
Sia $F=\inv p (x)$, dati $\tilde{x}, \tilde{y}\in F$ abbiamo visto che $\exists\varphi\in Aut(E)$ con $\vp(\tilde{x})=\tilde{y}$ se e solo se $p_\star(\pi_1(E, \tilde{x}))=p_\star(\pi_1(E, \tilde{y})$.\\
Inoltre, abbiamo visto, al variare di $\tilde{y}\in F$ i gruppi $p_\star (\pi_1(E,\tilde{y}))$ sono tutti e soli i coniugati di $p_\star (\pi_1(E,\tilde{x}))$.\\
Dunque l'azione su $F$ \`e transitiva se e solo se $p_\star (\pi_1(E,\tilde{x}))$ coincide con tutti i suoi coniugati (\`e dunque normale)
\end{itemize}
\endproof
\end{thm}
\end{document}