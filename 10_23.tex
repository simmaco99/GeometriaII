\documentclass[a4paper,12pt]{article}
\usepackage[a4paper, top=2cm,bottom=2cm,right=2cm,left=2cm]{geometry}

\usepackage{bm,xcolor,mathdots,latexsym,amsfonts,amsthm,amsmath,
					mathrsfs,graphicx,cancel,tikz-cd,hyperref,booktabs,caption,amssymb,amssymb,wasysym}
\hypersetup{colorlinks=true,linkcolor=blue}
\usepackage[italian]{babel}
\usepackage[T1]{fontenc}
\usepackage[utf8]{inputenc}
\newcommand{\s}[1]{\left\{ #1 \right\}}
\newcommand{\sbarra}{\backslash} %% \ 
\newcommand{\ds}{\displaystyle} 
\newcommand{\alla}{^}  
\newcommand{\implica}{\Rightarrow}
\newcommand{\iimplica}{\Leftarrow}
\newcommand{\ses}{\Leftrightarrow} %se e solo se
\newcommand{\tc}{\quad \text{ t. c .} \quad } % tale che 
\newcommand{\spazio}{\vspace{0.5 cm}}
\newcommand{\bbianco}{\textcolor{white}{,}}
\newcommand{\bianco}{\textcolor{white}{,} \\}% per andare a capo dopo 																					definizioni teoremi ...


% campi 
\newcommand{\N}{\mathbb{N}} 
\newcommand{\R}{\mathbb{R}}
\newcommand{\Q}{\mathbb{Q}}
\newcommand{\Z}{\mathbb{Z}}
\newcommand{\K}{\mathbb{K}} 
\newcommand{\C}{\mathbb{C}}
\newcommand{\F}{\mathbb{F}}
\newcommand{\p}{\mathbb{P}}

%GEOMETRIA
\newcommand{\B}{\mathfrak{B}} %Base B
\newcommand{\D}{\mathfrak{D}}%Base D
\newcommand{\RR}{\mathfrak{R}}%Base R 
\newcommand{\Can}{\mathfrak{C}}%Base canonica
\newcommand{\Rif}{\mathfrak{R}}%Riferimento affine
\newcommand{\AB}{M_\D ^\B }% matrice applicazione rispetto alla base B e D 
\newcommand{\vett}{\overrightarrow}
\newcommand{\sd}{\sim_{SD}}%relazione sx dx
\newcommand{\nvett}{v_1, \, \dots , \, v_n} % v1 ... vn
\newcommand{\ncomb}{a_1 v_1 + \dots + a_n v_n} %a1 v1 + ... +an vn
\newcommand{\nrif}{P_1, \cdots , P_n} 
\newcommand{\bidu}{\left( V^\star \right)^\star}

\newcommand{\udis}{\amalg}
\newcommand{\ric}{\mathfrak{U}}
\newcommand{\inclu}{\hookrightarrow }
%ALGEBRA

\newcommand{\semidir}{\rtimes}%semidiretto
\newcommand{\W}{\Omega}
\newcommand{\norma}{\vert \vert }
\newcommand{\bignormal}{\left\vert \left\vert}
\newcommand{\bignormar}{\right\vert \right\vert}
\newcommand{\normale}{\triangleleft}
\newcommand{\nnorma}{\vert \vert \, \cdot \, \vert \vert}
\newcommand{\dt}{\, \mathrm{d}t}
\newcommand{\dz}{\, \mathrm{d}z}
\newcommand{\dx}{\, \mathrm{d}x}
\newcommand{\dy}{\, \mathrm{d}y}
\newcommand{\amma}{\gamma}
\newcommand{\inv}[1]{#1^{-1}}
\newcommand{\az}{\centerdot}
\newcommand{\ammasol}[1]{\tilde{\gamma}_{\tilde{#1}}}
\newcommand{\pror}[1]{\mathbb{P}^#1 (\R)}
\newcommand{\proc}[1]{\mathbb{P}^#1(\C)}
\newcommand{\sol}[2]{\widetilde{#1}_{\widetilde{#2}}}
\newcommand{\bsol}[3]{\left(\widetilde{#1}\right)_{\widetilde{#2}_{#3}}}
\newcommand{\norm}[1]{\left\vert\left\vert #1 \right\vert \right\vert}
\newcommand{\abs}[1]{\left\vert #1 \right\vert }
\newcommand{\ris}[2]{#1_{\vert #2}}
\newcommand{\vp}{\varphi}
\newcommand{\vt}{\vartheta}
\newcommand{\wt}[1]{\widetilde{#1}}
\newcommand{\pr}[2]{\frac{\partial \, #1}{\partial\, #2}}%derivata parziale
%per creare teoremi, dimostrazioni ... 
\theoremstyle{plain}
\newtheorem{thm}{Teorema}[section] 
\newtheorem{ese}[thm]{Esempio} 
\newtheorem{ex}[thm]{Esercizio} 
\newtheorem{fatti}[thm]{Fatti}
\newtheorem{fatto}[thm]{Fatto}

\newtheorem{cor}[thm]{Corollario} 
\newtheorem{lem}[thm]{Lemma} 
\newtheorem{al}[thm]{Algoritmo}
\newtheorem{prop}[thm]{Proposizione} 
\theoremstyle{definition} 
\newtheorem{defn}{Definizione}[section] 
\newcommand{\intt}[2]{int_{#1}^{#2}}
\theoremstyle{remark} 
\newtheorem{oss}{Osservazione} 
\newcommand{\di }{\, \mathrm{d}}
\newcommand{\tonde}[1]{\left( #1 \right)}
\newcommand{\quadre}[1]{\left[ #1 \right]}
\newcommand{\w}{\omega}

% diagrammi commutativi tikzcd
% per leggere la documentazione texdoc

\begin{document}

\textbf{Lezione del 23 ottobre di Gandini}
\begin{ese}Il quoziente di uno spazio primo numerabile, in generale, non \`e primo numerabile.\\
Consideriamo $\R^2$ e $A =\{ (x,0) \, \vert \, x \in \R\}$.\\
$X=\frac{\R^2}{A}$ non \`e primo numerabile.\\
Sia $[A]\in X$ il punto definito da $A$, dimostriamo che tale punto non ha un sistema fondamentale di intorni numerabile.\\
Supponiamo per assurdo che $\ds \{ U_n \}_{n \in \N}$ sia un sistema fondamentali di intoni di $A$ e sia $V_n = \pi^{-1}(U_n)$.\\
$V_n$ \`e un aperto di $\R^2$ quindi 
$$ \forall n \in \N \quad \exists \varepsilon_n >0 \quad (n,y) \in V_n \quad \forall y \in [0,\varepsilon_n)$$
Sia 
$$f:\, R \to ( 0, + \infty) \text{ continua  e tale che } f(n) = \frac{\varepsilon_n}{2}\quad \forall n \in \N$$
ad esempio $f$ lineare a tratti ottenuta interpolando i punti $\left( n =, \frac{\varepsilon_n}{2}\right)$.\\
Poniamo 
$$ V = \{ (x,y) \, \vert \, \vert y \vert < f(x) \} \text{ aperto in } \R^2 \quad A \subseteq V $$
Sia $U= \pi (V)$ allora esso \`e un intorno aperto di $[A]$ in $X$ da cui 
$$  \exists \overline{n} \quad U_{\overline{n}} \subseteq U \quad \implica \quad V_n \subseteq V $$
L'ultima affermazione \`e assurda infatti, per costruzione, $\exists (x,y) \in V_n \sbarra V $
\end{ese}
\subsection{Quozienti per azioni di gruppi}
\begin{defn}Sia $X$ uno spazio topologico e $G$ un gruppo che agisce su $X$ tramite omeomorfismo, allora definiamo 
$$ \frac{X}{G} \text{ il quoziente ottenuto dalla relazione } x \sim y \quad \ses \quad \exists g \in G \quad g \centerdot x = y $$
ovvero le classi di equivalenza sono le orbite
\end{defn}
\begin{prop}$\Z$ agisce su $\R$ per traslazione allora
$$ \frac{\R}{\Z} \cong S^1$$
\proof Consideriamo
$$ f:\, \R \to S^1 \quad t \to (\cos (2\pi t) , \sin (2\pi t))$$
Se proviamo che $f$ \`e un identificazione ho finito, infatti le fibre di $f$ sono le orbite in cui si partiziona $\Z$.\\
Poich\`e $f$ \`e continua e suriettiva, proviamo che \`e aperta.\\
Poich\`e gli intervalli aperti $I$ sono una base, basta dimostrare che $f(I)$ \`e aperto.\\
Sia $a \in \R$ e sia $I=(a,a+1)$ allora
$$ f_{\vert (a,a+1)} : \, (a,a+1) \to S^1 \sbarra f(a) \text{ \`e omeomorfismo }$$
da cui $f(I)$ aperto.\\
Sia $A$ un generico aperto
\begin{itemize}
\item Se $A$ contenuto in un intervallo $(a,a+1)$ allora
$$ f(A) \text{ aperto in } S' \sbarra f(a) \quad\implica \quad f(A) \text{ aperto in } S^1$$
infatti $S'\sbarra f(a)$ \`e aperto in $S^1$
\item In generale
$$A = \bigcup_{j\in J} A_j \text{ con } A_j \text{ contenuti in intervallo di ampiezza }1 $$
dunque 
$$ f(A) = f \left( \bigcup_{j \in J } A_j \right) = \bigcup_{j\in J} f(A_j)$$
ora unione di aperti \`e aperta quindi $f(A)$ \`e aperto.
\end{itemize}
\endproof
\end{prop}
\begin{oss}\`E presente un ambiguit\`a nella notazione infatti $\frac{\R}{\Z}$ pu\`o indicare 2 cose differenti
\begin{itemize}
\item $\R$ facendo collassare $\Z$ in questo caso il quoziente \`e omeomorfo ad un bouquet infinito di circonferenze
\item $\Z$ che agisce su $\R$ ed in questo caso il quoziente \`e omeomorfo a $S^1$
\end{itemize}
\end{oss}
\spazio
\begin{ese}$\Q$ agisce su $\R$ con la traslazione allora il quoziente \`e pi\`u che numerabile con la topologia indiscreta.
\proof
Le orbite sono di cardinalit\`a numerabile dunque, in quantit\`a, devono essere pi\`u che numerabili.\\
Sia $A\subseteq \R$ un aperto saturo non vuoto dunque $x_0 \in A $ , essendo aperto $(x_0-\varepsilon, x_0 + \varepsilon ) \subseteq A $ ora essendo saturo
$$ A \supseteq \bigcup_{q \in \Q} (x_0+q-\varepsilon, x_0+q +\varepsilon) =\R \quad \implica \quad A = \R$$
\end{ese}
\spazio
\begin{prop}\label{az_gruppo_aperta}Sia $G$ un gruppo che agisce su $X$ spazio topologico, allora
$$ \pi:\, X \to \frac{X}{G}$$
\`e aperta.
\proof Sia $U \subseteq X$ un aperto.\\
Poich\`e $G$ agisce per omeomorfismo $g\centerdot U $ \`e aperto 
$$ \pi(U) = \pi \left( \bigcup_{g \in G } g\centerdot U \right)$$
infatti $U$ e $\bigcup g \centerdot U$ hanno la stessa orbita.\\
Ora $\bigcup g\centerdot U$ \`e un aperto saturo quindi $\pi(U)$\`e aperto.
\endproof
\end{prop}
\begin{oss}Se $G$ \`e finito allora $\pi$ \`e chiusa in quanto unione finite di chiusi \`e chiusa
\end{oss}
\spazio
\begin{prop}Siano $f_i :\, X_i \to Y_i$ identificazioni aperte $i=1,2$ allora
$$ f_1 \times f_2 :\, X_1 \times X_2 \to Y_1 \times Y_2$$ \`e un'identificazione
\proof $f_1\times f_2$ \`e iniettiva e suriettiva poich\`e lo sono $f_1$ e $f_2$ resta da provare che \`e aperta.\\
Sia $U_i\subseteq X_i$ aperto allora
$$ ( f_1 \times f_2) (U_1 \times U_2 ) = f_1(U_1) \times f(U_2) \text{ aperto della topologia prodotto}$$
dunque $f_1\times f_2$ \`e un'identificazione.
\end{prop}
\spazio
\begin{prop}Sia $X$ spazio topologico e $G$ gruppo che agisce su $X$ tramite omeomorfismo.\\
Sia $K = \{ (x,g\centerdot x) \, \vert \,  x\in X \, \, g \in X \}$ allora
$$ \frac{X}{G} \text{ \`e di Hausdorff } \quad \ses  K \text{ chiuso in } X \times X $$
\proof $$ \frac{X}{G} \text{ \`e di Hausdorff } \quad  \quad \Delta_{\frac{X}{G}} \text{ \`e chiusa}$$
Per la proposizione~\ref{az_gruppo_aperta} $\pi$ \`e un identificazione aperta quindi anche $\pi \times \pi$ lo \`e
$$ \Delta_{\frac{X}{G}} \text{ chiusa } \quad \ses \quad ( \pi \times \pi )^{-1} \left( \Delta_{\frac{X}{G}} \right)\subseteq X \times X \text{ chiuso } $$
D'altronte $ ( \pi \times \pi )^{-1} \left( \Delta_{\frac{X}{G}} \right)=K$
\endproof
\end{prop}
\newpage
\section{Ricoprimenti}
\begin{defn}[Ricoprimento]\bianco
Sia $X$ uno spazio topologico.\\
Un ricoprimento \`e una famiglia $\ric \subseteq \mathcal{P}(X)$ se
$$ X = \bigcup_{U \in \ric} U $$
se tutti gli $U \in \ric$ sono aperti, $\ric$ \`e detto ricoprimento aperto 
\end{defn}
\spazio
\begin{defn}[Localmente finito]\bianco $\ric \subseteq \mathcal{X}$\`e una famiglia localmente finita se 
$$ \forall x \in X \exists V \in I(x) \quad V \cap U \neq \emptyset \text{ solamente per finiti } U \in \ric $$
\end{defn}
\begin{ese} 
$$ \R= \bigcup_{n \in \Z} [n, n+1]$$
\`e un ricoprimento chiuso localmente finito
\end{ese}
\spazio
\begin{defn}[Ricoprimento fondamentale]\bianco
$\ric$ ricoprimento di $X$ \`e detto fondamentale se dato $A \subseteq X $ allora
$$ A \text{ aperto in } X \quad \ses \quad A \cap U \text{ aperto in } U \quad \forall U \in \ric $$
in modo equivalente
$$ A \text{ chiuso in } X \quad \ses \quad A \cap U \text{ chiuso in } U \quad \forall U \in \ric $$
\begin{oss}La freccia $\implica$ segue dalla definizione di topologia di sottospazio
\end{oss}
\end{defn}
\spazio
\begin{prop} $$\ric \text{ ricoprimento aperto di  } X \quad \implica \quad  \ric \text{ ricoprimento fondamentale}$$
\proof
Sia $A\subseteq X $ tale che $A\cap U $ aperto in $U$ $\forall U \in \ric$.\\
Ora $U$ \`e aperto in $X$ e poich\`e aperto di aperto \`e aperto\\
$$ A\cap U \text { aperto in } X \quad  \forall U \in \ric $$
dunque 
$$A = \bigcup_{U \in \ric} A \cap U \text{ aperto in } X $$
\end{prop}
\begin{oss}In generale, ricoprimenti chiusi non sono fondamentali.\\
Sia $\R$ con la topologia euclidea allora
$$ \R= \bigcup_{x \in \R} \{ x \} \text{ \`e un ricoprimento chiuso } $$
Ora  $ \forall A\subseteq \R$ allora $A\cap \{x\}$ \`e aperto in $\{x\}$ dunque se il ricoprimento fosse fondamentale
$$ \forall A \subseteq \R \quad A \text{ aperto } \quad \implica \quad \R \text{ con la topologia discreta}$$ 
\end{oss}
\spazio
\begin{prop}Sia $\ric$ un ricoprimento fondamentale di $X$ e $f:\, X \to Y$ funzione tra spazi metrici.
$$ f \text{ continua } \quad \ses \quad f_{\vert U } \text{ continua } \forall U \in \ric $$
\proof $\implica$ $$ \forall A \subseteq Y\text{ aperto} \quad 
\left( f_{\vert U } \right)^{-1}(A) = f^{-1}(A) \cap U $$
Ora essendo $f$ continua $f^{-1}(A)$ aperto e poich\`e $\ric$ fondamentale anche $f^{-1}(A) \cap U $ \`e un aperto \\
$\iimplica$ Sia $A\subseteq Y$ aperto 
$$ f^{-1}(A) \text{ aperto } \quad \ses \quad f^{-1}(A) \cap U \text{ aperto} \quad \forall U \in \ric $$
Ma $f_{\vert U} $ continua dunque
$$ \left( f_{\vert U } \right)^{-1}(A) = f^{-1}(A) \cap U \text{ aperto }\quad \forall U \in \ric$$
\endproof
\end{prop}
\end{document}