\documentclass[a4paper,12pt]{article}
\usepackage[a4paper, top=2cm,bottom=2cm,right=2cm,left=2cm]{geometry}

\usepackage{bm,xcolor,mathdots,latexsym,amsfonts,amsthm,amsmath,
					mathrsfs,graphicx,cancel,tikz-cd,hyperref,booktabs,caption,amssymb,amssymb,wasysym}
\hypersetup{colorlinks=true,linkcolor=blue}
\usepackage[italian]{babel}
\usepackage[T1]{fontenc}
\usepackage[utf8]{inputenc}
\newcommand{\s}[1]{\left\{ #1 \right\}}
\newcommand{\sbarra}{\backslash} %% \ 
\newcommand{\ds}{\displaystyle} 
\newcommand{\alla}{^}  
\newcommand{\implica}{\Rightarrow}
\newcommand{\iimplica}{\Leftarrow}
\newcommand{\ses}{\Leftrightarrow} %se e solo se
\newcommand{\tc}{\quad \text{ t. c .} \quad } % tale che 
\newcommand{\spazio}{\vspace{0.5 cm}}
\newcommand{\bbianco}{\textcolor{white}{,}}
\newcommand{\bianco}{\textcolor{white}{,} \\}% per andare a capo dopo 																					definizioni teoremi ...


% campi 
\newcommand{\N}{\mathbb{N}} 
\newcommand{\R}{\mathbb{R}}
\newcommand{\Q}{\mathbb{Q}}
\newcommand{\Z}{\mathbb{Z}}
\newcommand{\K}{\mathbb{K}} 
\newcommand{\C}{\mathbb{C}}
\newcommand{\F}{\mathbb{F}}
\newcommand{\p}{\mathbb{P}}

%GEOMETRIA
\newcommand{\B}{\mathfrak{B}} %Base B
\newcommand{\D}{\mathfrak{D}}%Base D
\newcommand{\RR}{\mathfrak{R}}%Base R 
\newcommand{\Can}{\mathfrak{C}}%Base canonica
\newcommand{\Rif}{\mathfrak{R}}%Riferimento affine
\newcommand{\AB}{M_\D ^\B }% matrice applicazione rispetto alla base B e D 
\newcommand{\vett}{\overrightarrow}
\newcommand{\sd}{\sim_{SD}}%relazione sx dx
\newcommand{\nvett}{v_1, \, \dots , \, v_n} % v1 ... vn
\newcommand{\ncomb}{a_1 v_1 + \dots + a_n v_n} %a1 v1 + ... +an vn
\newcommand{\nrif}{P_1, \cdots , P_n} 
\newcommand{\bidu}{\left( V^\star \right)^\star}

\newcommand{\udis}{\amalg}
\newcommand{\ric}{\mathfrak{U}}
\newcommand{\inclu}{\hookrightarrow }
%ALGEBRA

\newcommand{\semidir}{\rtimes}%semidiretto
\newcommand{\W}{\Omega}
\newcommand{\norma}{\vert \vert }
\newcommand{\bignormal}{\left\vert \left\vert}
\newcommand{\bignormar}{\right\vert \right\vert}
\newcommand{\normale}{\triangleleft}
\newcommand{\nnorma}{\vert \vert \, \cdot \, \vert \vert}
\newcommand{\dt}{\, \mathrm{d}t}
\newcommand{\dz}{\, \mathrm{d}z}
\newcommand{\dx}{\, \mathrm{d}x}
\newcommand{\dy}{\, \mathrm{d}y}
\newcommand{\amma}{\gamma}
\newcommand{\inv}[1]{#1^{-1}}
\newcommand{\az}{\centerdot}
\newcommand{\ammasol}[1]{\tilde{\gamma}_{\tilde{#1}}}
\newcommand{\pror}[1]{\mathbb{P}^#1 (\R)}
\newcommand{\proc}[1]{\mathbb{P}^#1(\C)}
\newcommand{\sol}[2]{\widetilde{#1}_{\widetilde{#2}}}
\newcommand{\bsol}[3]{\left(\widetilde{#1}\right)_{\widetilde{#2}_{#3}}}
\newcommand{\norm}[1]{\left\vert\left\vert #1 \right\vert \right\vert}
\newcommand{\abs}[1]{\left\vert #1 \right\vert }
\newcommand{\ris}[2]{#1_{\vert #2}}
\newcommand{\vp}{\varphi}
\newcommand{\vt}{\vartheta}
\newcommand{\wt}[1]{\widetilde{#1}}
\newcommand{\pr}[2]{\frac{\partial \, #1}{\partial\, #2}}%derivata parziale
%per creare teoremi, dimostrazioni ... 
\theoremstyle{plain}
\newtheorem{thm}{Teorema}[section] 
\newtheorem{ese}[thm]{Esempio} 
\newtheorem{ex}[thm]{Esercizio} 
\newtheorem{fatti}[thm]{Fatti}
\newtheorem{fatto}[thm]{Fatto}

\newtheorem{cor}[thm]{Corollario} 
\newtheorem{lem}[thm]{Lemma} 
\newtheorem{al}[thm]{Algoritmo}
\newtheorem{prop}[thm]{Proposizione} 
\theoremstyle{definition} 
\newtheorem{defn}{Definizione}[section] 
\newcommand{\intt}[2]{int_{#1}^{#2}}
\theoremstyle{remark} 
\newtheorem{oss}{Osservazione} 
\newcommand{\di }{\, \mathrm{d}}
\newcommand{\tonde}[1]{\left( #1 \right)}
\newcommand{\quadre}[1]{\left[ #1 \right]}
\newcommand{\w}{\omega}

% diagrammi commutativi tikzcd
% per leggere la documentazione texdoc

\newcommand{\vt}{\vartheta}
\begin{document}
\textbf{Lezione del 16  Marzo}

\begin{thm}Sia $p:\, E\to X$ rivestimento regolare, allora
$$ Aut(E) \cong \frac{\pi_1(X,x)}{p_\star(\pi_1(E, \tilde{x}))}$$
\proof Per prima cosa osserviamo che il termine di destra \`e un gruppo, il sottogruppo per cui stiamo quozientando \`e normale (essendo il rivestimento regolare).\\
Sia $\vt:\, \pi_1(X,x)\to Aut(E)$ tale che $\forall \alpha\in \pi_1(X,x)$ definiamo $\vt(\alpha)$ come l'unico elemento di $Aut(E)$ tale che $\vt(\alpha)(\tilde{x})=\tilde{x}\az \alpha$.\\
Osserviamo che tale isomorfismo esiste in quanto il rivestimento \`e regolare (azione \`e transitiva) ed \`e unico poich\`e l'azione \`e libera.\\
Mostriamo che $\vt$ \`e un omomorfismo 
$$ (\vt(\alpha_1)\vt(\alpha_2))(\tilde{x}= \vt(\alpha_1) ( \tilde{x}\az \alpha_2)= (\vt(\alpha_1) (\tilde{x}))\az \alpha_2=(\tilde{x}\az\alpha_1) \az \alpha_2$$
Dove la seconda uguaglianza deriva dal fatto che l'azione di monodromia \`e quella indotta da $Aut(E)$ commutano\\
Mostriamo che $\vt$ \`e suriettivo.\\
Se $\vp\in Aut(E)$ poich\`e l'azione di monodromia \`e transitiva, $\exists\alpha\in \pi_1(X, x)$ con $\tilde{x}\az \alpha=\vp(\tilde{x})$, dunque $\vt(\alpha)=vp$ in quanto i due isomorfismi coincidono su $\tilde{x}$ ($Aut(E)$ agisce in maniera libera).\\
Mostriamo che $Ker \vt = p_\star(\pi_1(E, \tilde{x}))$, da cui per il primo teorema di isomorfismo si ha la tesi.\\
Poich\`e $Aut(E)$ agisce liberamente si ha
$$\vt(\alpha)=Id \quad \ses \quad \vt(\alpha)(\tilde{x})=\tilde{x} \quad \ses \quad \tilde{x}\az \alpha =\tilde{x}\quad \ses 
\quad \alpha\in p_\star(\pi_1(E,\tilde{x}))$$
\endproof
\end{thm}
\begin{cor}$$p:\, E\to X \text{  regolare } \quad \implica \quad X\cong \frac{X}{Aut(E)}$$
\proof Essendo aperto e suriettivo $p$ \`e un'identificazione, per cui $X \cong \frac{X}{\sim}$ dove 
$$\tilde{x}\sim
\tilde{y} \quad \ses \quad p(\tilde{x})=p(\tilde{y}) \quad \ses \quad \tilde{x}=\vp (\tilde{y}) \, \exists \vp \in Aut(E)
$$
dove per l'ultima implicazione abbiamo usato il fatto che $\varphi\circ p = p $ e in quanto $Aut(E)$ agisce in maniera transitiva sulle fibre 
\end{cor}
Vale una sorta di viceversa
\begin{prop}Se $G$ agisce in maniera propriamente discontinua su uno spazi connesso $E$ allora la proiezione $p:\, E \to \frac{E}{G}$ \`e un rivestimento regolare
\proof Dato $x\in \frac{E}{G}$ devo costruire un intorno ben rivestito $U$.\\
Scelgo $\tilde{x}\in E$ con $p(\tilde{x})=x$, dalla definizione di azione propriamente discontinua, $\exists V \subseteq X$ aperto che contiene $\tilde{x}$ tale che $\gamma(V) \cap V=\emptyset \ , \, \forall \gamma \in G\sbarra \{ Id\}$\\
Pongo $U=p(V)$ che \`e aperto (le proiezioni al quoziente per azioni di gruppi sono aperte).\\
Per costruzione $\ds \inv p  (U) =\bigcup_{\gamma\in G} \gamma(V)$ ora $\gamma(V)$ \`e aperto ($\amma$ \`e omeomorfismo)  e l'unione \`e disgiunta.\\
$$\amma_1(V) \cap \amma_2(V)\neq \emptyset \quad \implica\quad V \cap (\inv {\gamma_1} \gamma_2(V)) \neq \emptyset \quad \implica \quad \inv{\gamma_1}\gamma_2=Id\quad \implica \quad \amma_1=\amma_2$$
Infine $\ris p  {\gamma(V)}$ \`e un omeomorfismo su $U$ in quanto \`e continua, aperta, suriettiva e iniettiva
\end{prop}
\spazio
\begin{oss} Un rivestimento universale \`e regolare.\\
$p_\star(\{1\})=\{ 1 \}\normale \pi_1(X,x)$
\end{oss}

\begin{cor}Sia $p:\, E \to X$ un rivestimento universale, allora $Aut(E) \cong \pi_1(X,x)$
\end{cor}
\begin{cor} Se $E$ \`e semplicemente connesso e $G$ agisce su $E$ in maniera propriamente discontinua, $\pi_1\left(\frac{E}{G}\right) \cong G$
\end{cor}
\begin{ese}\bbianco
\begin{enumerate}
\item $S^1=\frac{\R}{\Z}$ ora essendo $\R$ semplicemente connesso si ha $\pi_1(S^1)=\Z$
\item $(S^1)^n=\frac{\R^n}{\Z^n}$ da cui $\pi_1((S^1)^n)=\Z^n$
\item $\pror n= \frac{S^n}{\{ \pm Id\}}$ dunque se $n\geq 2 $ allora $\pi_1(\pror n) \cong \{ \pm Id \} = \Z_2$
\end{enumerate}
\end{ese}

\newpage
\begin{defn}\bianco 
$X$ spazio topologico connesso per archi si dice \textbf{semilocalemente semplicemente connesso} se 
$$\forall x \in X\, \, \exists U \subseteq X \text{ aperto  con } x \in X \quad i:\, U \to X \text{ induce il morfismo banale } i_\star:\, \pi_1(U,x) \to \pi_1(X,x)$$
ovvero $\forall x \in X \, \, \exists U\ni x$ tale che tutti i lacci basati in $x$ e contenuti in $U$ sono banali in $X$ 
\end{defn}
\begin{oss}La propiet\`a sopra definita \`e verificata, ad esempio, se ogni punto ha un intorno semplicemente connesso
\end{oss}
\begin{oss}Se $X$ ammette un rivestimento universale, allora \`e semilocalmente semplicemente connesso.\\
Dato $x\in X$, posso prendere un suo intorno $U$ connesso per archi e ben rivestito.\\
Se $p:\, E \to X$ \`e il rivestimento universale $V\subseteq \inv p(U)$ \`e un aperto con $\ris p V :\, V \to U $ omeomorfismo dunque abbiamo il seguente diagramma
$$ \begin{tikzcd}
V \arrow{d}{\ris p V} \arrow[hook,r,"j"] & E \arrow[d, "p"] \\
U \arrow[hook, r ,"i"] \arrow[bend left, u,"s"]& X 
\end{tikzcd}$$
da cui $i=p\circ j \circ s $ dunque $i_\star = p_\star \circ j_\star \circ s_\star$ ma $j_\star$ \`e banale in quanto lo \`e $\pi_1(E)$ da cui $i_\star$ \`e banale 
\end{oss}

\begin{lem}Quando esiste, il rivestimento universale \`e unico a meno di isomorfismi
\proof Siano $p_1:\, E_1 \to X$ e $p_2:\, E_2 \to X$ rivestimenti universali di $X$.\\
Abbiamo dunque:\\
$ \begin{tikzcd}
& E_2 \arrow{d}{p_2} \\
E_1 \arrow[dashed]{ru}{\vp} \arrow{r}{p_1} & X 
\end{tikzcd}$
dove l'esistenza di $\vp$ ($p_2 \circ \vp = p_1$) deriva dal fatto che  $E_1$ \`e semplicemente connesso.\\
Fissati $\tilde{x}_1\in \inv{p_1}(x)$ e $\tilde{x}_2 \in \inv{p_2}(x)$ posso richiedere $\vp(\tilde{x}_1) =\tilde{x}_2$.\\
In modo analogo $\exists \, \psi:\, E_2 \to E_1$ con $p_1\circ \psi=p_2$ e $\psi(\tilde{x}_2) =\tilde{x}_1$\\
Ne segue che $\vp $ e $\psi$ sono isomorfismi l'uno l'inverso dell'altro
\end{lem}

\begin{thm}$X$ spazio topologico localmente connesso per archi e connesso
$$ X \text{ ammette rivestimento universale } \quad \ses \quad X \text{ semilocalmente semplicemente connesso }$$
inoltre, in tal caso il rivestimento \`e unico
\proof $\implica$ gi\`a visto\\
$\iimplica$ Fissato $x\in X$. Si definisce $E = \ds \frac{\bigcup_{y \in X}\Omega(x,y)}{\sim}$ dove $\gamma_1\sim \gamma_2 $ se e solo se sono omotopi come cammini.\\ Si topologizza $\ds \bigcup_{y \in X}\Omega(x,y)$tale insieme con la topologia compatta-aperta e si dota $E$ della topologia quoziente.\\
Sia $p:\, E \to X$ dove $p([\gamma])=\gamma(1)$
\end{thm}

\begin{thm}$X$ connesso e semilocalmente semplicemente  connesso per archi, $x\in X$\\
$\forall H < \pi_1(X,x) $ esiste $p:\,E \to X$ tale che $p_\star (\pi_1(E, \tilde{x}))=H$ dove $\tilde{x}\in \inv p (x)$.\\
Tale rivestimento \`e unico
\proof Sia $g:\, \wt{X} \to X$ il rivestimento universale di $X$ e fissiamo $\wt{\wt{x}}\in \inv g (x)$.\\
Abbiamo un isomorfismo $\pi_(X, x) \to Aut(\wt{X})$, con un abuso di notazione denoto anche con $H$ la copia di $H$ in $Aut(\wt X)$, pongo $E=\frac{\wt X}{H}$

$$\begin{tikzcd}
\wt X\arrow{r}{\pi}  \arrow[bend right,rr,"g"]& E=\frac{\wt E}{H}\arrow[r,"p"] & X =\frac{\wt X}{\pi_1(X,x)} 
\end{tikzcd}$$
Si verifica facilmente che $p$ \`e un rivestimento (un aperto ben rivestito rispetto a $g$ lo \`e anche rispetto a $p$)\\
Inoltre se $\tilde{x}=\pi\left( \wt{\wt{x}}\right)\in E$ abbiamo $p_\star (\pi_1(E, \tilde{x}))=Stab(\tilde{x})$ rispetto all'azione di monodromia.\\
Dato $\alpha\in \pi_1(X,x)$ con $\alpha=[\amma]$ siano 

$\wt \gamma $ e $\wt{\wt{\gamma}}$ i sollevamenti di $\amma$ in $E$ a partire da $\tilde{x}$ e $\wt{\wt{x}}$ cos\`i che $\wt \amma = \pi\circ \wt{\wt{\amma}}$\\
Ora $\wt x \az \alpha =\wt \amma(1)= \pi \left( \wt{\wt{\amma}}(1)\right)$ che \`e uguale a $\tilde{x}$ se e solo se $\wt{\wt{\amma}}(1)$ \`e equivalente a $\wt{\wt{x}}$ tramite l'azione di $H$ che equivale alla tesi 
\end{thm}

\newpage

%AGGIUNGERE DISEGNI degli esempi 

Sia $\Gamma$ un grafo finito connesso avente $V$ vertici e $E$ lati
\begin{itemize}
\item $\Gamma$ \`e un albero se non contiene cicli, cio\`e loop iniettivi
\item Un albero \`e contraibile (induzione sul numero di vertici: un albero deve avere un vertice libero, \`e possibile retrarre per deformazione  il lato che lo contiene sul resto dell'albero)
\item Se $\Gamma$ \`e un albero allora $V-E=1$ (induzione: ogni retrazione toglie un vertice e un lato, si finisce con un vertice e 0 lati)
\item $\Gamma $ conesso allora contiene $\Gamma'$ albero massimale che contiene tutti i vertici di $\Gamma$
$$ \Gamma = \Gamma' \cup \{ \text{ qualche lato} \}$$
\item Definisco $\chi(\Gamma)= V-E$ che prende il nome di caratteristica di Eulero.\\
Se $\Gamma'\subset\Gamma $ massimale allora $\chi(\Gamma')=1$ e $\Gamma=\Gamma'\cup \{ (1-\chi(\Gamma)) \text{ lati}\}$

\end{itemize}

\begin{thm}$$\pi_1(\Gamma)\cong F_{1-\chi(\Gamma)}$$
\proof Si dimostra usando induttivamente Van Kampen e dal fatto che $\Gamma$ si ottiene da un albero massimale aggiungendo $1-\chi (\Gamma)$ lati 
\end{thm}

\begin{thm}$F$ gruppo libero su $n$ generatori, $H < F$ di indice $k$.\\
Allora $H$ \`e un gruppo libero su $k(n-1)+1$ generatori
\proof $F=\pi_1(\Gamma)$ con $\chi(\Gamma)=1-n$ per il teorema appena visto.\\
Sia $\wt \Gamma$ il rivestimento di $\Gamma$ associato ad $H$.\\
Poich\`e vertici e lati sono semplicemente connessi, se $V,E$ sono i vertici ed i lati di $\Gamma$ allora i vertici di $\wt \Gamma$ sono $kV$ e $kE$ in quanto il grado del rivestimento \`e $k$ da cui 
$H=\pi_1(\wt \Gamma)= F_{1-\chi(\wt\Gamma)}=F_{1+k(n-1)}$
\endproof
\end{thm}





































\end{document}