\documentclass[a4paper,12pt]{article}
\usepackage[a4paper, top=2cm,bottom=2cm,right=2cm,left=2cm]{geometry}

\usepackage{bm,xcolor,mathdots,latexsym,amsfonts,amsthm,amsmath,
					mathrsfs,graphicx,cancel,tikz-cd,hyperref,booktabs,caption,amssymb,amssymb,wasysym}
\hypersetup{colorlinks=true,linkcolor=blue}
\usepackage[italian]{babel}
\usepackage[T1]{fontenc}
\usepackage[utf8]{inputenc}
\newcommand{\s}[1]{\left\{ #1 \right\}}
\newcommand{\sbarra}{\backslash} %% \ 
\newcommand{\ds}{\displaystyle} 
\newcommand{\alla}{^}  
\newcommand{\implica}{\Rightarrow}
\newcommand{\iimplica}{\Leftarrow}
\newcommand{\ses}{\Leftrightarrow} %se e solo se
\newcommand{\tc}{\quad \text{ t. c .} \quad } % tale che 
\newcommand{\spazio}{\vspace{0.5 cm}}
\newcommand{\bbianco}{\textcolor{white}{,}}
\newcommand{\bianco}{\textcolor{white}{,} \\}% per andare a capo dopo 																					definizioni teoremi ...


% campi 
\newcommand{\N}{\mathbb{N}} 
\newcommand{\R}{\mathbb{R}}
\newcommand{\Q}{\mathbb{Q}}
\newcommand{\Z}{\mathbb{Z}}
\newcommand{\K}{\mathbb{K}} 
\newcommand{\C}{\mathbb{C}}
\newcommand{\F}{\mathbb{F}}
\newcommand{\p}{\mathbb{P}}

%GEOMETRIA
\newcommand{\B}{\mathfrak{B}} %Base B
\newcommand{\D}{\mathfrak{D}}%Base D
\newcommand{\RR}{\mathfrak{R}}%Base R 
\newcommand{\Can}{\mathfrak{C}}%Base canonica
\newcommand{\Rif}{\mathfrak{R}}%Riferimento affine
\newcommand{\AB}{M_\D ^\B }% matrice applicazione rispetto alla base B e D 
\newcommand{\vett}{\overrightarrow}
\newcommand{\sd}{\sim_{SD}}%relazione sx dx
\newcommand{\nvett}{v_1, \, \dots , \, v_n} % v1 ... vn
\newcommand{\ncomb}{a_1 v_1 + \dots + a_n v_n} %a1 v1 + ... +an vn
\newcommand{\nrif}{P_1, \cdots , P_n} 
\newcommand{\bidu}{\left( V^\star \right)^\star}

\newcommand{\udis}{\amalg}
\newcommand{\ric}{\mathfrak{U}}
\newcommand{\inclu}{\hookrightarrow }
%ALGEBRA

\newcommand{\semidir}{\rtimes}%semidiretto
\newcommand{\W}{\Omega}
\newcommand{\norma}{\vert \vert }
\newcommand{\bignormal}{\left\vert \left\vert}
\newcommand{\bignormar}{\right\vert \right\vert}
\newcommand{\normale}{\triangleleft}
\newcommand{\nnorma}{\vert \vert \, \cdot \, \vert \vert}
\newcommand{\dt}{\, \mathrm{d}t}
\newcommand{\dz}{\, \mathrm{d}z}
\newcommand{\dx}{\, \mathrm{d}x}
\newcommand{\dy}{\, \mathrm{d}y}
\newcommand{\amma}{\gamma}
\newcommand{\inv}[1]{#1^{-1}}
\newcommand{\az}{\centerdot}
\newcommand{\ammasol}[1]{\tilde{\gamma}_{\tilde{#1}}}
\newcommand{\pror}[1]{\mathbb{P}^#1 (\R)}
\newcommand{\proc}[1]{\mathbb{P}^#1(\C)}
\newcommand{\sol}[2]{\widetilde{#1}_{\widetilde{#2}}}
\newcommand{\bsol}[3]{\left(\widetilde{#1}\right)_{\widetilde{#2}_{#3}}}
\newcommand{\norm}[1]{\left\vert\left\vert #1 \right\vert \right\vert}
\newcommand{\abs}[1]{\left\vert #1 \right\vert }
\newcommand{\ris}[2]{#1_{\vert #2}}
\newcommand{\vp}{\varphi}
\newcommand{\vt}{\vartheta}
\newcommand{\wt}[1]{\widetilde{#1}}
\newcommand{\pr}[2]{\frac{\partial \, #1}{\partial\, #2}}%derivata parziale
%per creare teoremi, dimostrazioni ... 
\theoremstyle{plain}
\newtheorem{thm}{Teorema}[section] 
\newtheorem{ese}[thm]{Esempio} 
\newtheorem{ex}[thm]{Esercizio} 
\newtheorem{fatti}[thm]{Fatti}
\newtheorem{fatto}[thm]{Fatto}

\newtheorem{cor}[thm]{Corollario} 
\newtheorem{lem}[thm]{Lemma} 
\newtheorem{al}[thm]{Algoritmo}
\newtheorem{prop}[thm]{Proposizione} 
\theoremstyle{definition} 
\newtheorem{defn}{Definizione}[section] 
\newcommand{\intt}[2]{int_{#1}^{#2}}
\theoremstyle{remark} 
\newtheorem{oss}{Osservazione} 
\newcommand{\di }{\, \mathrm{d}}
\newcommand{\tonde}[1]{\left( #1 \right)}
\newcommand{\quadre}[1]{\left[ #1 \right]}
\newcommand{\w}{\omega}

% diagrammi commutativi tikzcd
% per leggere la documentazione texdoc

\begin{document}
\textbf{Lezione del 11 Dicembre del Prof. Frigerio}
\begin{oss}Con un abuso, notazionale, d'ora in poi indicheremo con $\alpha\star \beta \star \gamma$ il cammino $( \alpha \star \beta) \star \gamma$ o il cammino $\alpha \star ( \beta \star \gamma)$, il che non crea problemi a meno di riparametrazione, dunque a meno di omotopie di cammini, stessa convenzione per giunzioni multiple 
\end{oss}
\begin{lem}$1_a$ \`e l'elemento neutro
\proof
$1_a \star \alpha$ e $\alpha \star 1_a$ sono riparametrazione di $\alpha $ $\forall \alpha \in \Omega(a,a)$
$$ [1_a] \cdot [\alpha] = [1_a \star \alpha] = [\alpha] = [ \alpha \star 1_a] = [\alpha]\cdot  [1_a]$$
\end{lem}
\begin{lem}Sia $\alpha\in \Omega(a,a)$ allora $\overline{\alpha}$ \`e l'inverso di $\alpha$
\proof Mostriamo che $\alpha \star \overline{\alpha}\sim 1_a$.
$$H(t,s)=\begin{cases}\alpha(2t) \text{ se } t \leq \frac{s}{2}\\ \alpha(s) \text{ se } \frac{s}{2}\leq t \leq 1 - \frac{s}{2}\\
\overline{\alpha}(2t-1) \text{ se } t> 1 - \frac{s}{2}
\end{cases}$$
In modo analogo si prova che $\overline{\alpha}\star \alpha \sim 1_a$
\endproof
\end{lem}
\begin{thm}Abbiamo dimostrato che $\pi_1(X,a)$ dotato dell'operazione $[\alpha]\cdot [\beta] := [\alpha\star \beta]$ \`e un gruppo.\\
Tale gruppo prende il nome di gruppo fondamentale
\end{thm}

\spazio
\begin{oss}D'ora in avanti, se non diversamente esplicitato, assumiamo $X$ connesso per archi (in quanto se $Y$ \`e la componente connnessa per archi di $a$ in $X$ allora $\pi_1(X,a)\cong \pi_1(Y,a)$)
\end{oss}
\begin{defn} Siano $a, b \in X$ e sia $\gamma \in \Omega(a,b)$.\\
Poniamo 
$$\gamma_\sharp:\pi_1(X,a) \to \pi_1(X,b) \qquad \gamma_\sharp([\alpha]) = [ \overline{\gamma} \star  \alpha\star \gamma]$$
\end{defn}
\begin{oss}Osserviamo che $\gamma_\sharp$ \`e ben definita.\\
$$\alpha\sim \beta\quad \implica \quad \overline{\gamma} \star \alpha \sim \overline{\gamma}\star \beta \quad \implica \quad 
\quad \overline{\gamma}\star \alpha\star \gamma \sim \overline{\gamma}\star \beta \star \gamma $$
\end{oss}
\begin{thm}$\gamma_\sharp$ \`e un isomorfismo di gruppi
\proof Mostriamo che \`e un omomorfismo di gruppi
$$ \gamma_\sharp ([\alpha]\cdot [\beta]) = \gamma_\sharp( [\alpha\star \beta]) =  [ \overline{ \gamma}\star \alpha\star \beta \star \gamma] = [ \gamma \star \alpha \star ( \gamma \star\overline{\gamma}) \star \beta ]= [( \overline{\gamma}\star \alpha\star \gamma )\star ( \overline{\gamma} \star \beta\star \gamma)]= \gamma_\sharp([\alpha]) \cdot \gamma_\sharp([\beta ])$$
$\overline{\gamma}_\sharp$ \`e l'inversa di $\gamma_\sharp$ infatti
$$ \overline{\gamma}_\sharp
\left( \gamma_\sharp([\alpha]) \right) = \overline{\gamma}_\sharp \left( [\overline{\gamma}\star \alpha \star \gamma ] \right)= \left[ \left( \overline{\overline{\gamma}}\star \gamma  \right)\star \alpha \star \left( \gamma \star \overline{\gamma} \right) \right] = \left[ 1_a \star \alpha \star 1_a \right]= [\alpha]$$
Analogamente si mostra che vale $\gamma_\sharp \left( \overline{\gamma}_\sharp ([\beta])\right)=[\beta]$
\end{thm}
\begin{cor}Il tipo di isomorfismo trovato precedentemente non dipende da $a$, per cui a volte si parla di "gruppo fondamentale di $X$" e lo si denota con $\pi_1(X)$
\end{cor}
\newpage
\begin{defn}
$$ \Omega(S^1,a)= \{ \gamma:\, S^1 \to X \text{ con } \gamma(1)=a \}$$
dove $S^1\subseteq\C$ da cui $1\in S^1 $ \`e $(1,0) \in \R^2 $
\end{defn}
Esiste una bigezione canonica tra $\Omega(a,a)$ e $ \Omega(S^1, a)$.\\
Se $\alpha\in\Omega(a,a)$ poich\`e $\alpha(0)=\alpha(1)$, $\alpha$ definisce 
$$ \hat{\alpha}: \frac{[0,1]}{\{0,1\}} \to X $$ 
continua.\\
Identifichiamo $\ds \frac{[0,1]}{\{ 0, 1\}}$ con $S^1 $ $\left(  t \to e^{2\pi i t }\right)$ da cui 
$$ \hat{\alpha}:\, S^1 \to X$$ 
e$\hat{\alpha}(1)=a$.\\
L'invero di $ \alpha \to \hat{\alpha}$\`e  $ \alpha(t) =\hat{\alpha}\left( e^{2\pi i t } \right)$
\spazio
%\begin{ex}$\alpha\sim \beta \text{ (come cammini) } \quad \ses \quad \hat{\alpha}\sim \hat{\beta } \text{ tramite } H:\, S^1\times [0,1] \to X \text{ con } H(1,s)=a $$ \end{ex}
\begin{lem}Sia $Q=[0,1]\times [0,1]$ e $C=\{ s=1\} \cup \{ t=0\} \cup \{ t=1\}$ ($t,s$ sono le coordinate di $Q$) 
$$ \frac{Q}{C} \cong D^2$$ 
tramite un omeomorfismo che manda $[t,0]$ in $e^{2\pi i t}$
\end{lem}
\spazio
\begin{prop}$\alpha\in Omega (a,a)$
$$ [\alpha]=1 \quad \ses \quad \hat{\alpha} \text{ si estende in mondo continuo a } D^2 $$
\proof $\implica$ se $\alpha\sim 1_a$ allora  esiste 
$$ H:\, Q \to X \quad H(t,0)=\alpha(t) \text{ e } H(C)=\{a \}$$
$H$ definisce per passaggio al quoziente 
$$\tilde{H}: \frac{Q}{C} \to X$$
e tramite l'identificazione del lemma precedente otteniamo 
$$ \tilde{H}: D^2 \to X $$ 
Osserviamo che si ha $\tilde{H}_{\vert S^1}=\hat{\alpha}$ dunque $\hat{\alpha}$ si estende a $D^2$\\
$\iimplica$ Se $\hat{\alpha}$ si estende a $f:\, D^2 \to X$.\\
La mappa $H: \, Q \to X$ data da $H = f \circ \pi $ ( $\pi:Q \to \frac{Q}{C}=D^2$ ) da un'omotopia a estremi fissi tra $\alpha$ e $1_a$
\endproof
\end{prop}
\begin{cor}Sia $P\subseteq \R^2$ un poligono convesso con lati $l_1, \cdots, l_n $ parametrizzati da $\varphi_i :[0,1]\to l_i$ e sia $\theta: \partial P \to X$ e poniamo $\alpha_i=\theta \circ \varphi_i$
$$ \alpha_1 \star cdots \star \alpha_n \sim 1_{\alpha_1(0)} \quad \ses \quad \theta \text{ si estende in modo continuo a } P $$
\proof Esiste un omeomorfismo $f:\, P\to D^2$ con $f(\partial P) = S^1$ per cui la tesi segue da quanto gi\`a visto 
\endproof
\end{cor}
\end{document}