\documentclass[a4paper,12pt]{article}
\usepackage[a4paper, top=2cm,bottom=2cm,right=2cm,left=2cm]{geometry}

\usepackage{bm,xcolor,mathdots,latexsym,amsfonts,amsthm,amsmath,
					mathrsfs,graphicx,cancel,tikz-cd,hyperref,booktabs,caption,amssymb,amssymb,wasysym}
\hypersetup{colorlinks=true,linkcolor=blue}
\usepackage[italian]{babel}
\usepackage[T1]{fontenc}
\usepackage[utf8]{inputenc}
\newcommand{\s}[1]{\left\{ #1 \right\}}
\newcommand{\sbarra}{\backslash} %% \ 
\newcommand{\ds}{\displaystyle} 
\newcommand{\alla}{^}  
\newcommand{\implica}{\Rightarrow}
\newcommand{\iimplica}{\Leftarrow}
\newcommand{\ses}{\Leftrightarrow} %se e solo se
\newcommand{\tc}{\quad \text{ t. c .} \quad } % tale che 
\newcommand{\spazio}{\vspace{0.5 cm}}
\newcommand{\bbianco}{\textcolor{white}{,}}
\newcommand{\bianco}{\textcolor{white}{,} \\}% per andare a capo dopo 																					definizioni teoremi ...


% campi 
\newcommand{\N}{\mathbb{N}} 
\newcommand{\R}{\mathbb{R}}
\newcommand{\Q}{\mathbb{Q}}
\newcommand{\Z}{\mathbb{Z}}
\newcommand{\K}{\mathbb{K}} 
\newcommand{\C}{\mathbb{C}}
\newcommand{\F}{\mathbb{F}}
\newcommand{\p}{\mathbb{P}}

%GEOMETRIA
\newcommand{\B}{\mathfrak{B}} %Base B
\newcommand{\D}{\mathfrak{D}}%Base D
\newcommand{\RR}{\mathfrak{R}}%Base R 
\newcommand{\Can}{\mathfrak{C}}%Base canonica
\newcommand{\Rif}{\mathfrak{R}}%Riferimento affine
\newcommand{\AB}{M_\D ^\B }% matrice applicazione rispetto alla base B e D 
\newcommand{\vett}{\overrightarrow}
\newcommand{\sd}{\sim_{SD}}%relazione sx dx
\newcommand{\nvett}{v_1, \, \dots , \, v_n} % v1 ... vn
\newcommand{\ncomb}{a_1 v_1 + \dots + a_n v_n} %a1 v1 + ... +an vn
\newcommand{\nrif}{P_1, \cdots , P_n} 
\newcommand{\bidu}{\left( V^\star \right)^\star}

\newcommand{\udis}{\amalg}
\newcommand{\ric}{\mathfrak{U}}
\newcommand{\inclu}{\hookrightarrow }
%ALGEBRA

\newcommand{\semidir}{\rtimes}%semidiretto
\newcommand{\W}{\Omega}
\newcommand{\norma}{\vert \vert }
\newcommand{\bignormal}{\left\vert \left\vert}
\newcommand{\bignormar}{\right\vert \right\vert}
\newcommand{\normale}{\triangleleft}
\newcommand{\nnorma}{\vert \vert \, \cdot \, \vert \vert}
\newcommand{\dt}{\, \mathrm{d}t}
\newcommand{\dz}{\, \mathrm{d}z}
\newcommand{\dx}{\, \mathrm{d}x}
\newcommand{\dy}{\, \mathrm{d}y}
\newcommand{\amma}{\gamma}
\newcommand{\inv}[1]{#1^{-1}}
\newcommand{\az}{\centerdot}
\newcommand{\ammasol}[1]{\tilde{\gamma}_{\tilde{#1}}}
\newcommand{\pror}[1]{\mathbb{P}^#1 (\R)}
\newcommand{\proc}[1]{\mathbb{P}^#1(\C)}
\newcommand{\sol}[2]{\widetilde{#1}_{\widetilde{#2}}}
\newcommand{\bsol}[3]{\left(\widetilde{#1}\right)_{\widetilde{#2}_{#3}}}
\newcommand{\norm}[1]{\left\vert\left\vert #1 \right\vert \right\vert}
\newcommand{\abs}[1]{\left\vert #1 \right\vert }
\newcommand{\ris}[2]{#1_{\vert #2}}
\newcommand{\vp}{\varphi}
\newcommand{\vt}{\vartheta}
\newcommand{\wt}[1]{\widetilde{#1}}
\newcommand{\pr}[2]{\frac{\partial \, #1}{\partial\, #2}}%derivata parziale
%per creare teoremi, dimostrazioni ... 
\theoremstyle{plain}
\newtheorem{thm}{Teorema}[section] 
\newtheorem{ese}[thm]{Esempio} 
\newtheorem{ex}[thm]{Esercizio} 
\newtheorem{fatti}[thm]{Fatti}
\newtheorem{fatto}[thm]{Fatto}

\newtheorem{cor}[thm]{Corollario} 
\newtheorem{lem}[thm]{Lemma} 
\newtheorem{al}[thm]{Algoritmo}
\newtheorem{prop}[thm]{Proposizione} 
\theoremstyle{definition} 
\newtheorem{defn}{Definizione}[section] 
\newcommand{\intt}[2]{int_{#1}^{#2}}
\theoremstyle{remark} 
\newtheorem{oss}{Osservazione} 
\newcommand{\di }{\, \mathrm{d}}
\newcommand{\tonde}[1]{\left( #1 \right)}
\newcommand{\quadre}[1]{\left[ #1 \right]}
\newcommand{\w}{\omega}

% diagrammi commutativi tikzcd
% per leggere la documentazione texdoc

\begin{document}
\textbf{Lezione del 18 ottobre del Prof. Frigerio}
\begin{defn}Un insieme $Z\subseteq X $ \`e $f$-saturo se 
$$ Z = f^{-1}(f(Z))$$ 
in modo equivalente
$$ Z = f^{-1}(C) \quad C\subseteq Y$$

\begin{oss}In generale, vale $ Z \subseteq f^{-1}(f(Z))$
\end{oss}
\end{defn}
\begin{defn}[Identificazione]\bianco
$f:\, X \to Y$ si dice identificazione se \`e continua, suriettiva e se 
$$ A\subseteq X \text{ aperto } \quad \ses \quad f^{-1}(A)\subseteq X \text{ aperto }$$
in modo equivalente
$$ A\subseteq X \text{ aperto } \quad \ses \quad \exists B \text{ aperto saturo } \quad A= f(B)$$
\end{defn}
\begin{oss}La freccia "vera" \`e $\iimplica$ infatti l'altra \`e la definizione di continuit\`a
\end{oss}
\begin{oss}Sia $f:\, X \to Y $ continua \`e biettiva. Allora
$$ f\text{ immersione } \ses f \text{ identificazione } \ses f \text{ omeomorfismo}$$
\end{oss}
\spazio
\begin{prop}Sia $f:\, X \to Y$ identificazione.\\
Definendo la relazione $x\sim x' \ses f(x)=f(x')$ allora $\overline{f}$ che fa commutare il seguente diagramma
$$ \begin{tikzcd} X \arrow[r,"f"] 
\arrow{rd}{\pi}&Y\\ &\frac{X}{\sim} \arrow{u}{\overline{f}}
\end{tikzcd}$$
\`e un omeomorfismo 
\proof $\overline{f}$ \`e ben definita e biettiva per motivi insiemistici, inoltre \`e continua per la propiet\`a universale della topologia quoziente.\\
Devo vedere che $f$ \`e aperta.\\
Sia $A\in \frac{X}{\sim}$ aperto.\\
$$ \overline{f}(A)= f(\pi^{-1}(A))$$
Ma $f$ \`e un identificazione  
$$f(\pi^{-1}(A))  \subseteq Y \text{ aperto } \quad \ses  \quad f^{-1}(f(\pi^{-1}(A))) \subseteq X \text{ aperto }$$
Ora $\pi^{-1}(A)$ \`e $f$-saturo (per come \`e stata definita la relazione di equivalenza) per cui 
$$f^{-1}(f(\pi^{-1}(A)))=\pi^{-1}(A) \text{ aperto perch\`e  } \pi \text{ continua}$$
\begin{oss} $\overline{f} $ come sopra \`e un omeomorfismo $\ses$ $f$ \`e identificazione.\\
Basta leggere con attenzione la dimostrazione 
\end{oss} 
\end{prop}
\spazio
\begin{prop}Sia $f:\, X \to Y$ continua e suriettiva
\begin{itemize}
\item $f$ \`e aperta allora \`e un'identificazione
\item $f$ \`e chiusa allora \`e un'identificazione 
\end{itemize}
\proof Mostriamo solamente il caso aperto.\\
Dato $A\subseteq Y$, devo dimostrare che
$$A \text{ aperto } \quad \ses \quad f^{-1}(A) \subseteq X \text{ aperto } $$
$\implica$ deriva dalla continuit\`a di $f$\\
$\iimplica$ essendo $f$ aperta $f^{-1}(A)$ aperto $\implica$ $ f(f^{-1}(A))$ aperto.\\
Inoltre essendo $f$ surgettiva allora $f(f^{-1}(A))=A$
\endproof
\end{prop}
\spazio
Notazione: se $A \subseteq X $ si pone con $\frac{X}{A}=\frac{X}{\sim} $ dove 
$$ x\sim y \quad\ses \quad x=y  \text{ o }  (x\in A \text{ e } y \in A $$
Tale insieme \`e ottenuto da $X$ condensando $A$ ad un punto 
\begin{ese} $\frac{[0,1]}{\{ 0, 1\}}$\\
Cerco un identificazione 
$$ f:\, [0,1]\to S' =\{ x \in \R^2 \, \vert \, \norma x \norma =1 \}$$
tale che $f(x)=f(y) \ses x=y \text{ o } \{ x ,y \}=\{ 0,1\}$.\\
Pongo 
$$ f(t)= (\cos 2\pi t, \sin 2\pi t )$$ 
tale funzione \`e continua, suriettiva e induce la relazione di equivalenza.\\
Mostriamo che \`e chiusa 
\begin{itemize}
\item $f_\vert: \, \left[ 0, \frac{1}{2} \right] \to  C_1 =S' \cap \{ y \geq 0 \}$ con $C_1$ chiuso.\\
Tale restrizione \`e un omeomorfismo essendo $g=\frac{\arccos x}{2\pi}$ una sua inversa continua
\item In modo analogo $f_\vert:\, \left[ \frac{1}{2}, 1 \right] \to C_2=S'\cap \{ y<0\}$ \`e un omeomorfismo.\\
\item Sia $Z\subseteq [0,1]$ chiuso allora 
$$ f(Z) = f\left( Z \cap \left[ 0, \frac{1}{2}\right] \right) \cup f\left( Z \cap \left[ \frac{1}{2},1\right] \right)$$
Ora $Z \cap \left[ 0, \frac{1}{2}\right]$ \`e un chiuso di $\left[0, \frac{1}{2}\right]$ e poich\`e $f_\vert$ \`e un omo sul chiuso $C_1$ allora $$f\left(  Z \cap \left[ 0, \frac{1}{2}\right] \right) \text{ \`e un chiuso di } S' \text{ (chiuso di un chiuso)}$$
Analogamente
$$ f\left(  Z \cap \left[ \frac{1}{2},1\right] \right) \text{ \`e un chiuso di } S'$$
quindi $f(Z)$ \`e unione di $2$ chiusi quindi \`e chiuso
\end{itemize}
\begin{oss}La funzione decritta non \`e aperta infatti $[0,1)$ aperto 
in $[0,1]$ ma $f([0,1)$ non \`e aperto (arco di circonferenza) con un solo "estremo chiuso"
\end{oss}
\end{ese}
\spazio
Enunciamo $2$ teoremi che ci servono per generalizzare il risultato precedente
\begin{thm} $X$ compatto, $Y$ \`e $T2$ con $f:\, X \to Y$ continua. Allora $f$ \`e chiusa
\end{thm}
\begin{thm}I chiusi e limitati di $\R^n$ sono compatti
\end{thm}
\begin{defn}Su $\R^n$ definiamo i seguenti insiemi
$$ D^n = \{ x \in R^n \, \vert \, \norma x \norma  \leq 1 \} \overline{B^n(0,1)}$$
$$ S^{n-1} =\partial D^n = \{ x \in \R^n \, \vert \, \norma x \norma =1 \}$$
\end{defn}

\begin{thm}
$$ \frac{D^n}{\partial D^n} \text{ omeomorfo a } S^n$$
\proof Cerco $f:\, D^n \to S^n $ tale che
$$ f(x)=f(y) \quad \ses \quad x=y \text{ o } \norma x \norma = \norma y \norma =1 $$
continua (per il teorema precedente, chiusa) e surgettiva.\\
Pongo
$$ f(x) = ( \lambda x , 2  \norma x \norma ^2 -1 ) $$
Poich\`e $f(x)\in S^n$ allora
$$\lambda^2 \norma x \norma ^2 + ( 2\norma x \norma ^2 -1) ^2 = 1  $$
$$ \lambda = 2 \sqrt{ 1 - \norma x \norma ^2 }$$
Si verifichi che $f$ \`e continua (dunque chiusa), biettiva e ha la propiet\`a richiesta
\end{thm}
\spazio
\begin{oss}Esistono identificazioni che non sono chiuse n\`e aperte.\\
Sia  $$ X =\{ x \geq 0  \} \cap \{ y =0 \}  \subseteq \R^2 $$
Allora $\pi:X \to \R \quad \pi(x,y)= x$ \`e un identificazione.\\
Sia $ C \subseteq \R$  allora dobbiamo provare che 
$$ \pi^{-1} ( C) \text{ chiuso } \quad \implica  \quad C \text{ chiuso } $$
Sia $p \in \overline{C}$ allora essendo $\R$ Hausdorff 
$$ \exists \{ p_n \}\subseteq C \quad \lim p_n = p $$
Ora $(p_n, 0 ) \in \pi^{-1}(C) $ $\forall n $ quindi supponendo che la controimmagine sia chiusa
$$ \lim (p_n , 0 ) = (p,0) \in \overline{\pi^{-1}(C)} =\pi^{-1}(C)$$
Dunque $p=\pi(p,0) \in C$ ovvero abbiamo provato che $C=\overline{C}$.\\
Mostriamo che non \`e aperta.\\
Sia  $ Y = \{ x \in X \quad d( x,0)<1 \}$ $Y$ \`e aperta in $X$ infatti 
$Y = B^2(0,1) \cap X $ mentre $\pi(Y) =[0,1)$ non \`e aperto in $\R$\\
Se prendiamo come chiuso un ramo di iperbole (primo quadrante) otteniamo che essa \`e chiusa in $X$ mentre la sua immagine non \`e chiusa in $\R$
\end{oss}
\end{document}
