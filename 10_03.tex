\documentclass[a4paper,12pt]{article}
\usepackage[a4paper, top=2cm,bottom=2cm,right=2cm,left=2cm]{geometry}

\usepackage{bm,xcolor,mathdots,latexsym,amsfonts,amsthm,amsmath,
					mathrsfs,graphicx,cancel,tikz-cd,hyperref,booktabs,caption,amssymb,amssymb,wasysym}
\hypersetup{colorlinks=true,linkcolor=blue}
\usepackage[italian]{babel}
\usepackage[T1]{fontenc}
\usepackage[utf8]{inputenc}
\newcommand{\s}[1]{\left\{ #1 \right\}}
\newcommand{\sbarra}{\backslash} %% \ 
\newcommand{\ds}{\displaystyle} 
\newcommand{\alla}{^}  
\newcommand{\implica}{\Rightarrow}
\newcommand{\iimplica}{\Leftarrow}
\newcommand{\ses}{\Leftrightarrow} %se e solo se
\newcommand{\tc}{\quad \text{ t. c .} \quad } % tale che 
\newcommand{\spazio}{\vspace{0.5 cm}}
\newcommand{\bbianco}{\textcolor{white}{,}}
\newcommand{\bianco}{\textcolor{white}{,} \\}% per andare a capo dopo 																					definizioni teoremi ...


% campi 
\newcommand{\N}{\mathbb{N}} 
\newcommand{\R}{\mathbb{R}}
\newcommand{\Q}{\mathbb{Q}}
\newcommand{\Z}{\mathbb{Z}}
\newcommand{\K}{\mathbb{K}} 
\newcommand{\C}{\mathbb{C}}
\newcommand{\F}{\mathbb{F}}
\newcommand{\p}{\mathbb{P}}

%GEOMETRIA
\newcommand{\B}{\mathfrak{B}} %Base B
\newcommand{\D}{\mathfrak{D}}%Base D
\newcommand{\RR}{\mathfrak{R}}%Base R 
\newcommand{\Can}{\mathfrak{C}}%Base canonica
\newcommand{\Rif}{\mathfrak{R}}%Riferimento affine
\newcommand{\AB}{M_\D ^\B }% matrice applicazione rispetto alla base B e D 
\newcommand{\vett}{\overrightarrow}
\newcommand{\sd}{\sim_{SD}}%relazione sx dx
\newcommand{\nvett}{v_1, \, \dots , \, v_n} % v1 ... vn
\newcommand{\ncomb}{a_1 v_1 + \dots + a_n v_n} %a1 v1 + ... +an vn
\newcommand{\nrif}{P_1, \cdots , P_n} 
\newcommand{\bidu}{\left( V^\star \right)^\star}

\newcommand{\udis}{\amalg}
\newcommand{\ric}{\mathfrak{U}}
\newcommand{\inclu}{\hookrightarrow }
%ALGEBRA

\newcommand{\semidir}{\rtimes}%semidiretto
\newcommand{\W}{\Omega}
\newcommand{\norma}{\vert \vert }
\newcommand{\bignormal}{\left\vert \left\vert}
\newcommand{\bignormar}{\right\vert \right\vert}
\newcommand{\normale}{\triangleleft}
\newcommand{\nnorma}{\vert \vert \, \cdot \, \vert \vert}
\newcommand{\dt}{\, \mathrm{d}t}
\newcommand{\dz}{\, \mathrm{d}z}
\newcommand{\dx}{\, \mathrm{d}x}
\newcommand{\dy}{\, \mathrm{d}y}
\newcommand{\amma}{\gamma}
\newcommand{\inv}[1]{#1^{-1}}
\newcommand{\az}{\centerdot}
\newcommand{\ammasol}[1]{\tilde{\gamma}_{\tilde{#1}}}
\newcommand{\pror}[1]{\mathbb{P}^#1 (\R)}
\newcommand{\proc}[1]{\mathbb{P}^#1(\C)}
\newcommand{\sol}[2]{\widetilde{#1}_{\widetilde{#2}}}
\newcommand{\bsol}[3]{\left(\widetilde{#1}\right)_{\widetilde{#2}_{#3}}}
\newcommand{\norm}[1]{\left\vert\left\vert #1 \right\vert \right\vert}
\newcommand{\abs}[1]{\left\vert #1 \right\vert }
\newcommand{\ris}[2]{#1_{\vert #2}}
\newcommand{\vp}{\varphi}
\newcommand{\vt}{\vartheta}
\newcommand{\wt}[1]{\widetilde{#1}}
\newcommand{\pr}[2]{\frac{\partial \, #1}{\partial\, #2}}%derivata parziale
%per creare teoremi, dimostrazioni ... 
\theoremstyle{plain}
\newtheorem{thm}{Teorema}[section] 
\newtheorem{ese}[thm]{Esempio} 
\newtheorem{ex}[thm]{Esercizio} 
\newtheorem{fatti}[thm]{Fatti}
\newtheorem{fatto}[thm]{Fatto}

\newtheorem{cor}[thm]{Corollario} 
\newtheorem{lem}[thm]{Lemma} 
\newtheorem{al}[thm]{Algoritmo}
\newtheorem{prop}[thm]{Proposizione} 
\theoremstyle{definition} 
\newtheorem{defn}{Definizione}[section] 
\newcommand{\intt}[2]{int_{#1}^{#2}}
\theoremstyle{remark} 
\newtheorem{oss}{Osservazione} 
\newcommand{\di }{\, \mathrm{d}}
\newcommand{\tonde}[1]{\left( #1 \right)}
\newcommand{\quadre}[1]{\left[ #1 \right]}
\newcommand{\w}{\omega}

% diagrammi commutativi tikzcd
% per leggere la documentazione texdoc

\begin{document}
\textbf{Lezione del 3 Ottobre del prof. Frigerio}

\begin{prop}$f:\, (X,\tau)\to (Y,\tau')$ e siano $S$ e $\B$ una prebase e base di $\tau$.\\
I seguenti fatti sono tra loro equivalenti
\begin{itemize}
\item[(i)]$f$ \`e continua
\item[(ii)]$f^{-1}(A) $ \`e aperto $\forall A \in S$
\item[(iii)]$f^{-1}(A)$ \`e aperto $\forall A \in \B$
\end{itemize}
\proof \bbianco
\begin{itemize}
\item (i)$\implica$(ii) Ogni elemento di $S$ \`e aperto ed essendo $f$ continua la controimmagine di aperti \`e un aperto 
\item (ii)$\implica$(i) Sia $\overline{\B}=\{ \text{ intersezione finite di elementi di } S \cup \{ Y \} \} $\\
Dalla distributivit\`a di $f^{-1}$ rispetto all'intersezione e dal fatto che un' intersezione finite di aperti \`e un aperto 
$$ \forall B \in \overline{\B} \quad f^{-1}(B)\text{ \`e un aperto di } X $$
Dalla definizione di prebase $\forall A \in \tau'$ 
$$ A = \bigcup_{i\in I } B_i \text{ dove } B_i \in \B $$ 
Ora se $A=Y$ allora $f^{-1}(A)=X$ che \`e aperto altrimenti 
$$f^{-1}(A)= f^{-1} \left( \bigcup_{i \in I } B_i \right)$$ 
ora essendo $f^{-1}$ distributiva rispetto all'unione 
$$ f^{-1}(A)= \bigcup_{i \in I} f^{-1}(B_i)$$
Ma l'unione arbitraria di aperti \`e un aperto 
\item (i)$\ses$(iii)  Ogni base \`e in particolare una prebase
\end{itemize}
\endproof
\end{prop}

\newpage

\begin{prop}Sia $(X,\tau)$ uno spazio topologico e $B\subseteq X$ allora
$$ X = B^\circ \amalg \partial B \amalg (X\sbarra B)^\circ$$
\proof Poich\`e $\overline{B}$ \`e il pi\`u piccolo chiuso che contiene $B$  allora $X\sbarra \overline{B}$ \`e il piu grande aperto contenuto in $X\sbarra B$ dunque 
$$ (X\sbarra B)^\circ = X \sbarra \overline{B}$$
Inoltre, per definizione di frontiera e poich\`e $B^\circ \subseteq \overline{B}$
$$\overline{B}=B^\circ \amalg \partial B $$
Da cui segue
$$ X=\overline{B} \amalg ( X\sbarra \overline{B}) = B^\circ \amalg \partial B \amalg (X\sbarra B)^\circ$$
\endproof
\end{prop}
\spazio
\begin{defn}$Y \subseteq X $ si dice denso se $\overline{Y}=X$
\end{defn}
\begin{oss}$$ Y \text{ denso } \quad \ses \quad Y \cap A \neq \emptyset \quad \forall A \text{ aperto non vuoto}$$
Dalla decomposizione mostrata precedentemente $ X =  \overline{Y} \amalg (X\sbarra Y)^\circ$
$$ Y \text{\`e denso } \quad \ses \quad (X\sbarra Y)^\circ = \emptyset$$ 
L'ultima \`e equivalente a dire che gli unici aperti contenuti in $X\sbarra Y$ sono vuoti che \`e equivalente alla tesi.
\end{oss}
\spazio
\begin{defn}$X$ si dice separabile se ammette un sottoinsieme denso e numerabile
\end{defn}
\begin{prop} $X$ \`e a base numerabile $\implica$  $X$ separabile
\proof Sia $ \ds \B=\{ B_i \}_{i \in \N} $ una base numerabile di $X$ (posso supporre che  $B_i$ non vuoto per ogni $i$)\\
$\forall i $ scegliamo $x_i\in B_i$ e sia $Y=\{ x_i \, \vert \, i \in \N\}$\\
Osserviamo che $Y$ \`e al pi\`u numerabile, resta da provare che \`e denso.\\
Sia $A$ un aperto non vuoto, dunque, $A$ \`e unione di elementi della base $\B$ dunque 
$$ \exists j \quad B_j \subseteq A \quad \implica \quad \{ x_j \} \subseteq A \cap Y $$ 
Dunque grazie all'osservazione precedente $Y$ \`e denso  
\endproof
\end{prop}
\spazio
\begin{prop}Se $(X,\tau)$ \`e metrizzabile.
$$ X \text{ separabile} \quad \ses \quad X \text{ \`e a base numerabile} $$
\proof
$\iimplica$ \`e vera per un qualsiasi spazio topologico\\
$\implica$ Sia $Y$ un denso numerabile su $X$ e sia $d$ la distanza che induce $\tau$\\
Consideriamo l'insieme 
$$ \B=\{ B(x,R) \, \vert \, x \in Y , \, R \in \Q_+ \}$$ 
osserviamo che $\B$ \`e numerabile infatti \`e in bigezione con $\N \times \N $ ($\Q$ \`e in bigezione con $\N$ e $Y$ essendo numerabile \`e in bigezione con $\N$)\\
Occorre provare che $\B$ \`e una base.\\
Sia $A$ aperto, allora, per definzione $\exists R>0$ tale che $B(x,R)\subseteq A$.\\
L'insieme $\ds A'=B\left(x, \frac{R}{3} \right)$ \`e un aperto per cui essendo $Y$ denso  $A\cap Y\neq \emptyset $ dunque $\exists y\in Y$ tale che $\ds d(x,y)<\frac{R}{3}$\\
Sia $R'\in \Q$ tale che $\ds \frac{R}{3}<R' < \frac{2}{3}R $ allora 
$ B(y,R') \in \B $ \\
Osserviamo che $x\in B(y,R')$ e infatti $d(x,y)< \frac{R}{3}<R'$ .\\
Sia $z\in B(y,R')$ allora
$$ d(x,z) \leq d(x,y)+ d(y,z) < \frac{R}{3}+R' < R $$ dunque
$z\in B(x,R) \subseteq A$ e perci\`o $B(y,R') \subseteq A $
\endproof
\end{prop}
\newpage

\begin{defn}Sia $(X,\tau)$ uno spazio topologico e $x_0\in X$.\\
Un insieme $U\subseteq X$ \`e un intorno di $x_0$ se $x_0\in U^\circ$.\\
In modo equivalente, se $\exists V $ aperto con $x_0 \in V \subseteq U$.\\
Denoteremo con $I(x_0)$ l'insieme degli intorni di $x_0$
\end{defn}

\begin{defn}Un sistema fondamentale di intorni per $x_0$ \`e una famiglia $\mathfrak{F}\subseteq I(x_0)$ tale che $$\forall U \in I(x_0) \quad \exists V \in \mathfrak{F} \quad V \subseteq U$$
%contiene intorni suff. piccoli
\end{defn}
\begin{oss}Se $X$ \`e metrico.
$$ U \text{ intorno di } x_0 \quad \ses \quad \exists R>0 \quad B(x_0, R) \subseteq U $$
Dunque la famiglia
$$\mathfrak{F}=\left\{ B \left(x_0, \frac{1}{n}\right) ,\, n \in \N_+ \right\}$$
\`e un sistema fondamentale (numerabile) di intorni di $x_0$ 
\end{oss}
\spazio
\begin{defn}[Primo numerabile]\bianco$X$ soddisfa il primo assioma di numerabilit\`a se ogni $x_0\in X$ ha un sistema fondamentale di intorni numerabili
\end{defn}

\begin{fatto}
$$X \text{ metrizzabile} \quad \implica \quad X \text{ primo numerabile}$$
\end{fatto}
\begin{prop}
$$ \text{ II assioma di numerabilit\`a } \quad \implica \text{ I assioma di numerabilit\`a}$$
\proof Se $\B$ \`e una base numerabile, sia $x_0 \in X$ 
$$ \mathfrak{F}=\{ B \in \B , \vert \, x_0\in B \}$$
osserviamo che $\mathfrak{F}$ \`e un sistema fondamentale di intorni per $x_0$.\\
Sia $U$ un intorno di $x_0$, essendo  $U^\circ$ un aperto 
$$ \exists I \subseteq \N \quad \exists B_i \in \B \quad U^\circ = \bigcup_{i\in I } B_i$$
Dunque $\exists i_0 $ tale che 
$$ x_0 \in B_{i_0} \subseteq U^\circ \subseteq U \text{ in quanto } x_0 \in U^\circ$$
Da ci\`o segue la tesi in quanto $B_{i_0}\in \mathfrak{F}$ ed inoltre $\mathfrak{F}\subseteq \B$ dunque \`e numerabile
\endproof
\end{prop}
\spazio
\begin{prop}[Aperti e intorni]
$$ A\subseteq X \text{ \`e aperto } \quad \ses \quad \text{  \`e intorno di ogni suo punto } $$
\proof
$$ A\text{ \`e apeto } \ses A = A^\circ \ses ( x\in A \ses x\in A^\circ ) \ses$$
$$\ses ( x\in A \ses A\text{ intorno di x}) \ses A\text{ intorno di ogni suo punto } $$
\endproof
\end{prop}
\begin{prop}Sia $C\subseteq X$ generico
$$ x \in \overline{C} \quad \ses \quad \forall U \in I(x) \quad U \cap C \neq \emptyset$$
\proof$\implica$ 
$ x\in \overline{C}\implica x\not \in (C\sbarra X)^\circ \implica X \sbarra C \text{ non \`e un intorno di } X$\\
Se $\exists U \in I(x_0)$ tale che $U \cap C = \emptyset$ allora $U \subseteq X\sbarra C$ ovvero $X\sbarra C $ \`e un intorno di $x$ (assurdo).\\
$\iimplica$  $x\not \in \overline{C}\implica x\in (X\sbarra C)^\circ$ dunque 
$\exists U \in I(x_0)$ tale che $U \subseteq X \sbarra C $ dunque $X\sbarra C$ \`e l'intorno disgiunto cercato
\endproof

\end{prop}
\newpage

Sia $X=\R$ e $\B=\{ [a,b) , \, a<b \}$ allora
\begin{enumerate}
\item $\B$ \`e una base di una topologia $\tau$ 
\item $\tau$ \`e la pi\`u fine della topologia euclidea
\item $\tau$ \`e separabile
\item $\tau$ non \`e a base numerabile
\item $\tau$ non \`e metrizzabile
\end{enumerate}
\begin{enumerate}
\item Mostriamo che la base ricopre $\R$ 
$$ \R = \bigcup_{n \in \Z} [n , n+1)$$
inoltre 
$$ [a,b) \cap [c,d) =\begin{cases} \emptyset \\
[e,f) \text{ con } e=\max \{ a,c \} \text{ e } f = \min \{ b ,d \} 
\end{cases}$$
Valgono entrambi le richieste del criterio dunque $\B$ \`e una base
\item Basta vedere che ogni aperto della topologia euclidea \`e anche un aperto di $\tau$
$$ B(x_0,R)= (x_0-R, x_0+R) = \bigcup_{n \in \N_+}=\left[ x_0-R+\frac{1}{n}, x_0+R \right) $$
\item $\Q$ incontra tutti gli aperti non vuoti di $\B$, dunque di $\tau$ da ci\`o segue che \`e denso 
\item Sia $\B'$ una qualsiasi base di $\tau$.\\
$\forall	 B \in \B' $ sia $f(B)=\inf B \in \R \cup \{ - \infty \} $\\
Si dimostra che $\R \subseteq Im f $ perci\`o $\B'$ non \`e numerabile
\item Uno spazio separabile e a base numerabile se e solo se $\tau$ \`e metrizzabile. Dunque $\tau$ non \`e metrizzabile
\end{enumerate}
\newpage

\end{document}




