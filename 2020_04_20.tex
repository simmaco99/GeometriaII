\documentclass[a4paper,12pt]{article}
\usepackage[a4paper, top=2cm,bottom=2cm,right=2cm,left=2cm]{geometry}

\usepackage{bm,xcolor,mathdots,latexsym,amsfonts,amsthm,amsmath,
					mathrsfs,graphicx,cancel,tikz-cd,hyperref,booktabs,caption,amssymb,amssymb,wasysym}
\hypersetup{colorlinks=true,linkcolor=blue}
\usepackage[italian]{babel}
\usepackage[T1]{fontenc}
\usepackage[utf8]{inputenc}
\newcommand{\s}[1]{\left\{ #1 \right\}}
\newcommand{\sbarra}{\backslash} %% \ 
\newcommand{\ds}{\displaystyle} 
\newcommand{\alla}{^}  
\newcommand{\implica}{\Rightarrow}
\newcommand{\iimplica}{\Leftarrow}
\newcommand{\ses}{\Leftrightarrow} %se e solo se
\newcommand{\tc}{\quad \text{ t. c .} \quad } % tale che 
\newcommand{\spazio}{\vspace{0.5 cm}}
\newcommand{\bbianco}{\textcolor{white}{,}}
\newcommand{\bianco}{\textcolor{white}{,} \\}% per andare a capo dopo 																					definizioni teoremi ...


% campi 
\newcommand{\N}{\mathbb{N}} 
\newcommand{\R}{\mathbb{R}}
\newcommand{\Q}{\mathbb{Q}}
\newcommand{\Z}{\mathbb{Z}}
\newcommand{\K}{\mathbb{K}} 
\newcommand{\C}{\mathbb{C}}
\newcommand{\F}{\mathbb{F}}
\newcommand{\p}{\mathbb{P}}

%GEOMETRIA
\newcommand{\B}{\mathfrak{B}} %Base B
\newcommand{\D}{\mathfrak{D}}%Base D
\newcommand{\RR}{\mathfrak{R}}%Base R 
\newcommand{\Can}{\mathfrak{C}}%Base canonica
\newcommand{\Rif}{\mathfrak{R}}%Riferimento affine
\newcommand{\AB}{M_\D ^\B }% matrice applicazione rispetto alla base B e D 
\newcommand{\vett}{\overrightarrow}
\newcommand{\sd}{\sim_{SD}}%relazione sx dx
\newcommand{\nvett}{v_1, \, \dots , \, v_n} % v1 ... vn
\newcommand{\ncomb}{a_1 v_1 + \dots + a_n v_n} %a1 v1 + ... +an vn
\newcommand{\nrif}{P_1, \cdots , P_n} 
\newcommand{\bidu}{\left( V^\star \right)^\star}

\newcommand{\udis}{\amalg}
\newcommand{\ric}{\mathfrak{U}}
\newcommand{\inclu}{\hookrightarrow }
%ALGEBRA

\newcommand{\semidir}{\rtimes}%semidiretto
\newcommand{\W}{\Omega}
\newcommand{\norma}{\vert \vert }
\newcommand{\bignormal}{\left\vert \left\vert}
\newcommand{\bignormar}{\right\vert \right\vert}
\newcommand{\normale}{\triangleleft}
\newcommand{\nnorma}{\vert \vert \, \cdot \, \vert \vert}
\newcommand{\dt}{\, \mathrm{d}t}
\newcommand{\dz}{\, \mathrm{d}z}
\newcommand{\dx}{\, \mathrm{d}x}
\newcommand{\dy}{\, \mathrm{d}y}
\newcommand{\amma}{\gamma}
\newcommand{\inv}[1]{#1^{-1}}
\newcommand{\az}{\centerdot}
\newcommand{\ammasol}[1]{\tilde{\gamma}_{\tilde{#1}}}
\newcommand{\pror}[1]{\mathbb{P}^#1 (\R)}
\newcommand{\proc}[1]{\mathbb{P}^#1(\C)}
\newcommand{\sol}[2]{\widetilde{#1}_{\widetilde{#2}}}
\newcommand{\bsol}[3]{\left(\widetilde{#1}\right)_{\widetilde{#2}_{#3}}}
\newcommand{\norm}[1]{\left\vert\left\vert #1 \right\vert \right\vert}
\newcommand{\abs}[1]{\left\vert #1 \right\vert }
\newcommand{\ris}[2]{#1_{\vert #2}}
\newcommand{\vp}{\varphi}
\newcommand{\vt}{\vartheta}
\newcommand{\wt}[1]{\widetilde{#1}}
\newcommand{\pr}[2]{\frac{\partial \, #1}{\partial\, #2}}%derivata parziale
%per creare teoremi, dimostrazioni ... 
\theoremstyle{plain}
\newtheorem{thm}{Teorema}[section] 
\newtheorem{ese}[thm]{Esempio} 
\newtheorem{ex}[thm]{Esercizio} 
\newtheorem{fatti}[thm]{Fatti}
\newtheorem{fatto}[thm]{Fatto}

\newtheorem{cor}[thm]{Corollario} 
\newtheorem{lem}[thm]{Lemma} 
\newtheorem{al}[thm]{Algoritmo}
\newtheorem{prop}[thm]{Proposizione} 
\theoremstyle{definition} 
\newtheorem{defn}{Definizione}[section] 
\newcommand{\intt}[2]{int_{#1}^{#2}}
\theoremstyle{remark} 
\newtheorem{oss}{Osservazione} 
\newcommand{\di }{\, \mathrm{d}}
\newcommand{\tonde}[1]{\left( #1 \right)}
\newcommand{\quadre}[1]{\left[ #1 \right]}
\newcommand{\w}{\omega}

% diagrammi commutativi tikzcd
% per leggere la documentazione texdoc

\begin{document}
\textbf{Lezione del 20 aprile }
\begin{thm}[Lemma di Schwarz]\bianco
Sia $f(z)$ una funzione olomorfa nel disco aperto $\abs{z} < 1 $.\\
Assumiamo $f(0)=0$ e $\abs{f(z)}<1$ per $\abs{z} < 1 $ allora 
\begin{enumerate}
\item $\abs{f(z)}\leq \abs z $ e $\abs{f'(0)}\leq 1$
\item Se $\exists z_0\neq 0 $ tale che $\abs{f(z_0)} =z_0$ oppure $\abs{f'(0)}=1$ allora $f(z) = \lambda z $ con $\abs \lambda =1$
\end{enumerate}
\proof \bbianco
\begin{enumerate}
\item $f$ \`e olomorfa, dunque analitica per $\abs z <1$, sia 
$$ f(z) = \sum a_n z^n$$
l'approssimazione di Taylor nell'origine  ( con raggio di convergenza $\rho\geq 1$).\\
Definiamo la funzione 
$$ g(z) =\begin{cases}\frac{f(z)}{z} \text{ se }  0<\abs z <1 \\ 
a_1 = f'(0 \text{ se } z=0
\end{cases}$$
Dunque 
$$ g=\sum a_n z^{n-1}$$ 
da cui $g$ \`e analitica ovvero olomorfa.\\
Sia $0<r<1$ e $\abs  z=r$ abbiamo 
$$ \abs{g(z)} = \abs{\frac{f(z)}{z}} \leq \frac{1}{r} \quad \implica \quad \abs{g(z)} \leq 1 \text{ per } \abs{z} <1 $$
dove l'implicazione deriva dal principio del massimo modulo per funzioni olomorfe.\\
Abbiamo dunque provato 
$$ \abs{\frac{f(z)}{z}}< 1\quad \implica \quad \abs{f(z)}\leq \abs z$$
In particolare, vale anche $\abs{g(0)}<1$ da cui $\abs{f'(0)}<1$
\item Se $\exists a$ come nelle ipotesi, allora per il corollario al principio del massimo modulo, $g$ risulta costante da cui 
$$ g(z) = \lambda \quad \implica \frac{f(z)}{z} =\lambda \quad \implica f(z) = \lambda z$$
\end{enumerate}
\end{thm}
\begin{ex}Cosa succede se assumiamo $\abs{f(z)}\leq 1$ al posto di $\abs{f(z)} <1$ 
\end{ex}
\newpage
\section{Serie di Laurent}
\begin{defn}Una serie di Laurent \`e un'espressione della forma 
$$ \sum_{n \in \Z} a_n z^n$$
\end{defn}
Ad una serie di Laurent \`e possibile associare $2$ serie di potenze
$$ \sum_{n\geq 0 } a_n z^n \quad \sum_{n<0} a_n z^{-n}$$
assumiamo che entrambe le serie abbiano raggio di convergenza diverso da $0$ e $+\infty$ allora poniamo 
$$ \rho_1=\text{ raggio di convergenza della serie } \sum_{n\geq 0 } a_n z^n$$
$$ \rho_2=\frac{1}{\text{ raggio di convergenza della serie } \sum_{n< 0 } a_n z^{-n}}$$
Sia 
$$ f_2 =\sum_{n<0} a_n z^n$$
tale serie converge assolutamente per $\abs z >\rho_2$, mostriamo che tale funzione \`a anche olomorfa in $\abs z>\rho_2$.\\
Poniamo 
$$g(u)=f_2\tonde{\frac{1}{u}} =\sum_{n<0} a_n \tonde{\frac{1}{n}}^n =\sum_{n<0} a_n u^{-n} = \sum_{k>0} a_{-k} u^k$$
e tale serie converge assolutamente per $\abs u < \frac{1}{\rho_2}$.\\
Essendo $g$ analitica si ha 
$$ g'(z) = \sum_{k>0} k a_{-k} a^k$$
Notiamo ora  che $f_2(z) = g\tonde{\frac{1}{z}}$ dunque usando la formula di derivazione di funzioni composte otteniamo 
$$f_2'(z) = -g'\tonde{\frac{1}{z}} \frac{1}{z^2} =-\frac{1}{z^2} \left( \sum_{k>0} ka_{-k} \left(\frac{1}{z}\right)^{k-1} \right) = \sum_{n<0} na_n z^{n-1} $$
dunque esiste $f_2'(z)$ da cui $f_2$ \`e olomorfa.
\spazio
\begin{prop}Sia $\sum_{n\in \Z}a_n z^n$ una serie di Laurent.\\
Assumiamo che 
\begin{itemize}
\item Le serie $\sum_{n \geq 0 } a_n z^n$ e $\sum_{n<0} a_nz^{-n}$ sono assolutamente convergenti
\item I $\rho_1$ e $\rho_2$ definiti sopra soddisfano $\rho_2<\rho_1$
\end{itemize}
allora la somma $f(z)$ della serie di Laurent \`e olomorfa nella corona $\rho_2< \abs z <\rho_	1$ e la serie converge normalmente in $r_2\leq\abs z \leq r_1$ con $\rho_2<r_2<r_1<\rho_1$
\end{prop}
\begin{defn}Diciamo che una funzione definita sulla corona $\rho_2<\abs z<\rho_1$ ha un'espansione di Laurent se esiste una serie di Laurent che converge in questa corona e di cui $f(z)$ \`e la somma (per ogni $z$ nella corona)
\end{defn}
\begin{oss}Se $f$ ammette una serie di Laurent, allora $f$ \`e olomorfa nella corona.\\
Infatti se 
$$ f(z) = \sum_{z\in \Z} a_n z^n$$ 
sappiamo che le funzione $$f_1(z)=\sum_{z\geq 0 a_n z^n} \quad f_2(z)=\sum_{z<0} a_n z^n$$ sono olomorfe.\\
Concludiamo osservando che la somma di funzioni olomorfe \`e olomorfa da cui $f(z) = f_1(z) +f_2(z)$ \`e olomorfa 
\end{oss}

 \end{document}