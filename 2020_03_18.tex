\documentclass[a4paper,12pt]{article}
\usepackage[a4paper, top=2cm,bottom=2cm,right=2cm,left=2cm]{geometry}

\usepackage{bm,xcolor,mathdots,latexsym,amsfonts,amsthm,amsmath,
					mathrsfs,graphicx,cancel,tikz-cd,hyperref,booktabs,caption,amssymb,amssymb,wasysym}
\hypersetup{colorlinks=true,linkcolor=blue}
\usepackage[italian]{babel}
\usepackage[T1]{fontenc}
\usepackage[utf8]{inputenc}
\newcommand{\s}[1]{\left\{ #1 \right\}}
\newcommand{\sbarra}{\backslash} %% \ 
\newcommand{\ds}{\displaystyle} 
\newcommand{\alla}{^}  
\newcommand{\implica}{\Rightarrow}
\newcommand{\iimplica}{\Leftarrow}
\newcommand{\ses}{\Leftrightarrow} %se e solo se
\newcommand{\tc}{\quad \text{ t. c .} \quad } % tale che 
\newcommand{\spazio}{\vspace{0.5 cm}}
\newcommand{\bbianco}{\textcolor{white}{,}}
\newcommand{\bianco}{\textcolor{white}{,} \\}% per andare a capo dopo 																					definizioni teoremi ...


% campi 
\newcommand{\N}{\mathbb{N}} 
\newcommand{\R}{\mathbb{R}}
\newcommand{\Q}{\mathbb{Q}}
\newcommand{\Z}{\mathbb{Z}}
\newcommand{\K}{\mathbb{K}} 
\newcommand{\C}{\mathbb{C}}
\newcommand{\F}{\mathbb{F}}
\newcommand{\p}{\mathbb{P}}

%GEOMETRIA
\newcommand{\B}{\mathfrak{B}} %Base B
\newcommand{\D}{\mathfrak{D}}%Base D
\newcommand{\RR}{\mathfrak{R}}%Base R 
\newcommand{\Can}{\mathfrak{C}}%Base canonica
\newcommand{\Rif}{\mathfrak{R}}%Riferimento affine
\newcommand{\AB}{M_\D ^\B }% matrice applicazione rispetto alla base B e D 
\newcommand{\vett}{\overrightarrow}
\newcommand{\sd}{\sim_{SD}}%relazione sx dx
\newcommand{\nvett}{v_1, \, \dots , \, v_n} % v1 ... vn
\newcommand{\ncomb}{a_1 v_1 + \dots + a_n v_n} %a1 v1 + ... +an vn
\newcommand{\nrif}{P_1, \cdots , P_n} 
\newcommand{\bidu}{\left( V^\star \right)^\star}

\newcommand{\udis}{\amalg}
\newcommand{\ric}{\mathfrak{U}}
\newcommand{\inclu}{\hookrightarrow }
%ALGEBRA

\newcommand{\semidir}{\rtimes}%semidiretto
\newcommand{\W}{\Omega}
\newcommand{\norma}{\vert \vert }
\newcommand{\bignormal}{\left\vert \left\vert}
\newcommand{\bignormar}{\right\vert \right\vert}
\newcommand{\normale}{\triangleleft}
\newcommand{\nnorma}{\vert \vert \, \cdot \, \vert \vert}
\newcommand{\dt}{\, \mathrm{d}t}
\newcommand{\dz}{\, \mathrm{d}z}
\newcommand{\dx}{\, \mathrm{d}x}
\newcommand{\dy}{\, \mathrm{d}y}
\newcommand{\amma}{\gamma}
\newcommand{\inv}[1]{#1^{-1}}
\newcommand{\az}{\centerdot}
\newcommand{\ammasol}[1]{\tilde{\gamma}_{\tilde{#1}}}
\newcommand{\pror}[1]{\mathbb{P}^#1 (\R)}
\newcommand{\proc}[1]{\mathbb{P}^#1(\C)}
\newcommand{\sol}[2]{\widetilde{#1}_{\widetilde{#2}}}
\newcommand{\bsol}[3]{\left(\widetilde{#1}\right)_{\widetilde{#2}_{#3}}}
\newcommand{\norm}[1]{\left\vert\left\vert #1 \right\vert \right\vert}
\newcommand{\abs}[1]{\left\vert #1 \right\vert }
\newcommand{\ris}[2]{#1_{\vert #2}}
\newcommand{\vp}{\varphi}
\newcommand{\vt}{\vartheta}
\newcommand{\wt}[1]{\widetilde{#1}}
\newcommand{\pr}[2]{\frac{\partial \, #1}{\partial\, #2}}%derivata parziale
%per creare teoremi, dimostrazioni ... 
\theoremstyle{plain}
\newtheorem{thm}{Teorema}[section] 
\newtheorem{ese}[thm]{Esempio} 
\newtheorem{ex}[thm]{Esercizio} 
\newtheorem{fatti}[thm]{Fatti}
\newtheorem{fatto}[thm]{Fatto}

\newtheorem{cor}[thm]{Corollario} 
\newtheorem{lem}[thm]{Lemma} 
\newtheorem{al}[thm]{Algoritmo}
\newtheorem{prop}[thm]{Proposizione} 
\theoremstyle{definition} 
\newtheorem{defn}{Definizione}[section] 
\newcommand{\intt}[2]{int_{#1}^{#2}}
\theoremstyle{remark} 
\newtheorem{oss}{Osservazione} 
\newcommand{\di }{\, \mathrm{d}}
\newcommand{\tonde}[1]{\left( #1 \right)}
\newcommand{\quadre}[1]{\left[ #1 \right]}
\newcommand{\w}{\omega}

% diagrammi commutativi tikzcd
% per leggere la documentazione texdoc

\begin{document}
\textbf{Lezione del 18  Marzo}
\begin{defn}Se $U\subseteq \C$ \`e un aperto e $f:\, U \to \C$ \`e una funzione possiamo scrivere 
$$f(z)=u(z)+iv(z)$$
chiamiamo 
\begin{itemize}
\item $u:\, U \to \R$ la parte reale di $f$ 
\item $v:\, U \to \R$ la parte immaginaria di $f$ 
\end{itemize}
\end{defn}
\begin{fatto}Sia $U,f,u,v$ come sopra e $z_0\in U$
$$ f \text{ continua in } z_0 \quad \ses \quad u,v\text{ continue in } z_0 $$
\end{fatto}
\begin{defn}Sia $U \subseteq \C$.\\
La funzione $f:\, U \to \C$ \`e differenziabile in $z_0 \in U$ se $\exists A:\, \C \to \C$ applicazione $\R$-lineare tale che 
\begin{enumerate}
\item $f(z)=f(z_0)+A(z-z_0)+r(z)$
\item $\ds \frac{\abs{r(z)}}{\abs{z-z_0}} \to 0  \text{ per } z \to z_0$
\end{enumerate}
In questo caso chiameremo l'applicazione $A$ lo Jacobiano di $f$ e lo denoteremo con $J_{f_{z_0}}$ 
\end{defn}
\begin{oss}$f$ differenziabile in $z_0$ allora $f$ continua in $z_0$
\end{oss}
\begin{fatti}\bbianco
\begin{enumerate}
\item Se $f$ \`e differenziabile in $z_0\in U$ allora esistono le derivate parziali di $f\in z_0$ e si ha 
$$ J_{f_{z_0}} =\begin{pmatrix}
\pr u x (x_0,y_0) & \pr u y (x_0, y_0) \\
\pr v x (x_0,y_0) & \pr v y (x_0, y_0) \\
\end{pmatrix}$$ 
dove $u,v$ sono la parte reale ed immaginaria di $f$ e $z_0=x_0+ i y_0$
\item (Teorema del differenziale totale)\\
Siano $u,v$ come sopra. Se esistono $\pr u x ,\, \pr u y,\, \pr v x \text{ e } \pr v y $ in un intorno di $z_0$ e sono continue in $z_0$ allora $f$ \`e differenziabile in $z_0$
\end{enumerate}
\end{fatti}
\begin{ex}Sia $f:\, R^2 \to \R$ con $f((x,y))=\begin{cases} x \text{ se } y \neq x^2 \\ 0 \text{ se } y=x^2\end{cases}$\\
$f$\`e differenziabile in $(0,0)$?
\end{ex}
\begin{defn}\bianco 
Sia $U\subseteq \C$ un aperto e $f:\, U \to \C$ continua\\
$f$ si dice olomorfa in $z_0\in U$ se esiste il limite 
$$\lim_{h \to 0\atop{h\neq 0}}\frac{f(z_0+h)-f(z_0)}{h}$$
se tale limite esiste, lo chiamiamo derivata di $f$ in $z_0$ e lo denotiamo con $f'(z_0)$\\ \\ 
Diciamo che $f$ \`e olomorfa in $U$ se lo \`e in ogni punto di $U$\\ \\
Diciamo che $f$ \`e intera se \`e definita su $\C$ ed \`e olomorfa su $\C$
\end{defn}
\begin{prop}Sia $U\subseteq \C$ aperto e siano $f,g:\, U \to \C$ olomorfe in $z_0$ allora 
\begin{itemize}
\item $f+g$ \`e olomorfa in $z_0$ con $(f+g)'(z_0)=f'(z_0)+g'(z_0)$
\item $fg$ \`e olomorfa in $z_0$ con $(fg)'(z_0)=f'(z_0)g(z_0)+f(z_0)g'(z_0)$
\item se $g(z_0)\neq 0 $ allora $\frac{f}{g}$ \`e olomorfa in $z_0$ con 
$$ \left(\frac{f}{g}\right)'(z_0)=\frac{f'(z_0)g(z_0)-f(z_0)g'(z_0)}{g(z_0)^2}$$
\end{itemize}
\end{prop}
\begin{prop}
Siano $U,V$ aperti di $\C$. Siano $f:\, U \to V$ e $g:\, V\to \C$ funzioni.\\
Assumiamo $f$ olomorfa in $z_0\in U$ e $g$ olomorfa in $f(z_0)$, allora $g\circ f$ \`e olomorfa in $z_0$ e si ha
$$ (g\circ f)'(z_0)= g'(f(z_0))f'(z_0)$$
\end{prop}
\begin{ese}\bbianco
\begin{itemize}
\item $f(z)=z$ \`e intera
$$\frac{f(z+h)-f(z)}{h}=\frac{z+h-z}{h}=1 \to 1 $$
dunque $f'(z)=1 \, \forall z\in \C$
\item $f(z)=z^n$ \`e intera con $n>1$
$$ \frac{(z+h)^n - z^n}{h}= \frac{\sum_{k=0}^n {n\choose k} z^k h^{n-k} -z^n }{h} = \sum_{k=0}^{n-1}{n\choose k} z^k h^{n-k-1} \to n z^{n-1}$$
dunque $f'(z)= n z^{n-1}$
\item $f(z)=z^n $ con $n<0$ \`e olomorfa in $C\sbarra \{0\}$ (esercizio)
\item Dagli esempi precedenti segue che i polinomi sono interi e le funzioni razionali sono olomorfe dove definite (denominatore non nullo)
\item $f(z)=\overline{z}$ non \`e olomorfa\\
$$ \frac{f(z+h) - f(z)}{h}=\frac{\overline{h}}{h}$$
Se esiste la derivata, il limite dovrebbe esistere per ogni possibile direzione $h=0$ dunque dovrebbe succedere
$$ 1 = \lim_{h \to 0 \atop{Im(h) = 0}} \frac{\overline{h}}{h} = \lim_{h \to 0 \atop{Re (h)=0}}\frac{\overline{h}}{h}=-1$$
\item $f(z)=\overline{z}$ \`e differenziabile
\end{itemize}
\end{ese}
\begin{oss}Se $f$ \`e olomorfa in $z_0$ allora \`e anche continua.
$$ f(z)-f(z_0)= \left( \frac{f(z)-f(z_0)}{z-z_0}\right) (z-z_0) \to f'(z_0) \lim_{z\to z_0} (z-z_0)= 0$$
\end{oss}

\begin{thm}
Sia $U \subseteq \C$ un aperto e $f:\, U \to \C$ continua

$$ f\text{ olomorfa in } z_0 \quad \ses \quad \begin{cases} a)\, f \text{ differenziabile in } z_0 \\
																								  b)\, J_{f_{z_0}} \text{ corrisponde alla moltiplicazione per } a \in \C
\end{cases}$$
\proof $\implica$ La condizione di olomorfit\`a implica cche 
\begin{equation}
\label{eq1}
f(z_0+h)=f(z_0)+f'(z_0)h + r(h)
\end{equation}
con $\ds \frac{\abs{r(h)}}{\abs h} \to 0 $ per $h \to 0 $\\
Nella base $\{1 , i\}$ di $\C$ visto come spazio vettoriale reale, possiamo riscrivere la \ref{eq1} come 
$$ f(x_0+\alpha, y_0+\beta)=f(x_0,y_0) + f'(z_0)(\alpha+i\beta) + r(\alpha,\beta)$$ 
con $\ds \frac{\abs{r(\alpha,\beta)}}{\sqrt{\alpha^2 + \beta^2 }}\to 0 $ (intendiamo $z_0=x_0+iy_0$ e $h =\alpha+i\beta$)\\
La mappa $\C \to \C$ che manda $z$ in $f'(z_0)z$ \`e $\R$-lineare, dunque sono soddisfatte a) e b)\\
$\iimplica$ Si procede in modo analogo
\begin{itemize}
\item Si parte dalla condizione di derivabilit\`a nelle variabili $x$ e $y$
\item Si riscrive la condizione di sopra in termini della variabile $z=x+iy$
\item Si pone per b) lo jacobiano uguale alla moltiplicazione per $f'(z_0)$
\end{itemize}
\endproof
\end{thm}
\begin{lem}Sia $A:\, \C\to \C$ una funzione $\R$-lineare.\\
I seguenti fatti sono equivalenti
\begin{itemize}
\item $A$ indotta dalla moltiplicazione per un numero complesso: $A(z) = a\cdot z $ con $a\in \C$
\item $A$ \`e $\C$-lineare
\item $A(i)=1A(i)$
\item $A$ \`e la moltiplicazione per la matrice $\begin{pmatrix}
 \alpha & - \beta \\ \beta & \alpha
\end{pmatrix}$ con $\alpha, \beta \in \R$ (l'applicazione \`e rappresentata nella base $\{1 , i\}$)\\
$a$ del punto $1$ \`e $\alpha+i\beta$
\end{itemize}
\proof Esercizio
\end{lem}
\begin{thm}[di Cauchy-Riemann]\bianco
Sia $U\subseteq \C$ aperto e $f:\, U\to \C$ continua.\\
Denotiamo $f(x,y)=u(x,y)+iv(x,y)$ nella base $\{1 , i\}$ di $\C$
$$ f \text{ olomorfa in } z_0\in U \quad \ses \quad \begin{cases} a)\, f \text{ differenziabile in } z_0\\ 
b')\, \pr u x (x_0,y_0) = \pr v y (x_0, y_0)  \text{ e } \pr u y (x_0,y_0) = -\pr v x (x_0,y_0)\end{cases}$$
dove $z_0=x_0+iy_0$\\
Nel caso $f$ sia olomorfa allora $f'(z_0)=\pr u x (x_0,y_0) + i \pr v y (x_0,y_0)$
\proof Mostriamo che la condizione b) del precedente teorema implica la condizione b' (nota come condizione di Cauchy-Riemann)
Poich\`e $f$ \`e differenziabile il suo Jacobiano \`e della forma
$$J=\begin{pmatrix}
\pr u x (x_0,y_0) & \pr u y (x_0, y_0)\\
\pr v x (x_0, y_0) & \pr v y (x_0, y_0)
\end{pmatrix} $$
Per il Lemma precedente essendo lo Jacobiano $\R$-lineare e corrispondente alla moltiplicazione per un numero complesso si avr\`a 
$J=\begin{pmatrix}
 \alpha & -\beta \\ \beta & \alpha
\end{pmatrix}$ da cui
$$ \pr u x (x_0,y_0) = \pr v y (x_0, y_0) = \alpha \text{ e } \pr u y (x_0, y_0) = - \pr v x (x_0, y_0) =\beta $$
\endproof
\end{thm}
\begin{defn}$\forall z=x + i y \in \C$, chiamiamo esponenziale del numero complesso $z$ la quantit\`a 
$$e^z=e^x(\cos y + i \sin y)$$
\end{defn}
\begin{ese}$f(z)=e^z$ \`e intera con $f'(z) = e^z$\\
Osserviamo che $u(x,y)= e^x \cos x $ mentre $v(x,y) = e^x \sin y$, in modo ovvio sono soddisfatte le ipotesi del teorema di Cauchy-Riemann
\end{ese}

\begin{defn}
$$\sin z = \frac{e^{iz}-e^{iz}}{2i} \qquad \cos z= \frac{e^{iz}+e^{-iz}}{2}$$
$$\sinh z = \frac{e^{z}-e^{z}}{2} \qquad \cosh z= \frac{e^{z}+e^{-z}}{2}$$
\end{defn}













\end{document}