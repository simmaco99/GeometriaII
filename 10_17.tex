\documentclass[a4paper,12pt]{article}
\usepackage[a4paper, top=2cm,bottom=2cm,right=2cm,left=2cm]{geometry}

\usepackage{bm,xcolor,mathdots,latexsym,amsfonts,amsthm,amsmath,
					mathrsfs,graphicx,cancel,tikz-cd,hyperref,booktabs,caption,amssymb,amssymb,wasysym}
\hypersetup{colorlinks=true,linkcolor=blue}
\usepackage[italian]{babel}
\usepackage[T1]{fontenc}
\usepackage[utf8]{inputenc}
\newcommand{\s}[1]{\left\{ #1 \right\}}
\newcommand{\sbarra}{\backslash} %% \ 
\newcommand{\ds}{\displaystyle} 
\newcommand{\alla}{^}  
\newcommand{\implica}{\Rightarrow}
\newcommand{\iimplica}{\Leftarrow}
\newcommand{\ses}{\Leftrightarrow} %se e solo se
\newcommand{\tc}{\quad \text{ t. c .} \quad } % tale che 
\newcommand{\spazio}{\vspace{0.5 cm}}
\newcommand{\bbianco}{\textcolor{white}{,}}
\newcommand{\bianco}{\textcolor{white}{,} \\}% per andare a capo dopo 																					definizioni teoremi ...


% campi 
\newcommand{\N}{\mathbb{N}} 
\newcommand{\R}{\mathbb{R}}
\newcommand{\Q}{\mathbb{Q}}
\newcommand{\Z}{\mathbb{Z}}
\newcommand{\K}{\mathbb{K}} 
\newcommand{\C}{\mathbb{C}}
\newcommand{\F}{\mathbb{F}}
\newcommand{\p}{\mathbb{P}}

%GEOMETRIA
\newcommand{\B}{\mathfrak{B}} %Base B
\newcommand{\D}{\mathfrak{D}}%Base D
\newcommand{\RR}{\mathfrak{R}}%Base R 
\newcommand{\Can}{\mathfrak{C}}%Base canonica
\newcommand{\Rif}{\mathfrak{R}}%Riferimento affine
\newcommand{\AB}{M_\D ^\B }% matrice applicazione rispetto alla base B e D 
\newcommand{\vett}{\overrightarrow}
\newcommand{\sd}{\sim_{SD}}%relazione sx dx
\newcommand{\nvett}{v_1, \, \dots , \, v_n} % v1 ... vn
\newcommand{\ncomb}{a_1 v_1 + \dots + a_n v_n} %a1 v1 + ... +an vn
\newcommand{\nrif}{P_1, \cdots , P_n} 
\newcommand{\bidu}{\left( V^\star \right)^\star}

\newcommand{\udis}{\amalg}
\newcommand{\ric}{\mathfrak{U}}
\newcommand{\inclu}{\hookrightarrow }
%ALGEBRA

\newcommand{\semidir}{\rtimes}%semidiretto
\newcommand{\W}{\Omega}
\newcommand{\norma}{\vert \vert }
\newcommand{\bignormal}{\left\vert \left\vert}
\newcommand{\bignormar}{\right\vert \right\vert}
\newcommand{\normale}{\triangleleft}
\newcommand{\nnorma}{\vert \vert \, \cdot \, \vert \vert}
\newcommand{\dt}{\, \mathrm{d}t}
\newcommand{\dz}{\, \mathrm{d}z}
\newcommand{\dx}{\, \mathrm{d}x}
\newcommand{\dy}{\, \mathrm{d}y}
\newcommand{\amma}{\gamma}
\newcommand{\inv}[1]{#1^{-1}}
\newcommand{\az}{\centerdot}
\newcommand{\ammasol}[1]{\tilde{\gamma}_{\tilde{#1}}}
\newcommand{\pror}[1]{\mathbb{P}^#1 (\R)}
\newcommand{\proc}[1]{\mathbb{P}^#1(\C)}
\newcommand{\sol}[2]{\widetilde{#1}_{\widetilde{#2}}}
\newcommand{\bsol}[3]{\left(\widetilde{#1}\right)_{\widetilde{#2}_{#3}}}
\newcommand{\norm}[1]{\left\vert\left\vert #1 \right\vert \right\vert}
\newcommand{\abs}[1]{\left\vert #1 \right\vert }
\newcommand{\ris}[2]{#1_{\vert #2}}
\newcommand{\vp}{\varphi}
\newcommand{\vt}{\vartheta}
\newcommand{\wt}[1]{\widetilde{#1}}
\newcommand{\pr}[2]{\frac{\partial \, #1}{\partial\, #2}}%derivata parziale
%per creare teoremi, dimostrazioni ... 
\theoremstyle{plain}
\newtheorem{thm}{Teorema}[section] 
\newtheorem{ese}[thm]{Esempio} 
\newtheorem{ex}[thm]{Esercizio} 
\newtheorem{fatti}[thm]{Fatti}
\newtheorem{fatto}[thm]{Fatto}

\newtheorem{cor}[thm]{Corollario} 
\newtheorem{lem}[thm]{Lemma} 
\newtheorem{al}[thm]{Algoritmo}
\newtheorem{prop}[thm]{Proposizione} 
\theoremstyle{definition} 
\newtheorem{defn}{Definizione}[section] 
\newcommand{\intt}[2]{int_{#1}^{#2}}
\theoremstyle{remark} 
\newtheorem{oss}{Osservazione} 
\newcommand{\di }{\, \mathrm{d}}
\newcommand{\tonde}[1]{\left( #1 \right)}
\newcommand{\quadre}[1]{\left[ #1 \right]}
\newcommand{\w}{\omega}

% diagrammi commutativi tikzcd
% per leggere la documentazione texdoc

\begin{document}
\textbf{Lezione del 17 ottobre di Gandini}\\
Continuo dei controesempi della lezione precedente
\begin{lem}Sia $X$ uno spazio $T4$ separabile e $D\subseteq X$ chiuso e discreto.\\
Allora $D$ ha cardinalit\`a meno che continua.
\proof Sia $Q\subseteq X $ un denso numerabile.\\
Mostriamo che 
$$ \vert \mathcal{P}(D) \vert \leq \vert \mathcal{P}(Q) \vert \leq \vert \R\vert$$
Infatti $\vert D \vert < \vert \mathcal{P}(D) \vert$ quindi $ \vert D  \vert < \vert \R \vert $\\

Sia $S \subseteq D$, $D$  \`e discreto quindi  $S$ \`e  un chiuso, ora chiuso di chiuso \`e un chiuso da cui $S\subseteq X $ \`e un chiuso.\\
Ora anche $S$ \`e anche un aperto quindi $D\sbarra S $ \`e un chiuso in $X$.\\
Dal fatto che $X$ \`e $T4$ posso separare $D$ da $D \sbarra S$ ovvero 
$$ \exists U_S, V_S\subseteq X \text{ aperti disgiunti} \quad S \subseteq U_S \quad D\sbarra S \subseteq V_S$$
Consideriamo l'applicazione   
$$\mathcal{P}(D) \to \mathcal{P}(Q) \quad S \to Q \cap U_S$$
osserviamo che  $U_S \cap Q \neq \emptyset$ in quanto un denso interseca tutti gli aperti non vuoti.\\
Mostriamo che la funzione \`e iniettiva:
$$ S\neq T \quad \implica S\sbarra T \neq \emptyset $$
Allora
$$ \begin{cases} S \subseteq U_S \quad D \sbarra S \subseteq V_S\\
T \subseteq U_T \quad D \sbarra T \subseteq V_T 
\end{cases} \quad U_S \cap V_S = U_T \cap V_T$$
Ora per costruzione $S\sbarra T \subseteq U_S \cap V_T$ quindi $U_S \cap V_T$ \`e un aperto non vuoto da cui 
$$ U_S \cap V_T \cap Q \neq \emptyset \quad \implica \quad (U_S \cap Q ) \cap V_T \neq (U_T \cap Q ) \cap V_T \quad \implica \quad U_S \cap Q \neq U_T \cap Q$$
Abbiamo trovato un'applicazione iniettiva quindi vale la disuguaglianza cercata
\endproof
\end{lem}
\begin{cor}$\R_S^2$ non \`e $T4$ 
\proof Abbiamo provato che $\R_S^2$ \`e separabile.\\
Ora l'antidiagonale \`e discreta , xhiusa e ha cardinalit\`a continua
\end{cor}
\begin{oss}
\item Regolare $\not \implica$ normale. Sia $\R_S^2$ il piano di Sorgenfray.\\
Il piano di Sorgenfray \`e regolare in quanto $\R_S$ lo \`e ed il prodotto di regolari \`e regolare mentre per il corollario precedente non \`e normale
\end{oss}
\begin{oss}Abbiamo mostrato un esempio di prodotto di spazi normali che non \`e normale
\end{oss}
\begin{oss}$T2 \not \implica$ regolare.\\
Sia $$K=\left\{ \left. \frac{1}{n} \, \right\vert \, n \in \N \right\}$$ e consideriamo su $\R$ la topologia generata dagli aperti 
$$ (a,b) \quad a,b \in \R $$ 
$$ (a,b) \sbarra K \quad a,b \in \R$$
Questa topologia chiamata $\R_K$ \`e $T2$ essendo un raffinamento di quella euclidea ma non \`e regolare.\\
Infatti per costruzione $K$ \`e un chiuso, ma non posso separarlo da $0$.
$$ A \in I(0) \quad \implica \quad A \supseteq (-\varepsilon, \varepsilon) \sbarra K$$ 
Sia $B$ aperto con $K \subseteq B$ 
$$ \forall n \quad \frac{1}{n}\in K \quad \exists \varepsilon_n \quad \left(\frac{1}{n}-\varepsilon , \frac{1}{n}+\varepsilon \right) \subseteq B $$
quindi in particolare $A\cap B \neq \emptyset$
\end{oss}
\spazio
\begin{thm}
$$ \text{ regolare a base numerabile} \quad \implica \quad \text{ normale}$$
\proof Sia $\B$ una base numerabile per $X$ spazio regolare.\\
Siano $C,D\subseteq X $ chiusi disgiunti 
$$ \forall c \in C \quad \exists U \subseteq X \quad c \in U \subseteq \overline{U}\subseteq X\sbarra D $$
in particolare posso scegliere $U \in \B$.\\
Sia
$$ \mathfrak{C}=\{ U \in \B \, \vert \, \overline{U}\cap D =\emptyset \}$$
in particolare $U \in \mathfrak{C}$ esiste in quanto $$ \overline{U} \subseteq X\sbarra D \ses \overline{U}\cap D=\emptyset$$
Similmente
$$ \mathfrak{D}=\{ V \in \B \, \vert \, \overline{V}\cap C =\emptyset \}$$
Ora essendo $\B$ numerabile posso considerare $\mathfrak{C}=\{C_n\}$ e $\mathfrak{D}=\{ D_n\}$.\\
Dato $n$ siano 
$$ A_n = U_n \left\sbarra \bigcup_{j\leq n }\overline{V_j}\right.$$
$$ A = \bigcup A_n \text{ \`e un aperto, in quanto lo sono gli  } A_n  $$
Segue dalla costruzione che $C \subseteq A $ infatti $\overline{V_j}\cap C = \emptyset$\\
In modo analogo possiamo definire $B=\bigcup B_n$.\\
Osserviamo che 
$$ \forall n, m \quad A_n \cap B_n =\emptyset $$
Infatti se $m\leq n $ allora $B_n \subseteq V_n$ dunque $A_m \cap B_n =\emptyset$  similmente $A_n \cap B_m =\emptyset$.\\
Dunque abbiamo costruito $2$ aperti disgiunti che separano $C,D$
\endproof
\end{thm}
\spazio 
\begin{thm}[Teorema di Urysohn]\bianco
$$ \text{ regoalare a base numerabile } \implica \text{ metrizzabile}$$
\end{thm}
\begin{cor}
$$ \text{ regolare a base numerabile} \ses \text{ normale a base numerabile} \ses$$ $$ \ses \text{ metrizzabile a base numerabile } \ses \text{ metrizzabile separabile } $$
\end{cor}
\newpage
\section{Quozienti topologici}
\begin{defn}[Insieme saturo]\bianco
Sia $\pi:\, X \to Y$ proiezione al quoziente.\\
Sia $ Z\subseteq X $ allora $Z$ \`e detto saturo se \`e unione di classe di equivalenza.\\
In modo equivalente
$$ Z = \pi^{-1}(\pi(Z))$$
\end{defn}
\begin{oss}In generale \`e vera solo l'inclusione $\subset$
\end{oss}
\begin{defn}La topologia quoziente su $Y=\frac{X}{\sim}$ 
\`e la pi\`u fine topologia su $Y$ che rende continua la proiezione al quoziente $$\pi:\, X \to Y \quad x\to [x]$$

\end{defn}
\begin{oss}Non \`e detto che tale topologia esista.
\end{oss}


\begin{prop}$$ \{ \pi(B) \, \vert \, B \subseteq X \text{ aperto saturo }\}$$ 
\`e una topologia
\proof \bbianco 
\begin{itemize}
\item $\emptyset = \pi(\emptyset)$ ed essendo la proiezione  suriettiva $Y=\pi(X)$
\item 
$$ \bigcup (\pi(B_i))=\pi \left( \bigcup B_i \right) $$
ora l'unione dei $B_i$ \`e un aperto e poich\`e ogni $B_i$ \`e unione di classi di equivalenza anche $\bigcup B_i$ lo \`e
\item $\pi (B_1) \cap \pi(B_2)=\pi(B_1\cap B_2)$ essendo $B_1$ e $B_2$ saturi
\end{itemize}
\endproof
\end{prop}
\begin{prop}La topologia sopra descritta \`e la pi\`u fine topologia che rende la proiezione continua, tale topologia prende il nome di $\tau_{quot}$
\proof Infatti se $\tau$ \`e una qualsiasi topologia su $Y$ che rende continua $\pi$ allora
$$ \forall A \in \tau \quad B = \pi^{-1}(A) \text{ \`e un aperto sautro}$$
dunque $A= \pi(B)$ con $B$ aperto saturo di $X$
\endproof
\end{prop}
\begin{oss}
$$ A \subseteq Y \text{ aperto } \ses \pi^{-1}(A) \subseteq X\text{ aperto } \ses A = \pi(B) \text{ con } B \subseteq X \text{ aperto saturo}$$
similmente con i chiusi
$$ C \subseteq Y \text{ chiuso } \ses \pi^{-1}(C) \subseteq X\text{ chiuso }\ses C = \pi(B) \text{ con } B \subseteq X \text{ chiuso saturo}$$
\end{oss}
\newpage
\begin{prop}[Propiet\`a universale]\bianco
La topologia quoziente \`e l'unica topologia su $Y$ con la seguente propiet\`a
$$ \forall Z\text{ spazio topologico} \, \forall f:\, Y \to Z $$
$$ f\text{ continua } \quad \ses \quad f \circ \pi \text{ continua}$$
\proof Vediamo che la topologia quoziente soddisfa la propiet\`a (l'unicit\`a \`e simile a quanto visto per le altre propiet\`a universali)\\
$\implica$ La composizione di funzioni continue \`e continua
$\iimplica$ Sia $A \subseteq Z$ un aperto allora dobbiamo provare che 
$$ f \text{ continua } \quad \ses \quad f^{-1}(A)\subseteq  Y \text{ \`e aperto } \forall A\subseteq Z \text{ aperto }  $$
$$ f^{-1}(A) \subseteq Y \text{ aperto } \quad \ses \quad \pi^{-1} (f^{-1}(A)=(f\circ \pi)^{-1}(A))\subseteq X \text{ aperto } $$
\endproof
\end{prop}

\end{document}