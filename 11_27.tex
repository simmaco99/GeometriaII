\documentclass[a4paper,12pt]{article}
\usepackage[a4paper, top=2cm,bottom=2cm,right=2cm,left=2cm]{geometry}

\usepackage{bm,xcolor,mathdots,latexsym,amsfonts,amsthm,amsmath,
					mathrsfs,graphicx,cancel,tikz-cd,hyperref,booktabs,caption,amssymb,amssymb,wasysym}
\hypersetup{colorlinks=true,linkcolor=blue}
\usepackage[italian]{babel}
\usepackage[T1]{fontenc}
\usepackage[utf8]{inputenc}
\newcommand{\s}[1]{\left\{ #1 \right\}}
\newcommand{\sbarra}{\backslash} %% \ 
\newcommand{\ds}{\displaystyle} 
\newcommand{\alla}{^}  
\newcommand{\implica}{\Rightarrow}
\newcommand{\iimplica}{\Leftarrow}
\newcommand{\ses}{\Leftrightarrow} %se e solo se
\newcommand{\tc}{\quad \text{ t. c .} \quad } % tale che 
\newcommand{\spazio}{\vspace{0.5 cm}}
\newcommand{\bbianco}{\textcolor{white}{,}}
\newcommand{\bianco}{\textcolor{white}{,} \\}% per andare a capo dopo 																					definizioni teoremi ...


% campi 
\newcommand{\N}{\mathbb{N}} 
\newcommand{\R}{\mathbb{R}}
\newcommand{\Q}{\mathbb{Q}}
\newcommand{\Z}{\mathbb{Z}}
\newcommand{\K}{\mathbb{K}} 
\newcommand{\C}{\mathbb{C}}
\newcommand{\F}{\mathbb{F}}
\newcommand{\p}{\mathbb{P}}

%GEOMETRIA
\newcommand{\B}{\mathfrak{B}} %Base B
\newcommand{\D}{\mathfrak{D}}%Base D
\newcommand{\RR}{\mathfrak{R}}%Base R 
\newcommand{\Can}{\mathfrak{C}}%Base canonica
\newcommand{\Rif}{\mathfrak{R}}%Riferimento affine
\newcommand{\AB}{M_\D ^\B }% matrice applicazione rispetto alla base B e D 
\newcommand{\vett}{\overrightarrow}
\newcommand{\sd}{\sim_{SD}}%relazione sx dx
\newcommand{\nvett}{v_1, \, \dots , \, v_n} % v1 ... vn
\newcommand{\ncomb}{a_1 v_1 + \dots + a_n v_n} %a1 v1 + ... +an vn
\newcommand{\nrif}{P_1, \cdots , P_n} 
\newcommand{\bidu}{\left( V^\star \right)^\star}

\newcommand{\udis}{\amalg}
\newcommand{\ric}{\mathfrak{U}}
\newcommand{\inclu}{\hookrightarrow }
%ALGEBRA

\newcommand{\semidir}{\rtimes}%semidiretto
\newcommand{\W}{\Omega}
\newcommand{\norma}{\vert \vert }
\newcommand{\bignormal}{\left\vert \left\vert}
\newcommand{\bignormar}{\right\vert \right\vert}
\newcommand{\normale}{\triangleleft}
\newcommand{\nnorma}{\vert \vert \, \cdot \, \vert \vert}
\newcommand{\dt}{\, \mathrm{d}t}
\newcommand{\dz}{\, \mathrm{d}z}
\newcommand{\dx}{\, \mathrm{d}x}
\newcommand{\dy}{\, \mathrm{d}y}
\newcommand{\amma}{\gamma}
\newcommand{\inv}[1]{#1^{-1}}
\newcommand{\az}{\centerdot}
\newcommand{\ammasol}[1]{\tilde{\gamma}_{\tilde{#1}}}
\newcommand{\pror}[1]{\mathbb{P}^#1 (\R)}
\newcommand{\proc}[1]{\mathbb{P}^#1(\C)}
\newcommand{\sol}[2]{\widetilde{#1}_{\widetilde{#2}}}
\newcommand{\bsol}[3]{\left(\widetilde{#1}\right)_{\widetilde{#2}_{#3}}}
\newcommand{\norm}[1]{\left\vert\left\vert #1 \right\vert \right\vert}
\newcommand{\abs}[1]{\left\vert #1 \right\vert }
\newcommand{\ris}[2]{#1_{\vert #2}}
\newcommand{\vp}{\varphi}
\newcommand{\vt}{\vartheta}
\newcommand{\wt}[1]{\widetilde{#1}}
\newcommand{\pr}[2]{\frac{\partial \, #1}{\partial\, #2}}%derivata parziale
%per creare teoremi, dimostrazioni ... 
\theoremstyle{plain}
\newtheorem{thm}{Teorema}[section] 
\newtheorem{ese}[thm]{Esempio} 
\newtheorem{ex}[thm]{Esercizio} 
\newtheorem{fatti}[thm]{Fatti}
\newtheorem{fatto}[thm]{Fatto}

\newtheorem{cor}[thm]{Corollario} 
\newtheorem{lem}[thm]{Lemma} 
\newtheorem{al}[thm]{Algoritmo}
\newtheorem{prop}[thm]{Proposizione} 
\theoremstyle{definition} 
\newtheorem{defn}{Definizione}[section] 
\newcommand{\intt}[2]{int_{#1}^{#2}}
\theoremstyle{remark} 
\newtheorem{oss}{Osservazione} 
\newcommand{\di }{\, \mathrm{d}}
\newcommand{\tonde}[1]{\left( #1 \right)}
\newcommand{\quadre}[1]{\left[ #1 \right]}
\newcommand{\w}{\omega}

% diagrammi commutativi tikzcd
% per leggere la documentazione texdoc


\begin{document}
\textbf{Lezione del 27 Novembre tenuta dal Prof. Frigerio}
\begin{prop}Sia 
$$ U_i = \{ [x_0:\, \dots \, : x_n ] \in \p^n(\K) \, \vert \, x_i \neq 0 \} $$
allora esiste una biezione naturale tra $U_i$ e $\K^n$ che, nel caso $\K=\R$ \`e un omeomorfismo 
\proof Siano
$$ \varphi:\, U_i \to \K^n \quad \varphi([x_0:\, \dots \, : x_n) = \left( \frac{x_0}{x_i}, \dots , \frac{x_{i-1}}{x_i}, \frac{x_{i+1}}{x_i}, \dots , \frac{x_n }{x_i} \right)$$
e 
$$ \psi:\, \K^n \to U_i \quad \psi( x_1, \dots , x_n ) = [ x_1 : \dots :x_{i-1}: x_i : x_{i+1} : \dots : x_n]$$
Osserviamo che queste 2 funzioni sono una l'inversa dell'altra.\\
Supponiamo $\K=\R$.\\
$ \psi $ \`e composizione di 
$$ \R^n \inclu \R^{n+1}\sbarra \{ 0 \} \to \p^n(\R) \quad (x_1, \dots, x_n ) \inclu (x_1, \dots, 1 ,  \dots , x_n ) \to [ x_1 : \dots : 1: \dots : x_n ]$$
dunque \`e continua.\\
$\varphi$ si ottiene per passaggio al quoziente da $ \tilde{\varphi}:\, \pi^{-1}(U_i) \to \R^n $ ora $\tilde{\varphi}$ \`e continua dunque lo \`e anche $\varphi$.\endproof
\end{prop}
\begin{oss}Il proiettivo \`e unione degli $U_i$, inoltre 
gli $U_i$ sono aperti in $\p^n(\R)$.\\
$\forall p \in \p^n(\R)$ esiste $U$ aperto omeomorfo a $\R^n$ con $p\in U$
\end{oss}
\spazio
\begin{defn}Sia $X$ topologico. X si dice variet\`a $n$-dimensionale se 
\begin{itemize}
\item $X$ \`e di Hausdorff
\item $\forall p \in X$ $\exists U$ aperto omeomorfo ad un aperto $V$ di $\R^n$
\item $X$ \`e a base numerabile 
\end{itemize}
\begin{oss}Poich\`e le palle aperte sono una base della topologia di $\R^n$ e una palla aperta \`e omeomorfo a $\R^n$ la propiet\`a 2 \`e equivalente a richiedere che $U$ sia omoemorfa ad una palla aperta oppure a $\R^n$
\end{oss}
\end{defn}
\begin{oss}Per quanto osservato sul proiettivo $\p^n(\R)$ \`e una $n$-variet\`a (manca da dimostrare che \`e a base numerabile)
\end{oss}
\begin{oss}Le $3$ propiet\`a che definiscono una variet\`a sono indipendenti.\\
Se prendiamo $\frac{\R\times \{ -1, 1 \} }{\sim}$ dove $(x,t)\sim (y,s) \ses (x,t)=(y,s) \text{ o }  (x=y \text{ e } x \neq 0 )$.\\
Tale spazio verifica la seconda propiet\`a  ma non \`e di Hausdorff.
\end{oss}
\newpage
Studiamo ora i proiettivi complessi.\\
Con le dimostrazioni analoghe possiamo dimostrare che $ \p^n(\C)$ \`e compatto di Hausdorff e viene ricoperto dagli $U_i$, inoltre \`e a base numerabile.\\
Ora essendo $\C^n \cong \R^{2n}$ $\p^n(\C)$ \`e una variet\`a $2n$ dimensionale
\begin{prop}$\p^1( \C) \cong S^2$
\proof $$\p^1(\C) =\{ z_0 = 0\} \cup \{ z_0 \neq 0 \} = \{ [0:1]\} \cup U_0$$
Poich\`e $\p^1(\C)$ \`e compatto e di Hausdorff, per unicit\`a della compattificazione di Alexandross si ha 
$$ \p^1(\C) \cong \hat{U_0} \cong \hat{\R^2} = S^2$$
\begin{oss}In generale 
$$ \p^n(\K) = \{ x_0 = 0 \}  \cup U_0 \cong \p^{n-1}(\K) \cup U_0 \cong P^{n-1}\cup \K^n$$
\end{oss}
\end{prop}
\spazio
Consideriamo la mappa 
$$ f:\, S^{2n+1} \to \p^n(\C)$$ ottenuta restringendo $\pi$.\\
Tale mappa \`e suriettiva  perch\`e $\pi(v) = \pi\left(\frac{v}{\norma v \norma } \right)$  dunque $\p^n(\C)$ \`e compatto.\\
 $\forall p \in \p^n(\C)$ si ha $f^{-1}(P)\subseteq S^{2n+1}$.\\
 Ora se $v \in f^{-1}(p)$ si ha $f^{-1}(p)=\{ \lambda v \, \vert \, \lambda\in \C \, \, \vert \lambda\vert =1\}$ da cui $f^{-1}(p)$ \`e omeomorfo a $S^1$ tramite la mappa 
 $$ S^1=\{ \lambda \in \C \, \vert \, \vert \lambda \vert \} \to f^{-1}(p) \qquad \lambda \to \lambda v$$
 tale mappa \`e continua, biettiva e chiusa (va da un compatto ad uno spazio di Hausdorff)
 \newpage

 \begin{ese}Compatto per successione $\not \implica$ compatto
 \proof Sia $\ds X =[0,1]^{[0,1]}$ con la topologia prodotto (convergenza puntuale).\\
Definiamo $supp(f)=\{ x \in [0,1]\, \vert \, f(x)\neq 0\}$ e sia 
$$ Y = \{ f\in X \, \vert \, \vert supp(f) \vert \leq \vert \N \vert \}$$
Osserviamo che
\begin{enumerate}
\item $Y$ \`e denso.\\
Sia $U$ un aperto di $X$ allora posso prendere una funzione in $Y$ che appartiene a tale aperto 
\item $X$ \`e $T2$.\\
Prodotto di $T2$ \`e $T2$
\item $Y$ non \`e compatto .\\
Supponiamo $Y$ compatto allora $Y$ sarebbe chiuso, ma data la densit\`a di $Y$ si avrebbe  $Y=X$ ma ci\`o \`e assurdo
\item $Y$ \`e compatto per successione.\\
Sia $\{ f_n\}$ una successione in $Y$, devo trovare una sua estratta che converge puntualmente a $f\in Y$.\\
Sia $\ds A=\bigcup_{n \in \N} supp (f_n)$ dunque $\vert A \vert \leq \vert \N \vert $, dunque
$$ A = \{ a_1, \dots, a_n , \dots \}$$
La successione $\{f_n (a_0) \}\subseteq [0,1]$ 
ammette una scelta crescente di indici $k_0(n)$ tale che $f_{k_0(n)}(a_0) \to l_0 $ ($[0,1]$ \`e compatto).\\
Analogamente considerando la successione $\ds \left\{ f_{k_0(n)}(a_1) \right\} \subseteq [0,1]$ dunque esiste una sottosuccessione $k_1(n)$ estratta da $k_0(n)$  con 
$f_{k_1(n)}(a_1)\to l_1$.\\
Iterando la costruzione $\forall m \in \N$ costruisco una sottosuccessione  $k_{m+1}(n)$ estratta da $k_m(n)$ e tale che 
$$\ds \lim_{n \to + \infty} f_{k_{m+1}(n)}\left( a_{m+1} \right) \to l_{m+1}$$
Segue che $\forall m \in \N$ $\ds \lim_{i \to + \infty} f_{k_i(i)}(a_m)=l_m$.\\
Sia $f(a_i)=l_i$ dunque $f \in Y$ infatti il supporto di $f$ \`e $A$ che \`e numerabile inoltre $f_{k_i} \to f $
\end{enumerate}
 \end{ese}
\end{document}