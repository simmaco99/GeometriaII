\documentclass[a4paper,12pt]{article}
\usepackage[a4paper, top=2cm,bottom=2cm,right=2cm,left=2cm]{geometry}

\usepackage{bm,xcolor,mathdots,latexsym,amsfonts,amsthm,amsmath,
					mathrsfs,graphicx,cancel,tikz-cd,hyperref,booktabs,caption,amssymb,amssymb,wasysym}
\hypersetup{colorlinks=true,linkcolor=blue}
\usepackage[italian]{babel}
\usepackage[T1]{fontenc}
\usepackage[utf8]{inputenc}
\newcommand{\s}[1]{\left\{ #1 \right\}}
\newcommand{\sbarra}{\backslash} %% \ 
\newcommand{\ds}{\displaystyle} 
\newcommand{\alla}{^}  
\newcommand{\implica}{\Rightarrow}
\newcommand{\iimplica}{\Leftarrow}
\newcommand{\ses}{\Leftrightarrow} %se e solo se
\newcommand{\tc}{\quad \text{ t. c .} \quad } % tale che 
\newcommand{\spazio}{\vspace{0.5 cm}}
\newcommand{\bbianco}{\textcolor{white}{,}}
\newcommand{\bianco}{\textcolor{white}{,} \\}% per andare a capo dopo 																					definizioni teoremi ...


% campi 
\newcommand{\N}{\mathbb{N}} 
\newcommand{\R}{\mathbb{R}}
\newcommand{\Q}{\mathbb{Q}}
\newcommand{\Z}{\mathbb{Z}}
\newcommand{\K}{\mathbb{K}} 
\newcommand{\C}{\mathbb{C}}
\newcommand{\F}{\mathbb{F}}
\newcommand{\p}{\mathbb{P}}

%GEOMETRIA
\newcommand{\B}{\mathfrak{B}} %Base B
\newcommand{\D}{\mathfrak{D}}%Base D
\newcommand{\RR}{\mathfrak{R}}%Base R 
\newcommand{\Can}{\mathfrak{C}}%Base canonica
\newcommand{\Rif}{\mathfrak{R}}%Riferimento affine
\newcommand{\AB}{M_\D ^\B }% matrice applicazione rispetto alla base B e D 
\newcommand{\vett}{\overrightarrow}
\newcommand{\sd}{\sim_{SD}}%relazione sx dx
\newcommand{\nvett}{v_1, \, \dots , \, v_n} % v1 ... vn
\newcommand{\ncomb}{a_1 v_1 + \dots + a_n v_n} %a1 v1 + ... +an vn
\newcommand{\nrif}{P_1, \cdots , P_n} 
\newcommand{\bidu}{\left( V^\star \right)^\star}

\newcommand{\udis}{\amalg}
\newcommand{\ric}{\mathfrak{U}}
\newcommand{\inclu}{\hookrightarrow }
%ALGEBRA

\newcommand{\semidir}{\rtimes}%semidiretto
\newcommand{\W}{\Omega}
\newcommand{\norma}{\vert \vert }
\newcommand{\bignormal}{\left\vert \left\vert}
\newcommand{\bignormar}{\right\vert \right\vert}
\newcommand{\normale}{\triangleleft}
\newcommand{\nnorma}{\vert \vert \, \cdot \, \vert \vert}
\newcommand{\dt}{\, \mathrm{d}t}
\newcommand{\dz}{\, \mathrm{d}z}
\newcommand{\dx}{\, \mathrm{d}x}
\newcommand{\dy}{\, \mathrm{d}y}
\newcommand{\amma}{\gamma}
\newcommand{\inv}[1]{#1^{-1}}
\newcommand{\az}{\centerdot}
\newcommand{\ammasol}[1]{\tilde{\gamma}_{\tilde{#1}}}
\newcommand{\pror}[1]{\mathbb{P}^#1 (\R)}
\newcommand{\proc}[1]{\mathbb{P}^#1(\C)}
\newcommand{\sol}[2]{\widetilde{#1}_{\widetilde{#2}}}
\newcommand{\bsol}[3]{\left(\widetilde{#1}\right)_{\widetilde{#2}_{#3}}}
\newcommand{\norm}[1]{\left\vert\left\vert #1 \right\vert \right\vert}
\newcommand{\abs}[1]{\left\vert #1 \right\vert }
\newcommand{\ris}[2]{#1_{\vert #2}}
\newcommand{\vp}{\varphi}
\newcommand{\vt}{\vartheta}
\newcommand{\wt}[1]{\widetilde{#1}}
\newcommand{\pr}[2]{\frac{\partial \, #1}{\partial\, #2}}%derivata parziale
%per creare teoremi, dimostrazioni ... 
\theoremstyle{plain}
\newtheorem{thm}{Teorema}[section] 
\newtheorem{ese}[thm]{Esempio} 
\newtheorem{ex}[thm]{Esercizio} 
\newtheorem{fatti}[thm]{Fatti}
\newtheorem{fatto}[thm]{Fatto}

\newtheorem{cor}[thm]{Corollario} 
\newtheorem{lem}[thm]{Lemma} 
\newtheorem{al}[thm]{Algoritmo}
\newtheorem{prop}[thm]{Proposizione} 
\theoremstyle{definition} 
\newtheorem{defn}{Definizione}[section] 
\newcommand{\intt}[2]{int_{#1}^{#2}}
\theoremstyle{remark} 
\newtheorem{oss}{Osservazione} 
\newcommand{\di }{\, \mathrm{d}}
\newcommand{\tonde}[1]{\left( #1 \right)}
\newcommand{\quadre}[1]{\left[ #1 \right]}
\newcommand{\w}{\omega}

% diagrammi commutativi tikzcd
% per leggere la documentazione texdoc

\begin{document}
\textbf{Lezione del 4 e 9  Marzo}
\begin{fatto} Sia $R\subseteq F(S)$ e sia $\psi:\, F(S) \to H $ omomorfismo di gruppi, se $\psi(r)=e_H \, \, \forall r \in R$ allora la  mappa $\overline{\psi}:\,\langle R \, \vert \, S \rangle \to H$   tale che $\overline{\psi}([s])=\psi(s)$ \`e ben definita.
\end{fatto}

\begin{prop}Siano 
$$G_0=\langle S_0 \, \vert R_ \rangle \quad G_1=\langle S_1 \, \vert R_1 \rangle \quad G_2=\langle S_2 \, \vert R_2 \rangle$$
$$\psi_1:\, G_0\to G_1  \quad \psi_2:\, G_0 \to G_2 \text{ omomorfismi di gruppo } $$
$$N = N( \{ \phi_1(g) \phi_2(g)^{-1} \, \vert\, g \in G_0 \})$$
Allora $$G=\frac{G_1\star G_2}{N} \cong \langle S_1 \cup S_2 \, \vert \, R_1 \cup R_2 \cup R\rangle$$
dove  $R=\{ \tilde{\phi}_1 (s) \tilde{\phi}_2(s)^{-1} \, \vert \, s\in S_0\} $ con  $\tilde{\phi}_i : F(S_0) \to F(S_i)$ omomorfismo con  $ [ \tilde{\phi}_i(s)]
 =\phi_i ([s])$ per $i=1,2$
\end{prop}

\begin{thm}[di Van Kampen]\bianco
Sia $X$ uno spazio topologico e supponiamo che $X=A\cup B$ con $A,B$ aperti connessi per archi e con $\emptyset\neq A\cap B$ connesso per archi.\\
Siano 
$$\alpha:\, A \cap B \inclu A \quad \beta:\, A \cap B \inclu B  $$
$$ f: \, A \inclu X \quad g:\, B \inclu X$$
le inclusioni.\\
Sia $G$ un gruppo con  $h:\, \pi_1(A) \to G$ e $k:\, \pi_1(B) \to G$ omomorfismi tali che $ h \circ \alpha_\star = k \circ \beta_\star$ allora esiste unico $\pi:\, \pi_1(X) \to G $ tale che $\phi \circ f_\star=h$ e $ \phi \circ g_\star = k $\\
Tutti i gruppi fondamentali si intendono puntati in un medesimo $x_0\in A \cap B$ 
\proof 
Sia $x\in X$ allora $\alpha_x\in \Omega(x_0,x)$ \`e un cammino tale che 
\begin{itemize}
\item se $x=x_0$ allora $\alpha_x=C_{x_0}$
\item se $x\in A $ allora $\alpha_x( [0,1])\subset A $
\item se $x\in B$ allora $\alpha_x([0,1])\subseteq B$
\end{itemize}
come conseguenza se $x\in A\cap B$ allora $\alpha_x([0,1])\subseteq A\cap B$ .\\
Le ipotesi del teorema ci danno un diagramma del tipo 
$$ \begin{tikzcd}
															& \pi_1(A) \arrow{rd}{f_\star}  \arrow[bend left]{rrd}{h} \\
\pi(A\cup B) \arrow{ru}{\alpha_\star}	 \arrow{rd}{\beta_\star}											 & & \pi_1(X) & G \\
															&\pi_1(B) \arrow{ru}{g_\star} \arrow[bend right]{rru}{k}											
\end{tikzcd}$$
Dato $[\gamma]\in \pi_1(X)$ con $\gamma$ suo rappresentante, andiamo a definire chi dovrebbe essere $\pi([\gamma])$ affinch\`a la mappa $\phi:\, \pi_1(X) \to G$ renda i 2 triangoli commutativi.\\ \\ 
Poich\`e il ricoprimento $\{\gamma^{-1}(A), \gamma^{-1}(B)\}$ ammette un numero di Lesbegue, esiste una suddivisione $0=t_0< t_1< \dots < t_n=1$ tali che $\gamma([t_{i-1},t_i])$ \`e contenuto in $A$ oppure in $B$.\\
Poniamo, per convenzione, $\alpha_i= \alpha_{\gamma(t_i)}$ e andiamo a costruire una serie di cammini 
$$ \gamma_1= \gamma_{\vert [ 0,t_1]} $$
$$ \gamma_i= \alpha_{i-1} \star \gamma_{\vert [ t_{i-1}, t_i]} \star \overline{\alpha_i} \text{ per } i = 2, \dots , n-1 $$
$$ \gamma_n = \alpha{n-1} \star \gamma_{\vert [ t_{n-1}, t_n]}$$
Osserviamo che $\gamma \sim \gamma_1 \star \dots \star \gamma_n $ in quanto ogni qual volta giungo con un $\alpha_i$ giungo anche per $\overline{\alpha}_i$.
Pongo ora $\phi([\gamma])=g_1\dots g_n$ dove 
$$ g_i =\begin{cases} h([\gamma_i]) \text{ se } Imm \gamma_i \subseteq A \\
									k([\gamma_i]) \text{ se } Imm \gamma_i \subseteq B \\
\end{cases}$$
Abbiamo dunque provato l'unicit\`a di $\phi$ e per costruzione $\phi$ rende i triangoli commutativi
\begin{itemize}
\item[(a)]Proviamo che $\phi$ non dipende dalla suddivisione scelta.\\
Basta provare che se aggiungiamo un punto alla suddivisione allora il valore della funzione non cambia (se abbiamo 2 suddivisioni, scegliamo un loro raffinamento).\\
Sia $0=t_0 < t_1< \dots < t_i < t_{i+1} < \dots <t_n=1$ come nella definizione di $\phi$.\\
Sia $\overline{t}$ tale che $t_i< \overline{t} < t_{i+1}$\\
Come nella definizione di $\phi$ fatta precedentemente siano  $\beta= \alpha_{\gamma(\overline{t}}$ e 
$$\delta_1 =\alpha_i \star \gamma_{\vert [ t_i, \overline{t}]} \star \overline{\beta} \quad \delta_2 = \beta \star  \gamma_{\vert [\overline{t}, t_{i+1}} \star \overline{\alpha_{i+1}}$$
Sia 
$$ l_j= \begin{cases}k(\delta_j) \text{ se } Im \gamma_j \subseteq A \\
								h(\delta_j) \text{ se } Im \gamma_j \subseteq B 
\end{cases}$$
Osserviamo che $\gamma_{i+1}\sim \delta_1 \star \delta_2$ dunque
$$ \phi([\gamma])= g_1 \dots g_i g_{i+1}g_{i+2}\dots g_n= g_1\dots g_i ( l_1 l_2) g_{i+2} \dots g_n$$
dunque abbiamo provato che $\phi$ non dipende dalla suddivisione scelta
\item[(b)] Proviamo che $\phi$ non dipende dal rappresentante scelto\\
Sia $\beta \sim \gamma $.\\
Sia $H:[0,1]^2 \to X$ l'omotopia allora $\{ H^{-1}(A), H^{-1}(B)\}$ \`e un ricoprimento aperto di $[0,1]\times [0,1]$ da cui esiset $n \in \N$ con $H \left( \left[ \frac{k}{n}, \frac{k+1}{n} \right]\right) $ contenuto in $A$ oppure in $B$.\\
(da inserire disegnini)
\item[(c)]Il fatto che $\phi$ \`e omo \`e lasciato come esercizio

\end{itemize}
\end{thm}
\begin{cor} Sia $X,A,B$ come nell'epotesi del Teorema di Van Kamper allora 
$$\pi_1(X)= \frac{\pi_1(A)\star \pi_1(B)}{N} \text{ dove } N = N(\{i(\alpha_\star(g))j(\beta_\star(g) \, \vert \, g \in \pi_1(A\cap B)\}$$
con $i,j$ le ovvie inclusioni
\proof

$$ \begin{tikzcd} \pi_A \arrow[bend left] {rrd}{f_\star} \arrow[rd,"i"] \\   &\pi_1(A)\star \pi_1(B) & \pi_1(X) \\ 
\pi_1(B)\arrow[bend right]{rru}{g_\star} \arrow[ru,"j"]  \end{tikzcd}$$ 
Allora dalla propiet\`a universale del prodotto libero segue che esiste unica $$\psi:\, \pi_1(A) \star \pi_1(B) \to \pi_1(X)$$
che rende commutativi i 2 triangoli.\\
\`E di facile verifica che $\psi(n)=e \, \, \forall n \in N$ da cui $\psi$ passa al quoziente .\\
Usando il teorema di Van Kampen con $h=i,\, k=j$ e $G=\frac{\pi_1(A) \star \pi_1(B)}{N}$ si ottiene che esiste $\phi$.\\
Dall'unicit\`a di $\phi$ e $\psi$ segue che una \`e l'inversa dell'altra
\end{cor}
\spazio 
\begin{cor}Assumiamo le stesse ipotesi dei teorema di Van Kampen su $X$, se $A$ e $B$ sono semplicemente connessi allora anche $X$ lo \`e.
\end{cor}
\begin{ese}$S_n$ \`e semplicemente connesso per $n\geq 2$
\proof Siano $A=S^n \sbarra \{ (1, 0, \dots, 0)\}$ e $B=S^n \sbarra \{ (-1, 0, \dots, 0)\}$ ora $A,B$ sono omeomorfi a $\R^n$ che \`e semplicemente connsesso
\end{ese}
\begin{oss}$\pi_1(\mathbb{P}^n(\R)) \cong \Z_2\, \, \forall n \geq 2 $.\\
$S^n$ \`e rivestimento universale di $\mathbb{P}^n(\R))$ avente grado $2$ da cui il gruppo fondamentale \`e un gruppo con 2 elementi 
\end{oss}
\begin{ese}$\pi_1(\mathbb{P}^n(\C))= \{ e\} \, \, \forall n $
\proof Se $n=0$ allora $\mathbb{P}(\C)$ \`e un punto dunque non ho nulla da dimostrare\\
Sia $H=\{ [ x_0:\dots : x_n] \in \mathbb{P}^n(\C)\, \vert \, x_0=0\}$.\\
Siano $H=\mathbb{P}^n(\C)\sbarra A $ e $B=\mathbb{P}^n(\C) \sbarra \{ [1:0\dots : 0]\}$ ora 
$$ A \to \C^n \quad [x_0:\,\dots \, : x_n] \to \left( \frac{x_1}{x_0}: \, \dots \, : \frac{x_n}{x_0}\right) $$ 
\`e un omeomorfismo, dunque $\pi_1(A)=\{ e \} $.\\
Similmente $A\cap B$ \`e omeomorfo a $\C^n$ meno un punto dunque \`e connesso per archi.\\
Mostriamo che $B$ \`e omotopicamente equivalente a $\mathbb{P}^{n-1}(\C)$\\\
$H \cong \mathbb{P}^{n-1}(\C)$ \`e B si ritrae per deformazione su $H$.\\
Sia $r\in \C^{n+1} $ la retta generata da $(1,0,\dots, 0)$ e sia $$h:\, (C^{n+1} \sbarra r )\times [0,1]\to C^{n+1}\sbarra r \quad ((x_0, \dots, x_n), t) \to (tx_0, x_1,\dots, x_n)$$

Tale mappa passa al quoziente ottenendo $K:\, B \times [0,1] \to B$ che induce la retrazione cercata.\\
Usando l'ipotesi induttiva  $\pi_1(B)=\{ e \} $ e per un corollario del teorema di Van Kamper si giunge alla tesi
\end{ese}
\newpage
\section{Wedge di cerchi}
\begin{defn} Sia $X$ topologico. Diciamo che $X$ \`e un wedge di cerchi se esiste una famiglia $S_\alpha$ con $\alpha\in I$ con 
\begin{itemize}
\item $X=\bigcup S_\alpha$ , inoltre $\exists p $ con $S_\alpha\cap S_\beta = \{ p \} $ se $\alpha\neq \beta$
\item $S_\alpha\cong S^1$
\item $U \subseteq	 X $ aperto $\ses \, U \cap S_\alpha$ \`e aperto in $S_\alpha$ per ogni $\alpha\in I $
\end{itemize}
Denotiamo $X=\ds \wedge_{\alpha\in I } S_\alpha$
\end{defn}
\begin{prop}Sia $X$ un wedge di cerchi allora $\pi_1(X,p)$ \`e un gruppo libero inoltre se $[\gamma_\alpha]$ \`e un generatore di $\pi_1(S_\alpha, p)$ allora $\{ [\gamma_\alpha]\, \vert \, \alpha\in I \} $ \`e un insieme di generatori liberi 
\end{prop}
\begin{cor}
Sia $X=\ds \wedge_{i=1}^n S_i $ allora $\pi_1(X,p)=F_n$ dove $F_n$ \`e il gruppo libero generato da $n$ elementi
\end{cor} 
\begin{ese} Sia $X= C_1 \cup C_2$ con 
$$C_1 = \{ (x,y)\in \R^2\, \vert \, (x-1)^2+ y^2=1\}\text{ e } C_2=\{ (x,y) \in \R^2 \, \vert \, (x+1)^2 +y^2=1\}$$
allora $\pi_1(X)= \Z \star \Z$
\proof Sia $p=(-2,0)$ e $q=(2,0)$ allora \`e di facile verifica che $A=X\sbarra\{ p\}$ e $B=X \sbarra\{ q\}$ si retraggono per deformazione rispettivamente  su $C_1$ e $C_2$\\
Ora $A\cap B$ \`e omotopicamente equivalente a $S^1 \sbarra \{ pt\}\cong\R$ dunque \`e semplicemente connesso.
$$ \pi_1(X)= \frac{\pi_1(A)\star \pi_1(B)}{N}$$ 
ora $N$ dipende dall'intersezione ma essendo banale si ha $\pi_1(X)=\pi_1(C_1)\star \pi_1(C_2)\cong \Z\star \Z$
\end{ese}
\begin{ex}
Provare che $\pi_1(\R^2 \sbarra \{x_1, \dots , x_n\})
=F_n$ dove $x_i\in \R^2 $ e $x_i \neq x_j$ 
se $i=j$ 
[Suggerimento: Mostrare che $\R^2 \sbarra \{x_1, \dots , x_n\}$ 	\`e omotopicamente equivalente al wedge di $n$ cerchi
\end{ex}
\begin{ex}Siano $r_1, \dots, r_n \subseteq \R^3$ rette distinte passanti per l'origine.\\
$R^3\sbarra \ds \bigcup r_i$ si ritrae per deformazione su $S^2 \sbarra (S^n \cap \bigcup r_i)$ dunque $\pi_1 ( \R^3\sbarra \bigcup r_i) = F_{2n-1}$
\end{ex}
\end{document}