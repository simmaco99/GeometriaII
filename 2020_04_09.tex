\documentclass[a4paper,12pt]{article}
\usepackage[a4paper, top=2cm,bottom=2cm,right=2cm,left=2cm]{geometry}

\usepackage{bm,xcolor,mathdots,latexsym,amsfonts,amsthm,amsmath,
					mathrsfs,graphicx,cancel,tikz-cd,hyperref,booktabs,caption,amssymb,amssymb,wasysym}
\hypersetup{colorlinks=true,linkcolor=blue}
\usepackage[italian]{babel}
\usepackage[T1]{fontenc}
\usepackage[utf8]{inputenc}
\newcommand{\s}[1]{\left\{ #1 \right\}}
\newcommand{\sbarra}{\backslash} %% \ 
\newcommand{\ds}{\displaystyle} 
\newcommand{\alla}{^}  
\newcommand{\implica}{\Rightarrow}
\newcommand{\iimplica}{\Leftarrow}
\newcommand{\ses}{\Leftrightarrow} %se e solo se
\newcommand{\tc}{\quad \text{ t. c .} \quad } % tale che 
\newcommand{\spazio}{\vspace{0.5 cm}}
\newcommand{\bbianco}{\textcolor{white}{,}}
\newcommand{\bianco}{\textcolor{white}{,} \\}% per andare a capo dopo 																					definizioni teoremi ...


% campi 
\newcommand{\N}{\mathbb{N}} 
\newcommand{\R}{\mathbb{R}}
\newcommand{\Q}{\mathbb{Q}}
\newcommand{\Z}{\mathbb{Z}}
\newcommand{\K}{\mathbb{K}} 
\newcommand{\C}{\mathbb{C}}
\newcommand{\F}{\mathbb{F}}
\newcommand{\p}{\mathbb{P}}

%GEOMETRIA
\newcommand{\B}{\mathfrak{B}} %Base B
\newcommand{\D}{\mathfrak{D}}%Base D
\newcommand{\RR}{\mathfrak{R}}%Base R 
\newcommand{\Can}{\mathfrak{C}}%Base canonica
\newcommand{\Rif}{\mathfrak{R}}%Riferimento affine
\newcommand{\AB}{M_\D ^\B }% matrice applicazione rispetto alla base B e D 
\newcommand{\vett}{\overrightarrow}
\newcommand{\sd}{\sim_{SD}}%relazione sx dx
\newcommand{\nvett}{v_1, \, \dots , \, v_n} % v1 ... vn
\newcommand{\ncomb}{a_1 v_1 + \dots + a_n v_n} %a1 v1 + ... +an vn
\newcommand{\nrif}{P_1, \cdots , P_n} 
\newcommand{\bidu}{\left( V^\star \right)^\star}

\newcommand{\udis}{\amalg}
\newcommand{\ric}{\mathfrak{U}}
\newcommand{\inclu}{\hookrightarrow }
%ALGEBRA

\newcommand{\semidir}{\rtimes}%semidiretto
\newcommand{\W}{\Omega}
\newcommand{\norma}{\vert \vert }
\newcommand{\bignormal}{\left\vert \left\vert}
\newcommand{\bignormar}{\right\vert \right\vert}
\newcommand{\normale}{\triangleleft}
\newcommand{\nnorma}{\vert \vert \, \cdot \, \vert \vert}
\newcommand{\dt}{\, \mathrm{d}t}
\newcommand{\dz}{\, \mathrm{d}z}
\newcommand{\dx}{\, \mathrm{d}x}
\newcommand{\dy}{\, \mathrm{d}y}
\newcommand{\amma}{\gamma}
\newcommand{\inv}[1]{#1^{-1}}
\newcommand{\az}{\centerdot}
\newcommand{\ammasol}[1]{\tilde{\gamma}_{\tilde{#1}}}
\newcommand{\pror}[1]{\mathbb{P}^#1 (\R)}
\newcommand{\proc}[1]{\mathbb{P}^#1(\C)}
\newcommand{\sol}[2]{\widetilde{#1}_{\widetilde{#2}}}
\newcommand{\bsol}[3]{\left(\widetilde{#1}\right)_{\widetilde{#2}_{#3}}}
\newcommand{\norm}[1]{\left\vert\left\vert #1 \right\vert \right\vert}
\newcommand{\abs}[1]{\left\vert #1 \right\vert }
\newcommand{\ris}[2]{#1_{\vert #2}}
\newcommand{\vp}{\varphi}
\newcommand{\vt}{\vartheta}
\newcommand{\wt}[1]{\widetilde{#1}}
\newcommand{\pr}[2]{\frac{\partial \, #1}{\partial\, #2}}%derivata parziale
%per creare teoremi, dimostrazioni ... 
\theoremstyle{plain}
\newtheorem{thm}{Teorema}[section] 
\newtheorem{ese}[thm]{Esempio} 
\newtheorem{ex}[thm]{Esercizio} 
\newtheorem{fatti}[thm]{Fatti}
\newtheorem{fatto}[thm]{Fatto}

\newtheorem{cor}[thm]{Corollario} 
\newtheorem{lem}[thm]{Lemma} 
\newtheorem{al}[thm]{Algoritmo}
\newtheorem{prop}[thm]{Proposizione} 
\theoremstyle{definition} 
\newtheorem{defn}{Definizione}[section] 
\newcommand{\intt}[2]{int_{#1}^{#2}}
\theoremstyle{remark} 
\newtheorem{oss}{Osservazione} 
\newcommand{\di }{\, \mathrm{d}}
\newcommand{\tonde}[1]{\left( #1 \right)}
\newcommand{\quadre}[1]{\left[ #1 \right]}
\newcommand{\w}{\omega}

% diagrammi commutativi tikzcd
% per leggere la documentazione texdoc

\begin{document}
\textbf{Lezione del 9 aprile}
\begin{thm}[di Liouville]\spazio
Una funzione intera limitata \`e costante
\proof Essendo $f$ intera si ha $f(z) = \sum a_n z^n$ e tale serie ha raggio di convergenza infinito.\\
Dal disuguaglianza di Cauchy si ha $$\forall r, \forall n \geq 0 \quad  \abs{a_n} r^n \leq M(r)$$
Inoltre essendo $f$ limitata, esiste $M>0$ con $\abs{f(z)}\leq M$ dunque 
$$ \abs{a_n} r^n \leq M \quad \implica \quad \abs{a_n} \leq \frac{M}{r^n} \quad \forall r\geq 0,\, \forall n\geq 0$$ 
dunque per $r\to + \infty$ si ha $\abs{a_n}\to 0$ per $n \geq 1 $ da cui $f(z) =a_0$
\end{thm}
\begin{thm}[Teorema fondamentale dell'algebra]\bianco
Sia $P(z)\in \C[z]$ un polinomio non costante.\\
Allora ammette almeno uno zero.
\proof Supponiamo per assurdo $P(z)$ non si annulli da cui $\frac{1}{P(z)}$ \`e intera.\\
Assumiamo $$P(z) = a_n z^n + \dots + a_0 \text{ con }  a_n \neq 0$$ quindi
$$ P(z) =z^n \tonde{a_n + \frac{a_{n-1}}{z}+\dots +\frac{a_0}{z^n}}$$
Per $\abs z\to +\infty$ si ha $P(z) \to +\infty$ da cui $\abs{\frac{1}{P(z)}} \to 0 $\\
Dunque $\exists R>0$ con $\abs{\frac{1}{P(z)}}$ limitata per $\abs z >R$\\
D'altra parte anche $\abs{\frac{1}{P(z)}}$ definita su $\{ z\in \C\, \vert \, \abs{z} \leq R\}$ \`e limitata (funzione continua su un compatto).\\
Ora per il teorema di Louville otteniamo $\frac{1}{P(z)}$ \`e costante dunque anche $P(z)$ lo \`e, in contraddizione con l'ipotesi
\endproof
\end{thm}
\newpage
\section{Propriet\`a del valor medio}
\begin{defn}Sia $D\subseteq \C$ e $f:\, D \to \C$ continua.\\
Diciamo che $f$ ha la propriet\`a del valor medio (PVM) se 
$$\forall a\in D \quad \exists r_0 >0$$ 
tale che 
\begin{enumerate}
\item $$\{z\, \vert \, \abs{z-a} < r_0\} \subseteq D$$
\item $$f(a) = \frac{1}{2\pi} \int_0^{2\pi} f\tonde{ a+re^{i\vartheta}} \di \vartheta\text{ per ogni} 0\leq r<r_0$$
\end{enumerate}
\end{defn}
\begin{oss}Se $f$ ha la propriet\`a del valor medio, anche $Re(f)$ e $Im(f)$ lo hanno 
\end{oss}
\begin{oss}Se $f$ \`e olomorfa, $f$ ha la propriet\`a del valor medio.\\
Sia $f$ olomorfa in $D$ e $a\in D$, dunque  dalla formula integrale di Cauchy abbiamo 
$$f(a) I(\gamma,a) = \int_\gamma \frac{f(z)}{z-a} \dz$$
dove $\gamma:\, \vartheta \to a +r_0^{2i \vartheta }$ per $\vartheta \in [0,2\pi]$ e  $r_0$ \`e tale che $\{ z\, \vert \, \abs{z-a} \leq r_0\}\subseteq D$. \\
Tale curva ha indice di avvolgimento $1$ da cui 
$$ f(a) = \frac{1}{2\pi i } \int_\gamma \frac{f(z)}{z-a} =\frac{1}{2\pi i }\int_0^{2\pi} \frac{f\tonde{a+r_0e^{i\vartheta}}}{a+r_0e^{i\vartheta} -a } a+r_0i e^{i\vartheta} \di \vartheta$$
Concludiamo notando che la stessa formula vale per ogni $0\leq r< r_0$
\end{oss}
\begin{thm}[Principio del massimo modulo]\bianco
Sia $D\subseteq \C$ un aperto e sia $f:\, D\to \C$ una funzione continua che ha la propriet\`a del massimo modulo.\\
Se $\abs f $ ha un massimo relativo in un punto $a\in D$ allora $f$ \`e costante in un intorno di $a$ 
\proof Se $f(a) =0$ allora $\abs{f(z)} \leq \abs{f(a)}=0$ per $z$ sufficientemente vicino ad $a$ e dunque $f(z)=0$ in un intorno di $a$.\\
Consideriamo adesso il caso $f(a)\neq 0$, possiamo inoltre supporre $f(a) \in \R$ e $f(a) >0$ (se $f(a)=\alpha e^{i\beta}$ allora lo rimpiazzo con $e^{-i\beta} f(a)$.\\
Siano $u=Re(f)$ e $v=Im(f)$ e sia $r_0>0$ tale che 
\begin{enumerate}
\item $r_0$ \`e come nella definizione della propriet\`a del massimo modulo 
\item $\abs{f(z)}\leq \abs{f(a)}$ per $z\in B(a,r_0)$
\end{enumerate}
Dunque se pongo 
$$ M(r) =\sum\{ \abs{f(z)} \, : \, \abs{z-a} =r\} \text{ per } 0\leq r <r_0$$
si ha che $M(r) \leq \abs{f(a)}$  per ogni $0\leq r< r_0$ infatti ci\`o segue dalla propriet\`a $2$ sopra esposta.\\
Inoltre poich\`e $f$ soddisfa la propriet\`a del massimo modulo otteniamo 
$$ f(a) = \abs{f(a)} \leq \frac{1}{2\pi i }\int_0^{2\pi} \abs{f\tonde{a+re^{i\vartheta} }} \di \vartheta \leq\frac{1}{2\pi}\int_0^{2\pi} M(r) \di \vartheta =M(r) \quad \text{ per } 0\leq r <r_0$$
dunque otteniamo $f(a) = M(r)$ dunque poich\`e $f(a) \in \R$ otteniamo 
$$ \int_0^{2\pi} \left[ M(r) - u\tonde{ a+re^{i\vartheta}} \right]\di \vartheta =0$$
Sia 
$$ g(\vt) = M(r) - u\tonde{a+re^{i\vt}}$$
ora $g(\vt)\geq 0$ in quanto $M(r) \leq \abs{f(s)}$.\\
Abbiamo dunque definito una funzione $g$ non negativa su $[0,2\pi]$ e tale che il suo integrale su $[0,2\pi]$ sia nullo dunque $g$ \`e costante su tale intervallo ovvero 
\begin{equation}
\label{pmaxmod}
M(r) = u\tonde{a+re^{i\vt}}
\end{equation}
Usando la definizione di $M(r)$ otteniamo 
$$M(r) \geq \abs{f\tonde{a+re^{i\vt}}} = \tonde{ u\tonde{a+r e^{i\vt}}^2 + v\tonde{a+re^{i\vt}}^2}^{\frac{1}{2}}$$
dunque dall'uguaglianza~\ref{pmaxmod} otteniamo 
$$ M(r) \leq \tonde{M(r)^2 + v\tonde{a+re^{i\vt}}^2}^{\frac{1}{2}} \quad \implica \quad v\tonde{a+re^{i\vt}} =0 $$
Concludendo otteniamo che $\forall z$ tale che $\abs{z-a}<r_0$ vale 
$$ f(z) = u(z) = M(\abs z) = f(a) $$ 
\end{thm}
\begin{cor}Sia $D$ un aperto connesso e limitato.\\
Sia $f$ una funzione continua su $\overline{D}$ che la propriet\`a del valor medio su $D$.\\
Sia $M=\sup \{ \abs{f(z)} \, :\, z\in \partial D\}$ allora
\begin{itemize}
\item $\abs{f(z)} \leq M$  per ogni $z\in D$
\item Se $\exists a\in D$ tale che $\abs{f(a)}=M$ allora $f$ \`e costante su $D$ 
\end{itemize}
\proof Sia $$M'=\sup\{ \abs{f(z)} \, : \, z\in \overline{D}\}$$
allora chiaramente si ha $M'\geq M$ inoltre $M'$ \`e finito infatti $\abs{f}$ \`e continua su un compatto, da cui $\exists a\in \overline{D}$ con $\abs{f(a)}=M'$.\\
Andiamo a distinguere $2$ casi
\begin{itemize}
\item Se $a\in D$ allora per il principio del massimo modulo si ha che $f$ \`e costante su un intorno di $a$. Sia 
$$ D'=\{ z\in D \, : \, \abs{f(z)}=\abs{f(a)}$$
Ora $D'$ \`e preimmagine di $\abs{f(a)}$ dunque \`e chiuso.\\
Mostriamo che $D'$ \`e aperto, il che conclude $D'=D$.\\
Se $a'\in D$ dunque $\abs{f(a')}=M$ allora con gli stessi argomenti usati nella dimostrazione della propriet\`a del massimo modulo si osserva che 
$$ f(z)= f(a')\text{ per } \abs{z-a}<r_0' $$
dunque $D$ \`e aperto.\\
Essendo $f$ continua su $\overline{D}$ si ha $f$ costante su $\overline{D}$ da cui $M=M'$
\item $a\not \in D$ allora $a\in \partial D$ dunque otteniamo $M'\leq \abs{f(a)}=M'$ ma poich\`e $M\leq M'$ si ha la tesi
\end{itemize}
\end{cor}
\begin{cor}[Principio del massimo modulo per funzioni olomorfe]\bianco
Sia $f$ olomorfa su un aperto connesso $D$.\\
Se $f$ non \`e costante su $D$ allora $\abs f$ non ha massimo relativo in $D$\\
Inoltre se $D$ \`e limitata ed $f$ continua in $\overline{D}$ allora $\abs f$ ammette massimo nel bordo di $D$ 
\proof Essendo $f$ olomorfa, $f$ ha la propriet\`a del valor medio .\\
Se $\abs f$ ha un massimo relativo in $D$ allora per il principio del massimo modulo, otteniamo che $f$ \`e costante su un aperto di $D$, da cui, per il principio di continuazione analitica $f$ \`e costante in $D$ (il che \`e assurdo).\\
Assumendo che $D$ sia limitata, la tesi segue dal lemma precedente
\end{cor}
\begin{oss}Sia $f$ olomorfa in $D(0,r)=\{ z\in \C\, :\, \abs{z} <r\}$ e continua in $\overline{D(0,r)}$ allora 
$$ \abs{f(z)} \leq M(r) \quad\forall z\text{ con } \abs{z} \leq r $$
\end{oss}
\end{document}