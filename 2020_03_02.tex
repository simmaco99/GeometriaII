\documentclass[a4paper,12pt]{article}
\usepackage[a4paper, top=2cm,bottom=2cm,right=2cm,left=2cm]{geometry}

\usepackage{bm,xcolor,mathdots,latexsym,amsfonts,amsthm,amsmath,
					mathrsfs,graphicx,cancel,tikz-cd,hyperref,booktabs,caption,amssymb,amssymb,wasysym}
\hypersetup{colorlinks=true,linkcolor=blue}
\usepackage[italian]{babel}
\usepackage[T1]{fontenc}
\usepackage[utf8]{inputenc}
\newcommand{\s}[1]{\left\{ #1 \right\}}
\newcommand{\sbarra}{\backslash} %% \ 
\newcommand{\ds}{\displaystyle} 
\newcommand{\alla}{^}  
\newcommand{\implica}{\Rightarrow}
\newcommand{\iimplica}{\Leftarrow}
\newcommand{\ses}{\Leftrightarrow} %se e solo se
\newcommand{\tc}{\quad \text{ t. c .} \quad } % tale che 
\newcommand{\spazio}{\vspace{0.5 cm}}
\newcommand{\bbianco}{\textcolor{white}{,}}
\newcommand{\bianco}{\textcolor{white}{,} \\}% per andare a capo dopo 																					definizioni teoremi ...


% campi 
\newcommand{\N}{\mathbb{N}} 
\newcommand{\R}{\mathbb{R}}
\newcommand{\Q}{\mathbb{Q}}
\newcommand{\Z}{\mathbb{Z}}
\newcommand{\K}{\mathbb{K}} 
\newcommand{\C}{\mathbb{C}}
\newcommand{\F}{\mathbb{F}}
\newcommand{\p}{\mathbb{P}}

%GEOMETRIA
\newcommand{\B}{\mathfrak{B}} %Base B
\newcommand{\D}{\mathfrak{D}}%Base D
\newcommand{\RR}{\mathfrak{R}}%Base R 
\newcommand{\Can}{\mathfrak{C}}%Base canonica
\newcommand{\Rif}{\mathfrak{R}}%Riferimento affine
\newcommand{\AB}{M_\D ^\B }% matrice applicazione rispetto alla base B e D 
\newcommand{\vett}{\overrightarrow}
\newcommand{\sd}{\sim_{SD}}%relazione sx dx
\newcommand{\nvett}{v_1, \, \dots , \, v_n} % v1 ... vn
\newcommand{\ncomb}{a_1 v_1 + \dots + a_n v_n} %a1 v1 + ... +an vn
\newcommand{\nrif}{P_1, \cdots , P_n} 
\newcommand{\bidu}{\left( V^\star \right)^\star}

\newcommand{\udis}{\amalg}
\newcommand{\ric}{\mathfrak{U}}
\newcommand{\inclu}{\hookrightarrow }
%ALGEBRA

\newcommand{\semidir}{\rtimes}%semidiretto
\newcommand{\W}{\Omega}
\newcommand{\norma}{\vert \vert }
\newcommand{\bignormal}{\left\vert \left\vert}
\newcommand{\bignormar}{\right\vert \right\vert}
\newcommand{\normale}{\triangleleft}
\newcommand{\nnorma}{\vert \vert \, \cdot \, \vert \vert}
\newcommand{\dt}{\, \mathrm{d}t}
\newcommand{\dz}{\, \mathrm{d}z}
\newcommand{\dx}{\, \mathrm{d}x}
\newcommand{\dy}{\, \mathrm{d}y}
\newcommand{\amma}{\gamma}
\newcommand{\inv}[1]{#1^{-1}}
\newcommand{\az}{\centerdot}
\newcommand{\ammasol}[1]{\tilde{\gamma}_{\tilde{#1}}}
\newcommand{\pror}[1]{\mathbb{P}^#1 (\R)}
\newcommand{\proc}[1]{\mathbb{P}^#1(\C)}
\newcommand{\sol}[2]{\widetilde{#1}_{\widetilde{#2}}}
\newcommand{\bsol}[3]{\left(\widetilde{#1}\right)_{\widetilde{#2}_{#3}}}
\newcommand{\norm}[1]{\left\vert\left\vert #1 \right\vert \right\vert}
\newcommand{\abs}[1]{\left\vert #1 \right\vert }
\newcommand{\ris}[2]{#1_{\vert #2}}
\newcommand{\vp}{\varphi}
\newcommand{\vt}{\vartheta}
\newcommand{\wt}[1]{\widetilde{#1}}
\newcommand{\pr}[2]{\frac{\partial \, #1}{\partial\, #2}}%derivata parziale
%per creare teoremi, dimostrazioni ... 
\theoremstyle{plain}
\newtheorem{thm}{Teorema}[section] 
\newtheorem{ese}[thm]{Esempio} 
\newtheorem{ex}[thm]{Esercizio} 
\newtheorem{fatti}[thm]{Fatti}
\newtheorem{fatto}[thm]{Fatto}

\newtheorem{cor}[thm]{Corollario} 
\newtheorem{lem}[thm]{Lemma} 
\newtheorem{al}[thm]{Algoritmo}
\newtheorem{prop}[thm]{Proposizione} 
\theoremstyle{definition} 
\newtheorem{defn}{Definizione}[section] 
\newcommand{\intt}[2]{int_{#1}^{#2}}
\theoremstyle{remark} 
\newtheorem{oss}{Osservazione} 
\newcommand{\di }{\, \mathrm{d}}
\newcommand{\tonde}[1]{\left( #1 \right)}
\newcommand{\quadre}[1]{\left[ #1 \right]}
\newcommand{\w}{\omega}

% diagrammi commutativi tikzcd
% per leggere la documentazione texdoc

\begin{document}
\textbf{Lezione del 2 Marzo}
\begin{defn}[Prodotto libero]\bianco
Sia  $\ds (G_i)_{i\in I}$ una famiglia di gruppi.\\
Il prodotto libero dei $G_i$ \`e 
\begin{itemize}
\item[a)]un gruppo $G$
\item[b)]$ \left( \phi_i:\, G_i \to G \right)_{i\in I} $ omomorfismi di gruppi che soddisfano la seguente propiet\`a universale:\\
dati $\psi_{i}:\, G_i \to H$ omomorfismi di gruppi con $H$ gruppo arbitrario, $\exists ! \psi$ che fa commutare il diagramma $\forall i$
$$ \begin{tikzcd}G_i  \arrow{rr}{\phi_i}
								   \arrow{rd}{\psi_i} & & G \arrow[dashed]{dl}{\psi}  \\
& H
\end{tikzcd}$$
ovvero tale che $\psi\circ \phi_i = \psi_i$
\end{itemize}
In questo caso denotiamo $G= \star_{i\in I} G_i$
\end{defn}
\begin{ese}Nel caso di $I=\{ 1,2\}$ abbiamo la seguente situazione 
$$\begin{tikzcd}
{}  G_1	\arrow{rd}{\phi_1}
			\arrow[bend left]{rrd}{\psi_1}&                     &    \\
                                      & G \arrow[ dashed]{r}{\psi} &H \\
G_2\arrow{ru}{\psi_2} \arrow[bend right]{rru}{\psi_2} &                     &   
\end{tikzcd}$$
dove $\psi$ \`e tale che i due triangoli commutino 
\end{ese}

\begin{prop}[Unicit\`a del prodotto libero]\bianco
\proof Fissiamo una famiglia $\ds (G_i)_{i\in I}$ di gruppi e siano $(G,\phi_i)$ e $(G', \psi_i')$ due prodotti liberi.\\
Usando la propiet\`a universale ponendo $H=G'$ e $\psi_i=\phi_i'$ si ha $\exists ! \phi':\, G \to G'$ tale che $\phi'\circ \phi_i =\phi_i'$\\
Usando la propiet\`a universale ponendo $H=G$ e $\psi_i=\phi_i$ si ha $\exists ! \phi:\, G' \to G$ tale che $\phi\circ \phi_i' =\phi_i$\\
Dunque mettendo insieme le 2 relazioni otteniamo $$\phi_i=\phi\circ \phi_i' = \phi \circ \left( \phi' \circ \phi_i \right)=\left( \phi \circ \phi' \right)\circ \phi_i$$
dunque abbiamo che i seguenti diagrammi commutano
$$\begin{tikzcd} G_i \arrow{rd}{\phi_i} \arrow{r}{\phi_i} & G \arrow{d}{\phi\circ \phi'} & &G_i \arrow{rd}{\phi_i} \arrow{r}{\phi_i} & G \arrow{d}{id_G } 
\\
 &G & & & G
\end{tikzcd}$$
dunque dalla propiet\`a universale $\phi \circ \phi'=id_G$ da cui $\phi$ \`e l'isomorfismo cercato 
\endproof
\end{prop}
\spazio
\begin{prop}[Esistenza del prodotto libero]\bianco
\proof Denotiamo con $e_i \in G_i$ l'identit\`a $\forall i \in I$.\\
Definiamo $$W=\bigcup_{i\in I} ( G_i - \{e_i\})$$
e definiamo $G$ nel seguente modo
\begin{itemize}
\item come insieme
$$ G=\left\{\, (g_1, \dots, g_m) \, \left\vert \,
{ {m\geq 0\, \,  g_i \in W }\atop{ g_k\text{ e } g_{k+1} \text{ appartengono a insiemi diversi}}}\right.\right\}$$
dove $(g_1, \dots, g_m)$ \`e chiamata parola ridotta sull'alfabeto $W$
\item definiamo su $G$ un prodotto
$ (g_1,\dots, g_m)\cdot(h_1,\dots,h_s)= $ $$=\begin{cases} (g_1,\dots, g_m,h_1,\dots,h_s)  \text{ se } g_m \text{ e } h_1 \text{ non appartengono allo stesso } G_i\\
(g_1,\dots, g_mh_1,\dots,h_s)  \text{ se } g_m \text{ e } h_1 \text{ appartengono allo stesso } G_i \text{ e } g_m h_1\neq e_i \\
(g_1,\dots, g_{m-1},h_2,\dots,h_s)  \text{ se } g_m \text{ e } h_1 \text{ appartengono allo stesso } G_i \text{ e } g_m h_1= e_i 
\end{cases}$$
Nel terzo caso si procede induttivamente analizzando $g_{m-1}$ e $h_2$.\\
La parola verr\`a indicata senza tonde e senza virgole 
\end{itemize}
Definiamo 
$\phi_i:\, G_i \to G \quad x \to (x)$ dove $(x)$ \`e una parola.\\
Verifichiamo che tale coppia soddisfa la propiet\`a universale.\\
Dati $\psi_i:\, G_i \to H $ omomorfismi di gruppi  ($H$ arbitrario) definiamo $\psi:\, G \to H $ nel seguente modo
$$ \psi(g_1\dots g_m)=\psi_{i_1}(g_1) \cdots \psi_{i_m}(g_m)$$
dove $g_j \in G_{i_j}$.\\
\`E di facile verifica che con tale scelta si ha $\psi \circ \phi_i=\psi_i$
\endproof
\end{prop}
\begin{defn}[Gruppo libero generato da un insieme]\bianco
Sia $S$ un insieme.\\
Un gruppo libero generato da $S$ \`e il dato di un gruppo $F$ e di un'applicazione $\phi$ che gode della seguente propiet\`a universale.\\
Data $\psi:\, S \to H$ mappa con $H$ arbitrario, esiste unico omomorfismo $\vartheta:\, F \to H$ tale che $\psi=\vartheta \circ \phi$
\end{defn}
\begin{prop}Per ogni insieme, esiste un unico gruppo libero generato da $S$
\proof
Sia $S=\{ x_i \, i\in I\}$ allora definiamo
$$ F(S)= \star_{i\in I } G_i \text{ dove } G_i = \{ x_i^m \, \vert \, m \in \Z\}\cong\Z$$
ovvero \`e il prodotto libero di copie di $\Z$ indicizzate dagli elementi di $S$\\
Prendiamo come $\phi$ l'ovvia inclusione di $S$ in $F(S)$
Si pu\`o provare che due prodotti liberi generati da un insiemi sono canonicamente isomorfi (la dimostrazione \`e analoga a quanto osservato per i prodotti liberi)
\end{prop}
\begin{oss}Dalla propiet\`a universale di $F(S)$ possiamo concludere che dato $H$ gruppo arbitrario 
$$ Hom(F(S),H)=Mappe(S,H)$$
\end{oss}
\newpage
\begin{defn}[Sottogruppo generato da un insieme]\bianco
Sia $S\subseteq G$ sottoinsieme con $G$ gruppo allora
$$ \langle S\rangle =\bigcap_{{K < G}\atop{S\subseteq K}}K$$
\begin{oss}
$$\langle S \rangle=\left\{ a_1^{\pm 1} \dots a_k^{\pm 1} \, \vert \, a_i \in S \text{ e } k \in \N\right\}$$
\end{oss}
\end{defn}
\begin{defn}[Chiusura normale]\bianco
Sia $S\subseteq G$ sottoinsieme con $G$ gruppo allora
$$N(S)=\bigcap_{{K\lhd G}\atop{S\subseteq K}}K$$
\begin{oss}
$$N(S)= \{ g a g^{-1}  \, \vert \, g \in G ,\, \in S\}$$
\end{oss}
\end{defn}
\begin{oss}Vogliamo studiare la relazione tra $G=\langle S \rangle$ e $F(S)$\\
Consideriamo il seguente diagramma
$$\begin{tikzcd} S \arrow[r,"i"]  \arrow[hook,rd]& F(S) \\ &  G
\end{tikzcd}$$
Dunque per la propiet\`a universale di $F(S)$ esiste una mappa $\phi:\, F(S) \to G$ tale che il diagramma commuta.\\
Dalla definizione di $G$ segue che la mappa \`e suriettiva, da cui 
$$ G \cong \frac{F(S)}{Ker \, \phi} =\frac{F(S)}{N(R)}$$ 
Poniamo per notazione $G=\langle S \, \vert \, R \rangle$, dove $S$ \`e l'insieme dei generatori, mentre $R$ l'insieme dei generatori
\end{oss}
\begin{ese}$$\Z \cong \langle 1 \rangle$$
$$\Z\oplus \Z \cong \langle a, b \, \vert \, ab a^{-1}b^{-1}\rangle$$

\end{ese}
\end{document}