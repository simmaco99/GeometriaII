\documentclass[a4paper,12pt]{article}
\usepackage[a4paper, top=2cm,bottom=2cm,right=2cm,left=2cm]{geometry}

\usepackage{bm,xcolor,mathdots,latexsym,amsfonts,amsthm,amsmath,
					mathrsfs,graphicx,cancel,tikz-cd,hyperref,booktabs,caption,amssymb,amssymb,wasysym}
\hypersetup{colorlinks=true,linkcolor=blue}
\usepackage[italian]{babel}
\usepackage[T1]{fontenc}
\usepackage[utf8]{inputenc}
\newcommand{\s}[1]{\left\{ #1 \right\}}
\newcommand{\sbarra}{\backslash} %% \ 
\newcommand{\ds}{\displaystyle} 
\newcommand{\alla}{^}  
\newcommand{\implica}{\Rightarrow}
\newcommand{\iimplica}{\Leftarrow}
\newcommand{\ses}{\Leftrightarrow} %se e solo se
\newcommand{\tc}{\quad \text{ t. c .} \quad } % tale che 
\newcommand{\spazio}{\vspace{0.5 cm}}
\newcommand{\bbianco}{\textcolor{white}{,}}
\newcommand{\bianco}{\textcolor{white}{,} \\}% per andare a capo dopo 																					definizioni teoremi ...


% campi 
\newcommand{\N}{\mathbb{N}} 
\newcommand{\R}{\mathbb{R}}
\newcommand{\Q}{\mathbb{Q}}
\newcommand{\Z}{\mathbb{Z}}
\newcommand{\K}{\mathbb{K}} 
\newcommand{\C}{\mathbb{C}}
\newcommand{\F}{\mathbb{F}}
\newcommand{\p}{\mathbb{P}}

%GEOMETRIA
\newcommand{\B}{\mathfrak{B}} %Base B
\newcommand{\D}{\mathfrak{D}}%Base D
\newcommand{\RR}{\mathfrak{R}}%Base R 
\newcommand{\Can}{\mathfrak{C}}%Base canonica
\newcommand{\Rif}{\mathfrak{R}}%Riferimento affine
\newcommand{\AB}{M_\D ^\B }% matrice applicazione rispetto alla base B e D 
\newcommand{\vett}{\overrightarrow}
\newcommand{\sd}{\sim_{SD}}%relazione sx dx
\newcommand{\nvett}{v_1, \, \dots , \, v_n} % v1 ... vn
\newcommand{\ncomb}{a_1 v_1 + \dots + a_n v_n} %a1 v1 + ... +an vn
\newcommand{\nrif}{P_1, \cdots , P_n} 
\newcommand{\bidu}{\left( V^\star \right)^\star}

\newcommand{\udis}{\amalg}
\newcommand{\ric}{\mathfrak{U}}
\newcommand{\inclu}{\hookrightarrow }
%ALGEBRA

\newcommand{\semidir}{\rtimes}%semidiretto
\newcommand{\W}{\Omega}
\newcommand{\norma}{\vert \vert }
\newcommand{\bignormal}{\left\vert \left\vert}
\newcommand{\bignormar}{\right\vert \right\vert}
\newcommand{\normale}{\triangleleft}
\newcommand{\nnorma}{\vert \vert \, \cdot \, \vert \vert}
\newcommand{\dt}{\, \mathrm{d}t}
\newcommand{\dz}{\, \mathrm{d}z}
\newcommand{\dx}{\, \mathrm{d}x}
\newcommand{\dy}{\, \mathrm{d}y}
\newcommand{\amma}{\gamma}
\newcommand{\inv}[1]{#1^{-1}}
\newcommand{\az}{\centerdot}
\newcommand{\ammasol}[1]{\tilde{\gamma}_{\tilde{#1}}}
\newcommand{\pror}[1]{\mathbb{P}^#1 (\R)}
\newcommand{\proc}[1]{\mathbb{P}^#1(\C)}
\newcommand{\sol}[2]{\widetilde{#1}_{\widetilde{#2}}}
\newcommand{\bsol}[3]{\left(\widetilde{#1}\right)_{\widetilde{#2}_{#3}}}
\newcommand{\norm}[1]{\left\vert\left\vert #1 \right\vert \right\vert}
\newcommand{\abs}[1]{\left\vert #1 \right\vert }
\newcommand{\ris}[2]{#1_{\vert #2}}
\newcommand{\vp}{\varphi}
\newcommand{\vt}{\vartheta}
\newcommand{\wt}[1]{\widetilde{#1}}
\newcommand{\pr}[2]{\frac{\partial \, #1}{\partial\, #2}}%derivata parziale
%per creare teoremi, dimostrazioni ... 
\theoremstyle{plain}
\newtheorem{thm}{Teorema}[section] 
\newtheorem{ese}[thm]{Esempio} 
\newtheorem{ex}[thm]{Esercizio} 
\newtheorem{fatti}[thm]{Fatti}
\newtheorem{fatto}[thm]{Fatto}

\newtheorem{cor}[thm]{Corollario} 
\newtheorem{lem}[thm]{Lemma} 
\newtheorem{al}[thm]{Algoritmo}
\newtheorem{prop}[thm]{Proposizione} 
\theoremstyle{definition} 
\newtheorem{defn}{Definizione}[section] 
\newcommand{\intt}[2]{int_{#1}^{#2}}
\theoremstyle{remark} 
\newtheorem{oss}{Osservazione} 
\newcommand{\di }{\, \mathrm{d}}
\newcommand{\tonde}[1]{\left( #1 \right)}
\newcommand{\quadre}[1]{\left[ #1 \right]}
\newcommand{\w}{\omega}

% diagrammi commutativi tikzcd
% per leggere la documentazione texdoc

\begin{document}
\section{Sottospazio prodotto}
\textbf{Lezione del 10 ottobre di Gandini}
\begin{defn}[Prodotto cartesiano]\bianco
Sia $ \ds \{X_\alpha\}_{\alpha\in A}$ una famiglia di insiemi, il prodotto cartesiano della famiglia \`e 
$$ X=\prod_{\alpha\in A} X_\alpha = \left. \left\{ f:\, A \to \bigcup_{\alpha\in A} X_\alpha \, \right\vert \,  f(\alpha) \in X_\alpha \, \forall \alpha\in A  \right\}$$
Nel caso in cui $A=\{ 1, \dots , n \}$ 
$$ X = \prod_{i=1}^n X_i = X_1 \times \dots \times X_n$$
$$ x\in X \quad \implica \quad x= (x_1, \dots , x_n ) \text{ ovvero } f(i)=x_i$$
\end{defn}
\begin{defn}[Proiezioni]Lo spazio cartesiano $X$ ammette delle proiezioni naturali $\forall \alpha \in A $
$$ P_\alpha:\, X \to X_\alpha \quad f \to f(\alpha) $$
\end{defn}
\spazio
\begin{defn}La topologia prodotto su $X$ \`e la topologia meno fine  che rende  tutte le proiezioni $P_\alpha$ continue.
\end{defn}
\begin{prop}Una base per la topologia prodotto \`e data da 
$$\B=\left\{ \left. \prod_{\alpha\in A} U_\alpha \, \right \vert \, U_\alpha\subseteq X_\alpha \text{ aperto e } U_\alpha \neq X_\alpha \text{ per un numero finito di } \alpha \right\}$$
\proof Sia $U_\alpha \subseteq X_\alpha $ aperto allora
$$ P_\alpha^{-1} (U_\alpha)= \prod_{\beta \in A } V_\beta  \text{ dove } V_\beta = \begin{cases} U_\alpha \text{ se } \beta= \alpha\\ X_\beta \text{ se } \beta \neq \alpha \end{cases} $$
Dunque le $P_\alpha$ sono continue se e solo se tutte le controimmagini di tale forma sono aperte \\
Siano $\alpha_1, \dots, \alpha_n\in A$ e $U_{\alpha_i} \subseteq X_{\alpha_i} $ aperti $\forall i=1,\dots, n $ allora
$$ A=\bigcap_{i=1}^n P_{\alpha_i}^{-1}(U_{\alpha_i} ) \subseteq X \text{ aperto nella topologia prodotto} $$
intersezione finita di aperti \`e un aperto; ora 
$$ A=\prod_{\alpha \in A } V_\alpha \text{ dove } V_\alpha = \begin{cases}
U_{\alpha_i} \text{ se } \alpha=\alpha_i \\ X_\alpha \text{ se } \alpha\neq \alpha_i \end{cases}$$
Dunque osserviamo che  $A\in \B$ infatti solo un numero finito di aperti \`e diverso da tutto lo spazio $X_i$, ovvero ogni elemento di $\B$ \`e un aperto nella topologia prodotto.\\
Se $\B$ \`e una base di una topologia possiamo concludere in quanto, per definizione, cerchiamo la topologia meno fine che rende continue le proiezioni.\\
Mostriamo che $\B$ \`e una base.\begin{enumerate}
\item Se prendiamo $U_\alpha=X_\alpha $ $\forall \alpha\in A $ allora $\prod U_\alpha=X$ ovvero $\B$ ricopre $X$
\item  
$$ \left( \prod_{\alpha \in A} U_\alpha \right) \cap  \left( \prod_{\alpha \in A} V_\alpha \right)=   \prod_{\alpha \in A} (U_\alpha \cap V_\alpha)$$
Se $U_\alpha , V_\alpha$ sono aperti in $X_\alpha$ allora $U_\alpha \cap V_\alpha$ \`e un aperto di $X_\alpha$.\\
Inoltre se il numero dei $U_\alpha \neq X_\alpha $ e dei $V_\alpha \neq X_\alpha$ \`e finito allora sar\`a finito anche il numero dei $U_\alpha \cap V_\alpha \neq X_\alpha$
\end{enumerate}
Abbiamo provato che $\B$ verifica il criterio per essere una base 
\endproof
\end{prop}
\spazio
\begin{oss}Supponiamo $B_\alpha $ base per la topologia di $X_\alpha$ e assumiamo che $X_\alpha \in B_\alpha$ allora
$$\B'=\left\{ \left. \prod _{\alpha \in A} B_\alpha \, \right\vert\, B_\alpha \in \B_\alpha \text{ e } B_\alpha=X_\alpha \text{ tranne per finiti } \alpha \right\}$$
\`e una base della topologia prodotto su $X$\\
Segue direttamente dal fatto che $\B$ definita nella proposizione precedente \`e una base
\end{oss}
\begin{cor}Sia $A$ numerabile e $X_\alpha$ secondo-numerabile $\forall\alpha\in A$ allora $X$ \`e secondo-numerabile.\\
\proof Sia $\B_\alpha$ una base numerabile per $X_\alpha$ e supponiamo che $X_\alpha\in \B_\alpha$
Sia 
$$A_i= \left(  \left.\prod_{\alpha\in A} B_\alpha \, \right \vert B_\alpha \in \B_\alpha  \text{ e } B_\alpha\neq X_\alpha \text{ per le prime } i  \right\}$$
Allora $A_i$ \`e ovviamente numerabile inoltre 
$$\B'=\bigcup_{i \in A } A_i$$
quindi $\B'$ \`e numerabile essendo unione numerabile di insiemi numerabili
\endproof
\end{cor}
\begin{cor}Sia $A$ numerabile e $X_\alpha$ primo-numerabile $\forall\alpha\in A$ allora $X$ \`e primo-numerabile
\end{cor}
\begin{oss}Se $A$ non \`e numerabile, i corollari precedenti, in generale, sono falsi
\end{oss}
\newpage

\subsection{Propiet\`a della topologia prodotto}
\begin{prop}La proiezione $P_\alpha$  \`e un' applicazione aperta
\proof Sia $\B$ una base di $X$ allora 
$$ P_\alpha \text{ aperta } \ses P_ \alpha(B) \text{ aperta } \forall B \in \B$$
Per come abbiamo definito una base della topologia prodotto
$$ B = \prod_{\beta\in A }  U_\beta
 \text{ dove } U_\beta \subseteq X_\beta \text{ aperto e  } U_\beta \neq X_\beta \text{ per finiti } \beta $$
 quindi $P_\alpha(B) = U_\alpha$ che \`e aperto per definizione.
 \endproof
\end{prop}
\begin{oss}$P_\alpha$, in generale, non \`e chiusa.\\
Prendiamo $\R^2$ e consideriamo $P_1$.\\
Sia $Z=\{ (x,y) \, \vert xy=1 \}$ ovvero un iperbole equilatera, 
$Z$ \`e chiuso in quanto luogo di zeri di un polinomio $(Z=p^{-1}(\{ 0\} ))$, un polinomio \`e una funzione continua e $\{ 0 \} $ \`e chiuso.\\
$$P_1(Z) =\R \sbarra \{ 0 \} \text{ non \`e chiuso perch\`e in } \R \text{ } \{ 0 \} \text{ \`e aperto }$$
\end{oss}
\spazio
\begin{prop}Dato $\alpha\in A $, fissiamo $x_\beta \in X_\beta $ con $\beta \neq \alpha$, sia 
$$X(\alpha) = \{ f\in X \, \vert \, f(\beta)= x_\beta \} \subseteq X $$ 
Allora la restrizione 
$$ P_{\vert \alpha} :\, X(\alpha) \to X_\alpha \text{ \`e un omeomorfismo } $$
dove $X(\alpha)$ eredita la topologia di sottospazio
\proof \bbianco
\begin{itemize}
\item La restrizione \`e continua infatti $P_{\vert \alpha} = P_\alpha \circ i $ con $i:\, X(\alpha) \inclu X $, ora $i$ \`e continua per definizione di topologia di sottospazio 
\item La restrizione \`e biunivoca infatti l'inclusione \`e iniettiva 
\item Mostriamo che la funzione \`e aperta.\\
Gli aperti di $X(\alpha)$ sono gli insiemi della forma 
$$U= \left( \prod_{\gamma \in A } U_\gamma \right) \cap X(\alpha)$$
Se $U\neq \emptyset$ allora 
$$ U = \{ f\in X(\alpha) \, \vert \,  f(\alpha)\in U_\alpha\} $$
D'altra parte $P_{\vert \alpha}(U) = P_\alpha(U) = U_\alpha$ che \`e aperto 
\end{itemize}
\endproof
\end{prop}
\begin{oss}$P_{\vert \alpha} $ non \`e canonica come la proiezione infatti dipende dalla scelta di $x_\beta$
\end{oss}

\newpage
\subsection{Propiet\`a universale}
\begin{prop}\bianco
La topologia prodotto verifica la seguente propiet\`a:\\
dato $Z$ topologico e $f:\, Z \to X $ una funzione arbitraria
$$ f \text{ continua} \quad \ses \quad P_\alpha \circ f \text{ continua}$$
\proof$\implica$ Composizione di funzioni continue \`e una funzione continua.\\
$\iimplica$ Sia $U\subseteq X$ un aperto, dimostriamo che $f^{-1}(U)$ \`e aperto.\\
Basta vederlo per $U\in \B$ (base della topologia prodotto) ovvero per un $U$ della forma
$$ U = \prod_{\alpha\in A} U_\alpha \text{ dove } U_\alpha\subseteq X_\alpha \text{ aperto } $$
inoltre $\exists A_0 \subseteq A \text{ finito, tale che } \forall \alpha \not \in A_0 \quad U_\alpha = X_\alpha$\\
D'altra parte $\ds f^{-1}(U) =f^{-1} \left( \prod_{\alpha\in A} U_\alpha \right)$ e osservando che 
$U=\ds\bigcap_{\alpha\in A_0} P_\alpha^{-1}(U_\alpha)$ otteniamo
$$ f^{-1}(U)= f^{-1} \left( \bigcap_{\alpha\in A_0} P^{-1}_\alpha(U_\alpha)\right) = \bigcap_{\alpha\in A_0 } (P_\alpha \circ f )^{-1}(U_\alpha)$$
Ora intersezione finita di aperti \`e un aperto quindi 
$$ \forall U \in \B \quad f^{-1}(U) \text{ \`e aperto } \quad \implica f \text{ continua } $$
\endproof
\end{prop}
\begin{thm}La topologia prodotto \`e univocamente determinata dalla propiet\`a universale.\\
Vale a dire:\\
Se $\tau_X$ \`e una topologia sul prodotto con la propiet\`a 
$$ \forall Z \text{ topologico } \, f:\, Z \to X \text{ allora } $$
$$ f \text{ continua } \quad \ses \quad P_\alpha \circ f \text{ continua } \forall \alpha\in A  $$
Allora $\tau_X$ \`e la topologia prodotto
\proof Abbiamo dimostrato che la topologia prodotto soddisfa la propiet\`a universale.\\
Sia $\tau_X$ una topologia che soddisfa la propiet\`a, $\tau_\alpha$ la topologia su $\tau_X$ e $\tau_{prod}$ la topologia prodotto, abbiamo dunque il seguente diagramma 
$$\begin{tikzcd} & (X, \tau_X) \arrow{d}{P_\alpha}\\
(Z,\tau_Z) \arrow[bend left]{ur}{f} \arrow{r}{P_\alpha\circ f} &(X_\alpha,\tau_\alpha)
\end{tikzcd}$$
\begin{itemize}
\item Prendiamo $Z=(X, \tau_{prod})$ allora
$$\begin{tikzcd} & (X, \tau_X) \arrow{d}{P_\alpha}\\
(X,\tau_{prod}) \arrow[bend left]{ur}{id_X} \arrow{r}{P_\alpha} &(X_\alpha,\tau_\alpha)
\end{tikzcd}$$
$P_\alpha:\, (X,\tau_{prod})\to (X,\tau_\alpha)$ \`e continua per definizione allora per definizione \`e continua anche 
$$id_X:\, (X,\tau_{prod})\to (X,\tau_X) \quad \implica \quad \tau_X < \tau_{prod}$$
\item Proviamo che $P_\alpha:\,(X,\tau_X) \to (X_\alpha,\tau_\alpha)$ \`e continua e poi concludere per minimalit\`a.\\
Prendiamo $Z=(X,\tau_X)$  allora 
$$\begin{tikzcd} & (X, \tau_X) \arrow{d}{P_\alpha}\\
(X,\tau_{X}) \arrow[bend left]{ur}{id_X} \arrow{r}{P_\alpha} &(X_\alpha,\tau_\alpha)
\end{tikzcd}$$
e poich\`e $id_X:\, (X, \tau_X)\to (X,\tau_X)$ \`e continua allora per la propiet\`a $P_\alpha:\, (X,\tau_X) \to (X_\alpha, \tau_\alpha)$ 
\`e continua
\end{itemize}
\end{thm}

\end{document}