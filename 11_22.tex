\documentclass[a4paper,12pt]{article}
\usepackage[a4paper, top=2cm,bottom=2cm,right=2cm,left=2cm]{geometry}

\usepackage{bm,xcolor,mathdots,latexsym,amsfonts,amsthm,amsmath,
					mathrsfs,graphicx,cancel,tikz-cd,hyperref,booktabs,caption,amssymb,amssymb,wasysym}
\hypersetup{colorlinks=true,linkcolor=blue}
\usepackage[italian]{babel}
\usepackage[T1]{fontenc}
\usepackage[utf8]{inputenc}
\newcommand{\s}[1]{\left\{ #1 \right\}}
\newcommand{\sbarra}{\backslash} %% \ 
\newcommand{\ds}{\displaystyle} 
\newcommand{\alla}{^}  
\newcommand{\implica}{\Rightarrow}
\newcommand{\iimplica}{\Leftarrow}
\newcommand{\ses}{\Leftrightarrow} %se e solo se
\newcommand{\tc}{\quad \text{ t. c .} \quad } % tale che 
\newcommand{\spazio}{\vspace{0.5 cm}}
\newcommand{\bbianco}{\textcolor{white}{,}}
\newcommand{\bianco}{\textcolor{white}{,} \\}% per andare a capo dopo 																					definizioni teoremi ...


% campi 
\newcommand{\N}{\mathbb{N}} 
\newcommand{\R}{\mathbb{R}}
\newcommand{\Q}{\mathbb{Q}}
\newcommand{\Z}{\mathbb{Z}}
\newcommand{\K}{\mathbb{K}} 
\newcommand{\C}{\mathbb{C}}
\newcommand{\F}{\mathbb{F}}
\newcommand{\p}{\mathbb{P}}

%GEOMETRIA
\newcommand{\B}{\mathfrak{B}} %Base B
\newcommand{\D}{\mathfrak{D}}%Base D
\newcommand{\RR}{\mathfrak{R}}%Base R 
\newcommand{\Can}{\mathfrak{C}}%Base canonica
\newcommand{\Rif}{\mathfrak{R}}%Riferimento affine
\newcommand{\AB}{M_\D ^\B }% matrice applicazione rispetto alla base B e D 
\newcommand{\vett}{\overrightarrow}
\newcommand{\sd}{\sim_{SD}}%relazione sx dx
\newcommand{\nvett}{v_1, \, \dots , \, v_n} % v1 ... vn
\newcommand{\ncomb}{a_1 v_1 + \dots + a_n v_n} %a1 v1 + ... +an vn
\newcommand{\nrif}{P_1, \cdots , P_n} 
\newcommand{\bidu}{\left( V^\star \right)^\star}

\newcommand{\udis}{\amalg}
\newcommand{\ric}{\mathfrak{U}}
\newcommand{\inclu}{\hookrightarrow }
%ALGEBRA

\newcommand{\semidir}{\rtimes}%semidiretto
\newcommand{\W}{\Omega}
\newcommand{\norma}{\vert \vert }
\newcommand{\bignormal}{\left\vert \left\vert}
\newcommand{\bignormar}{\right\vert \right\vert}
\newcommand{\normale}{\triangleleft}
\newcommand{\nnorma}{\vert \vert \, \cdot \, \vert \vert}
\newcommand{\dt}{\, \mathrm{d}t}
\newcommand{\dz}{\, \mathrm{d}z}
\newcommand{\dx}{\, \mathrm{d}x}
\newcommand{\dy}{\, \mathrm{d}y}
\newcommand{\amma}{\gamma}
\newcommand{\inv}[1]{#1^{-1}}
\newcommand{\az}{\centerdot}
\newcommand{\ammasol}[1]{\tilde{\gamma}_{\tilde{#1}}}
\newcommand{\pror}[1]{\mathbb{P}^#1 (\R)}
\newcommand{\proc}[1]{\mathbb{P}^#1(\C)}
\newcommand{\sol}[2]{\widetilde{#1}_{\widetilde{#2}}}
\newcommand{\bsol}[3]{\left(\widetilde{#1}\right)_{\widetilde{#2}_{#3}}}
\newcommand{\norm}[1]{\left\vert\left\vert #1 \right\vert \right\vert}
\newcommand{\abs}[1]{\left\vert #1 \right\vert }
\newcommand{\ris}[2]{#1_{\vert #2}}
\newcommand{\vp}{\varphi}
\newcommand{\vt}{\vartheta}
\newcommand{\wt}[1]{\widetilde{#1}}
\newcommand{\pr}[2]{\frac{\partial \, #1}{\partial\, #2}}%derivata parziale
%per creare teoremi, dimostrazioni ... 
\theoremstyle{plain}
\newtheorem{thm}{Teorema}[section] 
\newtheorem{ese}[thm]{Esempio} 
\newtheorem{ex}[thm]{Esercizio} 
\newtheorem{fatti}[thm]{Fatti}
\newtheorem{fatto}[thm]{Fatto}

\newtheorem{cor}[thm]{Corollario} 
\newtheorem{lem}[thm]{Lemma} 
\newtheorem{al}[thm]{Algoritmo}
\newtheorem{prop}[thm]{Proposizione} 
\theoremstyle{definition} 
\newtheorem{defn}{Definizione}[section] 
\newcommand{\intt}[2]{int_{#1}^{#2}}
\theoremstyle{remark} 
\newtheorem{oss}{Osservazione} 
\newcommand{\di }{\, \mathrm{d}}
\newcommand{\tonde}[1]{\left( #1 \right)}
\newcommand{\quadre}[1]{\left[ #1 \right]}
\newcommand{\w}{\omega}

% diagrammi commutativi tikzcd
% per leggere la documentazione texdoc


\begin{document}
\textbf{Lezione del 22 Novembre del Prof. Frigerio}
\begin{defn}[Spazio proiettivo]\bianco
Sia $\K$ un campo e sia $V$ un $\K$-spazio vettoriale.\\
Definiamo su $V$ la seguente relazione di equivalenza
$$ v \sim w \quad \ses \quad \exists \lambda \in \K^\star \quad v - \lambda w $$
Lo spazio proiettivo associato a $V$ \`e
$$ \p (V) = \frac{V \sbarra \{ 0 \}}{\sim}= \frac{V \sbarra  \{ 0\}}{G} \quad G = \{ \lambda Id \, \lambda\neq 0 \}$$
$\p(V)$ \`e detto lo spazio delle rette di $V$
\end{defn}
\begin{defn}\bianco$W \subseteq \p(V)$ \`e un sottospazio do dimensione $k$ se 
$$ W = \pi ( H \sbarra \{ 0\})$$ dove $H \subseteq V $ \`e uno spazio vettoriale di dimensione $k+1$ e $\pi:\, V \sbarra \{ 0 \} \to \p(V)$
\end{defn}
\begin{oss}Lo spazio proiettivo $p(V)$ ha dimensione $\dim V +1$
\end{oss}
\begin{oss}Se $V=\K^n$ si pone $\p(\K^n)=\p^{n-1}(\K)$.\\
Nel caso in cui $\K=\R$ oppure $\K=\C$ assumiamo $\p^n(\K)$ dotato della topologia quoziente di $\R^{n+1}\sbarra \{ 0 \}$ e $\C^{n+1}\sbarra \{ 0 \} \cong \R^{2n+2} \sbarra \{ 0 \}$
\end{oss}
\spazio
\begin{oss}Se $S^n\subseteq \R^{n+1}$ la sfera unitaria $S^n$ interseca ogni retta in $2$ punti della forma $\pm v$ per cui COME INSIEME 
$$ \p^n (\R) = \frac{S^n}{pm Id}$$
\end{oss}
\begin{prop}$\p^n(\R) \cong \frac{S^n}{\pm Id}$
\proof 
Consideriamo il seguente diagramma
$$ \begin{tikzcd} S^n \arrow[hook, r,"i"] \arrow[d]& \R^{n+1} \sbarra \{ 0 \} \arrow{d}{\pi} \\ \frac{S^n}{\pm Id} \arrow{r}{\overline{i}} &\p^n(\R)
\end{tikzcd}$$
esso \`e commutativo inoltre dalla propiet\`a universale della topologia quoziente $\overline{i}$ \`e continua.\\
Consideriamo adesso il seguente diagramma commutativo
$$ \begin{tikzcd} R^{n+1}\sbarra \{ 0\} \arrow[ r,"r"] \arrow{d}{\pi}& \S^{n} \arrow[d] \\ \frac{S^n}{\pm Id} \arrow{r}{\overline{i}} &\p^n(\R)
\end{tikzcd}$$
dove $r(x)=\frac{x}{\vert \vert x \vert\vert}$.\\
Analogamente $\overline{r}$ \`e continua.\\
Inoltre $\overline{r}$ e $\overline{i}$ sono una l'inversa dell'altra duneq omeomorfismi\endproof
\end{prop}
\spazio
\begin{cor}$\p^n(\R)$ \`e compatto e di Hausdorff
\proof $\p^n(\R)$ \`e quoziente di $S^n$ che \`e compatto dunque anche $\p^n(\R)$ lo \`e.\\
$P^n(\R)$ \`e di Hausdorff per l'azione di un gruppo finito (dunque l'azione \`e propria) \endproof
\end{cor}
\begin{prop}Sia $$D^n=\{ x\in \R^n\, , \, \vert\vert x \vert \vert \leq 1 \}$$ e definiamo $\sim_D$ la relazione di equivalenza
$$ x \sim_D y \quad \ses \quad x =y \text{ o } ( \vert \vert x\vert \vert = \vert \vert y \norm =1 \text{ e } x = -y )$$
Allora $P^n (\R) \cong \frac{D^n }{\sim_D}$
\proof Sia $H =\{ (x_0, \dots, x_n ) \in S^n \, \vert \, x_0 \geq 0\}$ ovvero $H$ \`e "l'emisfero nord"\\
Definiamo inoltre la relazione di equivalenza
$$ v \sim_H w \quad \ses \quad v =w \text{ o } ( v = -w \text{ e } x_0(v)=x_0(w)=0$$
ovvero i punti sull'equatore stanno in una stessa classe di equivalenza.\\
La composizione $ H \inclu S^n \to \p^n(\R)$ induce una bigezione continua.\\
La continuit\`a deriva dalla propiet\`a universale della topologia quoziente, inoltre ogni punto del proiettivo ha un rappresentante nell'emisfero nord (surgettiva), l'iniettivit\`a deriva dal fatto che $\sim_H$ identifica tutti i punti che vanno nella stessa classe del proiettivo.\\
Ora $H$ compatto dunque $ \frac{H}{\sim H}$ compatto ed essendo $\p^n(\R)$ T2 si conclude che $ \p^n(\R) \cong \frac{H}{
\sim_H}$.\\
Infine l'omomorfismo $$H \to D^n \quad (x_0, \dots, x_n) \to (x_1, \dots, x_n)$$ ha come inversa $(x_1, \dots, x_n ) \to ( \sqrt{1-x_1^n - \dots - x_n^2}, x_1, \dots, x_n)$, dunque $\frac{H}{\sim_H} \cong\frac{D}{\sim_D}$\\
\endproof
\end{prop}
\begin{oss}$\p(\R)\cong S^1$ in quanto 
$$\p(\R) = \frac{D^1}{\sim} = \frac{[-1,1]}{\{ -1,1 \}}=S^1$$
\end{oss}
\begin{oss} $\p^2(\R)$ si ottiene attaccando un disco ad un nastro di Möbius lungo il suo bordo (per entrambi \`e $S^1$).\\
In generale l'inclusione $\R^n \sbarra \{ 0 \} \inclu \R^{n+1}\sbarra \{0 \}$ induce un inclusione $\p^{n-1}(\R) \inclu \p^n (\R)$ tale che $\p^n (\R) \sbarra \p^{n-1}(\R) \cong D^n$
\end{oss}
\begin{defn}[Coordinate omogenee]\bianco
 Un punto di $\p^n (\R)$ \`e descritto da una $(n+1)$-upla a meno di multipli.\\
 La classe di $(x_0, \dots, x_n)$ si denota con $[x_0:\dots : x_n ]$.

\begin{oss}$[2:0:1]=[4:0:2]$ ed inoltre $[0:0:0]$ non esiste
\end{oss}
\end{defn}
\begin{fatti}\bbianco
\begin{itemize} \item Se $p\in \K[x_0, \dots, x_n]$\`e un polinomio non ha senso calcolare il polinomio sulle coordinate omogenee ovvero chiedersi quanto vale $p([x_0:\dots :x_n])$.
\item
Se $p$ \`e omogeneo di grado $\forall \lambda\in \K^\star $ 
$$ p( \lambda x_0, \dots, \lambda x_n) = \lambda^d p(x_0, \dots, x_n)$$
\`e ben definito il fatto che $p$ si annulli in $[x_0: \dots : x_n]$, cosa che avviene per definizione se $p(x_0, \dots, x_n)=0$ per un rappresentante di $[x_0:\dots : x_n]$\\
In particolare \`e ben definito $U_i=\{ x_i \neq 0 \} \subseteq \p^n(\R)$.
\item Se $p,q\in\K[x_0, \dots, x_n]$ sono omogenei e dello stesso grado \`e ben definita la funzione 
$$ \frac{p}{q}:\, V \to \K  \text{ dove } V =\{ x \in \p^n(\R)\, \vert \, q(x) \neq 0 \}$$ 
\end{itemize}
\end{fatti}

\end{document}