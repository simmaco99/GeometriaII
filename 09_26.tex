 \documentclass[a4paper,12pt]{article}
\usepackage[a4paper, top=2cm,bottom=2cm,right=2cm,left=2cm]{geometry}

\usepackage{bm,xcolor,mathdots,latexsym,amsfonts,amsthm,amsmath,
					mathrsfs,graphicx,cancel,tikz-cd,hyperref,booktabs,caption,amssymb,amssymb,wasysym}
\hypersetup{colorlinks=true,linkcolor=blue}
\usepackage[italian]{babel}
\usepackage[T1]{fontenc}
\usepackage[utf8]{inputenc}
\newcommand{\s}[1]{\left\{ #1 \right\}}
\newcommand{\sbarra}{\backslash} %% \ 
\newcommand{\ds}{\displaystyle} 
\newcommand{\alla}{^}  
\newcommand{\implica}{\Rightarrow}
\newcommand{\iimplica}{\Leftarrow}
\newcommand{\ses}{\Leftrightarrow} %se e solo se
\newcommand{\tc}{\quad \text{ t. c .} \quad } % tale che 
\newcommand{\spazio}{\vspace{0.5 cm}}
\newcommand{\bbianco}{\textcolor{white}{,}}
\newcommand{\bianco}{\textcolor{white}{,} \\}% per andare a capo dopo 																					definizioni teoremi ...


% campi 
\newcommand{\N}{\mathbb{N}} 
\newcommand{\R}{\mathbb{R}}
\newcommand{\Q}{\mathbb{Q}}
\newcommand{\Z}{\mathbb{Z}}
\newcommand{\K}{\mathbb{K}} 
\newcommand{\C}{\mathbb{C}}
\newcommand{\F}{\mathbb{F}}
\newcommand{\p}{\mathbb{P}}

%GEOMETRIA
\newcommand{\B}{\mathfrak{B}} %Base B
\newcommand{\D}{\mathfrak{D}}%Base D
\newcommand{\RR}{\mathfrak{R}}%Base R 
\newcommand{\Can}{\mathfrak{C}}%Base canonica
\newcommand{\Rif}{\mathfrak{R}}%Riferimento affine
\newcommand{\AB}{M_\D ^\B }% matrice applicazione rispetto alla base B e D 
\newcommand{\vett}{\overrightarrow}
\newcommand{\sd}{\sim_{SD}}%relazione sx dx
\newcommand{\nvett}{v_1, \, \dots , \, v_n} % v1 ... vn
\newcommand{\ncomb}{a_1 v_1 + \dots + a_n v_n} %a1 v1 + ... +an vn
\newcommand{\nrif}{P_1, \cdots , P_n} 
\newcommand{\bidu}{\left( V^\star \right)^\star}

\newcommand{\udis}{\amalg}
\newcommand{\ric}{\mathfrak{U}}
\newcommand{\inclu}{\hookrightarrow }
%ALGEBRA

\newcommand{\semidir}{\rtimes}%semidiretto
\newcommand{\W}{\Omega}
\newcommand{\norma}{\vert \vert }
\newcommand{\bignormal}{\left\vert \left\vert}
\newcommand{\bignormar}{\right\vert \right\vert}
\newcommand{\normale}{\triangleleft}
\newcommand{\nnorma}{\vert \vert \, \cdot \, \vert \vert}
\newcommand{\dt}{\, \mathrm{d}t}
\newcommand{\dz}{\, \mathrm{d}z}
\newcommand{\dx}{\, \mathrm{d}x}
\newcommand{\dy}{\, \mathrm{d}y}
\newcommand{\amma}{\gamma}
\newcommand{\inv}[1]{#1^{-1}}
\newcommand{\az}{\centerdot}
\newcommand{\ammasol}[1]{\tilde{\gamma}_{\tilde{#1}}}
\newcommand{\pror}[1]{\mathbb{P}^#1 (\R)}
\newcommand{\proc}[1]{\mathbb{P}^#1(\C)}
\newcommand{\sol}[2]{\widetilde{#1}_{\widetilde{#2}}}
\newcommand{\bsol}[3]{\left(\widetilde{#1}\right)_{\widetilde{#2}_{#3}}}
\newcommand{\norm}[1]{\left\vert\left\vert #1 \right\vert \right\vert}
\newcommand{\abs}[1]{\left\vert #1 \right\vert }
\newcommand{\ris}[2]{#1_{\vert #2}}
\newcommand{\vp}{\varphi}
\newcommand{\vt}{\vartheta}
\newcommand{\wt}[1]{\widetilde{#1}}
\newcommand{\pr}[2]{\frac{\partial \, #1}{\partial\, #2}}%derivata parziale
%per creare teoremi, dimostrazioni ... 
\theoremstyle{plain}
\newtheorem{thm}{Teorema}[section] 
\newtheorem{ese}[thm]{Esempio} 
\newtheorem{ex}[thm]{Esercizio} 
\newtheorem{fatti}[thm]{Fatti}
\newtheorem{fatto}[thm]{Fatto}

\newtheorem{cor}[thm]{Corollario} 
\newtheorem{lem}[thm]{Lemma} 
\newtheorem{al}[thm]{Algoritmo}
\newtheorem{prop}[thm]{Proposizione} 
\theoremstyle{definition} 
\newtheorem{defn}{Definizione}[section] 
\newcommand{\intt}[2]{int_{#1}^{#2}}
\theoremstyle{remark} 
\newtheorem{oss}{Osservazione} 
\newcommand{\di }{\, \mathrm{d}}
\newcommand{\tonde}[1]{\left( #1 \right)}
\newcommand{\quadre}[1]{\left[ #1 \right]}
\newcommand{\w}{\omega}

% diagrammi commutativi tikzcd
% per leggere la documentazione texdoc

\begin{document}
\textbf{Lezioni del 26  Settembre del prof. Frigerio}

\begin{defn}[Topologicamente equivalenti]\bianco
Due distanze $d$ e $d'$ sullo stesso insieme $X$ sono topologicamente equivalenti se inducono la stessa famiglia di aperti
\end{defn}

\begin{lem} Siano $d$ e $d'$ distanze sullo stesso insieme $X$ tali che $\exists k\geq 1$ per cui
$$ \frac{1}{k} \leq \frac{d'(x,y)}{d(x,y)} \leq k \quad \forall x,y \in X$$ 
allora $d$ e $d'$ sono topologicamente equivalenti
\proof $\forall x_0 \in X$ $$B_d(x_0,R) \subseteq B_{d'}(x_0, kR)$$ 
in quanto $ d(x_0, y)< R \quad \implica \quad d'(x_0, y) \leq k d(x_0,y) < k R$.\\
Sia $A$ un aperto di $X$ rispetto a $d'$, se $x_0 \in A $ allora per definizione
$$ \exists \varepsilon>0 \quad B_{d'}(x_0, \varepsilon)\subseteq A$$
allora 
$$ B_d \left( x_0, \frac{\varepsilon}{k}\right) \subseteq B_{d'}(x_0, \varepsilon)\subseteq A $$
dunque $A$ \`e aperto rispetto a $d$ (per arbitariet\`a di $x_0$)\\
La tesi segue dal fatto che l'ipotesi \`e simmetrica in $d$ e $d'$
\end{lem}

\begin{cor}
 Le distanze $d_1, d_E, d_\infty $ sono topologicamente equivalenti
 \proof Per la disuguaglianza tra media aritmetica e quadratica 
 $$ \frac{1}{n} \sum_{i=1}^n \vert x_i - y_i \vert \leq \sqrt{ \frac{\sum_{i=1}^n (x_i-y_i)^2}{n}}
 \quad \implica \quad \sum_{i=1}^n \vert x_i - y_i \vert \leq \sqrt{n} \sqrt{\sum_{i=1}^n (x_i-y_i)^2} \quad \implica \quad d_1 \leq \sqrt{n} d_E$$
$$ \sqrt{\sum_{i=1}^n (x_i-y_i)^2 } \leq \sqrt{\sum_{i=1}^n \left( \max_j \{ \vert x_j-y_j\vert ^2 \} \right) } =\sqrt{\sum_{i=1}^n d_\infty (x,y)^2}= \sqrt{n}d_\infty(x,y) \quad \implica $$
$$\implica \quad d_2 \leq \sqrt{n} d_\infty(x,y)$$
Infine ovviamente $d_\infty \leq d_1$ (negli addendi che si sommando per ottenere $d_1$ \`e presente $d_\infty$ e gli altri addendi sono non negativi)

\end{cor}
\newpage	
\section{Spazi topologici}
\begin{defn}[Spazio topologico]\bianco
Uno spazio topologico \`e una coppia $(X,\tau)$ dove $
\tau \subseteq \mathcal{P}(X)$ soddisfa:
\begin{enumerate}
	\item $\emptyset \in \tau $ e $X\in \tau$
	\item Se $A_1,A_2 \in \tau$ allora $A_1\cap A_2\in \tau$
	\item Se $I$ \`e un qualsiasi insieme e $A_i\in \tau\, \forall i\in I $ allora $\ds \cup_{i\in I} A_i \in \tau $
\end{enumerate}	
$\tau$ si chiama topologia di $X$ e gli elementi di $\tau$ si chiamano aperti di $\tau$

\begin{oss}
Con una facile induzione si prova che un intersezione finita di aperti \`e un aperto, mentre l'unione pu\`o essere fatta su una famiglia di aperti anche infinita	
\end{oss}
\end{defn}
\begin{defn}[Chiuso]\bianco
Sia $(X,\tau)$ uno spazio topologico, $C\subseteq X$ \`e chiuso se $X \sbarra C$ \`e aperto 	
\end{defn}
\begin{fatti}
	\begin{itemize}
		\item $\exists A \subseteq X $ n\`e aperti n\`e chiusi (es. $\mathbb{Q}\subseteq \R $ con la topologia euclidea)
		\item $X=X\sbarra \emptyset $ e $ \emptyset=X\sbarra X$ sono sempre aperti e chiusi
		\item Un'unione finita di chiusi \`e chiusa
		\item Un'intersezione arbitraria di chiusi \`e chiusa
	\end{itemize}
\end{fatti}
\spazio
\begin{prop}Se $(X,d)$ \`e uno spazio metrico, gli aperti rispetto a $d$  definiscono una topologia
\proof Sia $\tau$ la famiglia di aperti rispetto a $d$
\begin{enumerate}
	\item $\emptyset \in \tau$ per motivi di logica, $X\in \tau$ per definizione di palla $\forall x_0\in X \, B(x_0,1)\subseteq X $
	\item $A_1,A_2 \in \tau$.\\
		  Sia $x_0\in A_1\cap A_2 $ dunque per definizione di aperto rispetto a $d$
		  $$ \exists \delta_1 >0 \quad B(x_0, \delta_1) \subseteq A_1 $$
		  $$ \exists \delta_2 >0 \quad B(x_0, \delta_2) \subseteq A_2 $$
		  $$ \exists \delta=\min(\delta_1, \delta_2) \quad B(x_0,\delta)\subseteq B(x_0,\delta_1)\cap B(x_0,\delta_2)\subseteq A_1\cap A_2$$
		  Ovvero $A_1\cap A_2 \in \tau$
	\item Se $A_i \in \tau \, \forall i \in I $ e $ \ds x_0\in \cup_{i\in I} A_i$ allora 
	$$ \exists i_0 \in I \tc x_0 \in A_{i_0} \quad \implica \quad \exists \delta>0 \quad B(x_0, \delta )\subseteq A_{i_0}$$
	$$B(x_0,\delta) \subseteq \cup_{i\in I} A_i \quad \cup_{i\in I} A_i \text{\`e aperto}$$ 
\end{enumerate}
\endproof
\begin{oss}
Ogni spazio metrico \`e "naturalmente" uno spazio topologico	
\end{oss}
\end{prop}
\begin{defn}
Uno spazio topologico $(X,\tau)$ \`e metrizzabile se $\tau$ \`e indotto da una distanza su $X$	
\end{defn}

\spazio
Alcuni esempi di topologia
\begin{enumerate}
	\item Topologia metrizzabile
	\item Topologia discreta.\\
	 $\tau_D=\mathcal{P}(X)$ o equivalentemente ogni punto di $X$ (singoletto) \`e un intorno.\\
	 Tale topologia \`e metrizzabile essendo indotta dalla distanza discreta
	 \item Topologia indiscreta $\tau_I=\{ \emptyset , X \}$
	 \item Topologia confinata ($\tau_C$)$A\subseteq X $ \`e aperto se $X\sbarra A $ \`e finito o $X$
\end{enumerate}
\begin{ex}
La topologia indiscreta se $\vert X \vert  >1 $ non \`e metrizzabile	
\end{ex}
\begin{prop}
La topologia confinata \`e una topologia
\proof \bbianco
\begin{enumerate}
	\item $X\sbarra X=\emptyset$ ma $\emptyset$ \`e finito quindi $X$ \`e aperto\\
		 $ X\sbarra \emptyset = X $ quindi $\emptyset$ \`e aperto
	\item $A_1, A_2 \in \tau $ \\
	Se almeno uno dei 2 \`e vuoto $A_1\cap A_2 =\emptyset\in \tau$.\\
	Altrimenti $X\sbarra A_1$ e $X\sbarra A_2$ sono finiti 
	$$ X\sbarra (A_1\cap A_2 ) = \left( X\sbarra A_1 \right) \cup \left( X\sbarra A_2 \right) \text{ che \`e finito } \quad \implica \quad A_1\cap A_2 \in \tau$$
	\item Se $A_1\in \tau \, \forall i\in I $ si possono verificare 2 casi:\begin{itemize}
		\item $A_i=\emptyset \, \forall i\in I \quad \implica \ds \cup_{i\in I} A_i=\emptyset\in \tau$
		\item  $\exists A_{i_0}$ tale che $X\sbarra A_{i_0}$ \`e finito 
		$$ X \sbarra \left( \cup_{i\in I} A_i  \right)\subseteq X\sbarra A_{i_0} $$
		Ora $X\sbarra A_{i_0}$ \`e finito quindi l'unione \`e un aperto di $\tau$
 	\end{itemize}
\end{enumerate}	
\end{prop}
\newpage
\begin{defn}[Funzione continua]\bianco
$$ f:\, (X,\tau)\to (Y, \tau') \text{ \`e continua se } f^{-1}(A)\in \tau \quad \forall A\in \tau'$$	
\end{defn}
\begin{thm}\bbianco
\begin{enumerate}
	\item $Id:\, (X,\tau)\to (X,\tau)$ \`e continua
	\item Se $f:\, X\to Y $ e $g:\, Y \to Z $ sono continue allora $ g\circ f :\, X \to Z $ \`e continua
\end{enumerate}
\proof\bbianco
\begin{enumerate}
	\item Ovvia
	\item Sia $A$ un aperto di $Z$.
	$$ \text{ per continuit\'a di } g \quad g^{-1}(A) \text{ \`e aperto di } Y $$
	$$ \text{ per continuit\'a di } f \quad f^{-1}\left( g^{-1}(A)  \right) \text{ \`e aperto di } Y $$
	dunque $(f\circ g)^{-1}(A)$ \`e aperto di $X$
\end{enumerate}
\endproof
\end{thm}
\end{document}