\documentclass[a4paper,12pt]{article}
\usepackage[a4paper, top=2cm,bottom=2cm,right=2cm,left=2cm]{geometry}

\usepackage{bm,xcolor,mathdots,latexsym,amsfonts,amsthm,amsmath,
					mathrsfs,graphicx,cancel,tikz-cd,hyperref,booktabs,caption,amssymb,amssymb,wasysym}
\hypersetup{colorlinks=true,linkcolor=blue}
\usepackage[italian]{babel}
\usepackage[T1]{fontenc}
\usepackage[utf8]{inputenc}
\newcommand{\s}[1]{\left\{ #1 \right\}}
\newcommand{\sbarra}{\backslash} %% \ 
\newcommand{\ds}{\displaystyle} 
\newcommand{\alla}{^}  
\newcommand{\implica}{\Rightarrow}
\newcommand{\iimplica}{\Leftarrow}
\newcommand{\ses}{\Leftrightarrow} %se e solo se
\newcommand{\tc}{\quad \text{ t. c .} \quad } % tale che 
\newcommand{\spazio}{\vspace{0.5 cm}}
\newcommand{\bbianco}{\textcolor{white}{,}}
\newcommand{\bianco}{\textcolor{white}{,} \\}% per andare a capo dopo 																					definizioni teoremi ...


% campi 
\newcommand{\N}{\mathbb{N}} 
\newcommand{\R}{\mathbb{R}}
\newcommand{\Q}{\mathbb{Q}}
\newcommand{\Z}{\mathbb{Z}}
\newcommand{\K}{\mathbb{K}} 
\newcommand{\C}{\mathbb{C}}
\newcommand{\F}{\mathbb{F}}
\newcommand{\p}{\mathbb{P}}

%GEOMETRIA
\newcommand{\B}{\mathfrak{B}} %Base B
\newcommand{\D}{\mathfrak{D}}%Base D
\newcommand{\RR}{\mathfrak{R}}%Base R 
\newcommand{\Can}{\mathfrak{C}}%Base canonica
\newcommand{\Rif}{\mathfrak{R}}%Riferimento affine
\newcommand{\AB}{M_\D ^\B }% matrice applicazione rispetto alla base B e D 
\newcommand{\vett}{\overrightarrow}
\newcommand{\sd}{\sim_{SD}}%relazione sx dx
\newcommand{\nvett}{v_1, \, \dots , \, v_n} % v1 ... vn
\newcommand{\ncomb}{a_1 v_1 + \dots + a_n v_n} %a1 v1 + ... +an vn
\newcommand{\nrif}{P_1, \cdots , P_n} 
\newcommand{\bidu}{\left( V^\star \right)^\star}

\newcommand{\udis}{\amalg}
\newcommand{\ric}{\mathfrak{U}}
\newcommand{\inclu}{\hookrightarrow }
%ALGEBRA

\newcommand{\semidir}{\rtimes}%semidiretto
\newcommand{\W}{\Omega}
\newcommand{\norma}{\vert \vert }
\newcommand{\bignormal}{\left\vert \left\vert}
\newcommand{\bignormar}{\right\vert \right\vert}
\newcommand{\normale}{\triangleleft}
\newcommand{\nnorma}{\vert \vert \, \cdot \, \vert \vert}
\newcommand{\dt}{\, \mathrm{d}t}
\newcommand{\dz}{\, \mathrm{d}z}
\newcommand{\dx}{\, \mathrm{d}x}
\newcommand{\dy}{\, \mathrm{d}y}
\newcommand{\amma}{\gamma}
\newcommand{\inv}[1]{#1^{-1}}
\newcommand{\az}{\centerdot}
\newcommand{\ammasol}[1]{\tilde{\gamma}_{\tilde{#1}}}
\newcommand{\pror}[1]{\mathbb{P}^#1 (\R)}
\newcommand{\proc}[1]{\mathbb{P}^#1(\C)}
\newcommand{\sol}[2]{\widetilde{#1}_{\widetilde{#2}}}
\newcommand{\bsol}[3]{\left(\widetilde{#1}\right)_{\widetilde{#2}_{#3}}}
\newcommand{\norm}[1]{\left\vert\left\vert #1 \right\vert \right\vert}
\newcommand{\abs}[1]{\left\vert #1 \right\vert }
\newcommand{\ris}[2]{#1_{\vert #2}}
\newcommand{\vp}{\varphi}
\newcommand{\vt}{\vartheta}
\newcommand{\wt}[1]{\widetilde{#1}}
\newcommand{\pr}[2]{\frac{\partial \, #1}{\partial\, #2}}%derivata parziale
%per creare teoremi, dimostrazioni ... 
\theoremstyle{plain}
\newtheorem{thm}{Teorema}[section] 
\newtheorem{ese}[thm]{Esempio} 
\newtheorem{ex}[thm]{Esercizio} 
\newtheorem{fatti}[thm]{Fatti}
\newtheorem{fatto}[thm]{Fatto}

\newtheorem{cor}[thm]{Corollario} 
\newtheorem{lem}[thm]{Lemma} 
\newtheorem{al}[thm]{Algoritmo}
\newtheorem{prop}[thm]{Proposizione} 
\theoremstyle{definition} 
\newtheorem{defn}{Definizione}[section] 
\newcommand{\intt}[2]{int_{#1}^{#2}}
\theoremstyle{remark} 
\newtheorem{oss}{Osservazione} 
\newcommand{\di }{\, \mathrm{d}}
\newcommand{\tonde}[1]{\left( #1 \right)}
\newcommand{\quadre}[1]{\left[ #1 \right]}
\newcommand{\w}{\omega}

% diagrammi commutativi tikzcd
% per leggere la documentazione texdoc

\begin{document}

\textbf{Lezione del 2 ottobre del prof. Frigerio}
\begin{defn}[Omeomorfismo]\bianco
  Una funzione $f:\, X \to Y$ tra spazi topologici \`e un omeomorfismo se \`e continua e se esiste $g:\, Y \to X $ continua tale che $f\circ g =Id_y$, $ g\circ f=Id_X$.\\
  In modo equivalente: $f$ \`e continua, bigettiva e $f^{-1}$ \`e continua
 \end{defn}
 \spazio
 \begin{defn}
  Due spazi legati da un omeomorfismo si dicono omeomorfi
 \end{defn}

\begin{oss} \bbianco
\begin{enumerate}
 \item Composizione di omeomorfismi \`e un omeomorfismo
 \item Essere omeomorfi \`e una relazione di equivalenza
 \item L'insime degli omeomorfismi da $(X,\tau)$ in se \`e un gruppo con la composizione
 \end{enumerate}

\end{oss}
\begin{oss}
Se $f:\, X \to Y$ \`e continua e bigettiva, non \`e detto che sia un omeomorfismo ovvero $f^{-1}$ pu\`o non essere continua.\\
 Prendiamo come esempio $X=Y=\R$  allora le seguenti mappe sono continue 
  \begin{itemize}
   \item $ Id:\, ( \R, \tau_D ) \to (\R,\tau_E)$ \\
   $\forall A \in \tau_E \quad Id^{-1}(A)=A \in \tau_c$ infatti ogni sottoinsieme \`e un aperto nella topologia discreta
   \item $Id:\, ( \R, \tau_E ) \to (\R,\tau_C)$ \\
   $\forall A \in \tau_C $  allora  $A=\R\sbarra \{ \text{ insieme finito} \}$ che \`e un aperto nella topologia euclidea
   \item $Id:\, ( \R, \tau_C ) \to (\R,\tau_I) $ \\
   $\forall A \in \tau_I$ allora $A=\emptyset$ oppure $A=\R$ ed in entrambi i casi $A\in \tau_C$
  \end{itemize}
Nessuna delle seguenti mappe \'e continua
\begin{itemize}
   \item $ Id:\, ( \R, \tau_E ) \to (\R,\tau_D)$ \\
   $\{ 0\} $ \` un aperto in $\tau_D$ ma $Id^{-1}(\{ 0\})$ non \'e un aperto della topologia euclidea
   \item $Id:\, ( \R, \tau_C ) \to (\R,\tau_E)$ \\
   $B(0,1)$ \`e un aperto nella topologia euclidea ma non in quella cofinita ($\R\sbarra B(0,1)$ \`e infinito)
   \item $Id:\, ( \R, \tau_I) \to (\R,\tau_C) $ \\
   $X\sbarra\{0\}$ \`e un aperto nella topologia cofinita ma non \`e un aperto in quella indiscreta
  \end{itemize}
\end{oss}
\newpage
\subsection{Finitezza}
\begin{defn}
 Date $\tau$ e $\tau'$ topologie su un insieme $X$ si dice che $\tau$ \`e \textbf{meno fine} di $\tau'$ se $\tau\subseteq \tau'$ (ogni aperto di $\tau$ \`e un aperto di $\tau'$).\\
 In modo equivalente 
 $$ \tau \text{\`e meno fine di } \tau' \quad \ses \quad Id:\, (X,\tau') \to (X,\tau) \text{ \`e continua}$$
 In questo caso scriveremo $\tau < \tau'$
\end{defn}
\begin{oss}
Essere meno fini \`e una relazione di ordine parziale.\\
In generale $\tau_D$ \`e la pi\`u fine, mentre $\tau_I$ \'e la meno fine.\\
Su $\R$ vale $ \tau_I < \tau_C < \tau_E < \tau_D $
\end{oss}
\spazio
\begin{lem}
 Un intersezione arbitaria di topologie su $X$ \`e una topologia su $X$ 
 \proof Sia $\tau_I $ con $i \in I $ topologie su $X$ e sia $\ds=\bigcap_{i\in I} \tau_i $.\\
 Mostriamo che $\tau$\`e una topologia
 \begin{itemize}
  \item Poich\`e $\emptyset, X \in \tau_i \, \forall i $ allora $\emptyset, X \in \tau $
  \item Se $A,B\in\tau $ allora  $A, B \in \tau_i \, \forall i $ ed essendo $\tau_i$ una topologia $A\bigcap B \in \tau_i \, \forall i $ quindi $A\bigcap B\in \tau$
  \item Se $A_j \in \tau  \, \forall j \in J $ allora $A_j \in \tau_i \, \forall i \, \forall j $ ed essendo $\tau_i$ una topologia $\ds \left( \bigcup_{j\in J} A_j\right) \in \tau_i \, \forall i $ quindi  $\ds \left( \bigcup_{j\in J} A_j\right) \in \tau$
 \end{itemize}
\endproof
\end{lem}

\begin{cor}Data una famiglia $\ds \{ \tau_i\}_{i \in I } $ di topologie su $X$ esiste la pi\`u fine tra le topologie meno fini di ogni $\tau_i$
$$ \tau=\bigcap_{i \in I } \tau_i $$ 
\proof  Poic\`e $\tau$ deve essere la meno fine ovviamente $$\tau \subseteq \tau_i \, \forall i \quad \implica \quad \tau\subseteq \bigcap_{i \in I} \tau_i $$ 
Ogni altra topologia meno fine di tutte le $\tau_i$ deve essere contenuta nell'intersezione, dunque, volendo la pi\`u fine (pi\`u grande rispetto l'inclusione) deve essere proprio l'intersezione la topologia voluta infatti essa \`e una topologia per il lemma precedente
\endproof
  \end{cor}
\begin{cor}
 Sia $X$ un insieme, $S \subseteq \mathcal{P}(X)$.\\
 Allora esiste la topologia meno fine tra quelle che contengono $S$
 \proof Sia $\Omega$ l'insieme delle topologie che contengono $S$.\\
 $\Omega \neq \emptyset $ infatti la topologia discreta vi appartiene dunque esiste $\ds \bigcap_{\tau\in \Omega} \tau $\\
 L'intersezione \`e una topologia per il lemma precedente ed ovviamente \`e la meno fine possibile
\end{cor}

\begin{defn}
 Sia $X$ un insieme e $S\subseteq\mathcal{P}(X)$.\\
 La topologia meno fine tra quelle che contengono $S$ si dice generate da $S$ e $S$ viene chiamata \textbf{prebase} 
\end{defn}
\spazio

\begin{defn}[Base di una topologia]\bianco
Sia $(X,\tau)$ uno spazio topologico.\\\
Una base  di $\tau$ \`e un sottoinsieme $\B\subseteq \tau$  tale che 
$$ \forall A \in A \quad \exists B_i \in \B, \, \,  i \in I \quad A=\bigcup_{i \in I } B_i $$
\end{defn}
\spazio
\begin{defn}
 $X$ si dice a base numerabile oppure che $X$ soddisfa il secondo assioma di numerabilit\`a se ammette una base numerabile 
\end{defn}
\spazio

\begin{prop}[Criterio per una base]\bianco
 Sia $X$ un insieme (senza topologia).
 $$ \B \subseteq \mathcal{P}(X) \text{ \'e una base di una topologia su } X  \ses  \begin{cases} 
                                                                                              (i) X =\bigcup_{B \in \B} B \\ 
                                                                                              (ii) \forall A, A' \in \B \quad \exists B_i \in \B , \, \, i \in I \quad A\cap A' =\bigcup_{i\in I } B_i 
                                                                                             \end{cases}$$
\proof $\implica$ Discende direttamente dagli assiomi di topologia infatti supponendo che $\B$ sia una base di $\tau$ topologia: 
\begin{itemize}
 \item[(i)] $X\in \tau $ quindi si esprime come unione di $B \in \B$ 
 \item[(ii)] Se $A,A'\in \B $ allora essi sono aperti di $\tau$, anche $A\cap A'$ \`e un aperto della topologia e quindi anche $A\cap A'$ si esprime come unione di $B\in\B$
\end{itemize}
$\Leftarrow$ Definiamo $\tau$ nell'unico modo possibile
$$ A\in \tau \quad \ses \quad A = \bigcup_{i \in I} B_i \text{ per qualche } B_i \in \B , \, \, i\in I $$ 
Verifichiamo che $\tau$ \`e una topologia
\begin{itemize}
 \item $\emptyset \in \tau$ perch\`e $\tau$ contiene l'unione nulla\\
 $ X\in \tau $ per la propiet\`a (i)
 \item Se $A,A' \in \tau $ allora per definizone 
 $$ A= \bigcup_{i \in I } B_i \quad A'=\bigcup_{j \in J} B_j  \quad \implica \quad A\cap A'= \left( \bigcup_{i\in I } B_i \right)  \cap \left( \bigcup_{j\in J } B_j \right)= \bigcup_{i\in I \atop{ j \in J }} (B_i \cap B_j )$$
 Ciascun $B_i \cap B_j$ \`e unione di elementi di $\B$ per (ii) dunque $A\cap A'$ \`e unione di elementi di $\B$ come voluto
 \item Se $A_i\in \tau \, \, \forall i \in I $ allora $A_i $ si scrive come unione di elementi di $\B$ dunque $\ds \bigcap_{i\in I} A_i$ \`e unione di unione di elementi di $\B$
 \end{itemize}

\endproof

\end{prop}
\spazio
\begin{prop}Siano $X$ un  insieme e $S\subseteq \mathcal{P}(X)$ prebase di $\tau$, allora
\begin{enumerate}
 \item Le intersezione finite di elementi di $S \cup \{ X\} $ sono una base di $\tau$
 \item $A\in \tau $ $ \ses $ $ A$ \`e unione arbitraria di intersezioni finite di elementi di $S \cup \{ X \} $
\end{enumerate}
\proof Mostriamo 1.
\begin{itemize}
\item Verifichiamo innanzitutto che $\B=\{ \text{ intersezione finita di elementi di } S\cup \{ X \} \}$ \`e una base di qualche topologia utilizzando il criterio precedente
\begin{itemize}
 \item $X$ \`e banalmente unione di elementi di $\B$ 
 \item Se $B_1,B_2\in \B$ allora sia $B_1$ che $B_2$ sono intersezione finita di elementi di $S \cup \{ X\} $ dunque anche $B_1\cap B_2$ lo \`e 
\end{itemize}
Poich\`e sono verificate entrambe le propiet\`a $\B$ \`e base di una topologia $\tau'$
\item Mostriamo che $\tau=\tau'$\\
Per costruzione $\tau'$ contiene $S$ dunque essendo $\tau$ la meno fine topologia che contiene $S$ $ \tau< \tau' $.\\
D'altronte una qualsiasi topologia che contiene $S$ deve contenere $\tau'$ (intersezione finita e unione arbitraria di elementi di $S$) quindi $\tau'< \tau $ 
\end{itemize}
2. segue da 1. per definizione di base
\endproof
\end{prop}
\end{document}