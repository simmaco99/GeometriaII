\documentclass[a4paper,12pt]{article}
\usepackage[a4paper, top=2cm,bottom=2cm,right=2cm,left=2cm]{geometry}

\usepackage{bm,xcolor,mathdots,latexsym,amsfonts,amsthm,amsmath,
					mathrsfs,graphicx,cancel,tikz-cd,hyperref,booktabs,caption,amssymb,amssymb,wasysym}
\hypersetup{colorlinks=true,linkcolor=blue}
\usepackage[italian]{babel}
\usepackage[T1]{fontenc}
\usepackage[utf8]{inputenc}
\newcommand{\s}[1]{\left\{ #1 \right\}}
\newcommand{\sbarra}{\backslash} %% \ 
\newcommand{\ds}{\displaystyle} 
\newcommand{\alla}{^}  
\newcommand{\implica}{\Rightarrow}
\newcommand{\iimplica}{\Leftarrow}
\newcommand{\ses}{\Leftrightarrow} %se e solo se
\newcommand{\tc}{\quad \text{ t. c .} \quad } % tale che 
\newcommand{\spazio}{\vspace{0.5 cm}}
\newcommand{\bbianco}{\textcolor{white}{,}}
\newcommand{\bianco}{\textcolor{white}{,} \\}% per andare a capo dopo 																					definizioni teoremi ...


% campi 
\newcommand{\N}{\mathbb{N}} 
\newcommand{\R}{\mathbb{R}}
\newcommand{\Q}{\mathbb{Q}}
\newcommand{\Z}{\mathbb{Z}}
\newcommand{\K}{\mathbb{K}} 
\newcommand{\C}{\mathbb{C}}
\newcommand{\F}{\mathbb{F}}
\newcommand{\p}{\mathbb{P}}

%GEOMETRIA
\newcommand{\B}{\mathfrak{B}} %Base B
\newcommand{\D}{\mathfrak{D}}%Base D
\newcommand{\RR}{\mathfrak{R}}%Base R 
\newcommand{\Can}{\mathfrak{C}}%Base canonica
\newcommand{\Rif}{\mathfrak{R}}%Riferimento affine
\newcommand{\AB}{M_\D ^\B }% matrice applicazione rispetto alla base B e D 
\newcommand{\vett}{\overrightarrow}
\newcommand{\sd}{\sim_{SD}}%relazione sx dx
\newcommand{\nvett}{v_1, \, \dots , \, v_n} % v1 ... vn
\newcommand{\ncomb}{a_1 v_1 + \dots + a_n v_n} %a1 v1 + ... +an vn
\newcommand{\nrif}{P_1, \cdots , P_n} 
\newcommand{\bidu}{\left( V^\star \right)^\star}

\newcommand{\udis}{\amalg}
\newcommand{\ric}{\mathfrak{U}}
\newcommand{\inclu}{\hookrightarrow }
%ALGEBRA

\newcommand{\semidir}{\rtimes}%semidiretto
\newcommand{\W}{\Omega}
\newcommand{\norma}{\vert \vert }
\newcommand{\bignormal}{\left\vert \left\vert}
\newcommand{\bignormar}{\right\vert \right\vert}
\newcommand{\normale}{\triangleleft}
\newcommand{\nnorma}{\vert \vert \, \cdot \, \vert \vert}
\newcommand{\dt}{\, \mathrm{d}t}
\newcommand{\dz}{\, \mathrm{d}z}
\newcommand{\dx}{\, \mathrm{d}x}
\newcommand{\dy}{\, \mathrm{d}y}
\newcommand{\amma}{\gamma}
\newcommand{\inv}[1]{#1^{-1}}
\newcommand{\az}{\centerdot}
\newcommand{\ammasol}[1]{\tilde{\gamma}_{\tilde{#1}}}
\newcommand{\pror}[1]{\mathbb{P}^#1 (\R)}
\newcommand{\proc}[1]{\mathbb{P}^#1(\C)}
\newcommand{\sol}[2]{\widetilde{#1}_{\widetilde{#2}}}
\newcommand{\bsol}[3]{\left(\widetilde{#1}\right)_{\widetilde{#2}_{#3}}}
\newcommand{\norm}[1]{\left\vert\left\vert #1 \right\vert \right\vert}
\newcommand{\abs}[1]{\left\vert #1 \right\vert }
\newcommand{\ris}[2]{#1_{\vert #2}}
\newcommand{\vp}{\varphi}
\newcommand{\vt}{\vartheta}
\newcommand{\wt}[1]{\widetilde{#1}}
\newcommand{\pr}[2]{\frac{\partial \, #1}{\partial\, #2}}%derivata parziale
%per creare teoremi, dimostrazioni ... 
\theoremstyle{plain}
\newtheorem{thm}{Teorema}[section] 
\newtheorem{ese}[thm]{Esempio} 
\newtheorem{ex}[thm]{Esercizio} 
\newtheorem{fatti}[thm]{Fatti}
\newtheorem{fatto}[thm]{Fatto}

\newtheorem{cor}[thm]{Corollario} 
\newtheorem{lem}[thm]{Lemma} 
\newtheorem{al}[thm]{Algoritmo}
\newtheorem{prop}[thm]{Proposizione} 
\theoremstyle{definition} 
\newtheorem{defn}{Definizione}[section] 
\newcommand{\intt}[2]{int_{#1}^{#2}}
\theoremstyle{remark} 
\newtheorem{oss}{Osservazione} 
\newcommand{\di }{\, \mathrm{d}}
\newcommand{\tonde}[1]{\left( #1 \right)}
\newcommand{\quadre}[1]{\left[ #1 \right]}
\newcommand{\w}{\omega}

% diagrammi commutativi tikzcd
% per leggere la documentazione texdoc

\begin{document}
\textbf{Lezione del 30 aprile}
\begin{defn}Sia $\dim V = n+1$ e sia $\s{e_0, \dots, e_n}$ una base di $V$.\\
Diremo che $\s{e_0, \dots,e_n}$ \`e un \textbf{riferimento proiettivo}
\end{defn}
Fissato un riferimento proiettivo $\s{e_0, \dots, e_n}$ sia $v\in V \setminus\s 0$ allora 
$$ P = [v] = \s{\lambda v \, \vert \, \lambda\in \K} \in \p(V)$$
Si ha che $v=x_0e_0+\dots + x_n e_n$. Chiamiamo $x_0, \dots, x_n$ le \textbf{ coordinate omogenee } di $P$ rispetto al riferimento proiettivo $\s{e_0, \dots, e_n}$, per notazione $P=[x_0, \dots, x_n]$\\
Chiamiamo i \textbf{ punti fondamentali } rispetto al riferimento proiettivo fissato, i seguenti punti dello spazio proiettivo
$$ F_0=[e_0]\, \dots \, F_n = [e_n]$$
Il \textbf{ punto unit\`a }  rispetto al riferimento fissato 
$$ U = [ e_0+\dots + e_n]$$
\begin{oss}Le coordinate omogenee non sono uniche.\\
$\forall \mu \in \K, \, \mu\neq 0$ si ha $[v]=[\mu v]\in \p(R)$ dunque
$$ v=x_0e_0+\dots + x_n e_n \quad \implica \quad \mu v =(\mu x_0) e_0+\dots + (\mu x_n) e_n$$
dunque sia $x_0, \dots, x_n$ che $\mu x_0, \dots , \mu x_n$ sono coordinate omogenee del punto $[v]$.\\ \\
$\forall\mu \in \K^\star$ si ha $\s{\mu e_0, \dots , \mu w_n}$ \`e ancora una base di $V$, in  particolare, le $2$ basi danno luogo allo stesso sistema di coordinate omogenee
\end{oss}
\begin{defn}Nel caso in cui $V =\K^{n+1}$ chiamiamo \textbf{ riferimento proiettivo standard } di $\p(V)$  quello dato dalla base canonica di $\K^{n+1}$.\\
Se $P=[x_0, \dots, x_n] \in \p(V)$ chiamiamo $x_0, \dots, x_n$ coordinate proiettive standard di $O$
\end{defn}
\begin{defn}Sia $W$ un sottospazio vettoriale di $V$, chiamiamo $\p(W)$ il sottospazio proiettivo associato a $W$.\\
Estendiamo la notazione di dimensione anche a $\p(W)$ ponendo 
$$ \dim \p(W) = \dim W -1 $$
Nel caso in cui
\begin{itemize}
\item Se $\dim \p(W) = 0$ allora $\p(W)$ lo denotiamo punto  proiettivo
\item Se $\dim \p(W) = 1$ allora $\p(W)$ lo denotiamo retta proiettiva
\item Se $\dim \p(W) =2$  allora $\p(W)$ lo denotiamo piano  proiettivo
\item Se $\dim \p(W) =1-\dim \p(V)$ allora $\p(W)$ lo denotiamo iperpiano proiettivo
\end{itemize}
\begin{oss}Il vuoto ha dimensione pari a $-1$
\end{oss}
\end{defn}
\spazio
\begin{oss}Sia $\s{e_0, \dots, e_n}$ una base di $V$ e siano $a_0,\dots, a_n \in \K$ con $(a_0, \dots, a_n) \neq (0, \dots, 0)$.\\
Consideriamo l'equazione omogenea
\begin{equation}
\label{eq_om}
a_0X_0+ \dots + a_n X_n =0
\end{equation}
Tale equazione definisce un sottospazio vettoriale $W$ di $V$ che \`e un iperpiano.\\
Notiamo che i punti $P=[v]\in\p(V)$ le cui coordinate omogenee soddisfano l'equazione~\ref{eq_om} sono esattamente quelle per cui $v\in W$ dunque l'equazione~\ref{eq_om} \`e l'equazione dell'iperpiano $\p(W)$
\end{oss}
\begin{defn}Nel caso in cui $V = \K^n$ per ogni $i=0, \dots, n$ definiamo l'i-esimo iperpiano coordinato $H_i$ di $p(V)$ l'iperpiano definito da $X_i=0$
\end{defn}
\spazio
\begin{ese}Consideriamo i seguenti punti in $\pror 2 $ 
$$ P =\quadre{\frac{1}{2},1,1} \quad Q = \quadre{1, \frac{1}{3}, \frac{4}{3}} \quad R =[2,-1,2]$$
esiste una retta proiettiva che  li contiene?\\
Osserviamo che le rette sono iperpiani in $\pror 2 $.\\
Prima di tutto determiniamo la retta passante per $P$ e $Q$, poi ci chiediamo se $R$ \`e contenuta in tale retta.\\
Siano $P=[v]$ e $Q=[w]$
$$P\neq Q \quad \ses \quad \not\exists \lambda\in \K^\star \, \, w =\lambda	 v\quad \ses \quad v,w \text{ sono linearmente indipendenti}$$
Sia $W=Span_\R(v,w)$ dunque $\p(W)$ \`e la retta che passa per $P,Q$.\\
Essendo i punti equivalenti a meno di scalari prendo $v=(1,2,2)$ e $w=(3,1,4)$.\\
Sia $[x_0, x_1, x_2]\in\pror 2(V)$
$$ [x_0, x_1, x_2]\in\p(W) \quad \ses \quad (x_0,x_1,x_2)\in W\quad \ses \quad v,w,(x_0,x_1,x_2)\text{ sono linearmente indipendenti} \quad \ses $$
$$\ses \quad \det \begin{pmatrix}
x_0 & 1 & 3 \\
x_1 &  2 & 1 \\
x_2 & 2 & 4 
\end{pmatrix}=0 \quad \ses \quad 6x_0+ 2x_1-5x_2=0$$
Osserviamo che se $R=[z]$ allora $z\in W$ dunque $R$ appartiene alla retta, i 3 punti sono allineati
\end{ese}
\begin{oss}Pi\`u in generale, data una matrice $A\in M(t,n+1,\K)$ possiamo definire un sistama lineare
\begin{equation}
\label{sis_lin}
A \begin{pmatrix}
x_0 \\ \vdots \\ x_n
\end{pmatrix} =0
\end{equation}
Tali equazioni sono le equazioni cartesiane nella base $\s{e_0, \dots, e_n}$ di un sottospazio vettoriale $W$ di $V$ e sono anche le equazioni cartesiane del sottospazio proiettivo $\p(W)$ nel riferimento proiettivo $\s{e_0, \dots, e_n}$.\\
Osserviamo inoltre che $\dim W = \dim V -rk(A) $ da cui $\dim \p W =\dim \p V -rk(A)$\\
Attenzione: un sottospazio proiettivo non ammette un unico sistema di equazioni cartesiane.\\
Ino
\end{oss}
\spazio
\begin{lem}Siano $\p(W_1)$ e $\p(W_2)$ sottospazi proiettivi di $\p(V)$.
$$ \p(W_1) \cap \p(W_2)=\p(W_1\cap W_2)$$
\proof Fissato un riferimento proiettivo siano 
$$ A_1 \begin{pmatrix}
x_0 \\ \vdots \\ x_n 
\end{pmatrix}=0 \text{ un sistema di equazioni di }W_1$$
$$ A_2 \begin{pmatrix}
x_0 \\ \vdots \\ x_n 
\end{pmatrix}=0 \text{ un sistema di equazioni di }W_2$$
Ora 
$$ P=[x_0, \dots, x_n] \in \p(W_1)\cap \p(W_2) \quad \ses \quad   
\tonde{\begin{array}{c}
A_1 \\
\hline A_2 
\end{array}}  \tonde{\begin{array}{c} x_0 \\ \vdots \\ x_n \\ \hline  x_0 \\ \vdots \\ x_n
\end{array}} =0 \quad \ses \quad  $$
$$ \quad \ses \quad v=x_0e_0 +\dots + x_n e^n \in W_1\cap W_2 \quad \ses \quad P\in \p(W_1\cap W_2)$$
\end{lem}
\spazio
\begin{ex}In $\proc 2$ consideriamo le rette
$$ r_1: ax_1 -x_2 +3i x_0=0$$
$$r_2: -iax_0 +x_1 =ix_2=0$$
$$ r_3 =3ix_2 + 5x_0+x_1 =0$$
Calcolare la loro intersezione al variare del parametro $a$\\
Occorre calcolare il rango della matrice $A$ e osservare che $\dim( r_1 \cap r_2 \cap r_3) = 2 -rk(A)$
$$A=\begin{pmatrix} 
3i & a & -1 \\
-ia & 1 & i \\
5 & 1 & 3i 
\end{pmatrix}$$
\end{ex}
\spazio
\begin{defn}Diciamo che due sottospazi $\p(W_1)$ e $\p(W_2)$ di $\p(V)$ sono
\begin{itemize}
\item incidenti se $\p(W_1) \cap \p(W_2) \neq \emptyset$
\item sghembi se $\p(W_1) \cap \p(W_2) = \emptyset$
\end{itemize}
\begin{oss}Similmente possiamo dare la seguente definizione:\\
Due sottospazi si dicono sghembi se $\dim(\p(W_1) \cap \p(W_2))=-1$ e incidenti se tale dimensione \`e non negativa
\end{oss}
\end{defn}
\end{document}