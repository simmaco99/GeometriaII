\documentclass[a4paper,12pt]{article}
\usepackage[a4paper, top=2cm,bottom=2cm,right=2cm,left=2cm]{geometry}

\usepackage{bm,xcolor,mathdots,latexsym,amsfonts,amsthm,amsmath,
					mathrsfs,graphicx,cancel,tikz-cd,hyperref,booktabs,caption,amssymb,amssymb,wasysym}
\hypersetup{colorlinks=true,linkcolor=blue}
\usepackage[italian]{babel}
\usepackage[T1]{fontenc}
\usepackage[utf8]{inputenc}
\newcommand{\s}[1]{\left\{ #1 \right\}}
\newcommand{\sbarra}{\backslash} %% \ 
\newcommand{\ds}{\displaystyle} 
\newcommand{\alla}{^}  
\newcommand{\implica}{\Rightarrow}
\newcommand{\iimplica}{\Leftarrow}
\newcommand{\ses}{\Leftrightarrow} %se e solo se
\newcommand{\tc}{\quad \text{ t. c .} \quad } % tale che 
\newcommand{\spazio}{\vspace{0.5 cm}}
\newcommand{\bbianco}{\textcolor{white}{,}}
\newcommand{\bianco}{\textcolor{white}{,} \\}% per andare a capo dopo 																					definizioni teoremi ...


% campi 
\newcommand{\N}{\mathbb{N}} 
\newcommand{\R}{\mathbb{R}}
\newcommand{\Q}{\mathbb{Q}}
\newcommand{\Z}{\mathbb{Z}}
\newcommand{\K}{\mathbb{K}} 
\newcommand{\C}{\mathbb{C}}
\newcommand{\F}{\mathbb{F}}
\newcommand{\p}{\mathbb{P}}

%GEOMETRIA
\newcommand{\B}{\mathfrak{B}} %Base B
\newcommand{\D}{\mathfrak{D}}%Base D
\newcommand{\RR}{\mathfrak{R}}%Base R 
\newcommand{\Can}{\mathfrak{C}}%Base canonica
\newcommand{\Rif}{\mathfrak{R}}%Riferimento affine
\newcommand{\AB}{M_\D ^\B }% matrice applicazione rispetto alla base B e D 
\newcommand{\vett}{\overrightarrow}
\newcommand{\sd}{\sim_{SD}}%relazione sx dx
\newcommand{\nvett}{v_1, \, \dots , \, v_n} % v1 ... vn
\newcommand{\ncomb}{a_1 v_1 + \dots + a_n v_n} %a1 v1 + ... +an vn
\newcommand{\nrif}{P_1, \cdots , P_n} 
\newcommand{\bidu}{\left( V^\star \right)^\star}

\newcommand{\udis}{\amalg}
\newcommand{\ric}{\mathfrak{U}}
\newcommand{\inclu}{\hookrightarrow }
%ALGEBRA

\newcommand{\semidir}{\rtimes}%semidiretto
\newcommand{\W}{\Omega}
\newcommand{\norma}{\vert \vert }
\newcommand{\bignormal}{\left\vert \left\vert}
\newcommand{\bignormar}{\right\vert \right\vert}
\newcommand{\normale}{\triangleleft}
\newcommand{\nnorma}{\vert \vert \, \cdot \, \vert \vert}
\newcommand{\dt}{\, \mathrm{d}t}
\newcommand{\dz}{\, \mathrm{d}z}
\newcommand{\dx}{\, \mathrm{d}x}
\newcommand{\dy}{\, \mathrm{d}y}
\newcommand{\amma}{\gamma}
\newcommand{\inv}[1]{#1^{-1}}
\newcommand{\az}{\centerdot}
\newcommand{\ammasol}[1]{\tilde{\gamma}_{\tilde{#1}}}
\newcommand{\pror}[1]{\mathbb{P}^#1 (\R)}
\newcommand{\proc}[1]{\mathbb{P}^#1(\C)}
\newcommand{\sol}[2]{\widetilde{#1}_{\widetilde{#2}}}
\newcommand{\bsol}[3]{\left(\widetilde{#1}\right)_{\widetilde{#2}_{#3}}}
\newcommand{\norm}[1]{\left\vert\left\vert #1 \right\vert \right\vert}
\newcommand{\abs}[1]{\left\vert #1 \right\vert }
\newcommand{\ris}[2]{#1_{\vert #2}}
\newcommand{\vp}{\varphi}
\newcommand{\vt}{\vartheta}
\newcommand{\wt}[1]{\widetilde{#1}}
\newcommand{\pr}[2]{\frac{\partial \, #1}{\partial\, #2}}%derivata parziale
%per creare teoremi, dimostrazioni ... 
\theoremstyle{plain}
\newtheorem{thm}{Teorema}[section] 
\newtheorem{ese}[thm]{Esempio} 
\newtheorem{ex}[thm]{Esercizio} 
\newtheorem{fatti}[thm]{Fatti}
\newtheorem{fatto}[thm]{Fatto}

\newtheorem{cor}[thm]{Corollario} 
\newtheorem{lem}[thm]{Lemma} 
\newtheorem{al}[thm]{Algoritmo}
\newtheorem{prop}[thm]{Proposizione} 
\theoremstyle{definition} 
\newtheorem{defn}{Definizione}[section] 
\newcommand{\intt}[2]{int_{#1}^{#2}}
\theoremstyle{remark} 
\newtheorem{oss}{Osservazione} 
\newcommand{\di }{\, \mathrm{d}}
\newcommand{\tonde}[1]{\left( #1 \right)}
\newcommand{\quadre}[1]{\left[ #1 \right]}
\newcommand{\w}{\omega}

% diagrammi commutativi tikzcd
% per leggere la documentazione texdoc


\begin{document}
\textbf{Lezione del 6 Novembre di Gandini}
\begin{thm}[di Weistrass]\bianco
Sia $f:\, X \to \R$ continua con $X$ spazio compatto. AllorA $f$ ammette massimo e minimo in $X$
\proof Poich\`e $f(X)$ \`e un compatto di $\R$ \`e chiuso e limitato dunque ammette massimo e minimo 
\end{thm}

\newpage
\section{Norme}
\begin{defn}Sia $V$ un $\R$-spazio vettoriale, una norma su $V$ \`e una funzione 
$$ \nnorma :\, V \to \R$$ con le seguenti propiet\`a 
\begin{itemize}
\item[(i)]$\norma v \norma \geq 0 $ $\forall v \in V$ inoltre $\norma v \norma =0 \, \ses \, v =0 $
\item[(ii)]$\norma \lambda v \norma = \vert \lambda \vert \cdot \norma v \norma $ $\forall \lambda\in \R$ e $ \forall v \in V $
\item[(iii)]$\norma v + w \norma \leq \norma v \norma + \norma w \norma $ $\forall v, w \in V $
\end{itemize}

\end{defn}
\begin{defn}Sia $(V, \nnorma)$ uno spazio normato, allora definiamo la distanza associata alla norma come 
$$ d(v,w) = \norma v - w\norma $$
\begin{oss}Non tutte le distanze sono indotte da una norma, prendiamo ad esempio la distanza discreta su $\R^n$ infatti non \`e omogenea
\end{oss}
\end{defn}
\begin{defn}Due norme si dicono topologicamente equivalenti se lo sono le distanze associate
\end{defn}
\spazio
\begin{lem}Sia $\vert \vert \, \cdot \, \vert \vert :\,\R^n \to \R^n$ una norma, allora tale funzione \`e continua nella topologia euclidea
\proof Sia $\{ e_1,\dots, e_n \}$ la base canonica di $\R^n$ e sia $M =\max_{i=1,\dots, n } \norma e_1 \norma $ allora
$$ \bignormal \sum_{i=1}^n x_i e_1\bignormar \leq \sum_{i=1}^n \vert x_1 \vert \cdot \norma e_1 \norma \leq M \sum_{i=1}^n \vert x_1 \vert $$
Ora essendo $d_1$ e $d_2$ topologicamente equivalenti basta dimostrare che $\nnorma$ \`e continua nella topologia indotta da $d_1$.\\
Sia $d$ la distanza indotta da $\nnorma$ e siano $x= \left( x_1 , \, \dots , \, x_n \right)$ e $y= \left( y_1 , \, \dots , \, y_n \right)$ allora 
$$ d(x-y) \leq \norma x  -   y \norma \leq \bignormal \sum_{i=1}^n ( x_i - y_1) e_1 \bignormar \leq M \sum_{i=1}^n \vert x_1 - y_1 \vert = M d_1(x,y)$$
Dunque $\nnorma$ \`e continua rispetto alla distanza $d_1$ dunque rispetto a $d_2$\\
\endproof
\end{lem}
\begin{thm}Sia $V$ un $\R$-spazio vettoriale di dimensione finita, allora tutte le norme su $V$ sono topologicamente equivalenti.
\proof Fissando un qualunque isomorfismo lineare $\phi:\, \R^n \to V$ posso ricondurmi al caso $V=\R^n$.\\
Sia $\nnorma$ una norma su $\R^n$, vediamo che tale norma \`e equivalente a $\nnorma_2$.\\
Per il lemma precedente $\nnorma:\, \R^n \to \R^n$ \`e continua rispetto a $d_2$.\\
Sia $S^n \subseteq \R^n$ allora essendo chiuso e limitato \`e compatto da cui per il Teorema di Weistrass: $\nnorma $ ammette massimo $M$ e minimo $m$ su $S^n$.\\
Sia $v \in \R^n$ allora $$ \norma v \norma  = \bignormal \norma v \norma_2 \cdot \frac{v}{\norma v \norma_2} \bignormar = \norma v \norma_2 \cdot \bignormal \frac{v}{\norma v \norma_2} \bignormar \quad \implica \quad m \norma v \norma_2 \leq \norma v \norma \leq M \norma v \norma_2$$
dove l'implicazione deriva dal fatto che $\frac{\norma v \norma}{\norma v \norma 2}=1$.\\
Posto $k=\max \left( M, \frac{1}{m}\right)\geq 1 $ si ha 
$$ \frac{1}{k} \norma v \norma_2 \leq \norma v \norma \leq k \norma v \norma_2 $$
Sia $d$ la distanza indotta da $\nnorma$ allora
$$\forall v, w \quad  \frac{1}{k} d_2(v,w) \leq d(v,w) \leq k d_2(v,w)$$ 
Dunque le distanze indotte sono topologicamente equivalenti \endproof
\begin{oss}Se la dimensione di $V$ \`e infinita il teorema non vale.\\
Si pensi ad esempio $V=C^0([a,b])$ allora le distanze $d_1$ e $d_\infty$ non sono topologicamente equivalenti, tale distanze sono indotte da $\nnorma_1$ e $\nnorma_\infty$
\end{oss}
\end{thm}
\newpage
\section{Proiezione stereografica}
Il 18 ottobre abbiamo dimostrato il seguente teorema
\begin{thm}$$ \frac{D^n}{S^{n-1}}\cong S^n$$
\end{thm}
\begin{cor}$$S^n - \{ pt \} \cong \R^n$$
\proof Dalla definizione di collassamento di un insieme ad un punto, $\frac{D^n}{S^{n-1}}$ contiene un aperto omeomorfo a $D^n \sbarra S^{n-1}$ tale aperto \`e l'insieme 
$$ \{ x \in \R \, \vert \, \norma x \norma < 1\} $$
Se consideriamo la seguente funzione 
$$ B\left( 0 , \frac{\pi}{2} \right) \to \R^n \qquad v \to \tan ( \norma v ) v $$ 
tale funzione \`e un omeomorfismo da cui $D^n \sbarra S^{n-1} \cong \R^n $.\\
Ora dalla definizione di collassamento $D^n \sbarra S^{n-1}$ \`e omeomorfo a $\frac{D^n}{S^{n-1}}$ privato del punto ottenuto dal collasso di $S^{n-1}$ dunque abbiamo l'omeomorfismo cercato. \endproof
\end{cor}
\spazio
Mostriamo un omeomorfismo esplicito  $S^n - \{ pt \} \to \R^n$.\\
Fissiamo il punto $A=(1, 0, \dots, 0)$ e identifichiamo $\R^n$ con $\{ (x_0, \dots, x_n) \, \vert \, x_0=0 \}\subseteq \R^{n+1}$
$$ f:\, S^n \sbarra (1, 0, \dots, 0) \to \R^n$$
Sia $B=(x_0, \dots, x_n) \in S^n$ con $B\neq A $ allora definiamo $f(B)$ come  il punto di intersezione tra lo spazio $\{ x_0 =0\}$ e la retta passante per $A$  e $B$.\\
Esplicitiamo tale funzione.\\
La retta passante per $A$ e $B$ \`e 
$$ t(x_0-1, x_1, \dots, x_n) + ( 1, 0, \dots, 0)$$ dunque 
$$ f((x_0, \dots, x_n)) = \left( 0, \frac{x_1}{1-x_0}, \dots, \frac{x_n}{1-x_0}\right)$$
Mostriamo che $f$ \`e omeomorfismo.\\
Poniamo $y_i= \frac{x_i}{1-x_0}$ dunque  $f((x_0, \dots, x_n)) = (0, y_1, \dots, y_n )$.\\
Vediamo come possiamo ricavare $x_0$ da $y_1, \dots, y_n$

$$ \sum_{i=1}^n y_1^2+1 = \frac{\sum_{i=1}^n x_i^2}{(1-x_0)^2}+1 = \frac{\sum_{i=1}^n x_i^2+1 +x_0^2 +2x_0}{(1-x_0)^n}$$
Ma $(x_0, \dots, x_n) \in S^n$ dunque $\ds \sum_{i=0}^n x_i ^2 =1$ da cui
$$ \sum_{i=1}^n y_1^2+1 = \frac{2(1-x_0)}{(1-x_0)^2}=\frac{2}{1-x_0}$$
dunque dagli $y_i$ ricavo $x_0$ 
$$ x_0 = \frac{\left( \sum_{i=1}^n y_i^2\right)-1}{\left( \sum_{i=1}^n y_i \right) +1}$$
Definiamo $g: \, \R^n \to S^n \sbarra (1, 0, \dots 0)$  come 
$$ g( (y_1, \dots, y_n)) = \left(   x_0 = \frac{\left( \sum_{i=1}^n y_i^2\right)-1}{\left( \sum_{i=1}^n y_i \right) +1}, y_1(1-x_0), \dots, y_n (1-x_0) \right)$$
Ora $f$ e $g$ sono continue ed inverse l'una dell'altra dunque $f$ \`e un omeomorfismo 
\newpage
\section{Compattificazione di Alexandross}
Sia $X$ uno spazio topologico definiamo 
$$ \hat{X}= X \cup \{ \infty\}$$ 
dove $\infty \not \in X $.\\
Definiamo su $\hat{X}$ una topologia $\tau$
$$ \tau =\{ A  \, \vert \, A\subseteq X \text{ aperto } \} \cup \{ \hat{X} \sbarra K \, \vert \, K \subseteq X \text{ chiuso e compatto} \}$$
\begin{prop}La famiglia appena descritta \`e una topologia
\proof \bbianco
\begin{enumerate}
\item $\emptyset$ \`e un aperto di $X$.\\$\emptyset$ \`e chiuso e compatto in $X$ dunque $\hat{X} \in \tau$
\item L'intersezione di $2$ aperti di $X$  \`e un aperto di $X$.\\
$ ( \hat{X}\sbarra K_1) \cap ( \hat{X}\sbarra K_2) = \hat{X}\sbarra ( K_1 \cup K_2)$ ora $K_1 \cup K_2 $ \`e un chiuso e compatto di $X$\\
$A \cap (\hat{X}\sbarra K)= A \sbarra K $ \`e un aperto in $X$ essendo $A$ aperto e $K$ chiuso
\item L'unione di aperti di $X$ \`e un aperto di $X$.\\
$\ds \bigcup_{i\in I } \hat{X}\sbarra K_i = \hat{X}\sbarra \left( \bigcap_{i\in I } K_i\right)$ ora $\ds \bigcap_{i\in I } K_i$ \`e un chiuso e compatto di  $X$\\
$A \cup (\tilde{X}\sbarra K) = \hat{X}\sbarra ( K \sbarra A )$ ora chiuso in un compatto \`e compatto ed \`e chiuso
\end{enumerate}
\endproof
\end{prop}
\spazio
\begin{prop}[Propiet\`a]\bianco
\begin{enumerate}
\item $i:\, X \inclu \hat{X}$ \`e un immersione aperta
\item $\hat{X}$ \`e compatto
\end{enumerate}
\proof \bbianco
\begin{enumerate}
\item Da come abbiamo definito la topologia su $\hat{X}$ 
$$ A \subseteq X \text{ aperto } \quad \ses \quad i(A) \subseteq \hat{X}\text{ aperto}$$
dunque l'immersione \`e aperta
\item  Sia $\ric= \ds \{ U_i\}_{i\in I } $ un ricoprimento aperto di $\hat{X}$.\\
Essendo un ricoprimento $ \exists i_0\in I $ tale che $\infty \in U_{i_0}$ da cui $\exists K \subseteq X $ chiuso e compatto con $U_{i_0}=\hat{X}\sbarra K $
Ora essendo $K$ compatto 
$$ K \subseteq \bigcup_{i\in I } U_i \quad \implica \quad \exists i_1, \dots, i_n \in I \text{ con } K \subseteq U_{i_1} \cup \dots \cup U_{i_n}$$ 
da cui 
$$ \hat{X}= U_{i_0} \cup   U_{i_1} \cup \dots \cup U_{i_n}$$
\end{enumerate}
\endproof
\end{prop}
\begin{defn}$\hat{X}$ \`e detta compattificazione di Alexandross di $X$
\end{defn}
\end{document}