\documentclass[a4paper,12pt]{article}
\usepackage[a4paper, top=2cm,bottom=2cm,right=2cm,left=2cm]{geometry}

\usepackage{bm,xcolor,mathdots,latexsym,amsfonts,amsthm,amsmath,
					mathrsfs,graphicx,cancel,tikz-cd,hyperref,booktabs,caption,amssymb,amssymb,wasysym}
\hypersetup{colorlinks=true,linkcolor=blue}
\usepackage[italian]{babel}
\usepackage[T1]{fontenc}
\usepackage[utf8]{inputenc}
\newcommand{\s}[1]{\left\{ #1 \right\}}
\newcommand{\sbarra}{\backslash} %% \ 
\newcommand{\ds}{\displaystyle} 
\newcommand{\alla}{^}  
\newcommand{\implica}{\Rightarrow}
\newcommand{\iimplica}{\Leftarrow}
\newcommand{\ses}{\Leftrightarrow} %se e solo se
\newcommand{\tc}{\quad \text{ t. c .} \quad } % tale che 
\newcommand{\spazio}{\vspace{0.5 cm}}
\newcommand{\bbianco}{\textcolor{white}{,}}
\newcommand{\bianco}{\textcolor{white}{,} \\}% per andare a capo dopo 																					definizioni teoremi ...


% campi 
\newcommand{\N}{\mathbb{N}} 
\newcommand{\R}{\mathbb{R}}
\newcommand{\Q}{\mathbb{Q}}
\newcommand{\Z}{\mathbb{Z}}
\newcommand{\K}{\mathbb{K}} 
\newcommand{\C}{\mathbb{C}}
\newcommand{\F}{\mathbb{F}}
\newcommand{\p}{\mathbb{P}}

%GEOMETRIA
\newcommand{\B}{\mathfrak{B}} %Base B
\newcommand{\D}{\mathfrak{D}}%Base D
\newcommand{\RR}{\mathfrak{R}}%Base R 
\newcommand{\Can}{\mathfrak{C}}%Base canonica
\newcommand{\Rif}{\mathfrak{R}}%Riferimento affine
\newcommand{\AB}{M_\D ^\B }% matrice applicazione rispetto alla base B e D 
\newcommand{\vett}{\overrightarrow}
\newcommand{\sd}{\sim_{SD}}%relazione sx dx
\newcommand{\nvett}{v_1, \, \dots , \, v_n} % v1 ... vn
\newcommand{\ncomb}{a_1 v_1 + \dots + a_n v_n} %a1 v1 + ... +an vn
\newcommand{\nrif}{P_1, \cdots , P_n} 
\newcommand{\bidu}{\left( V^\star \right)^\star}

\newcommand{\udis}{\amalg}
\newcommand{\ric}{\mathfrak{U}}
\newcommand{\inclu}{\hookrightarrow }
%ALGEBRA

\newcommand{\semidir}{\rtimes}%semidiretto
\newcommand{\W}{\Omega}
\newcommand{\norma}{\vert \vert }
\newcommand{\bignormal}{\left\vert \left\vert}
\newcommand{\bignormar}{\right\vert \right\vert}
\newcommand{\normale}{\triangleleft}
\newcommand{\nnorma}{\vert \vert \, \cdot \, \vert \vert}
\newcommand{\dt}{\, \mathrm{d}t}
\newcommand{\dz}{\, \mathrm{d}z}
\newcommand{\dx}{\, \mathrm{d}x}
\newcommand{\dy}{\, \mathrm{d}y}
\newcommand{\amma}{\gamma}
\newcommand{\inv}[1]{#1^{-1}}
\newcommand{\az}{\centerdot}
\newcommand{\ammasol}[1]{\tilde{\gamma}_{\tilde{#1}}}
\newcommand{\pror}[1]{\mathbb{P}^#1 (\R)}
\newcommand{\proc}[1]{\mathbb{P}^#1(\C)}
\newcommand{\sol}[2]{\widetilde{#1}_{\widetilde{#2}}}
\newcommand{\bsol}[3]{\left(\widetilde{#1}\right)_{\widetilde{#2}_{#3}}}
\newcommand{\norm}[1]{\left\vert\left\vert #1 \right\vert \right\vert}
\newcommand{\abs}[1]{\left\vert #1 \right\vert }
\newcommand{\ris}[2]{#1_{\vert #2}}
\newcommand{\vp}{\varphi}
\newcommand{\vt}{\vartheta}
\newcommand{\wt}[1]{\widetilde{#1}}
\newcommand{\pr}[2]{\frac{\partial \, #1}{\partial\, #2}}%derivata parziale
%per creare teoremi, dimostrazioni ... 
\theoremstyle{plain}
\newtheorem{thm}{Teorema}[section] 
\newtheorem{ese}[thm]{Esempio} 
\newtheorem{ex}[thm]{Esercizio} 
\newtheorem{fatti}[thm]{Fatti}
\newtheorem{fatto}[thm]{Fatto}

\newtheorem{cor}[thm]{Corollario} 
\newtheorem{lem}[thm]{Lemma} 
\newtheorem{al}[thm]{Algoritmo}
\newtheorem{prop}[thm]{Proposizione} 
\theoremstyle{definition} 
\newtheorem{defn}{Definizione}[section] 
\newcommand{\intt}[2]{int_{#1}^{#2}}
\theoremstyle{remark} 
\newtheorem{oss}{Osservazione} 
\newcommand{\di }{\, \mathrm{d}}
\newcommand{\tonde}[1]{\left( #1 \right)}
\newcommand{\quadre}[1]{\left[ #1 \right]}
\newcommand{\w}{\omega}

% diagrammi commutativi tikzcd
% per leggere la documentazione texdoc


\begin{document}
\textbf{Lezione del 27 Settembre di Gandini}
\begin{defn}[Palle chiuse]\bianco
$$C(x_0, R)=\{ x \in X \, \vert \, d(x,x_0) \leq R \}$$
\end{defn}
\begin{prop} Sia $(X,d)$ uno spazio metrico, allora
\begin{enumerate}
\item Le palle aperte sono degli aperti
\item Le palle chiuse sono dei chiusi
\end{enumerate}
\proof \bbianco
\begin{enumerate}
\item  Sia $x_0 \in X $ e $r>0$ allora vogliamo provare che 
$$ \forall y\in B(x_0,r) \quad \exists \varepsilon>0 \quad B(y_0, \varepsilon) \subseteq B(x_0,r)$$
Sia $\varepsilon= r-d(x_0,y)$ e poich\`e $y\in B(x_0,r) $ $\varepsilon>0$\\
Sia $z\in B(y,\varepsilon) $ allora $d(x_0,z)< \varepsilon $ e usando la disuguaglianza triangolare
$$ d(x_0,z)\leq d(x_0,y)+ d(y,z) < d(x_0,y) + \varepsilon <r $$
\item $C(x_0,R)$ \`e chiuso $\ses$ $X\sbarra C(x_0,R)$ \`e un aperto $\ses $ $A=\{ x\in X \, \vert \, d(x,x_0)>r\} $ \`e aperto.\\
Sia $y\in A$ allora per definizione $d(y,x_0)>r $ e dunque ponendo $\varepsilon=d(y,x_0)-r>0$ otteniamo 
$$ d(z,x_0) \geq \vert d(y,x_0) - d(z,y) \vert = d(y,x_0)-d(z,y)>r$$
\end{enumerate}
\endproof
\end{prop}
\spazio
\begin{ese}Sia $(X,d)$ spazio metrico, $Z\subseteq X $ arbitrario, allora 
$$ d_Z:\, X \to \R \quad x \to \inf_{z\in Z} d(x,z)$$
\`e continua
\proof Dobbiamo dimostrare che $\forall x,y\in X$ vale 
$$ \vert d_Z(x) -d_Z(y) \vert \leq d(x,y)$$
infatti da ci\`o segue la continuit\`a.\\
Mostriamo la disuguaglianza; data la simmetria della distanza possiamo togliere il valore assoluto.\\
Dalla definizione di estremo inferiore
$$ \forall \varepsilon>0 \exists z \in Z \quad d(x,z) \leq d_Z(x) + \varepsilon$$
Ora $$d_Z(y)\leq d(y,z)\leq d(y,x)+d(y,z)\leq d(x,y) +d_Z(x)+\varepsilon$$
dunque otteniamo la disuguaglianza
\end{ese}
\newpage
\begin{ese}\bianco 
\begin{itemize}
\item[(i)]L'intersezione di famiglie arbitrarie di aperti non \`e aperto 
\item[(ii)]L'unione di famiglie arbitrarie di chiusi non \`e un chiuso 
\end{itemize}
In $\R$ 
$$ \bigcap_{n \in \N} \left( -\frac{1}{n}, \frac{1}{n}\right) =\{ 0\} \text{ che non \`e aperto } $$
$$ \bigcup_{n\in \N}  \left[ \frac{1}{n}, 1-\frac{1}{n}\right]=(0,1)\text{ che non \`e chiuso} $$
\end{ese}
\newpage
\begin{defn}\bianco 
Siano $X$ uno spazio topologico e $Z\subseteq X $ allora 
\begin{itemize}
\item La chiusura di $Z$ ($\overline{Z}$) \`e il pi\`u piccolo chiuso contenente $Z$ .\\
In modo equivalente $$\overline{Z}=\bigcap_{ C\subseteq  X \text{ chiuso} \atop{ Z\subseteq C}} C $$
\item La parte interna di $Z$ $\left( Z^\circ \right) $ \`e il pi\`u grande aperto contenuto in $Z$.
$$ Z^\circ =\bigcup_{A\subseteq \text{ aperto}\atop{A\subseteq Z}} A$$
\item La frontiera di $Z$ ($\partial Z $) \`e l'insieme $\overline{Z}\sbarra Z^\circ $  ed \`e un chiuso della topologia
\end{itemize}
\end{defn}
\begin{ese}In $\R$ 
 \begin{itemize}
\item $\overline{(a,b)}=\overline{[a,b)}=\overline{(a,b]}=\overline{[a,b]}=[a,b]$ 
\item $(a,b)^\circ=[a,b)^\circ=(a,b]^\circ=[a,b]^\circ=(a,b)$ 
\item Nei casi precedenti $\partial ( ...) = \{ a,b \} $
\end{itemize}
\end{ese}
\spazio
\begin{oss}Sia $(X,d)$ uno spazio metrico allora 
$$ B(x,r)\subseteq C(x,r) \quad \implica \quad \overline{B(x,r)}\subseteq C(x,r)$$
in $\R^2$ con la distanza euclidea (anche con $d_2, \, d_\infty$) vale l'uguaglianza.\\
In generale \`e falso infatti se prendiamo $d$ la distanza discreta:
$$ B(x,1)=\{x\} \text{ chiuso} \quad \implica \overline{B(x,1)}=B(x,1)$$
Mentre $C(x,1)=X$
\end{oss}
\spazio 

\spazio
\begin{prop}Sia $(X,d)$ spazio metrico, $Z\subseteq X $ arbitrario, allora 
$$ \overline{Z}=\{ x\in X \quad \vert \quad \inf_{z\in Z } d(x,z)=0\} = d_Z^{-1}( \{ 0 \} )$$
Sia $x\in X$ allora
$$ \inf_{z\in Z} d(x,z)=0 \quad \ses \quad \forall \varepsilon>0 \quad B(x,\varepsilon) \cap Z \neq \emptyset \quad \ses \quad x\in \overline{Z}$$
\end{prop}
\newpage
\begin{prop}Consideriamo l'insieme $C^0 ( [a,b])$ con le funzioni
\begin{enumerate}
\item $d_1(f,g)= \int_{a}^b \vert f(t)-g(t) \vert \dt $
\item $d_2(f,g)=  \left( \int_{a}^b \vert f(t)-g(t) \vert ^2 \right)^{\frac{1}{2}}\dt $
\item $d_\infty(f,g)= \sup_{t\ in [a,b]} \vert f(t)-g(t) \vert  $
\end{enumerate}
allora le 3 funzioni sono distanze
\proof Mostriamolo solamente nel caso di $d_1$.\begin{itemize}
\item 
$d_1(f,g)= d_1(g,f)$  segue dalle propiet\`a del valore assoluto 
\item  $d_1(f,g)\geq 0)$ infatti l'integrale di una funzione non negativa \`e non negativo.\\
Sia $d_1(f,g)=0$ e supponiamo $f\neq g$.\\
Essendo le funzioni diverse sia $t_0\in [a,b]$ tale che $f(t_0)\neq g(t_0)$ dunque essendo le funzioni continue\\
$$ \text{ Sia } \delta>0 \quad \exists \varepsilon \quad \vert f(t)-g(t) \vert \leq \delta \text{ per } t \in [t_0-\varepsilon , t_0+\varepsilon]$$
Dunque
$$ d_1(f,g) > \int_{t_0-\varepsilon}^{t_0+\varepsilon} \vert f(t)-g(t)\varepsilon \dt >0 $$
\item La disuguaglianza triangolare deriva dalla linearit\`a dell'integrale e dalla disuguaglianza triangolare del modulo
\end{itemize}
\endproof
\end{prop}

\begin{oss}Le distanza sopra definite non sono equivalenti.\\
Siano $\tau_1, \tau_2, \tau_\infty$ le topologie indotte.\\
Sia $f_n \in C^{0}([0,1])$ definita come 
$$ f_n(t) = \begin{cases} 1-nt \text{ se } t \leq \frac{1}{n}\\ 0 \text{ se } t > \frac{1}{n}
\end{cases}$$
Calcoliamo la distanza di $f_n$ da $0$(funzione nulla)
$$ d_\infty(f_n, 0)= \sup_{t\in [0,1]} \vert f_n(t)\vert = 1\quad \forall n \in \N$$
$$d_1(f_n,0) =\frac{1}{2n} \quad \implica \quad \forall	 \varepsilon >0 \quad \exists n_0 \quad f_n \in B_{d_1}(0, \varepsilon) \quad \forall n \geq n_0$$
Dunque ogni aperto di $\tau_1$ che contiene la funzione nulla contiene anche infinite funzioni $f_n$.\\
Consideriamo $B=B_{d_\infty}(0,1) $ essa \`e un aperto di $\tau_\infty$ ma non contiene nessuna $f_n$ dunque $B$ non \`e un aperto di $\tau_1$
\end{oss}
\newpage

\begin{ex}[Topologia della semicontinuit\`a]\bianco
Su $\R$ definiamo una topologia come segue
$$ \tau= \{ (-\infty , a) \, \vert a \in \R \} \cup \{ \emptyset ,\R\}$$
\`e una topologia e prende il nome di topologia della semicontinuit\`a
\proof \bbianco
\begin{itemize}
\item L'insieme stesso e il vuoto sono aperti per definizione
\item Siano $A$, $B$ aperti.\\
Se $A=\emptyset$  allora $A\cap B=\emptyset$ quindi \`e aperto.\ Se  $A=\R$ allora $A\cap B = B$ che \`e aperto.\\
Supponiamo allora $A=(-\infty , a ) $ e $B=(-\infty,  b)$ dunque $A\cap B = (-\infty, \min(a,b)) \in \tau$
\item Consideriamo  $A\subseteq \R$  non vuoto allora sia
$$ X=\bigcup_{a\in A} (-\infty, a)$$
se $A$ \`e limitato allora $\exists \sup A $ dunque $X=(-\infty , \sup A) \in \tau$\\
altrimenti $X=\R$
\end{itemize}
\end{ex}
\spazio
\begin{ex}Sia $f:\, X\to \R$ con $X$ spazio topologico e $\R$ con la topologia della semicontinuit\`a.
$$ f\text{ continua } \ses \forall x\in X  \, \forall \varepsilon>0 \, \exists U \subseteq X \text{ aperto che contiene } x \quad f(y)<f(x)+\varepsilon \quad \forall y\in U$$
\proof$\implica$ Sia $x\in X$ e sia $\varepsilon >0$  $$f(y) < f(x)+ \varepsilon \quad \implica \quad y \in f^{-1}((-\infty, f(x)+\varepsilon)) \text{ che \`e aperto in quanto controimmagine di aperto}$$
$\iimplica$ Dobbiamo provare che $A=f^{-1}((-\infty, a))$ \`e aperto $\forall a \in \R$.\\
Sia $x\in f^{-1}((-\infty, a))$ allora $\exists \varepsilon>0$ tale che $f(x) + \varepsilon < a$.\\
allora dalla tesi segue che 
$$ \exists U_x \subseteq X \text{ aperto tale che } f(U_x) \subseteq (-\infty,a)$$
allora $x\in U_x $ dunque $U_x\subseteq A$.\\
Dunque $A$ \`e un aperto in quanto unione di aperti $A = \ds \bigcup_{x\in f^{-1}((-\infty, a))} U_x$
\endproof
\end{ex}
\end{document}
