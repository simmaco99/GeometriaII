 \documentclass[a4paper,12pt]{article}
\usepackage[a4paper, top=2cm,bottom=2cm,right=2cm,left=2cm]{geometry}

\usepackage{bm,xcolor,mathdots,latexsym,amsfonts,amsthm,amsmath,
					mathrsfs,graphicx,cancel,tikz-cd,hyperref,booktabs,caption,amssymb,amssymb,wasysym}
\hypersetup{colorlinks=true,linkcolor=blue}
\usepackage[italian]{babel}
\usepackage[T1]{fontenc}
\usepackage[utf8]{inputenc}
\newcommand{\s}[1]{\left\{ #1 \right\}}
\newcommand{\sbarra}{\backslash} %% \ 
\newcommand{\ds}{\displaystyle} 
\newcommand{\alla}{^}  
\newcommand{\implica}{\Rightarrow}
\newcommand{\iimplica}{\Leftarrow}
\newcommand{\ses}{\Leftrightarrow} %se e solo se
\newcommand{\tc}{\quad \text{ t. c .} \quad } % tale che 
\newcommand{\spazio}{\vspace{0.5 cm}}
\newcommand{\bbianco}{\textcolor{white}{,}}
\newcommand{\bianco}{\textcolor{white}{,} \\}% per andare a capo dopo 																					definizioni teoremi ...


% campi 
\newcommand{\N}{\mathbb{N}} 
\newcommand{\R}{\mathbb{R}}
\newcommand{\Q}{\mathbb{Q}}
\newcommand{\Z}{\mathbb{Z}}
\newcommand{\K}{\mathbb{K}} 
\newcommand{\C}{\mathbb{C}}
\newcommand{\F}{\mathbb{F}}
\newcommand{\p}{\mathbb{P}}

%GEOMETRIA
\newcommand{\B}{\mathfrak{B}} %Base B
\newcommand{\D}{\mathfrak{D}}%Base D
\newcommand{\RR}{\mathfrak{R}}%Base R 
\newcommand{\Can}{\mathfrak{C}}%Base canonica
\newcommand{\Rif}{\mathfrak{R}}%Riferimento affine
\newcommand{\AB}{M_\D ^\B }% matrice applicazione rispetto alla base B e D 
\newcommand{\vett}{\overrightarrow}
\newcommand{\sd}{\sim_{SD}}%relazione sx dx
\newcommand{\nvett}{v_1, \, \dots , \, v_n} % v1 ... vn
\newcommand{\ncomb}{a_1 v_1 + \dots + a_n v_n} %a1 v1 + ... +an vn
\newcommand{\nrif}{P_1, \cdots , P_n} 
\newcommand{\bidu}{\left( V^\star \right)^\star}

\newcommand{\udis}{\amalg}
\newcommand{\ric}{\mathfrak{U}}
\newcommand{\inclu}{\hookrightarrow }
%ALGEBRA

\newcommand{\semidir}{\rtimes}%semidiretto
\newcommand{\W}{\Omega}
\newcommand{\norma}{\vert \vert }
\newcommand{\bignormal}{\left\vert \left\vert}
\newcommand{\bignormar}{\right\vert \right\vert}
\newcommand{\normale}{\triangleleft}
\newcommand{\nnorma}{\vert \vert \, \cdot \, \vert \vert}
\newcommand{\dt}{\, \mathrm{d}t}
\newcommand{\dz}{\, \mathrm{d}z}
\newcommand{\dx}{\, \mathrm{d}x}
\newcommand{\dy}{\, \mathrm{d}y}
\newcommand{\amma}{\gamma}
\newcommand{\inv}[1]{#1^{-1}}
\newcommand{\az}{\centerdot}
\newcommand{\ammasol}[1]{\tilde{\gamma}_{\tilde{#1}}}
\newcommand{\pror}[1]{\mathbb{P}^#1 (\R)}
\newcommand{\proc}[1]{\mathbb{P}^#1(\C)}
\newcommand{\sol}[2]{\widetilde{#1}_{\widetilde{#2}}}
\newcommand{\bsol}[3]{\left(\widetilde{#1}\right)_{\widetilde{#2}_{#3}}}
\newcommand{\norm}[1]{\left\vert\left\vert #1 \right\vert \right\vert}
\newcommand{\abs}[1]{\left\vert #1 \right\vert }
\newcommand{\ris}[2]{#1_{\vert #2}}
\newcommand{\vp}{\varphi}
\newcommand{\vt}{\vartheta}
\newcommand{\wt}[1]{\widetilde{#1}}
\newcommand{\pr}[2]{\frac{\partial \, #1}{\partial\, #2}}%derivata parziale
%per creare teoremi, dimostrazioni ... 
\theoremstyle{plain}
\newtheorem{thm}{Teorema}[section] 
\newtheorem{ese}[thm]{Esempio} 
\newtheorem{ex}[thm]{Esercizio} 
\newtheorem{fatti}[thm]{Fatti}
\newtheorem{fatto}[thm]{Fatto}

\newtheorem{cor}[thm]{Corollario} 
\newtheorem{lem}[thm]{Lemma} 
\newtheorem{al}[thm]{Algoritmo}
\newtheorem{prop}[thm]{Proposizione} 
\theoremstyle{definition} 
\newtheorem{defn}{Definizione}[section] 
\newcommand{\intt}[2]{int_{#1}^{#2}}
\theoremstyle{remark} 
\newtheorem{oss}{Osservazione} 
\newcommand{\di }{\, \mathrm{d}}
\newcommand{\tonde}[1]{\left( #1 \right)}
\newcommand{\quadre}[1]{\left[ #1 \right]}
\newcommand{\w}{\omega}

% diagrammi commutativi tikzcd
% per leggere la documentazione texdoc

\begin{document}
\textbf{Lezioni del 30  Ottobre del prof. Frigerio}
\begin{defn}[Compatto]\bianco
Uno spazio topologico $X$ \`e \textbf{compatto} se ogni suo ricoprimento aperto un sottoricoprimento finito, cio\`e se dato 
$$ \ric =\{ U_i \}_{i \in I} \quad U_i \text{ aperto } \forall i \in I \quad X = \bigcup_{i \in I} U_i $$
$$ \exists i_0, \, \dots , \, i_n \quad U = U_{i_0} \cup \dots \cup U_{i_n}$$
\end{defn}
\begin{defn}Un sottospazio si dice compatto se \`e compatto con la topologia di sottospazio
\end{defn}
\begin{defn}Uno spazio metrico \`e limitato se $$\exists x_0 \in X \text{ e } \exists R >0 \quad X=B(x_0,R)$$
In modo equivalente 
$$ text{diam}(X) = \sup_{x,y \in X} \{ d( x,y)\} < + \infty$$
\begin{oss}Mostriamo l'equivalenza delle definizioni.\\
Supponiamo $X=B(x_0,R) $ allora $ \forall x ,y \in X$  vale $d(x,y) < d(x_0,x)+d(x_0,y)<2R$\\
se $diam(X)=d<+\infty$ allora $\forall x_0 \in X$ vale $X=B(x_0,d+1)$
\end{oss}
\end{defn}
\spazio
\begin{lem}$X$ metrico compatto $\implica$ $X$ limitato
\proof Scelto $x_0 \in X $ e posto $U_n =B(x_0,n)$ con $n\in \N$ allora $\ds \ric =\{ U_n \}_{n\in \N}$ \`e un ricoprimento aperto.\\
Dalla compattezza di $X$ segue
$$ X \subseteq U_{n_0} \cup \cdots \cup U_{n_k}$$
da cui $X  \subseteq B(x_0,R)$ con $R=\max \{ n_0, \cdots , n_k \}$
\endproof
\end{lem}
\begin{cor}$\R$ non \`e compatto
\end{cor}
\spazio 
\begin{thm}$[0,1]$ \`e compatto
\proof Sia $\ric= \{ U_i\}_{i \in I } $ un ricoprimento aperto di $[0,1]$
$$ A=\left\{   t \in [0,1] \, \left | \, \exists J \subseteq I \text{ finito con } [0,t] \subseteq \bigcup_{i \in J } U_i \right\} \right.$$
La tesi, \`e dunque equivalente a $1\in A$.\\
Poich\`e $0\in U_{i_0}$ e $U_{i_0}$ \`e aperto abbiamo $[0,\varepsilon) \subseteq U_{i_0}$ per un qualche $\varepsilon>0$.\\
se $t\in [0,\varepsilon)$ si ha $[0,t]\subseteq U_{i_0}$  dunque $[0,\varepsilon) \subseteq A$.\\
Ora $A$ non \`e vuoto ed \`e limitato superiormente dunque \`e ben definito $t_0=\sup A>0$.\\
Mostriamo che $t_0$ \`e un massimo per $A$\\
Per definizione di ricoprimento 
$ \exists\, U_{i} $ aperto con  $ t_0 \in U_i  $ e poich\`e $t_0>0$
$$\exists\, \delta >0 \quad (t_0 -\delta , t_0] \subseteq U_i $$
e dalla definizione di estremo superiore
$$ \exists \,\overline{t} \in (t_0-\delta, t_0] \cap A \quad \implica \quad [0, \overline{t}] \subseteq U_{i_1} \cup \cdots \cup U_{i_n}$$
dunque 
$$ [0,t_0] \subseteq U_{i_1} \cup \cdots \cup U_{i_n} \cup U_i \quad \implica \quad t_0\in A   $$
Proviamo che $t_0=1$, supponiamo che $t_0<1$.\\
Come prima $\exists \, U_i $ con $t_0\in U_i$ e $[t_0, t_0+\delta) \subseteq U_i $ per qualche $\delta>0$.\\
Poich\`e $t_0 \in A$ 
$$ [0,t_0] \subseteq U_{i_1} \cup \cdots \cup U_{i_n} \quad \implica \quad  \left[ 0 , t_0 + \frac{\delta}{2}\right] \subseteq U_{i_1}\cup \cdots \cup U_{i_n} \cup U_i \quad \implica \quad t_0 +\frac{\delta}{2} \in A $$
Ma ci\`o \`e un assurdo poich\`e $t_0 =\max A$\\
\endproof
\end{thm}
\spazio
\begin{thm}$f:\, X \to Y $ continua.
$$ X \text{ compatto } \quad \implica \quad f(X) \text{ compatto }$$
\proof  Sia $\ds \ric=\{ U_i \}_{i \in I } $ un ricoprimento aperto di $f(X)$\\
Ora essendo $f$ continua, dalla propiet\`a universale della topologia di sottospazio, anche la funzione $\tilde{f}:\, X \to f(X)$ \`e continua.\\
Osserviamo che $\ds \{ \tilde{f}^{-1} (U_i) \}_{i \in I} $ \`e un ricoprimento aperto di $X$.\\
Dalla compattezza di $X$ segue $\exists J \subseteq I$ finito con
$$ X = \bigcup_{i \in J } \tilde{f}^{-1}(U_i) \quad \implica \quad f(X) =\bigcup_{i \in J } U_i $$
dunque $f(X)$ \`e compatto
\end{thm}
\spazio
\begin{fatti}\bbianco
\begin{itemize}
\item Ogni spazio finito \`e compatto 
\item Unione finita di sottospazi compatti \`e compatto
\end{itemize}
\end{fatti}
\spazio
\begin{thm}\label{chiuso_in_compatto}Sia $X$ uno spazio compatto
$$ Y \subseteq X \text{ chiuso } \quad \implica \quad Y \text{ compatto}$$
\proof Sia $\ds \ric=\{ U_i \}_{i \in I} $ un ricoprimento aperto di $Y$.\\
Dalla topologia di sottospazio posso supporre che $\forall i \in I $ si ha $V_i$ aperto di $X$ con $U_i = Y \cap V_i$.\\
Poich\`e $\{ U_i\}$ \`e un ricoprimento di $X$ si ha $\ds Y \subseteq \bigcup_{i \in I } V_i$.\\
Poich\`e $Y$ \`e chiuso $W=X\sbarra Y$ \`e aperto allora
$$ \{ V_i \}_{i \in I} \cup W \text{ \`e un ricoprimento aperto di } X $$
Per compattezza $X=V_{i_1} \cup \cdots \cup V_{i_n} \cup W$.\\
Poich\`e $W \cap Y = \emptyset$ se ne deduce che 
$$ Y \subseteq V_{i_1}\cup \cdots \cup V_{i_n} \quad \implica \quad Y= U_{i_1} \cup \cdots \cup U_{i_n}$$
da cui la tesi \endproof
\end{thm}
\begin{oss}
Abbiamo visto che $$Y \subseteq X \text{  compatto } \quad \ses $$
$$\ses \forall \{ U_i\}_{i \in I } \text{ famiglia di aperti di } X \quad  Y \subseteq \bigcup_{i\in I } U_i \text{ si ha }  \exists J\subseteq I \text{ finito } \quad Y \subseteq \bigcup_{i \in J } U_i$$
\end{oss}
\spazio
\begin{oss}Un sottospazio compatto non \`e chiuso.\\
Prendiamo $X$ con la topologia cofinita, mostriamo che ogni sottoinsieme \`e un compatto.\\
Sia $Y \subseteq X $ e supponiamo che $ \ds Y \subseteq \bigcup_{i \in I } U_i$ con $U_i $ aperto di $X$.\\
Se $Y = \emptyset$ allora \`e compatto, altrimenti $\exists\, y_0 \in Y$ e  dunque $y_0 \in U_{i_0}$ per un $i_0 \in I$.\\
Ora essendo $U_{i_0}$ aperto $X\sbarra U_{i_0}$ \`e finito dunque a maggior ragione anche $Y \sbarra U_{i_0}$ \`e finito.
$$ Y \sbarra  U_{i_0} =\{ y_1, \dots, y_n\} \quad \forall j=1,\dots,n \quad \exists i_j \text{ con } y_j \subseteq U_{i_j}$$
$$ Y = U_{i_0} \cup U_{i_1} \cup \dots  \cup U_{i_n}$$
ovvero $Y$ \`e compatto.\\
\\
Potevamo prendere come controesempio un generico $X$ con la topologia indiscreta.\\
Tutti i ricoprimenti aperti sono fatti da $X$ stesso dunque ogni sottoinsieme \`e compatto ma nessuno escluso il vuoto e $X$ stesso sono chiusi
\end{oss}
\spazio

\begin{thm}Sia $X$ spazio $T2$ e $ Y \subseteq X$
$$ Y \text{ compatto } \quad \implica \quad Y \text{ chiuso}$$
\proof Dobbiamo mostrare che $X\sbarra Y$ \`e aperto ovvero intorno di ogni suo punto.\\
Fissiamo $x_0 \in X \sbarra Y$ 
$$ \forall y \in Y \quad \exists \, U_y, \, V_y \text{ aperti  disgiunti di } X \quad x_0 \in U_y \quad y \in V_j $$
Ora $\ds \{ V_y\}_{y\in Y}$ \`e un ricoprimento di $Y$ con aperti di $X$.\\
Per compattezza di $Y$, $\exists \, y_1, \cdots , y_n \in Y$ con
$$Y \subseteq V_{y_1}\cup \cdots \cup V_{y_n}$$
Pongo  $$ V = V_{y_1}\cap \cdots \cap V_{y_n}$$
$$ U = U_{y_1}\cup \cdots \cup U_{y_n}$$
ora $U$ aperto e $x_0\in V$ aperto essendo intersezione finita di aperti.\\
Per costruzione $U \cap V =\emptyset$ dunque 
$x_0 \in U \subseteq X \sbarra V \subseteq X\sbarra Y$
\end{thm}

\begin{lem}$X$ compatto $T2$ $\implica$ $X$ regolare
\proof $T2 \implica T1$ per cui basta vedere che vale $T3$\\
Siano $x_0 \in X$ e $Y\subseteq X $ chiuso con $x_0 \not \in Y$\\
Poich\`e chiuso in un compatto \`e compatto (Teorema~\ref{chiuso_in_compatto}), $Y$ \`e compatto.\\
Gli aperti $U$ e $V$ della dimostrazione precedente verificano $x_0 \in U$,  $Y \subseteq V $ ed inoltre $U\cap V =\emptyset$
\endproof
\end{lem}
\begin{thm}$X$ compatto $T2$ $\implica$ $X$ normale
\proof $T2\implica T1$ dunque basta dimostrare che vale $T4$\\
Siano $C,D$ chiusi di $X$ con $C\cap D=\emptyset$ \\
Poich\`e $X$ regolare per il lemma precedente 
$$ \forall x\in C \quad \exists U_x,\, V_x \text{ aperti disgiunti di } X \text{ con } x \in U_x \quad D \subseteq V_x $$
$$ C \subseteq \bigcup_{x \in C } U_x $$
inoltre $C$ \`e chiuso in un compatto dunque compatto 
$$ \exists x_1, \cdots , x_n \in C \quad C \subseteq U_{x_1} \cup \cdots \cup U_{x_n }$$
Pongo
$$U = U_{x_1} \cup \cdots \cup U_{x_n}$$
$$V = V_{x_1} \cap \cdots \cap V_{x_n}$$
$U$ e $V$ sono aperti con la propiet\`a richiesta da $T4$ 
\endproof
\end{thm}
\begin{oss}Abbiamo dimostrato di pi\`u.\\
Se $X$ \`e $T2$ con $C,D\subseteq X $ compatti disgiunti allora essi si possono separare con aperti.
\end{oss}
\end{document}