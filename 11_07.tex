\documentclass[a4paper,12pt]{article}
\usepackage[a4paper, top=2cm,bottom=2cm,right=2cm,left=2cm]{geometry}

\usepackage{bm,xcolor,mathdots,latexsym,amsfonts,amsthm,amsmath,
					mathrsfs,graphicx,cancel,tikz-cd,hyperref,booktabs,caption,amssymb,amssymb,wasysym}
\hypersetup{colorlinks=true,linkcolor=blue}
\usepackage[italian]{babel}
\usepackage[T1]{fontenc}
\usepackage[utf8]{inputenc}
\newcommand{\s}[1]{\left\{ #1 \right\}}
\newcommand{\sbarra}{\backslash} %% \ 
\newcommand{\ds}{\displaystyle} 
\newcommand{\alla}{^}  
\newcommand{\implica}{\Rightarrow}
\newcommand{\iimplica}{\Leftarrow}
\newcommand{\ses}{\Leftrightarrow} %se e solo se
\newcommand{\tc}{\quad \text{ t. c .} \quad } % tale che 
\newcommand{\spazio}{\vspace{0.5 cm}}
\newcommand{\bbianco}{\textcolor{white}{,}}
\newcommand{\bianco}{\textcolor{white}{,} \\}% per andare a capo dopo 																					definizioni teoremi ...


% campi 
\newcommand{\N}{\mathbb{N}} 
\newcommand{\R}{\mathbb{R}}
\newcommand{\Q}{\mathbb{Q}}
\newcommand{\Z}{\mathbb{Z}}
\newcommand{\K}{\mathbb{K}} 
\newcommand{\C}{\mathbb{C}}
\newcommand{\F}{\mathbb{F}}
\newcommand{\p}{\mathbb{P}}

%GEOMETRIA
\newcommand{\B}{\mathfrak{B}} %Base B
\newcommand{\D}{\mathfrak{D}}%Base D
\newcommand{\RR}{\mathfrak{R}}%Base R 
\newcommand{\Can}{\mathfrak{C}}%Base canonica
\newcommand{\Rif}{\mathfrak{R}}%Riferimento affine
\newcommand{\AB}{M_\D ^\B }% matrice applicazione rispetto alla base B e D 
\newcommand{\vett}{\overrightarrow}
\newcommand{\sd}{\sim_{SD}}%relazione sx dx
\newcommand{\nvett}{v_1, \, \dots , \, v_n} % v1 ... vn
\newcommand{\ncomb}{a_1 v_1 + \dots + a_n v_n} %a1 v1 + ... +an vn
\newcommand{\nrif}{P_1, \cdots , P_n} 
\newcommand{\bidu}{\left( V^\star \right)^\star}

\newcommand{\udis}{\amalg}
\newcommand{\ric}{\mathfrak{U}}
\newcommand{\inclu}{\hookrightarrow }
%ALGEBRA

\newcommand{\semidir}{\rtimes}%semidiretto
\newcommand{\W}{\Omega}
\newcommand{\norma}{\vert \vert }
\newcommand{\bignormal}{\left\vert \left\vert}
\newcommand{\bignormar}{\right\vert \right\vert}
\newcommand{\normale}{\triangleleft}
\newcommand{\nnorma}{\vert \vert \, \cdot \, \vert \vert}
\newcommand{\dt}{\, \mathrm{d}t}
\newcommand{\dz}{\, \mathrm{d}z}
\newcommand{\dx}{\, \mathrm{d}x}
\newcommand{\dy}{\, \mathrm{d}y}
\newcommand{\amma}{\gamma}
\newcommand{\inv}[1]{#1^{-1}}
\newcommand{\az}{\centerdot}
\newcommand{\ammasol}[1]{\tilde{\gamma}_{\tilde{#1}}}
\newcommand{\pror}[1]{\mathbb{P}^#1 (\R)}
\newcommand{\proc}[1]{\mathbb{P}^#1(\C)}
\newcommand{\sol}[2]{\widetilde{#1}_{\widetilde{#2}}}
\newcommand{\bsol}[3]{\left(\widetilde{#1}\right)_{\widetilde{#2}_{#3}}}
\newcommand{\norm}[1]{\left\vert\left\vert #1 \right\vert \right\vert}
\newcommand{\abs}[1]{\left\vert #1 \right\vert }
\newcommand{\ris}[2]{#1_{\vert #2}}
\newcommand{\vp}{\varphi}
\newcommand{\vt}{\vartheta}
\newcommand{\wt}[1]{\widetilde{#1}}
\newcommand{\pr}[2]{\frac{\partial \, #1}{\partial\, #2}}%derivata parziale
%per creare teoremi, dimostrazioni ... 
\theoremstyle{plain}
\newtheorem{thm}{Teorema}[section] 
\newtheorem{ese}[thm]{Esempio} 
\newtheorem{ex}[thm]{Esercizio} 
\newtheorem{fatti}[thm]{Fatti}
\newtheorem{fatto}[thm]{Fatto}

\newtheorem{cor}[thm]{Corollario} 
\newtheorem{lem}[thm]{Lemma} 
\newtheorem{al}[thm]{Algoritmo}
\newtheorem{prop}[thm]{Proposizione} 
\theoremstyle{definition} 
\newtheorem{defn}{Definizione}[section] 
\newcommand{\intt}[2]{int_{#1}^{#2}}
\theoremstyle{remark} 
\newtheorem{oss}{Osservazione} 
\newcommand{\di }{\, \mathrm{d}}
\newcommand{\tonde}[1]{\left( #1 \right)}
\newcommand{\quadre}[1]{\left[ #1 \right]}
\newcommand{\w}{\omega}

% diagrammi commutativi tikzcd
% per leggere la documentazione texdoc

\begin{document}
\textbf{Lezione del 7 Novembre di Gandini}
\begin{thm}$S^n $ \`e omeomorfo alla compattificazione di Alexandross di $\R^n$
\proof Identifichiamo $R^n$ con $S^n \sbarra \{ (1,0, \dots,  0\}$.\\
Mostriamo che $\hat{\R^n} \cong S^n$, Vale a dire che la topologia euclidea su $S^n$ coincide con la topologia di Alexandross  dove  $\infty=(1,0, \dots, 0)$.\\
Sia $A\subseteq S^n$ 
$$ A \text{ aperto } \quad\ses \quad \begin{cases} A \subseteq S^n \sbarra (1, 0,\dots , 0 )  \text{ aperto }\\
A = S^n \sbarra K \text{ con } K \subseteq S^n \sbarra (1, 0, \dots, 0) \text{ compatto} \end{cases}$$
Osserviamo che basta richiedere che $K$ sia compatto infatti lo spazio \`e di Hausdorff dunque compatto implica chiuso.
\begin{itemize}
\item $(1, 0, \dots,0) \not \in A $ dunque  $A \subseteq S^n \sbarra (1, 0,\dots , 0 ) $.\\
Consideriamo $i:\, R^n \to S^n$ dove $i$ con codominio ristretto  a $S^n\sbarra (1, 0, \dots, 0) $\`e l'inversa della proiezione stereografica, dunque immersione aperta.\\
Ora essendo $i$ aperta 
$$ A \subseteq S^n \text{ aperto } \quad \ses \quad A\subseteq S^n \sbarra (1,0, \dots , 0) \text{ aperto }$$
\item Supponiamo, adesso, $(1, 0, \dots, 0)\in A$ 
$$ A \text{ aperto } \quad \ses \quad S^n \sbarra A \text{ chiuso } \quad \ses \quad K = S^n \sbarra A \text{ compatto } \quad \ses $$ $$ \ses \quad A = S^n \sbarra K \text{ con } K \subseteq S^n \sbarra (1,0, \dots, 0) \text{ compatto } $$
\end{itemize}
\endproof
\end{thm}
\spazio
\begin{prop}\label{compatt_Haus} $\hat{X}$ Hausdorff $\ses$ $\begin{cases}\ X \text{ Hausdorff} \\ \forall x\in X \quad  x\text{ ammette un intorno compatto } \end{cases}$
\proof \bbianco
\begin{itemize}
\item Caso $1$: $x,y\in X$.\\Gli aperti di $\hat{X}$ sono aperti di $X$  e viceversa quindi 
$$ \exists U,V \text{ aperti di } \hat{X} \text{ che separano } x \text{ e } y \quad \ses \quad  U,V \text{ aperti di } {X} \text{ che separano } x \text{ e } y$$
\item Caso $2$: $y = \infty$
$$ \exists U \ni x \quad V \ni \infty \text{ aperti di } \hat{X} \text{ con } U \cap V = \emptyset \quad 
\ses \quad \begin{cases} x \in U \subseteq X \text{ aperto } \\ \infty \in V = \hat{X}\sbarra K \text{ K compatto }  \\ U \cap V = \emptyset
\end{cases} \quad \ses$$
$$ \ses \quad \begin{cases}\exists U \ni x \text{ aperto di } X \\
 y\in V \subseteq K  \text{ con } K \text{ intorno compatto di } y\\
 U \cap V = \emptyset
\end{cases}$$
\end{itemize}
\endproof
\end{prop}
\spazio
\begin{prop}Sia $f:\, X \to Y $ immersione aperta tra spazi di Hausdorff.\\
Definiamo $g:\, Y \to \hat{X}$ via $g(y) = \begin{cases} x \text{ se } x \in f^{-1}(y)=\{ x \}\\
\infty \text{ se } f^{-1}(y)=\emptyset
\end{cases}$.\\
Allora $g$ \`e continua
\proof Sia $U \subseteq \hat X $ aperto 
\begin{itemize}
\item Caso $1$: $\infty \not \in U $.\\
$ U\subseteq X \quad \implica \quad g^{-1}(U)=f(U)$ e poich\`e $f$ \`e aperta $g^{-1}(U)$ \`e un aperto 
\item Caso $2$: $\infty \in U$.\\
Allora $ U = \hat{X}\sbarra K \, \exists K \text{ compatto }$
$$g^{-1}(U) = Y \sbarra g^{-1}(K) = Y \sbarra f(K)$$
ora $K$ \`e compatto dunque anche $f(K)$ lo \`e, ora un compatto in un Hausdorff \`e chiuso quindi $g^{-1}(U)$ \`e aperto
\end{itemize}
\endproof
\end{prop}
\spazio
\begin{cor}$X$ compatto di Hausdorff, $x_0\in X$  allora  $X$ \`e omeomorfo alla compattificazione di Alexandross di $X \sbarra \{ x_0\}$
\proof Mostriamo che $\hat{X\sbarra \{ x_0\}}$ \`e di Hausdorff utilizzando la proposizione~\ref{compatt_Haus}.\\
Essendo $X$ di Hausdorff anche $X\sbarra \{ x_0\} $ lo \`e, vediamo ora che ogni punto $x\neq x_0$ ammette un intorno compatto.\\
Essendo $X$ di Hausdorff $\exists U, V $ aperti con $x \in U $ e $x_0 \in V$ tali che $U \cap V = \emptyset$ dunque $x_0 \not \in \overline{U} $  ora essendo $K$ chiuso in un compatto \`e compatto.\\
$\overline{U}$ \`e l`intorno compatto di $x$ in $X_0 \sbarra \{x_0\}$ ( $x\in U \subseteq \overline{U}$ con $U$ aperto).\\

Sia  $Y=X\sbarra \{ x_0 \sbarra$, l'immersione $i:\, Y \inclu X$ \`e un immersione aperta, dunque per la proposizione precedente
$$ g:\, X \to \hat{Y} \quad g(x) =\begin{cases}
                                   x \quad \text{ se } x \neq x_0
                                   \infty \text{ se } x = x_0
                                  \end{cases}$$
\`e continua e biettiva.\\
Ora essendo $X$ compatto $\forall C \subseteq X$ con $C$ chiuso, $C$ \`e compatto, dunque $f(C)$ \`e compatto in Hausdorff dunque $f(C)$ chisuo.\\
$g$ \`e una funzione chiusa e biettiva dunque un omeomorfismo.
\endproof
\end{cor}
\begin{ex}$f:\, X \to Y$ continua con $Y$ di Hausdorff.\\
Sia $\hat{f}:\, \hat{X} \to \hat{Y}$ via $\hat{f}(\infty)=\infty$ e $\hat{f}(x)=f(x)$ se $x \neq \infty$
\end{ex}
\newpage
\section{Esaustione in compatti}
\begin{defn}
 Sia $X$ uno spazio topologico, una esaustione in compatti \`e una famiglia di compatti $\ds \{ K_n\}_{n \in \N}$ di $X$ tale che 
 $$ X = \bigcup_{n \in \N} K_n$$
 $$ K_n \subseteq K_{n+1}^\circ \quad \forall n \in \N$$
\end{defn}

\end{document}