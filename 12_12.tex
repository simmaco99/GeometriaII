\documentclass[a4paper,12pt]{article}
\usepackage[a4paper, top=2cm,bottom=2cm,right=2cm,left=2cm]{geometry}

\usepackage{bm,xcolor,mathdots,latexsym,amsfonts,amsthm,amsmath,
					mathrsfs,graphicx,cancel,tikz-cd,hyperref,booktabs,caption,amssymb,amssymb,wasysym}
\hypersetup{colorlinks=true,linkcolor=blue}
\usepackage[italian]{babel}
\usepackage[T1]{fontenc}
\usepackage[utf8]{inputenc}
\newcommand{\s}[1]{\left\{ #1 \right\}}
\newcommand{\sbarra}{\backslash} %% \ 
\newcommand{\ds}{\displaystyle} 
\newcommand{\alla}{^}  
\newcommand{\implica}{\Rightarrow}
\newcommand{\iimplica}{\Leftarrow}
\newcommand{\ses}{\Leftrightarrow} %se e solo se
\newcommand{\tc}{\quad \text{ t. c .} \quad } % tale che 
\newcommand{\spazio}{\vspace{0.5 cm}}
\newcommand{\bbianco}{\textcolor{white}{,}}
\newcommand{\bianco}{\textcolor{white}{,} \\}% per andare a capo dopo 																					definizioni teoremi ...


% campi 
\newcommand{\N}{\mathbb{N}} 
\newcommand{\R}{\mathbb{R}}
\newcommand{\Q}{\mathbb{Q}}
\newcommand{\Z}{\mathbb{Z}}
\newcommand{\K}{\mathbb{K}} 
\newcommand{\C}{\mathbb{C}}
\newcommand{\F}{\mathbb{F}}
\newcommand{\p}{\mathbb{P}}

%GEOMETRIA
\newcommand{\B}{\mathfrak{B}} %Base B
\newcommand{\D}{\mathfrak{D}}%Base D
\newcommand{\RR}{\mathfrak{R}}%Base R 
\newcommand{\Can}{\mathfrak{C}}%Base canonica
\newcommand{\Rif}{\mathfrak{R}}%Riferimento affine
\newcommand{\AB}{M_\D ^\B }% matrice applicazione rispetto alla base B e D 
\newcommand{\vett}{\overrightarrow}
\newcommand{\sd}{\sim_{SD}}%relazione sx dx
\newcommand{\nvett}{v_1, \, \dots , \, v_n} % v1 ... vn
\newcommand{\ncomb}{a_1 v_1 + \dots + a_n v_n} %a1 v1 + ... +an vn
\newcommand{\nrif}{P_1, \cdots , P_n} 
\newcommand{\bidu}{\left( V^\star \right)^\star}

\newcommand{\udis}{\amalg}
\newcommand{\ric}{\mathfrak{U}}
\newcommand{\inclu}{\hookrightarrow }
%ALGEBRA

\newcommand{\semidir}{\rtimes}%semidiretto
\newcommand{\W}{\Omega}
\newcommand{\norma}{\vert \vert }
\newcommand{\bignormal}{\left\vert \left\vert}
\newcommand{\bignormar}{\right\vert \right\vert}
\newcommand{\normale}{\triangleleft}
\newcommand{\nnorma}{\vert \vert \, \cdot \, \vert \vert}
\newcommand{\dt}{\, \mathrm{d}t}
\newcommand{\dz}{\, \mathrm{d}z}
\newcommand{\dx}{\, \mathrm{d}x}
\newcommand{\dy}{\, \mathrm{d}y}
\newcommand{\amma}{\gamma}
\newcommand{\inv}[1]{#1^{-1}}
\newcommand{\az}{\centerdot}
\newcommand{\ammasol}[1]{\tilde{\gamma}_{\tilde{#1}}}
\newcommand{\pror}[1]{\mathbb{P}^#1 (\R)}
\newcommand{\proc}[1]{\mathbb{P}^#1(\C)}
\newcommand{\sol}[2]{\widetilde{#1}_{\widetilde{#2}}}
\newcommand{\bsol}[3]{\left(\widetilde{#1}\right)_{\widetilde{#2}_{#3}}}
\newcommand{\norm}[1]{\left\vert\left\vert #1 \right\vert \right\vert}
\newcommand{\abs}[1]{\left\vert #1 \right\vert }
\newcommand{\ris}[2]{#1_{\vert #2}}
\newcommand{\vp}{\varphi}
\newcommand{\vt}{\vartheta}
\newcommand{\wt}[1]{\widetilde{#1}}
\newcommand{\pr}[2]{\frac{\partial \, #1}{\partial\, #2}}%derivata parziale
%per creare teoremi, dimostrazioni ... 
\theoremstyle{plain}
\newtheorem{thm}{Teorema}[section] 
\newtheorem{ese}[thm]{Esempio} 
\newtheorem{ex}[thm]{Esercizio} 
\newtheorem{fatti}[thm]{Fatti}
\newtheorem{fatto}[thm]{Fatto}

\newtheorem{cor}[thm]{Corollario} 
\newtheorem{lem}[thm]{Lemma} 
\newtheorem{al}[thm]{Algoritmo}
\newtheorem{prop}[thm]{Proposizione} 
\theoremstyle{definition} 
\newtheorem{defn}{Definizione}[section] 
\newcommand{\intt}[2]{int_{#1}^{#2}}
\theoremstyle{remark} 
\newtheorem{oss}{Osservazione} 
\newcommand{\di }{\, \mathrm{d}}
\newcommand{\tonde}[1]{\left( #1 \right)}
\newcommand{\quadre}[1]{\left[ #1 \right]}
\newcommand{\w}{\omega}

% diagrammi commutativi tikzcd
% per leggere la documentazione texdoc

\begin{document}
\textbf{Lezione del 12 Dicembre del Prof. Frigerio}
\section{$\pi_1$ come funtore}
Sia $f:\, X\to Y$ continua.\\
Se $\alpha:\, [0,1]\to X $ \`e continua allora anche $f \circ \alpha$ lo \`e.\\
$f$ induce $$f_\star:\, \Omega(a,a) \to \Omega(f(a),f(a))$$
Se $\alpha\sim \beta$ in $\Omega(a,a)$ e $H:\, [0,1]\times [0,1]\to X$ \`e un'omotopia di cammini tra $\alpha$ e $ \beta$ allora $f\circ H : [0,1]\times [0,1]\to Y$ \`e un omotopia di cammini tra $f_\star(\alpha)$ e $f_\star(\beta)$.\\
Possiamo dunque osservare che $f_\star$ induce
$$ f_\star:\, \pi_1(X,a) \to \pi_1(Y,f(a))$$
\begin{enumerate}
\item $f_\star$ \`e un omomorfismo di gruppi
$$ f_\star( [\alpha] \cdot [\beta]) =f_\star( [\alpha\star \beta])= [f\circ ( \alpha\star \beta)]= [ ( f\circ \alpha) \star ( f\circ \beta)]=f_\star([\alpha])\cdot f_\star([\beta])$$
\item se $g:Y \to Z $ allora $g_\star \circ f_\star =(g\circ f)_\star$
$$ (g\circ f)([\alpha])= [g\circ f \circ \alpha] = g_\star ([f\circ \alpha])= g_\star \left( f_\star( [\alpha]) \right)$$
\item Data $Id:X\to X$ allora $Id_\star:\, \pi_1(X,a) \to \pi_1(X,a)$ \`e l'identit\`a
\end{enumerate}
Dalle 3 propiet\`a sopra enunciate, ho realizzato un funtore tra la categorieadegli spazi topologici puntati con le funzioni continue che preservano la puntatura e quella dei gruppi con gli omomorfismi di gruppi
\begin{cor}
$$f:\, X \to Y \text{  omeomorfismo } \quad \implica \quad f_\star :\, \pi_1(X,a) \to \pi_1(Y, f(a))\text{ isomorfismo }$$
\proof sia $g:\, Y \to Y$ l'inversa continua di $f$ allora $g_\star$ \`e l'inversa di $f_\star$
\end{cor}
\begin{oss}Se $X$ e $Y$ sono omeomorfi allora $X$ e $Y$ sono omotopicamente equivalenti
\end{oss}
\begin{prop}Sia $f:\, X \to X $ omotopa all'identit\`a tramite $H$.\\
Fissato $a\in X$, sia $\gamma:\, [0,1]\to X$ data da $\gamma(s)=H(a,s)$\\
La mappa $f_\star :\, \pi_1(X,a) \to \pi_1(X, f(a))$ e $\gamma_\sharp$ coincidono
\proof $\forall \alpha\in \Omega(a,a)$ sia 
$$ K:\, [0,1]\times [0,1] \to X \qquad K(t,s)=H(\alpha(t),s)$$
Osserviamo che $\alpha\star \gamma \star \overline{f_\star(\alpha)}\star \overline{\gamma}$ \`e omotopa all'identit\`a, dunque la giunzione si estende da $\partial I^2$ a $I^2$.\\
Lo stesso vale per $f_\star(\alpha) \star \overline{\gamma}\star \overline{\alpha}\star \gamma$ che \`e perci\`o omotopa a $1_{f(a)}$ dunque 
$$ 1=[ 
f_\star(\alpha)\star ( \overline{\gamma}
\star \overline{\alpha}\star \gamma ) ]
 = \left[ f_\star(\alpha) \right]\cdot [ \overline{\gamma}\star \overline{\alpha}\star \gamma]$$
ovvero 
$$ [f_\star(\alpha)]=[\overline{\gamma}\star \overline{\alpha}\star \gamma]^{-1}=[\overline{\gamma}\star \alpha\star \gamma]=\gamma_\sharp([\alpha])$$
\endproof
\end{prop}
\begin{thm}
$$f:\, X \to Y\text{ equivalenza omotopica } \quad \implica \quad f_\star \text{ isomorfismo}$$
\proof Sia $g$ un'inversa omotopica di $f$.\\
Poich\`e $g \circ f \sim Id_X$ allora $(g\circ f)_\star:\, \pi_1(X,a)\to \pi_1(X,g(f(a))$ \`e un isomorfismo in quanto coincide con $\gamma_\sharp$ per un certo $\gamma \in\Omega(a, g(f(a)))$.\\
Analogamente $f_\star \circ g_\star$ \`e un isomorfismo, $f_\star$ \`e iniettiva e surgettiva, da cui la tesi
\endproof
\end{thm}
\begin{defn}$X$ si dice semplicemente connesso se $X$ connesso per archi e $\pi_1(X,a)=\{ 1\} $ $\forall a\in X$
\end{defn}
\begin{cor}$X$ contraibile  $\implica$ $X$  semplicemente connesso
\end{cor}
\spazio
\begin{prop}$A\subseteq X$ retratto con inclusione $i$ e retrazione $r$.
\begin{enumerate}
\item $i_\star$ iniettiva e $r_\star$ surgettiva (per ogni scelta del punto base in $A$)
\item se la retrazione \`e per deformazione $i_\star$ e $r_\star$ sono isomorfismi

\end{enumerate}
\proof \bbianco
\begin{enumerate}
\item Dato $a\in A$ si ha $r\circ i=Id_A$, dunque $r_\star \circ i_\star= Id_{\pi_1(A,a)}$ da cui la tesi
\item Segue dal fatto che $i$ e $r$ sono equivalenze omotopiche
\end{enumerate}
\end{prop}
\begin{oss}Se $A\subseteq X$  $i_\star:\pi_1(X,a)\to \pi_1(X,a)$ pu\`o non essere iniettiva.\\
Ad esempio se $X=D^2$ e $A=S^1$
\end{oss}
\begin{ex} Sia 
$$ \psi:\, \pi_1(X,a)\to [S',X] \qquad \psi([\alpha])=[\hat{\alpha}]$$
Mostrare che 
$$ X \text{ connesso per archi }\quad \ses \quad \psi \text{ surgettiva}$$
e
$$ \psi([\alpha])=\psi([\beta])\quad \ses \quad [\alpha] \text{ coniugato a } [\beta] \text{ in } \pi_1(X,a)$$
\end{ex}
\end{document}