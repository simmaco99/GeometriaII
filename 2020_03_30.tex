\documentclass[a4paper,12pt]{article}
\usepackage[a4paper, top=2cm,bottom=2cm,right=2cm,left=2cm]{geometry}

\usepackage{bm,xcolor,mathdots,latexsym,amsfonts,amsthm,amsmath,
					mathrsfs,graphicx,cancel,tikz-cd,hyperref,booktabs,caption,amssymb,amssymb,wasysym}
\hypersetup{colorlinks=true,linkcolor=blue}
\usepackage[italian]{babel}
\usepackage[T1]{fontenc}
\usepackage[utf8]{inputenc}
\newcommand{\s}[1]{\left\{ #1 \right\}}
\newcommand{\sbarra}{\backslash} %% \ 
\newcommand{\ds}{\displaystyle} 
\newcommand{\alla}{^}  
\newcommand{\implica}{\Rightarrow}
\newcommand{\iimplica}{\Leftarrow}
\newcommand{\ses}{\Leftrightarrow} %se e solo se
\newcommand{\tc}{\quad \text{ t. c .} \quad } % tale che 
\newcommand{\spazio}{\vspace{0.5 cm}}
\newcommand{\bbianco}{\textcolor{white}{,}}
\newcommand{\bianco}{\textcolor{white}{,} \\}% per andare a capo dopo 																					definizioni teoremi ...


% campi 
\newcommand{\N}{\mathbb{N}} 
\newcommand{\R}{\mathbb{R}}
\newcommand{\Q}{\mathbb{Q}}
\newcommand{\Z}{\mathbb{Z}}
\newcommand{\K}{\mathbb{K}} 
\newcommand{\C}{\mathbb{C}}
\newcommand{\F}{\mathbb{F}}
\newcommand{\p}{\mathbb{P}}

%GEOMETRIA
\newcommand{\B}{\mathfrak{B}} %Base B
\newcommand{\D}{\mathfrak{D}}%Base D
\newcommand{\RR}{\mathfrak{R}}%Base R 
\newcommand{\Can}{\mathfrak{C}}%Base canonica
\newcommand{\Rif}{\mathfrak{R}}%Riferimento affine
\newcommand{\AB}{M_\D ^\B }% matrice applicazione rispetto alla base B e D 
\newcommand{\vett}{\overrightarrow}
\newcommand{\sd}{\sim_{SD}}%relazione sx dx
\newcommand{\nvett}{v_1, \, \dots , \, v_n} % v1 ... vn
\newcommand{\ncomb}{a_1 v_1 + \dots + a_n v_n} %a1 v1 + ... +an vn
\newcommand{\nrif}{P_1, \cdots , P_n} 
\newcommand{\bidu}{\left( V^\star \right)^\star}

\newcommand{\udis}{\amalg}
\newcommand{\ric}{\mathfrak{U}}
\newcommand{\inclu}{\hookrightarrow }
%ALGEBRA

\newcommand{\semidir}{\rtimes}%semidiretto
\newcommand{\W}{\Omega}
\newcommand{\norma}{\vert \vert }
\newcommand{\bignormal}{\left\vert \left\vert}
\newcommand{\bignormar}{\right\vert \right\vert}
\newcommand{\normale}{\triangleleft}
\newcommand{\nnorma}{\vert \vert \, \cdot \, \vert \vert}
\newcommand{\dt}{\, \mathrm{d}t}
\newcommand{\dz}{\, \mathrm{d}z}
\newcommand{\dx}{\, \mathrm{d}x}
\newcommand{\dy}{\, \mathrm{d}y}
\newcommand{\amma}{\gamma}
\newcommand{\inv}[1]{#1^{-1}}
\newcommand{\az}{\centerdot}
\newcommand{\ammasol}[1]{\tilde{\gamma}_{\tilde{#1}}}
\newcommand{\pror}[1]{\mathbb{P}^#1 (\R)}
\newcommand{\proc}[1]{\mathbb{P}^#1(\C)}
\newcommand{\sol}[2]{\widetilde{#1}_{\widetilde{#2}}}
\newcommand{\bsol}[3]{\left(\widetilde{#1}\right)_{\widetilde{#2}_{#3}}}
\newcommand{\norm}[1]{\left\vert\left\vert #1 \right\vert \right\vert}
\newcommand{\abs}[1]{\left\vert #1 \right\vert }
\newcommand{\ris}[2]{#1_{\vert #2}}
\newcommand{\vp}{\varphi}
\newcommand{\vt}{\vartheta}
\newcommand{\wt}[1]{\widetilde{#1}}
\newcommand{\pr}[2]{\frac{\partial \, #1}{\partial\, #2}}%derivata parziale
%per creare teoremi, dimostrazioni ... 
\theoremstyle{plain}
\newtheorem{thm}{Teorema}[section] 
\newtheorem{ese}[thm]{Esempio} 
\newtheorem{ex}[thm]{Esercizio} 
\newtheorem{fatti}[thm]{Fatti}
\newtheorem{fatto}[thm]{Fatto}

\newtheorem{cor}[thm]{Corollario} 
\newtheorem{lem}[thm]{Lemma} 
\newtheorem{al}[thm]{Algoritmo}
\newtheorem{prop}[thm]{Proposizione} 
\theoremstyle{definition} 
\newtheorem{defn}{Definizione}[section] 
\newcommand{\intt}[2]{int_{#1}^{#2}}
\theoremstyle{remark} 
\newtheorem{oss}{Osservazione} 
\newcommand{\di }{\, \mathrm{d}}
\newcommand{\tonde}[1]{\left( #1 \right)}
\newcommand{\quadre}[1]{\left[ #1 \right]}
\newcommand{\w}{\omega}

% diagrammi commutativi tikzcd
% per leggere la documentazione texdoc

\begin{document}
\textbf{Lezione del 30 Marzo}
\begin{cor}Su $C^\star$ la forma $\frac{\di z}{z}$ \`e chiusa ma non esatta
\proof $\forall z_0\in C^\star$ esiste un aperto $z_0\in U\subset \C^\star$ su cui \`e definita una branca $F$ del logaritmo, ora $\di F = F' \di z =\frac{\di z}{z}z$ su $U$ per cui tale forma \`e chiusa.\\
Come abbiamo osservato $\gamma(t)=e^{2\pi i t} $ \`e un loop ma $\int_\amma \frac{\di z }{z}\neq 0 $.\\
Dunque per caratterizzazione la forma non \`e esatta
\end{cor}
\begin{cor}Non esiste un "logaritmo" definito su tutto $\C^\star$, altrimenti $\frac{\di z }{z}$ sarebbe esatto su $\C^\star$
\end{cor}
\newpage
\section{Integrazioni su rettangoli}
\begin{oss}Un rettangolo $R$ in $\C$ \`e caratterizzato da $4$ vertici della forma
$$a_1 + i b_1 \quad a_2 + i b_1 \quad a_2 + ib_2  \quad a_1 + i b_2 $$
Possiamo parametrizzare il bordo del rettangolo con il cammino $\amma = \amma_1 \star \amma_2\star \amma_3 \star \amma_4$ 
dove 
$$\gamma_1(t) =a_1+t(a_2-a_1) \quad \gamma_2(t)=a_2 + i (b_1+t(b_2-b_1))$$
similmente si definisce $\amma_3$ e $\amma_4$.\\
D'ora in poi poniamo per ogni $1$-forma 
$$\int_{\partial R} \w = \int_\amma \w$$
Se $\w=P\di x + Q \di y$ poich\`e $\gamma_1'(t) = -\gamma_3'(t) =1 $ e $\gamma_2'(t) = -\gamma_4'(t)=i$ si ha
$$ \int_{\amma R} \w = \int_{a_1}^{a_2} P(t,b_1) \dt + \int_{b_1}^{b_2}Q(a_2, t) \dt -\int_{a_1}^{a_2}P(t,b_2)\dt - \int_{b_1}^{b_2}Q(a_1, t) \dt $$
\end{oss}

\begin{prop}Se $D=B(z_0,r)$ \`e il disco aperto e $\w$ \`e una $1$-forma su $D$
$$\w \text{ esatta } \quad \ses \quad  \int_{\partial R} \w = 0 \,\, \forall \text{ rettangolo } R \subseteq D $$
\proof $\implica$ $\partial R $ \`e un cammino chiuso, per quanto visto sulla caratterizzazioni delle forme esatte, l'integrale lungo un cammino chiuso di una forma esatta \`e nulla\\
$\iimplica$ Sia $z_0\in D$ qualsiasi.\\
 Costruiamo una primitiva, integrando $\w$ lungo un cammino differenziabile a tratti fatti da un tratto orizzontale seguito da un tratto verticale che collega $z_0$ a $z$.\\
$\forall z\in D$ sia $\amma_z$ un tale cammino (l'esistenza deriva dal fatto che $D$ \`e un disco). Pongo 
$$F(z) =\int_{\amma_z} \w$$
Poich\`e $\amma_z$ \`e univocamente determinato da $z$  (a meno di riparametrizzazioni), $F$ \`e ben definita.\\
Devo mostrare che se $\w=P\di x + Q\di y $ allora $\pr F x =  P$ e $\pr F y=Q$.\\
Il fatto che $\pr F y = Q$ segue dallo stesso ragionamento fatto per costruire la primitiva la volta scorsa (integrale lungo un loop generico).\\
Infatti se $z'=z+ih$ allora $\amma_{z+ih}= \gamma_z\star \gamma_h$ ($\amma_h$ \`e un cammino verticale) dunque
$$F(z+ih)-F(z)=\int_{\amma_h} \w = \int_0^h Q(z+it) \dt$$
Ora se dividiamo per $h$, il termine di destra tende a $Q(z)$ per $h\to 0$.\\
Per mostrare che $\pr F x=P$ devo usare l'ipotesi, ci\`o ci consente di dire che 
$$F(z) = \int_{\alpha_z} \w$$ 
dove $\alpha_z$ \`e il cammino da $z_0$ a $z$ che va prima in verticale e poi in orizzontale, infatti  $\amma_z\star \overline{\alpha}_z$ \`e il bordo di un rettangolo dunque 
$$0=\int_{\amma_z\star \overline{\alpha}_z} \w = \int_{\amma_z} \w - \int_{\alpha_z} \w \quad \implica \quad \int_{\amma_z} \w = \int_{\alpha_z}\w$$ 
A questo punto la dimostrazione che $\pr F x = P$ \`e identica a quella fatta per $\pr F y = Q$ usando gli $\alpha_z$ al posto di $\amma_z$\\
\endproof
\begin{oss}Per verificare l'esattezza di $\w$ su un disco, basta controllare i bordi dei rettangoli, per $D$ qualsiasi devo considerare tutti i lacci chiusi
\end{oss}
\end{prop}
\begin{cor}$D$ aperto qualsiasi, $\w$ una $1$-forma differenziale su $D$.
$$ \int_{\partial R } \w \, \forall \text{ rettangolo } R \subset D \quad \implica \quad \w \text{ chiusa}$$
\proof Dato $z_0\in D$, sia $U$ una palla centrata in $z_0$ (esiste in quanto un aperto \`e intorno di ogni suo punto).\\
Ora $\forall$ rettangolo $R\subset D$ si ha $\int_{\partial R} \w=0$, per cui per la proposizione precedente $\w$ ha una primitiva su $U$ dunque $\w$ \`e chiusa
\end{cor}
\newpage
\section{Integrazioni lungo curvo continue}
\begin{defn}Sia $\w$ una $1$-forma CHIUSA su un aperto $D\subset\C$.\\
Se $\amma:\, [0,1]\to D$ \`e una curva continua (non necessariamente $C^1$ a tratti), allora una primitiva di $\w$ lungo $\amma$ \`e una funzione $f:\, [0,1]\to \C$ tale che 
$$\forall t_0\in [0,1] \, \, \exists \varepsilon>0 \text{ e } U \text{ intorno di } \amma(t_0) \quad f(t_0) = F(\amma(t)) \, \, \forall t \in (t_0-\varepsilon,t_0+\varepsilon)$$
dove $F:\, U \to \C$ \`e una primitiva locale di $\w$
\end{defn}
\begin{prop}Nelle ipotesi di sopra, una primitiva lungo $\amma$ esiste.\\
Due primitive lungo $\amma$ differiscono per una costante.
\proof Mostriamo l'esistenza.\\
Per compattezza di $[0,1]$, posso considerare una suddivisione $0=t_0<t_1<\dots <t_n=1$di $[0,1]$ tale che $\gamma([t_i,t_{i+1}])\subseteq U_i$ per $i=0, \dots, n-1$, $U_i$ \`e una palla in $D$ su cui $\w$ ammette una primitiva.\\
Sia  $F_0:\, U_0\to \C$ e pongo 
$$f(t) =F_0(\amma(t))\quad \text{ per } t\in [t_0,t_1]$$
Scelgo $F_1:\, U_1 \to \C$ con $F_1(\amma(t_1))=F_0(\amma(t_0))$ (esiste perch\`e data $F_1$ una primitiva qualsiasi, ne ottengo un'altra tale che $F_1(\amma(t_1))=F_0(\amma(t_1))$ sommando un'opportuna costante). Pongo 
$$f(t) = F_1(\amma(t))\text{ per } t\in [t_1, t_2]$$
Iterando costruisco una funzione $f$ continua (lo \`e su $[t_i,t_{i+1}]$ ed \`e ben definita sui $t_i$) ed inoltre \`e una primitiva lungo $\amma$.\\
Mostriamo ora l'unicit\`a a meno di costanti.\\
Siano $f_1,f_2:[0,1]\to \C$ due primitive lungo $\amma$ di $\w$.\\
Allora per definizione di primitiva lungo $\amma$ si ha 
$$\forall t_0\in[0,1]\, \, \exists\varepsilon>0 \text{ e } U \text{ intorno connesso di } \amma(t_0) \text{ tali che } $$
$$f_1(t) = F(\amma(t)) \text{ e } f_2(t) =G(\amma(t))\text{ per } t \in (t-\varepsilon, t+\varepsilon)$$ dove $F,G:\, U\to \C$ sono primitive locali di $\w$ 
Essendo $F$ e $G$ primitive di $\w$  su un connesso  differiscono per una costante dunque anche $f_1$ e $f_2$ differiscono per una costante su $(t-\varepsilon, t+\varepsilon)$.\\
Ora essendo $[0,1]$ connesso allora $f_1-f_2$ \`e costante su tutto $[0,1]$
\end{prop}
\begin{defn}$\w$ una $1$-forma differenziale chiusa su $D$, $\amma:\, [0,1]\to D$ continua
$$\int_\amma \w = f(1) - f(0)$$
dove $f$ \`e una primitiva di $\w$ lungo $\amma$
\begin{oss}La definizione \`e ben posta in quanto se $g$ \`e un altra primitiva, $g$ e $f$ differiscono per una costante dunque $f(1)-f(0)=g(1) - g(0)$
\end{oss}
\end{defn}
\begin{fatti}Sia $\w$ una $1$-forma chiusa e $\gamma_1, \gamma_2, \amma$ cammini continui
\begin{enumerate}
\item 
$$\int_{\amma_1\star \amma_2} \w = \int_{\amma_1} \w + \int_{\amma_2} \w$$
\item $$\int_{\overline{\amma}}\w=-\int_{\amma}\w$$
\item Se $\amma$ \`e differenziabile a tratti allora riotteniamo la definizione gi\`a data.\\
Infatti prendendo una suddivisione $0=t_0< t_1< \dots <t_n=1$ allora $\forall i=0, \dots , n-1$  si ha $\amma_i =\ris \amma {[t_i,t_{i+1}]}$ \`e differenziabile e $\amma_i([t_1, t_{i+1}]) \subseteq U$ su cui $\w$ ha primitiva $F_i$.\\
Usando la vecchia definizione
$$\int_\amma \w = \sum_{i=0}^{n-1} \int_{\amma_i} \w = \sum_{i=0}^{n-1} F_i(\amma(t_{i+1})) - F_i(\amma(t_{i})) $$ 
il che \`e equivalente a $\int_\amma \w =\sum_{i=0}^n \int_{\amma_i} \w$ con la nuova definizione 
\end{enumerate}
\end{fatti}
\begin{thm}[Invarianza omotopica]\bianco 
Sia $\w$ una $1$-forma differenziale chiusa su $D\subseteq \C$ aperto.\\
Siano $\amma_1,\amma_2:\, [0,1]\to D$ due cammini 
$$\amma_1 \sim \amma_2 \quad \implica \quad \int_{\amma_1}\w = \int_{\amma_2}\w$$
\proof Sia $H:[0,1]\times [0,1]\to \C$ un'omotopia ad estremi fissi tra $\amma_1$ e $\amma_2$, voglio costruire $G:\, [0,1]\times[0,1]\to \C$ continua tale che
$$\forall (t_0,s_0) \in [0,1]^2 \, \, \exists U \text{ intorno connesso in D } di H(t_0,s_0) \text{ e } F:\, U \to \C \text{ primitiva locale di }  \w $$
con $ G(t,s) =F(H(t,s))$ per $(t,s)$ in un intorno di $(t_0,s_0)$ \\
(l'intento \`e quello di costruire una primitiva di $\w$ lungo $H$)\\
Per compattezza di $[0,1]\times [0,1]$ esistono suddivisioni 
$$0=t_0< t_1< \dots < t_n =1 $$
$$0=s_0< t_1< \dots < s_m =1 $$
tali che $H([t_i,t_{i+1}]\times[s_j,s_{j+1}])\subseteq U_{i,j}$ su cui $\w$ ammette una primitiva.\\
Fissiamo $s_0=0$ e  occupiamoci della prima riga di "quadratini".\\
Scelgo $F_{0,0}:\, U_{0,0}\to \C$ una primitiva di $\w$ su $U_{0,0}$ e pongo 
$$G(t,s)=F_{0,0}(H(t,s)) \text{ per } (t,s)\in [t_0,t_1]\times [s_0,s_1]$$
Tra tutte le primitive di $\w$ su $U_{1,0}$ scelgo quello che coincide con $F_{0,0}$ su $H(t_1,s_0)$ (tale primitiva \`e unica perch\`e $U_{1,0}$ \`e connesso) e la chiamo $F_{1,0}$.\\
\`E facile verificare che $F_{0,0}(t_1,s) = F_{1,0}(t_1,s)$ per $s\in [s_0,s_1]$ (sfruttando la connessione) dunque posso porre
$$G(t,s) = F_{1,0}(H(t,s)) \text{ per } (t,s)\in [ t_1, t_2]\times[s_0,s_1]$$
Proseguo cos\`i definendo $G$ su tutti i quadratini della forma $[t_i,t_{i+1}]\times [s_0,s_1]$ ottenendo $G$ definita in $[0,1]\times[s_0,s_1]$.\\
Poi proseguo passando alla seconda striscia orizzontale $[0,1]\times [s_1,s_2]$ cominciando dal quadratino $[t_0,t_1]\times [s_1,s_2]$ e proseguo in questo modo definendo $G$ su tutto $[0,1]\times [0,1]$.\\
Notiamo che $G(0,s)$ \`e una primitiva di $\w$ lungo il cammino costante $H(0,s)$ dunque \`e costante ($G(0,0)=G(0,1)$), in modo analogo $G(1,0)=G(1,1)$, inoltre $G(t,0)$ \`e una primitiva di $\w$ lungo $\amma_1 $ mentre $G(t,1)$ lo \`e lungo $\amma_2$ abbiamo dunque 
$$\int_{\amma_1} \w = G(1,0)-G(0,0) = G(1,1)-G(0,1)=\int_{\amma_2}\w$$
\endproof
\end{thm}
\begin{cor}$D\subseteq \C$ semplicemente connesso, $\w$ una $1$-forma su $D$
$$\w \text{ chiusa } \quad \ses \quad \w \text{ esatta} $$
\proof $\iimplica$ discende direttamente dalla definizione\\
$\implica$ Dato $\amma:[0,1]\to D$ dalla semplice connessione segue che $\amma\sim C_{\amma(0)}$ dunque per quanto appena visto 
$$\int_\amma \w = \int_{C_{\amma(0)}}=0 \quad \implica \quad \w \text{ esatta}$$ 
\begin{oss}Se $D$ \`e semplicemente connesso, riesco ad incollare le primitive locali di $\w$ per creare una primitiva globale.\\
Abbiamo visto che ci\`o pu\`o non capitare se $\pi_1(D) \neq \{ 1 \}$ (e.g $\w=\frac{\di z }{z}$ su $D=\C^\star$ in quanto $\pi_1(C^\star)=\Z$)
\end{oss}
\end{cor}
\end{document}