\documentclass[a4paper,12pt]{article}
\usepackage[a4paper, top=2cm,bottom=2cm,right=2cm,left=2cm]{geometry}

\usepackage{bm,xcolor,mathdots,latexsym,amsfonts,amsthm,amsmath,
					mathrsfs,graphicx,cancel,tikz-cd,hyperref,booktabs,caption,amssymb,amssymb,wasysym}
\hypersetup{colorlinks=true,linkcolor=blue}
\usepackage[italian]{babel}
\usepackage[T1]{fontenc}
\usepackage[utf8]{inputenc}
\newcommand{\s}[1]{\left\{ #1 \right\}}
\newcommand{\sbarra}{\backslash} %% \ 
\newcommand{\ds}{\displaystyle} 
\newcommand{\alla}{^}  
\newcommand{\implica}{\Rightarrow}
\newcommand{\iimplica}{\Leftarrow}
\newcommand{\ses}{\Leftrightarrow} %se e solo se
\newcommand{\tc}{\quad \text{ t. c .} \quad } % tale che 
\newcommand{\spazio}{\vspace{0.5 cm}}
\newcommand{\bbianco}{\textcolor{white}{,}}
\newcommand{\bianco}{\textcolor{white}{,} \\}% per andare a capo dopo 																					definizioni teoremi ...


% campi 
\newcommand{\N}{\mathbb{N}} 
\newcommand{\R}{\mathbb{R}}
\newcommand{\Q}{\mathbb{Q}}
\newcommand{\Z}{\mathbb{Z}}
\newcommand{\K}{\mathbb{K}} 
\newcommand{\C}{\mathbb{C}}
\newcommand{\F}{\mathbb{F}}
\newcommand{\p}{\mathbb{P}}

%GEOMETRIA
\newcommand{\B}{\mathfrak{B}} %Base B
\newcommand{\D}{\mathfrak{D}}%Base D
\newcommand{\RR}{\mathfrak{R}}%Base R 
\newcommand{\Can}{\mathfrak{C}}%Base canonica
\newcommand{\Rif}{\mathfrak{R}}%Riferimento affine
\newcommand{\AB}{M_\D ^\B }% matrice applicazione rispetto alla base B e D 
\newcommand{\vett}{\overrightarrow}
\newcommand{\sd}{\sim_{SD}}%relazione sx dx
\newcommand{\nvett}{v_1, \, \dots , \, v_n} % v1 ... vn
\newcommand{\ncomb}{a_1 v_1 + \dots + a_n v_n} %a1 v1 + ... +an vn
\newcommand{\nrif}{P_1, \cdots , P_n} 
\newcommand{\bidu}{\left( V^\star \right)^\star}

\newcommand{\udis}{\amalg}
\newcommand{\ric}{\mathfrak{U}}
\newcommand{\inclu}{\hookrightarrow }
%ALGEBRA

\newcommand{\semidir}{\rtimes}%semidiretto
\newcommand{\W}{\Omega}
\newcommand{\norma}{\vert \vert }
\newcommand{\bignormal}{\left\vert \left\vert}
\newcommand{\bignormar}{\right\vert \right\vert}
\newcommand{\normale}{\triangleleft}
\newcommand{\nnorma}{\vert \vert \, \cdot \, \vert \vert}
\newcommand{\dt}{\, \mathrm{d}t}
\newcommand{\dz}{\, \mathrm{d}z}
\newcommand{\dx}{\, \mathrm{d}x}
\newcommand{\dy}{\, \mathrm{d}y}
\newcommand{\amma}{\gamma}
\newcommand{\inv}[1]{#1^{-1}}
\newcommand{\az}{\centerdot}
\newcommand{\ammasol}[1]{\tilde{\gamma}_{\tilde{#1}}}
\newcommand{\pror}[1]{\mathbb{P}^#1 (\R)}
\newcommand{\proc}[1]{\mathbb{P}^#1(\C)}
\newcommand{\sol}[2]{\widetilde{#1}_{\widetilde{#2}}}
\newcommand{\bsol}[3]{\left(\widetilde{#1}\right)_{\widetilde{#2}_{#3}}}
\newcommand{\norm}[1]{\left\vert\left\vert #1 \right\vert \right\vert}
\newcommand{\abs}[1]{\left\vert #1 \right\vert }
\newcommand{\ris}[2]{#1_{\vert #2}}
\newcommand{\vp}{\varphi}
\newcommand{\vt}{\vartheta}
\newcommand{\wt}[1]{\widetilde{#1}}
\newcommand{\pr}[2]{\frac{\partial \, #1}{\partial\, #2}}%derivata parziale
%per creare teoremi, dimostrazioni ... 
\theoremstyle{plain}
\newtheorem{thm}{Teorema}[section] 
\newtheorem{ese}[thm]{Esempio} 
\newtheorem{ex}[thm]{Esercizio} 
\newtheorem{fatti}[thm]{Fatti}
\newtheorem{fatto}[thm]{Fatto}

\newtheorem{cor}[thm]{Corollario} 
\newtheorem{lem}[thm]{Lemma} 
\newtheorem{al}[thm]{Algoritmo}
\newtheorem{prop}[thm]{Proposizione} 
\theoremstyle{definition} 
\newtheorem{defn}{Definizione}[section] 
\newcommand{\intt}[2]{int_{#1}^{#2}}
\theoremstyle{remark} 
\newtheorem{oss}{Osservazione} 
\newcommand{\di }{\, \mathrm{d}}
\newcommand{\tonde}[1]{\left( #1 \right)}
\newcommand{\quadre}[1]{\left[ #1 \right]}
\newcommand{\w}{\omega}

% diagrammi commutativi tikzcd
% per leggere la documentazione texdoc

\begin{document}
\textbf{Lezione del 8 Novembre del Prof. Frigerio}
\begin{thm}[di Alexander]\bianco
Sia $X$ spazio topologico con una prebase $P$.\\
Se da ogni ricoprimento con elementi di $P$ si pu\`o estrarre un sottoricoprimento finito, $X$ \`e compatto
\proof Supponiamo per assurdo che $X$ non sia compatto.\\
Sia $\Omega$ l'insieme dei ricoprimenti aperti da cui non \`e possibile estrarre sottoricoprimenti finiti, e su $\Omega$ definiamo una relazione d'ordine tramite l'inclusione.\\
$X$ non compatto $\implica$ $\Omega\neq \emptyset$.\\  \\
Proviamo che ogni catena di $\Omega$ ammette un maggiorante.\\
Sia $C \subseteq \Omega$ una catena, cio\`e $C= \ds \{ \ric_i \}_{i\in I} $ dove gli $\ric_i$ sono ricoprimenti aperti in $\Omega$.\\
Pongo $\ric=\ds \bigcup_{i \in I } \ric_i$, sicuramente $\ric$ \`e un ricoprimento e $\ric > \ric_i $ $\forall i \in I $.\\
Proviamo che $\ric \in \Omega$.\\
Supponiamo, per assurdo, che da $\ric$ si estrae un sottoricoprimento finito ovvero
$$ \exists A_i, \dots, A_n \in \ric \quad X = A_1 \cup \dots \cup A_n \text{ dove } A_j \in \ric_{i(j)} \text{ per qualche } i(j)\in I $$
Poich\`e gli $\ric_{i(j)}$ appartengono ad una catena e sono finiti, esiste il massimo, poniamo $\ric_{\overline{i}}\in C$ tale massimo.\\
Allora $A_J \in \ric_{\overline{i}}$ $\forall j=1, \dots n $ dunque da $\ric_{\overline{i}}$ si estrae un sottoricoprimento finito, il che \`e assurdo. $ \ric \in \Omega$ \`e il maggiorante della catena cercato.\\ \\
Per il Lemma di Zorn, esiste un elemento $\ds Z =\{ Z_i\}_{i \in I } \in \Omega$ massimale, ovvero un ricoprimento $Y$ che contiene $Z$ e un aperto non contenuto in $Z$ non appartiene ad $\Omega$ dunque da $Y$ si estrae un sottoricoprimento finito.\\ \\
Mostriamo che $P \cap Z$ \`e un ricoprimento di $X$.\\
Mostrando ci\`o, concludo la dimostrazione  infatti da $P\cap Z$ non posso estrarre sottoricoprimenti finiti (non posso da $Z$). Dunque $P\cap Z $ \`e un ricoprimento con aperti di $P$ dal quale non posso estrarre sottoricoprimenti finiti, il che \`e assurdo per ipotesi.\\
Sia $x\in X$, per definizione di ricoprimento $\exists i \in I $ con $ x\in Z_i$.\\
Ora $Z_i$ \`e aperto, dunque per definizione di prebase, 
$$ \exists P_1, \dots, P_n \in P \quad x \in P_1 \cap \dots \cap P_n \subseteq Z_i$$
Mostriamo che un $P_j \in Z$.\\
Supponiamo, per assurdo, che $P_j \not \in Z $ $\forall j =1, \dots, n $ dunque per massimalit\`a di $Z$,  $Z\cup \{ P_j\} \not \in \Omega$ ovvero
$$ \exists I_j \subseteq	 I \text{ finito tale che } X = P_j \cup \bigcup_{i\in I_j} Z_i $$
Poich\`e ci\`o vale $\forall j$ allora
$$ X = \left( \bigcap_{i=1}^n P_j \right) \cup \left( \bigcup_{j=1}^n \bigcup_{i \in I_j }Z_i \right)$$ infatti 
se $x \not \in \bigcap P_j$ allora appartiene ad un certo $Z_i$ dove $\ds i \in \bigcup_{j=1}^n I_j$.\\
Ora $\ds \bigcap_{j=1}^n P_j \subseteq Z_i$ dunque abbiamo trovato un sottoricoprimento finito di $Z$, da cui $P\cap Z$ \`e un ricoprimento, da cui la tesi.
\endproof
\end{thm}
\newpage
\begin{thm}[di Tychonoff]\bianco
Sia $X_i$, $i\in I$ una famiglia di spazi topologici compatti
$$ X=\prod_{i \in I } X_i \text{ \`e compatto}$$
\proof Per il teorema di Alexander, basta vedere che ogni  ricoprimenti $\ric$ fatto con aperti di una prebase ammette un sottoricoprimento finito.\\
Scegliendo la prebase canonica, sia $\ric=\bigcup_{i\in I } A_i$ un ricoprimento di $X$ dove 
$$ A_i =\{ \pi_i^{-1}(D)\, \vert \, D\in \D_i\} \text{ dove } \D_i \text{ famiglia di aperti di } X_i$$
Ora $\exists i \in I $ per cui $\D_i$ \`e un  ricoprimento di $X_i$.\\
Supponiamo, per assurdo, che
$$ \forall i \in I \quad \exists \overline{x_i} \in X_i \quad x_i \not \in \bigcup_{D \in \D_i} D $$
dunque l'elemento $\ds (\overline{x_i}_{i \in I } $  non appartiene ad alcun elemento di $\ric$ il che \`e assurdo.\\
Dunque $\exists i_0 \in I $ tale che $\D_{i_0}$ \`e un ricoprimento di $X_{i_0}$, ma $\D_{i_0}$ ammette un sottoricoprimento finito $B_1, \dots, B_n $ dunque 
$$ \{ \pi_{i_0}^{-1}(B_1), \dots , \pi_{i_0}^{-1}(B_n)\}$$
\`e il sottoricoprimento di $\ric$ cercato.\endproof
\end{thm}
\spazio
\begin{ese}Sia $X$ un insieme, $A\subseteq \R$ 
$$ A^X=\{ f :\, X \to A\}$$
Dotando $A^X$ della topologia prodotto ($A$ con topologia euclidea) si ottiene la topologia della convergenza puntuale
\end{ese}
\begin{oss}Gli elementi di $A^X$, spesso, si denotano con $(a_x)_{x\in X}$, pensandoli come "stringhe di elementi di $A$"
\end{oss}
\begin{lem}Sia $\{f_n\}$ una successione di funzioni con $f_n:\, X \to A $.
$$f_n \to f \text{ in } A^X \quad \implica \quad f_n \to f \text{ puntualmente} $$
\proof Sia $x_0 \in X $. Per definizione di topologia prodotto, se $f(x_0)=a \in A$ allora
$$\forall \varepsilon>0 \quad  \{ (a_x)_{x\in X }\text{ con } \vert a_{x_0} -a \vert < \varepsilon \} = \pi_{x_0}^{-1}((a-\varepsilon, a+\varepsilon))$$ 
\`e un aperto, dunque un intorno di $f$.\\
Poich\`e $f_n \to f $ allora
$$ \exists n_0 \in \N \quad \forall n \geq n_0  \quad \vert f(x_{n_0}) - a\vert = \vert f_n(x_0) - f(x_0) \vert \leq \varepsilon $$ 
ovvero $f_n \to f $ puntualmente 
\endproof
\begin{oss}Vale anche il viceversa, verr\`a dimostrato nelle successive lezioni
\end{oss}
\end{lem}
\end{document}