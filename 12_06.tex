\documentclass[a4paper,12pt]{article}
\usepackage[a4paper, top=2cm,bottom=2cm,right=2cm,left=2cm]{geometry}

\usepackage{bm,xcolor,mathdots,latexsym,amsfonts,amsthm,amsmath,
					mathrsfs,graphicx,cancel,tikz-cd,hyperref,booktabs,caption,amssymb,amssymb,wasysym}
\hypersetup{colorlinks=true,linkcolor=blue}
\usepackage[italian]{babel}
\usepackage[T1]{fontenc}
\usepackage[utf8]{inputenc}
\newcommand{\s}[1]{\left\{ #1 \right\}}
\newcommand{\sbarra}{\backslash} %% \ 
\newcommand{\ds}{\displaystyle} 
\newcommand{\alla}{^}  
\newcommand{\implica}{\Rightarrow}
\newcommand{\iimplica}{\Leftarrow}
\newcommand{\ses}{\Leftrightarrow} %se e solo se
\newcommand{\tc}{\quad \text{ t. c .} \quad } % tale che 
\newcommand{\spazio}{\vspace{0.5 cm}}
\newcommand{\bbianco}{\textcolor{white}{,}}
\newcommand{\bianco}{\textcolor{white}{,} \\}% per andare a capo dopo 																					definizioni teoremi ...


% campi 
\newcommand{\N}{\mathbb{N}} 
\newcommand{\R}{\mathbb{R}}
\newcommand{\Q}{\mathbb{Q}}
\newcommand{\Z}{\mathbb{Z}}
\newcommand{\K}{\mathbb{K}} 
\newcommand{\C}{\mathbb{C}}
\newcommand{\F}{\mathbb{F}}
\newcommand{\p}{\mathbb{P}}

%GEOMETRIA
\newcommand{\B}{\mathfrak{B}} %Base B
\newcommand{\D}{\mathfrak{D}}%Base D
\newcommand{\RR}{\mathfrak{R}}%Base R 
\newcommand{\Can}{\mathfrak{C}}%Base canonica
\newcommand{\Rif}{\mathfrak{R}}%Riferimento affine
\newcommand{\AB}{M_\D ^\B }% matrice applicazione rispetto alla base B e D 
\newcommand{\vett}{\overrightarrow}
\newcommand{\sd}{\sim_{SD}}%relazione sx dx
\newcommand{\nvett}{v_1, \, \dots , \, v_n} % v1 ... vn
\newcommand{\ncomb}{a_1 v_1 + \dots + a_n v_n} %a1 v1 + ... +an vn
\newcommand{\nrif}{P_1, \cdots , P_n} 
\newcommand{\bidu}{\left( V^\star \right)^\star}

\newcommand{\udis}{\amalg}
\newcommand{\ric}{\mathfrak{U}}
\newcommand{\inclu}{\hookrightarrow }
%ALGEBRA

\newcommand{\semidir}{\rtimes}%semidiretto
\newcommand{\W}{\Omega}
\newcommand{\norma}{\vert \vert }
\newcommand{\bignormal}{\left\vert \left\vert}
\newcommand{\bignormar}{\right\vert \right\vert}
\newcommand{\normale}{\triangleleft}
\newcommand{\nnorma}{\vert \vert \, \cdot \, \vert \vert}
\newcommand{\dt}{\, \mathrm{d}t}
\newcommand{\dz}{\, \mathrm{d}z}
\newcommand{\dx}{\, \mathrm{d}x}
\newcommand{\dy}{\, \mathrm{d}y}
\newcommand{\amma}{\gamma}
\newcommand{\inv}[1]{#1^{-1}}
\newcommand{\az}{\centerdot}
\newcommand{\ammasol}[1]{\tilde{\gamma}_{\tilde{#1}}}
\newcommand{\pror}[1]{\mathbb{P}^#1 (\R)}
\newcommand{\proc}[1]{\mathbb{P}^#1(\C)}
\newcommand{\sol}[2]{\widetilde{#1}_{\widetilde{#2}}}
\newcommand{\bsol}[3]{\left(\widetilde{#1}\right)_{\widetilde{#2}_{#3}}}
\newcommand{\norm}[1]{\left\vert\left\vert #1 \right\vert \right\vert}
\newcommand{\abs}[1]{\left\vert #1 \right\vert }
\newcommand{\ris}[2]{#1_{\vert #2}}
\newcommand{\vp}{\varphi}
\newcommand{\vt}{\vartheta}
\newcommand{\wt}[1]{\widetilde{#1}}
\newcommand{\pr}[2]{\frac{\partial \, #1}{\partial\, #2}}%derivata parziale
%per creare teoremi, dimostrazioni ... 
\theoremstyle{plain}
\newtheorem{thm}{Teorema}[section] 
\newtheorem{ese}[thm]{Esempio} 
\newtheorem{ex}[thm]{Esercizio} 
\newtheorem{fatti}[thm]{Fatti}
\newtheorem{fatto}[thm]{Fatto}

\newtheorem{cor}[thm]{Corollario} 
\newtheorem{lem}[thm]{Lemma} 
\newtheorem{al}[thm]{Algoritmo}
\newtheorem{prop}[thm]{Proposizione} 
\theoremstyle{definition} 
\newtheorem{defn}{Definizione}[section] 
\newcommand{\intt}[2]{int_{#1}^{#2}}
\theoremstyle{remark} 
\newtheorem{oss}{Osservazione} 
\newcommand{\di }{\, \mathrm{d}}
\newcommand{\tonde}[1]{\left( #1 \right)}
\newcommand{\quadre}[1]{\left[ #1 \right]}
\newcommand{\w}{\omega}

% diagrammi commutativi tikzcd
% per leggere la documentazione texdoc

\begin{document}
\textbf{Lezione del 6 Dicembre del Prof. Frigerio}

\begin{ese}$GL_+(n,\R)$ si ritrae per deformazione su $SL(n,\R)$\\
Sia $A\in  GL_+(n,\R)$ con $\det A \geq 0$.\\
Consideriamo
$$ A(t)= tA + \frac{(1-t)A}{\sqrt[n]{\det A}} = \left( t + \frac{1-t}{\sqrt[n]{\det A}} \right)A$$ 
dunque $A(t) \in GL_+(n,\R)$.\\
Possiamo definire l'omotopia
$$ H :\, GL_+(n,\R) \\times [0,1] \to GL_+(n,\R) \qquad H(A,t)=A(t)$$
Osserviamo che $A\in SL(n, \R)$ se e solo se $H(A,t)=A  \, \forall t \in [0,1]$.\\
Dunque $H$ \`e l'omotopia tra $Id$ e la retrazione $r=H_0$\\
In modo analogo si prova che $GL_+(n, \R)$ si ritrae per deformazione su $SO(n)$
\end{ese}

\begin{ese}[Pettine infinito]Sia $X=\left( \R \times \{0\} \right) \cup \Q \times \R$ 

\begin{itemize}
\item $\{(0,0)\}$ \`e un retratto di deformazione di $X$ che perci\`o \`e contraibile.\\

Sia $H : \, X\to [0,1] \to X $ dato da 
$$ H((x,y),t) = \begin{cases} (x, (1-2t)y) \text{ se } t \in \left[ 0, \frac{1}{2} \right] \\
((2-2t)x, 0) \text{ se } t \in \left[ \frac{1}{2},1 \right]
\end{cases}$$

La funzione \`e ben definita e $H((x,y),t) \in X \, \forall (x,y) \in X$ e $\forall t \in [0,1]$.\\
$H$ \`e continua lo sono le restrizioni sui 2 intervalli su cui \`e definita (sono ricoprimento fondamentale).\\
$$H((x,y),0)=(x,y)\, \, \forall (x,y) \in X$$
$$H((x,y),1)=(0,0)\, \, \forall (x,y) \in X$$
$$H((0,0),t)=(0,0)\, \, \forall t \in [0,1]$$
Dunque possiamo concludere che $\{(0,0)\}$ \`e un retratto di deformazione di $X$ 
\item $\{ (0,1)\}$ non \`e un retratto di deformazione di $X$\\
Supponiamo per assurdo, che esista $$H: X \times [0,1]\to X$$ omotopia tra $Id$ e la costante $(0,1)$ tale che $H((0,1), t)=(0,1)\, \, \forall t \in [0,1]$.\\
Presa $x\neq 0$ i punti $(0,1)$ e $ (x,1)$ giacciono in componenti connesse distinte di $X \sbarra \{ \R \times \{ 0 \} \}$ (in particolare, giacciono in componenti connesse per archi distinte).\\
Ora se consideriamo l'arco $ \gamma_x:[0,1]\to X$ definita da $ \gamma_x(t)=H((x,1),t)$.\\
Tale arco \`e continuo e connette $(0,1)$ a $(x,1)$ dunque deve passare per la retta $\{ y=0\}$  dunque 
$$ \forall x \neq 0 \quad \exists t(x)\in (0,1) \text{  tale che } H((x,1),t(x)) = (x',0)$$
Consideriamo ora la successione $\{t_n\} \subseteq [0,1]$ definita da $t_n=t\left( \frac{1}{n} \right)$.\\
Essendo $[0,1]$ compatto per successioni, a meno di estrarre una sottosuccessione, posso supporre $t_n \to \overline{t}$.\\
Se pongo $y:\R^n \to \R$ la proiezione sulla seconda coordinata, dalla continnuit\`a di $H$ e $y$ ottengo
$$ 0 = \lim_{n \to \infty} y \left( H\left( \frac{1}{n},1 \right), t_n \right) =y \left( H((0,1),\overline{t}) \right)=y(0,1)=1$$
dove abbiamo utilizzato il fatto che $H((0,1),t)=(0,1)\, \, \forall t \in [0,1]$
\end{itemize}
\end{ese}
\newpage
\section{Gruppo fondamentale}
\begin{defn}Sia $X$ topologico, $a,b \in X$ allora definiamo con
$$ \Omega(a,b)=\{ \gamma :[0,1]\to X  \text{ continua con } \gamma(0)=a \, \, \gamma(1)=b\}$$
% archi con estremi a e b
\end{defn}
\begin{defn}$\alpha, \beta \in \Omega(a,b)$ sono omotopi (come cammini o a estremi fissi) se $$ \exists H : [0,1] \times [0,1] \to X $$ omotopia tra $\alpha$ e $\beta$ tale che $$H(0,t)=a \text{ e } H(1,t)=b \, \,\forall t \in [0,1]$$
\end{defn}
\begin{defn}Il gruppo fondamentale con punto base $a$ \`e l'insieme $$\pi_1(X,a) = \frac{\Omega(a,a)}{\sim}$$
dove $\sim$ indica la relazione di omotopia a estremi fissi.\\
Dove l'operazione \`e data dalla giunzione $\star$
\end{defn}
\begin{oss}Da ora utilizzeremo le seguenti notazioni 
\begin{itemize}
\item   $1_a\in \Omega(a,a)$ denota il cammino costante in $a$.
\item $\alpha\in \Omega(a,b)$ allora denotiamo con $\overline{\alpha} \in \Omega(b,a)$ il cammino $\overline{\alpha}(t)=\alpha(1-t)$
\end{itemize}
\end{oss}
La giunzione di cammini non \`e associativa, dunque l'omotopia di cammini serve per avere l'associativit\`a, grazie al seguente
\begin{lem}Sia $\varphi:\, [0,1]\to [0,1]$ continua e tale che $\varphi(0)=0$ e $\varphi(1)=1$.\\
$$ \gamma\in \Omega(a,b) \quad \implica \quad \gamma \sim \gamma \circ \varphi$$
\proof $H(t,s)=\gamma(st + (1-s) \varphi(t))$ \
\`e l'omotopia ad estremi fissi cercata \endproof
\end{lem}
\begin{cor}\label{opass} Se $ \alpha\in \Omega(a,b)$, $\beta \in \Omega(b,c)$ e $ \gamma \in \Omega(c,d)$ allora 
$$ (\alpha\star \beta) \star \gamma \sim \alpha\star ( \beta \star \gamma)$$
\proof Un cammino \`e la riparametrazione dell'altro
\end{cor}
\begin{lem}\label{opdef}$\alpha, \alpha' \in \Omega	(a,b)$ e $\beta, \beta' \in \Omega(b,c)$
$$ \alpha\sim \alpha' 
\text{ e } \beta \sim 
\beta' \quad \implica \quad \alpha\star \beta \sim \alpha' \star \beta' \text{ in } \Omega (a,c)$$
\proof Se $H$ e $K$  sonio rispettivamente  omotopia  a estremi fissi tra $\alpha$, $\alpha'$ e $\beta$, $\beta'$
$$ (t,s) \to \begin{cases} H(2t,s) \text{ se }  t \in \left[ 0, \frac{1}{2} \right] \\
K(2t-1,s)  \text{ se } t \in \left[ \frac{1}{2},1 \right]
\end{cases}$$
\`e l'omotopia cercata. \endproof
\end{lem}
\begin{cor}L'operazione
$$ \pi_1(X,a) \times \pi_1(X,a) \to \pi_1(X,a)$$
$$ ( [\alpha], [\beta] ) \to [ \alpha\star \beta]$$
\`e ben definita  (\ref{opdef}) e associativa (\ref{opass})
\end{cor}
\end{document}