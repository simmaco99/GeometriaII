\documentclass[a4paper,12pt]{article}
\usepackage[a4paper, top=2cm,bottom=2cm,right=2cm,left=2cm]{geometry}

\usepackage{bm,xcolor,mathdots,latexsym,amsfonts,amsthm,amsmath,
					mathrsfs,graphicx,cancel,tikz-cd,hyperref,booktabs,caption,amssymb,amssymb,wasysym}
\hypersetup{colorlinks=true,linkcolor=blue}
\usepackage[italian]{babel}
\usepackage[T1]{fontenc}
\usepackage[utf8]{inputenc}
\newcommand{\s}[1]{\left\{ #1 \right\}}
\newcommand{\sbarra}{\backslash} %% \ 
\newcommand{\ds}{\displaystyle} 
\newcommand{\alla}{^}  
\newcommand{\implica}{\Rightarrow}
\newcommand{\iimplica}{\Leftarrow}
\newcommand{\ses}{\Leftrightarrow} %se e solo se
\newcommand{\tc}{\quad \text{ t. c .} \quad } % tale che 
\newcommand{\spazio}{\vspace{0.5 cm}}
\newcommand{\bbianco}{\textcolor{white}{,}}
\newcommand{\bianco}{\textcolor{white}{,} \\}% per andare a capo dopo 																					definizioni teoremi ...


% campi 
\newcommand{\N}{\mathbb{N}} 
\newcommand{\R}{\mathbb{R}}
\newcommand{\Q}{\mathbb{Q}}
\newcommand{\Z}{\mathbb{Z}}
\newcommand{\K}{\mathbb{K}} 
\newcommand{\C}{\mathbb{C}}
\newcommand{\F}{\mathbb{F}}
\newcommand{\p}{\mathbb{P}}

%GEOMETRIA
\newcommand{\B}{\mathfrak{B}} %Base B
\newcommand{\D}{\mathfrak{D}}%Base D
\newcommand{\RR}{\mathfrak{R}}%Base R 
\newcommand{\Can}{\mathfrak{C}}%Base canonica
\newcommand{\Rif}{\mathfrak{R}}%Riferimento affine
\newcommand{\AB}{M_\D ^\B }% matrice applicazione rispetto alla base B e D 
\newcommand{\vett}{\overrightarrow}
\newcommand{\sd}{\sim_{SD}}%relazione sx dx
\newcommand{\nvett}{v_1, \, \dots , \, v_n} % v1 ... vn
\newcommand{\ncomb}{a_1 v_1 + \dots + a_n v_n} %a1 v1 + ... +an vn
\newcommand{\nrif}{P_1, \cdots , P_n} 
\newcommand{\bidu}{\left( V^\star \right)^\star}

\newcommand{\udis}{\amalg}
\newcommand{\ric}{\mathfrak{U}}
\newcommand{\inclu}{\hookrightarrow }
%ALGEBRA

\newcommand{\semidir}{\rtimes}%semidiretto
\newcommand{\W}{\Omega}
\newcommand{\norma}{\vert \vert }
\newcommand{\bignormal}{\left\vert \left\vert}
\newcommand{\bignormar}{\right\vert \right\vert}
\newcommand{\normale}{\triangleleft}
\newcommand{\nnorma}{\vert \vert \, \cdot \, \vert \vert}
\newcommand{\dt}{\, \mathrm{d}t}
\newcommand{\dz}{\, \mathrm{d}z}
\newcommand{\dx}{\, \mathrm{d}x}
\newcommand{\dy}{\, \mathrm{d}y}
\newcommand{\amma}{\gamma}
\newcommand{\inv}[1]{#1^{-1}}
\newcommand{\az}{\centerdot}
\newcommand{\ammasol}[1]{\tilde{\gamma}_{\tilde{#1}}}
\newcommand{\pror}[1]{\mathbb{P}^#1 (\R)}
\newcommand{\proc}[1]{\mathbb{P}^#1(\C)}
\newcommand{\sol}[2]{\widetilde{#1}_{\widetilde{#2}}}
\newcommand{\bsol}[3]{\left(\widetilde{#1}\right)_{\widetilde{#2}_{#3}}}
\newcommand{\norm}[1]{\left\vert\left\vert #1 \right\vert \right\vert}
\newcommand{\abs}[1]{\left\vert #1 \right\vert }
\newcommand{\ris}[2]{#1_{\vert #2}}
\newcommand{\vp}{\varphi}
\newcommand{\vt}{\vartheta}
\newcommand{\wt}[1]{\widetilde{#1}}
\newcommand{\pr}[2]{\frac{\partial \, #1}{\partial\, #2}}%derivata parziale
%per creare teoremi, dimostrazioni ... 
\theoremstyle{plain}
\newtheorem{thm}{Teorema}[section] 
\newtheorem{ese}[thm]{Esempio} 
\newtheorem{ex}[thm]{Esercizio} 
\newtheorem{fatti}[thm]{Fatti}
\newtheorem{fatto}[thm]{Fatto}

\newtheorem{cor}[thm]{Corollario} 
\newtheorem{lem}[thm]{Lemma} 
\newtheorem{al}[thm]{Algoritmo}
\newtheorem{prop}[thm]{Proposizione} 
\theoremstyle{definition} 
\newtheorem{defn}{Definizione}[section] 
\newcommand{\intt}[2]{int_{#1}^{#2}}
\theoremstyle{remark} 
\newtheorem{oss}{Osservazione} 
\newcommand{\di }{\, \mathrm{d}}
\newcommand{\tonde}[1]{\left( #1 \right)}
\newcommand{\quadre}[1]{\left[ #1 \right]}
\newcommand{\w}{\omega}

% diagrammi commutativi tikzcd
% per leggere la documentazione texdoc


\begin{document}
\textbf{Lezione del 4 Ottobre di Gandini.}
\begin{ex}[Topologia di Zariski]\bianco
Sia $\K$ un campo  e $\mathfrak{F} \subseteq \K[x_1, \dots, x_n]$ allora definiamo $$V(\mathfrak{F}) = \{ (a_1, \dots, a_n ) \in \K^n \, \vert \, f(a_1,\dots, a_n)=0 \quad \forall f\in \mathfrak{F}$$
ovvero $\mathfrak{F}$ \`e una famiglia di polinomi in $n$ indeterminate con coefficienti in $\K$ e $V(\mathfrak{F}$ \`e il luogo di zeri della famiglia di polinomi.\\
Definiamo una topologia su $\K^n$ nel seguente modo
$$ C\subseteq \K^n \text{ chiuso } \quad \ses \quad \exists \mathfrak{F}\subseteq \K[x_1, \dots, x_n] \quad C=V(\mathfrak{F})$$
ovvero i chiusi sono tutti e soli i luoghi di zeri di famiglie di polinomi.\\
Mostriamo che \`e una topologia
\begin{itemize}
\item $\emptyset=V(1) $ ovvero del polinomio sempre costante ad $1$\\
$\K^n=V(0)$ ovvero del polinomio costantemente nullo
\item $\
\ds \bigcap_{i\in I } V (\mathfrak{F_i})= V \left( \bigcup_{i \in I } \mathfrak{F_i} \right)$
\item $V(\mathfrak{F_1})\cup V(\mathfrak{F_2})=V (\mathfrak{F_1}\cdot \mathfrak{F_2}) $ dove $\mathfrak{F_1}\cdot \mathfrak{F_2}=\{ f_1f_2 \, \vert \, f_1\in \mathfrak{F_1} , \, f_2 \in \mathfrak{F_2} \}$\\
$\subseteq$ Sia $a\in V(\mathfrak{F_1})\cup V(\mathfrak{F_2})$ allora 
$$ f_1(a)=0 \quad \forall f_1 \in \mathfrak{F_1} \quad  f_2(a)=0 \quad \forall f_2 \in \mathfrak{F_2} \quad \implica \quad (f_1f_2)(a)= 0 \quad \forall f_1 \in \mathfrak{F_1} \,  \forall f_2 \in \mathfrak{F_2} $$
$\supseteq$ Sia $a\in \K^n \sbarra ( V(\mathfrak{F_1}) \cup  V(\mathfrak{F_2})$ allora
$$\exists f_1 \in \mathfrak{F_1} \, \, f_1(a)\neq 0 \quad \exists f_2 \in \mathfrak{F_2} \, \, f_2(a)\neq 0 \quad  \implica \quad (f_1f_2)(a)\neq 0 \quad \implica \quad a \not \in V(\mathfrak{F_1}\mathfrak{F_2}) $$
\end{itemize}
\end{ex}
\begin{oss}
$$D(f) =\{ a \in \K^n \, \vert \, f(a)\neq 0 \}$$
\`E un aperto in quanto $D(f)=\K^n\sbarra V(f)$\\
Una base per questa topologie \`e 
$$ \{ D(f) \, \vert \, f \in \K[x_1, \dots, x_n]\}$$
infatti $\ds \K^n\sbarra V(\mathfrak{F}) = \bigcup_{f\in \mathfrak{F}} D(f)$ 
\end{oss}
\begin{oss}Se prendiamo $\K=\R$ allora la topologia di Zariski \`e meno  fine della topologia euclidea infatti le funzioni polinomiali sono continue nella topologia quindi 
$$ D(f)=f^{-1}( \R \sbarra \{ 0\}) \text{ ora } \R \sbarra \{0\} \text{ \`e un aperto quindi } D(f) \in \tau_{eucl}$$
\end{oss}
\begin{oss}Per $n=1$ la topologia di Zariski \`e la topologia cofinita\\

Se $C$ \`e un chiuso nella topologia cofinita diverso da $\K$ allora $C$ \`e finito ovvero $C=\{ a_1, \dots , a_n\}$.\\
Presa $\mathfrak{F}=\{ x-a_1, \dots, x-a_n\}$ otteniamo $C=V(\mathfrak{F}$ dunque \`e un chiuso in Zariski.\\
Sia $C$ un chiuso nella topologia di Zariski diverso da $\K$ allora $C=V(\mathfrak{F})$ allora per $f\in \mathfrak{F}$ abbiamo $V(f)$ finito infatti $f$ ha al pi\`u $\deg f$ radici
\end{oss}
\newpage
\begin{ex}$\R$ con la topologia di Sorgenfray \`e primo-numerabile ma non secondo-numerabile\\
Sia $x\in \R$ allora $$\B_x=\left\{ \left[x, x+\frac{1}{n}\right) \, \vert \, n \in \N \right\}$$ \`e un sistema fondamentale di intorni per $x$ numerabile.\\
Mostriamo che $\R_S$ non \`e secondo numerabile.\\
Sia $\B$ una base di $\R_S$\\
Sia $a\in \R$ allora $[a,a+1) $ \`e aperto dunque \`e unione di elementi di $\B$ ovvero $$\exists B_a \in \B \quad a\in B_a\subseteq [a,a+1)$$ dunque
$\B \supset B'=\{ B_a \, \vert \, a \in \R\}$ ora $\B'$ non \`e numerabile essendo i $B_a$ disgiunti dunque anche $\B$ non \`e numerabile
\begin{oss}Abbiamo anche dimostrato che $\R_s$ non \`e metrizzabile, se $X$ \`e metrico:\\$X$ primo-numerabile $\implica $ $X$ secondo-numerabile
\end{oss}
\end{ex}
\spazio
\begin{oss} La topologia cofinita su un insieme più che numerabile non soddisfa il primo assioma di numerabilit\`a.
\proof Sia $x\in X$ e supponiamo che $\ds \{ U_n\}_{n \in \N}$ sia un sistema fondamentale di intorni dunque $ X\sbarra U_n $ \`e finito infatti 
$$U_n \text{ intorno } \implica \exists A_n \subseteq U_n\text{ aperto} \quad \implica \quad X\sbarra A_n \text{ chiuso dunque finito}$$
Ora $ X\sbarra A_n \subseteq X\sbarra U_n$ dunque finito.\\
$$ X\sbarra U_n \text{ finito } \implica \quad \bigcup_{n \in \N} X \sbarra U_n \text{ numerabile }$$
Ora essendo $X$ pi\`u che numerabile esiste $y$ nel complementare di $\ds \bigcup_{n\in \N} X \sbarra U_n$.\\
$X\sbarra \{ y\}$ \`e un aperto che contiene $x$ dunque  \`e un intorno di $x$ da cui 
$$ \exists n \quad U_n \subseteq U$$
Inoltre $y\in X\sbarra U \subseteq X\sbarra U_n$ \\
$$ X\sbarra U =\{y \}\subseteq X\sbarra U_n \not \ni  y$$
ma ci\`o \`e un assurdo
\end{oss}
\newpage
\section{Intorni e continuit\`a}
\begin{defn}Sia $f:\, X \to Y$ una funzione tra spazi topologici e sia $x_0\in X$.\\
$f$ si dice continua in $x_0$ se 
$$ \forall V \text{ intorno di } f(x_0) \quad \exists U \text{ intorno di } x_0 \quad f(U)\subseteq V$$
ovvero 
$$ \forall V \text{ intorno di } f(x_0) \quad f^{-1} (V) \text{ \`e intorno di } x_0 $$
\end{defn}
\begin{prop}
$$ f\text{ continua } \quad \ses\quad f \text{ continua in ogni suo punto}$$
dove $f$ continua \`e intesa con la definizione "La controimmagine di aperti \`e un aperto"
\proof $\iimplica$ Sia $V$ un aperto 
$$ f^{-1}(V)\text{ aperto } \ses f^{-1}(V)\in I(x) \quad \forall x\in f^{-1}(V)$$
infatti $A$ \`e aperto se e solo se \`e intorno di ogni suo punto.\\
$\forall x \in f^{-1}(V)$
$$ V\text{ aperto }  \quad \implica \quad V\in I(f(x)) \quad \implica \quad f^{-1}(V) \in I(x)$$
$\implica$ Sia $x_0\in X$.\\
Sia $V\in I(f(x_0))$ dobbiamo provare che $f^{-1}(V)\in I(x_0)$
$$ V \in I(f(x_0)) \quad \implica \quad \exists V'\subseteq Y \text{ aperto } \quad f(x_0) \in V'\subseteq V$$
Ora $f^{-1}(V')$ essendo $f$ continua \`e un aperto quindi 
$$ x_0 \in f^{-1}(V') \subseteq f^{-1}(V) \quad \implica \quad f^{-1}(V) \in I(x_0)$$
\endproof
\end{prop}
\newpage
\begin{defn}Sia $X$ uno spazio topologico e $\ds \{ x_n \}_{n \in \N}$ una successione allora
$$ \{ x_n\} \text{ converge a } x \in X \text{ se } \forall V \text{ intorno di } x \quad \exists n_0 \tc x_n \in V \quad \forall n \geq n_0$$
\end{defn}
\begin{prop}Sia $X$ primo-numerabile e sia $C\subseteq X$ allora
$$ C \text{ chiuso } \quad \ses \quad \text{ per ogni successione } \{ x_n \} \subseteq C  \text{ convergente a } x \in X \text{ allora } x\in C $$
\proof $$\overline{C}=\{ x\in X \, \vert \, U \cap C \neq \emptyset \quad \forall U \in I(x)\}$$
$\implica$ sia $\{ x_n \} $ convergente a $x_0\in X$.\\
Basta dimostrare che $x_0 \in C = \overline{C}$ ovvero che $U \cap C \neq \emptyset $ con $U \in I(x_0)$.\\
Dalla definizione di convergenza
$$ \forall U \in I(x_0) \quad \exists n_0 \quad x_n \in U \quad \forall n \geq n_0$$
dunque $x_n\in U $ per infiniti $n$, ora la successione ha valori in $C$ dunque $U_n \cap C \neq \emptyset$\\
$\iimplica$ Sia $x\in \overline{C}$, vediamo che $x\in C$.\\
Per ipotesi basta costruire una successione $\{x_n \} \subseteq C$ convergente a $x$.\\
Sia $\ds\{ U_n\}_{n \in \N}$ un sistema fondamentale numerabile di intorni di $x_0$.\\
$$V_n =\ds \bigcap_{i=1}^n U_i \quad \implica \{ V_n \} \text{ \`e un sistema fondamentale di } x_0 \text{ con } V_1  \supseteq V_2 \supseteq \dots $$
Poich\`e $x\in \overline{C}$ allora $V_n \cap C \neq \emptyset \quad \forall n \in \N$.\\
Sia $x_n \in V_n \cap C$ e sia $\{ x_n\}$ la successione cos\`i costruita allora
\begin{itemize}
\item $\{ x_n \} \subseteq C $
\item $\{x_n \} $ converge a $x$ infatti
$$ \forall U \in I(x) \quad \exists n_0 \quad V_{n_0} \subseteq U \text{ allora } x_n \in V_n \subseteq V_{n_0} \subseteq U \quad \forall n \geq n_0$$
\end{itemize}
\endproof
\end{prop}
\begin{prop}Sia $X$ primo-numerabile e sia $A\subseteq X$ allora
$$ A \text{ aperto } \quad \ses \quad \text{ per ogni successione } \{ x_n \} \subseteq X  \text{ convergente a } x \in A \text{ allora } x_n\in A \quad \forall n \geq n_0 $$
\end{prop}
\begin{prop}
 Siano $X,Y$ spazi topologici primo-numerabile e $f:\, X \to Y$ 
$$ f \text{ continua } \quad \ses \quad \forall \{ x_n \} \subseteq X \text{ convergente a } x \in X \text{ allora } \{ f(x_n)\} \subseteq \text{ \`e convergente a } f(x) \in Y$$
\end{prop}
\end{document}