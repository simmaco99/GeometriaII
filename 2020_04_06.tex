
\documentclass[a4paper,12pt]{article}
\usepackage[a4paper, top=2cm,bottom=2cm,right=2cm,left=2cm]{geometry}

\usepackage{bm,xcolor,mathdots,latexsym,amsfonts,amsthm,amsmath,
					mathrsfs,graphicx,cancel,tikz-cd,hyperref,booktabs,caption,amssymb,amssymb,wasysym}
\hypersetup{colorlinks=true,linkcolor=blue}
\usepackage[italian]{babel}
\usepackage[T1]{fontenc}
\usepackage[utf8]{inputenc}
\newcommand{\s}[1]{\left\{ #1 \right\}}
\newcommand{\sbarra}{\backslash} %% \ 
\newcommand{\ds}{\displaystyle} 
\newcommand{\alla}{^}  
\newcommand{\implica}{\Rightarrow}
\newcommand{\iimplica}{\Leftarrow}
\newcommand{\ses}{\Leftrightarrow} %se e solo se
\newcommand{\tc}{\quad \text{ t. c .} \quad } % tale che 
\newcommand{\spazio}{\vspace{0.5 cm}}
\newcommand{\bbianco}{\textcolor{white}{,}}
\newcommand{\bianco}{\textcolor{white}{,} \\}% per andare a capo dopo 																					definizioni teoremi ...


% campi 
\newcommand{\N}{\mathbb{N}} 
\newcommand{\R}{\mathbb{R}}
\newcommand{\Q}{\mathbb{Q}}
\newcommand{\Z}{\mathbb{Z}}
\newcommand{\K}{\mathbb{K}} 
\newcommand{\C}{\mathbb{C}}
\newcommand{\F}{\mathbb{F}}
\newcommand{\p}{\mathbb{P}}

%GEOMETRIA
\newcommand{\B}{\mathfrak{B}} %Base B
\newcommand{\D}{\mathfrak{D}}%Base D
\newcommand{\RR}{\mathfrak{R}}%Base R 
\newcommand{\Can}{\mathfrak{C}}%Base canonica
\newcommand{\Rif}{\mathfrak{R}}%Riferimento affine
\newcommand{\AB}{M_\D ^\B }% matrice applicazione rispetto alla base B e D 
\newcommand{\vett}{\overrightarrow}
\newcommand{\sd}{\sim_{SD}}%relazione sx dx
\newcommand{\nvett}{v_1, \, \dots , \, v_n} % v1 ... vn
\newcommand{\ncomb}{a_1 v_1 + \dots + a_n v_n} %a1 v1 + ... +an vn
\newcommand{\nrif}{P_1, \cdots , P_n} 
\newcommand{\bidu}{\left( V^\star \right)^\star}

\newcommand{\udis}{\amalg}
\newcommand{\ric}{\mathfrak{U}}
\newcommand{\inclu}{\hookrightarrow }
%ALGEBRA

\newcommand{\semidir}{\rtimes}%semidiretto
\newcommand{\W}{\Omega}
\newcommand{\norma}{\vert \vert }
\newcommand{\bignormal}{\left\vert \left\vert}
\newcommand{\bignormar}{\right\vert \right\vert}
\newcommand{\normale}{\triangleleft}
\newcommand{\nnorma}{\vert \vert \, \cdot \, \vert \vert}
\newcommand{\dt}{\, \mathrm{d}t}
\newcommand{\dz}{\, \mathrm{d}z}
\newcommand{\dx}{\, \mathrm{d}x}
\newcommand{\dy}{\, \mathrm{d}y}
\newcommand{\amma}{\gamma}
\newcommand{\inv}[1]{#1^{-1}}
\newcommand{\az}{\centerdot}
\newcommand{\ammasol}[1]{\tilde{\gamma}_{\tilde{#1}}}
\newcommand{\pror}[1]{\mathbb{P}^#1 (\R)}
\newcommand{\proc}[1]{\mathbb{P}^#1(\C)}
\newcommand{\sol}[2]{\widetilde{#1}_{\widetilde{#2}}}
\newcommand{\bsol}[3]{\left(\widetilde{#1}\right)_{\widetilde{#2}_{#3}}}
\newcommand{\norm}[1]{\left\vert\left\vert #1 \right\vert \right\vert}
\newcommand{\abs}[1]{\left\vert #1 \right\vert }
\newcommand{\ris}[2]{#1_{\vert #2}}
\newcommand{\vp}{\varphi}
\newcommand{\vt}{\vartheta}
\newcommand{\wt}[1]{\widetilde{#1}}
\newcommand{\pr}[2]{\frac{\partial \, #1}{\partial\, #2}}%derivata parziale
%per creare teoremi, dimostrazioni ... 
\theoremstyle{plain}
\newtheorem{thm}{Teorema}[section] 
\newtheorem{ese}[thm]{Esempio} 
\newtheorem{ex}[thm]{Esercizio} 
\newtheorem{fatti}[thm]{Fatti}
\newtheorem{fatto}[thm]{Fatto}

\newtheorem{cor}[thm]{Corollario} 
\newtheorem{lem}[thm]{Lemma} 
\newtheorem{al}[thm]{Algoritmo}
\newtheorem{prop}[thm]{Proposizione} 
\theoremstyle{definition} 
\newtheorem{defn}{Definizione}[section] 
\newcommand{\intt}[2]{int_{#1}^{#2}}
\theoremstyle{remark} 
\newtheorem{oss}{Osservazione} 
\newcommand{\di }{\, \mathrm{d}}
\newcommand{\tonde}[1]{\left( #1 \right)}
\newcommand{\quadre}[1]{\left[ #1 \right]}
\newcommand{\w}{\omega}

% diagrammi commutativi tikzcd
% per leggere la documentazione texdoc

\begin{document}
\textbf{Lezione del 4 aprile}
\section{Indice di avvolgmento}
Fissato $a\in \C$, sappiamo che $\C\setminus \{a  \}$ si ritrae per deformazione su un cerchio di raggio $1$ e centro $a$, preso $x_0$ sul bordo del cerchio ($x_0\in \partial B(a,1)$), fissiamo un isomorfismo canonico
$$ f:\, \pi_1(\C\setminus \{a \}, x_0) \to \Z \quad [\gamma] \to 1 $$
dove $\gamma$ \`e una parametrizzazione di $\partial B(a,1)$ che percorre la circonferenza in senso antiorario.\\
Come gi\`a osservato esiste  una bigezione naturale
$$ \Omega(x_0,x_0) \to \W(S^1, x_0) \quad \gamma \to \hat{\gamma}$$
Questa bigezione induce un omomorfismo 
$$\partial:\, \pi_1(\C\setminus\{ a \}, x_0) \to [ S^1, \C \setminus\{ a \} \quad [\gamma]\to [\hat{\gamma}]$$
dove con $[S', \C \setminus \{a \}]$ intendiamo le classi di omotopie di mappe continue $S^1\to \C\setminus\{a\}$
Tale mappa risulta suriettiva ed inoltre $\partial([\gamma]) = \partial([\gamma'])$ se e solo se $[\gamma]$ e $[\gamma']$ sono coniugati.\\
Ora $\pi_1(\C\setminus\{a\}, x_0)$ \`e abeliano, dunque la mappa \`e iniettiva.\\
Definiamo per composizione la mappa 
$$ \psi:\,  \partial^{-1}\circ f:\,[S^1, \C\setminus\{a\}] \to \Z$$
\begin{defn}Sia $\gamma:\, [0,1]\to \C\setminus \{a\}$ un cammino chiuso in $x_0$.\\
Denotiamo indice di $\gamma$ rispetto ad $a$ l'intero $\psi([\hat{\gamma}])$ e lo denotiamo con $I(\gamma, a)$
\end{defn}
\begin{thm}Sia $\gamma:\, [0,1]\to \C\setminus\{a\}$ un cammino chiuso. Allora 
$$ I(\gamma, a) = \frac{1}{2\pi 1 }\int_\gamma \frac{dz}{z-a}$$
\proof Come sappiamo $\frac{1}{z-a}$ \`e olomorfa in $\C\setminus \{a\}$ dunque la forma $\w=\frac{\dz}{z-a}$ \`e chiusa.\\
Possiamo integrare $\w$ lungo curve $\gamma$ continue, e il valore dell'integrale non dipende dal rappresentante nella classe di omotopia, dunque \`e ben definita la funzione 
$$ \varphi:\, [S^1, \C\setminus\{a \}] \to \C$$ 
$$ [\hat{\gamma}]\to \frac{1}{2\pi i }\int_\gamma \w $$
Mostriamo che $\varphi = \psi$ il che conclude la dimostrazione.\\
Notiamo che 
$$ [S^1, \C\setminus\{ a\}]  =\{ [\hat{\gamma_n}]\, : \, \gamma_n:\, [0,1]\to \C\setminus \{a\} \text{ dove } t \to a + e^{2\pi i n t}\}$$
dunque $\psi([\hat{\gamma_n}])  =n$ (avvolto $n$ volte).\\
Ora abbiamo visto che 
$$ \int_{\gamma_n} \frac{\dz}{z-a} =2\pi i n $$ 
da cui la tesi 
\end{thm}
\begin{oss}Banalmente, 2 curve chiuse continue in $\C\setminus\{a \}$ liberamente omotope hanno lo stesso indice
\end{oss}
Vediamo come varia l'indice al variare del punto $a$
\begin{prop}Sia $\gamma:\, [0,1]\to \C$ una curva chiusa continua.\\
Allora la funzione 
$$ \C\setminus  Im \gamma \to \C \quad a\to I(\gamma,a)$$
\`e localmente costante (costante su ogni componente connessa del  dominio)
\proof Fissato $a$ nel dominio. Mostriamo che $\forall h \in \C$ sufficientemente piccolo si ha $I(\gamma, a+h) = I(\gamma, a)$.\\
Sia $$0<\gamma_0< \min \{ | \gamma(t) - a| \, :\, t\in [0,1]\}$$
\`e ben posto in quanto $|\gamma(t) - a|  \neq  0 $ poich\`e $a\not \in Imm \gamma $.\\
Per ogni $h\in \C$ con $\abs{h}\leq \gamma_0$ abbiamo 
$$ I(\gamma, a+h) = \frac{1}{2\pi i }\int_{\gamma } \frac{\dz}{z-(a+h)} =\frac{1}{2\pi i }\int_\gamma \frac{\dz}{(z-h)-a}$$
Con il cambio di variabili $z'= z-h$ otteniamo 
$$ I(\gamma, a +h) = \frac{1}{2\pi i }\int_{\gamma '} \frac{\dz'}{z'-a} \text{ dove } \gamma'(t) =\gamma(t) + h$$
Il che conclude la dimostrazione infatti 
$$ F(t,s) = \gamma(t) - sh $$ 
\`e un omotopia tra $\gamma $ e $\gamma'$
\end{prop}
\spazio
\begin{prop}Sia $a\in \C$ e sia $\gamma$ una curva chiusa continua tale che $Im\gamma \subseteq D \subseteq \C \setminus \{ a \}$ dove $D$ aperto semplicemente connesso.\\
Allora $I(\gamma,a) =0$
\proof  La forma $\w=\frac{\dz}{z-a}$ \`e chiusa in un semplicemente connesso, dunque esatta da cui 
$$ I(\gamma,a) = \int_\gamma \w = 0$$
\endproof
\end{prop}
\spazio
\begin{prop}Sia $\gamma$ una curva chiusa continua e sia $a\in \C\setminus Imm \gamma$.\\
$I(\gamma,a) = 0$ per ogni $a$ in una componente connessa illimitata di $\C\setminus Imm \gamma$
\end{prop}
\begin{ese}Sia $\gamma:\, t\to Re^{it} $ con $R>0$ allora
\begin{itemize}
\item $\abs z <R$ allora $I(\gamma, z) =1$
\item $\abs z >R$ allora $I(\gamma,z) = 0$
\end{itemize}
\end{ese}
\spazio
\begin{prop}Sia 
$$ f:\, \{ z\in \C \, \vert \,\abs z \leq R \} \to \C$$
una mappa continua e sia $\gamma(t) = f(Re^{2\pi i t})$ per $t\in [0,1]$.\\
Se $a\not\in Imm\gamma$ e $I(\gamma,a) \neq 0 $ allora esiste $z$ con $\abs z <R$ tale che $f(z) = a$
\proof Assumiamo, per assurdo $f(z) \neq a$ per ogni $\abs z <R$, dunque di conseguenza $f(z) \neq a$ per ogni $\abs \leq R$ (abbiamo supposto $a\not \in Imm \gamma$).\\
Definiamo 
$$ F(t,s) = f(tRe^{2\pi i }) \quad \forall t,s\in [0,1]$$
$F$ \`e un omotopia tra $\gamma $ e il cammino costante $f(0)$ dunque $\gamma $ \`e omotop ad un cammino costante da cui 
$$ \int_\gamma \frac{\dz}{z-a} = 0$$
contro l'ipotesi $I(\gamma, a) \neq 0 $
\end{prop}
\newpage
\begin{defn}Siano $\gamma_1, \gamma_2$ due curve continue, allora
$$ \gamma_1\gamma_2:\, t \to \gamma_1(t) \gamma_2(t)$$
$$ \gamma_1+ \gamma_2 :\, t\to \gamma_1(t)+\gamma_2(t)$$
\end{defn}
\begin{thm}Siano $\gamma_1,\gamma_2$ due curve continue chiuse con $0\not \in Imm \gamma_1, \gamma_2$ allora 
$$I(\gamma_1\gamma_2,0) =I(\gamma_1,0)+ I(\gamma_2,0)$$
\proof La forma $\w=\frac{\dz}{z}$ ammette una primitiva locale, sia 
$$f_i:\, [0,1]\to \C \quad e^{f_i(t)} =\gamma_i(t)$$
allora $$\gamma_1\gamma_2(t) =e^{f_1(t)} e^{f_2(t)} = e^{f_1(t) + f_2(t)}$$
dunque $f=f_1+f_2$ \`e una primitiva di $\w$ lungo $\gamma_1\gamma_2$ da cui 
$$ I(\gamma_1\gamma_2,0) = \frac{f_1(1) +f_2(1) - f_1(0) - f_2(0)}{2\pi i} = I(\gamma_1,0) +I(\gamma_2,0)$$
\end{thm}
\begin{thm}Siano $\gamma,\gamma_1$ curve chiuse continue tali che $0\not \in Imm\gamma_1,\gamma$.\\
Assumiamo che $0\leq \abs{\gamma_1(t)}\leq\abs \gamma(t)$ per ogni $t\in [0,1]$. Allora
$$ I(\gamma_1+\gamma,0) = I(\gamma,0)$$
\proof 
$$ \gamma(t) +\gamma_1(t) = \gamma(t) \tonde{1+\frac{\gamma_1(t)}{\gamma(t)}}  =\gamma(t) \beta(t)$$
dunque 
$$ I(\gamma_1+\gamma,0) = I(\gamma\beta,0) =I(\gamma,0) + I(\beta,0)$$
Mostriamo che $I(\beta,0)=0$ infatti $Imm \beta \subseteq D(1,1)$ dunque $Im \beta $ contenuto in un aperto  semplicemente connesso .\endproof
\end{thm}
\spazio
\begin{thm}[Formula integrale di Cauchy]\bianco
Sia $D\subseteq \C$ un aperto con $a\in D$ sia 
$$ \gamma:\, [0,1]\to D\setminus \{a\}$$ 
un cammino chiuso omotopicamente banale in $D$ e tale che $a\not \in Imm \gamma$.\\
Sia $f:\, D\to \C$ olomorfa allora 
$$ \frac{1}{2\pi i }\int_\gamma \frac{f(z)}{z-a}\dz = I(\gamma,a) f(a)$$
\proof Per ogni $z\in D$ definisco la funzione 
$$ g(z) =\begin{cases}\frac{f(z) - f(a)}{z-a} \text{ se } z\neq a \\
f(a) \text{ se } z=a
\end{cases}$$
essendo $f$ olomorfa, $g$ \`e continua in $a$ ed \`e olomorfa in $D\setminus r$ con $r$ la retta orizzontale che passa per $a$.\\
Per un teorema visto $g(z) \dz$ \`e chiusa in $D$.\\
Visto che $\gamma$ \`e omotopicamente banale in $D$  e $g(z) \dz $ \`e chiusa si ha 
$$ 0 =\int_\gamma g(z) \dz= \int_\gamma \frac{f(z) -f(a)}{z-a}\dz = \int_\gamma \frac{f(z)}{z-a}\dz - f(a) \int_\gamma \frac{1}{z-a}\dz $$
dividendo per $2\pi i$ si ha la tesi \endproof 
\end{thm}
\end{document}