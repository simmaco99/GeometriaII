\documentclass[a4paper,12pt]{article}
\usepackage[a4paper, top=2cm,bottom=2cm,right=2cm,left=2cm]{geometry}

\usepackage{bm,xcolor,mathdots,latexsym,amsfonts,amsthm,amsmath,
					mathrsfs,graphicx,cancel,tikz-cd,hyperref,booktabs,caption,amssymb,amssymb,wasysym}
\hypersetup{colorlinks=true,linkcolor=blue}
\usepackage[italian]{babel}
\usepackage[T1]{fontenc}
\usepackage[utf8]{inputenc}
\newcommand{\s}[1]{\left\{ #1 \right\}}
\newcommand{\sbarra}{\backslash} %% \ 
\newcommand{\ds}{\displaystyle} 
\newcommand{\alla}{^}  
\newcommand{\implica}{\Rightarrow}
\newcommand{\iimplica}{\Leftarrow}
\newcommand{\ses}{\Leftrightarrow} %se e solo se
\newcommand{\tc}{\quad \text{ t. c .} \quad } % tale che 
\newcommand{\spazio}{\vspace{0.5 cm}}
\newcommand{\bbianco}{\textcolor{white}{,}}
\newcommand{\bianco}{\textcolor{white}{,} \\}% per andare a capo dopo 																					definizioni teoremi ...


% campi 
\newcommand{\N}{\mathbb{N}} 
\newcommand{\R}{\mathbb{R}}
\newcommand{\Q}{\mathbb{Q}}
\newcommand{\Z}{\mathbb{Z}}
\newcommand{\K}{\mathbb{K}} 
\newcommand{\C}{\mathbb{C}}
\newcommand{\F}{\mathbb{F}}
\newcommand{\p}{\mathbb{P}}

%GEOMETRIA
\newcommand{\B}{\mathfrak{B}} %Base B
\newcommand{\D}{\mathfrak{D}}%Base D
\newcommand{\RR}{\mathfrak{R}}%Base R 
\newcommand{\Can}{\mathfrak{C}}%Base canonica
\newcommand{\Rif}{\mathfrak{R}}%Riferimento affine
\newcommand{\AB}{M_\D ^\B }% matrice applicazione rispetto alla base B e D 
\newcommand{\vett}{\overrightarrow}
\newcommand{\sd}{\sim_{SD}}%relazione sx dx
\newcommand{\nvett}{v_1, \, \dots , \, v_n} % v1 ... vn
\newcommand{\ncomb}{a_1 v_1 + \dots + a_n v_n} %a1 v1 + ... +an vn
\newcommand{\nrif}{P_1, \cdots , P_n} 
\newcommand{\bidu}{\left( V^\star \right)^\star}

\newcommand{\udis}{\amalg}
\newcommand{\ric}{\mathfrak{U}}
\newcommand{\inclu}{\hookrightarrow }
%ALGEBRA

\newcommand{\semidir}{\rtimes}%semidiretto
\newcommand{\W}{\Omega}
\newcommand{\norma}{\vert \vert }
\newcommand{\bignormal}{\left\vert \left\vert}
\newcommand{\bignormar}{\right\vert \right\vert}
\newcommand{\normale}{\triangleleft}
\newcommand{\nnorma}{\vert \vert \, \cdot \, \vert \vert}
\newcommand{\dt}{\, \mathrm{d}t}
\newcommand{\dz}{\, \mathrm{d}z}
\newcommand{\dx}{\, \mathrm{d}x}
\newcommand{\dy}{\, \mathrm{d}y}
\newcommand{\amma}{\gamma}
\newcommand{\inv}[1]{#1^{-1}}
\newcommand{\az}{\centerdot}
\newcommand{\ammasol}[1]{\tilde{\gamma}_{\tilde{#1}}}
\newcommand{\pror}[1]{\mathbb{P}^#1 (\R)}
\newcommand{\proc}[1]{\mathbb{P}^#1(\C)}
\newcommand{\sol}[2]{\widetilde{#1}_{\widetilde{#2}}}
\newcommand{\bsol}[3]{\left(\widetilde{#1}\right)_{\widetilde{#2}_{#3}}}
\newcommand{\norm}[1]{\left\vert\left\vert #1 \right\vert \right\vert}
\newcommand{\abs}[1]{\left\vert #1 \right\vert }
\newcommand{\ris}[2]{#1_{\vert #2}}
\newcommand{\vp}{\varphi}
\newcommand{\vt}{\vartheta}
\newcommand{\wt}[1]{\widetilde{#1}}
\newcommand{\pr}[2]{\frac{\partial \, #1}{\partial\, #2}}%derivata parziale
%per creare teoremi, dimostrazioni ... 
\theoremstyle{plain}
\newtheorem{thm}{Teorema}[section] 
\newtheorem{ese}[thm]{Esempio} 
\newtheorem{ex}[thm]{Esercizio} 
\newtheorem{fatti}[thm]{Fatti}
\newtheorem{fatto}[thm]{Fatto}

\newtheorem{cor}[thm]{Corollario} 
\newtheorem{lem}[thm]{Lemma} 
\newtheorem{al}[thm]{Algoritmo}
\newtheorem{prop}[thm]{Proposizione} 
\theoremstyle{definition} 
\newtheorem{defn}{Definizione}[section] 
\newcommand{\intt}[2]{int_{#1}^{#2}}
\theoremstyle{remark} 
\newtheorem{oss}{Osservazione} 
\newcommand{\di }{\, \mathrm{d}}
\newcommand{\tonde}[1]{\left( #1 \right)}
\newcommand{\quadre}[1]{\left[ #1 \right]}
\newcommand{\w}{\omega}

% diagrammi commutativi tikzcd
% per leggere la documentazione texdoc

\begin{document}

\textbf{Lezione del 20 Novembre di Gandini}
\begin{defn}Sia $X$ uno spazio topologico e $Y \subseteq X$ allora 
\begin{itemize}
\item $Y$ \`e detto \textbf{ raro } in $X$ se $\overline{Y}^\circ =\emptyset$
\item $Y$ \`e detto \textbf{ magro } o di I categoria  se \`e unione numerabile di rari 
\end{itemize}
\end{defn}
\begin{oss}Un chiuso $Z\subseteq X $ \`e raro $\ses$ $Z^\circ= \emptyset$\\
in modo equivalente se e solo se $X \sbarra Z$ \`e aperto denso
\end{oss}
\begin{defn}$X$ \`e detto di \textbf{ Baire } se $Y^\circ = \emptyset$ per ogni magro $Y \subseteq X$
\end{defn}
\begin{prop}I seguenti fatti sono equivalenti
\begin{itemize}
\item[(i)]$X$ \`e uno spazio di Baire
\item[(ii)]Unione numerabile di chiusi rari ha parte interna vuota
\item[(iii)]Intersezione numerabile di aperti densi \`e densa
\end{itemize}
\proof \bbianco
\begin{itemize}
\item (ii)$\ses$(iii) Usando l'osservazione di sopra e passando al complementare
\item (i)$\implica$(ii) \`E un indebolimento della definizione di spazio di Baire
\item (ii)$\implica$(i) Sia $Y \subseteq X$ magro allora 
$$ Y = \bigcup_{n \in \N} Y_n \quad Y_n \text{ raro }$$
Ora 
$$ Y^\circ = \left( \bigcup Y_n \right)^\circ \subseteq \left( \bigcup \overline{Y_n} \right)^\circ$$
Ora $\overline{Y_n}$ \`e chiuso raro dunque per la propiet\`a  (ii) si ha  $Y^\circ \subseteq \emptyset$ dunque $Y^\circ = \emptyset$ 
\end{itemize}
\end{prop}
\begin{thm}$X$ spazio metrico completo $\implica$ $X$ spazio di Baire
\proof Usiamo la caratterizzazione  (ii).\\
Sia $\{F_n\} $ una famiglia numerabile di chiusi rari.\\
Supponiamo per assurdo che $\ds \left( \bigcup F_n \right)^\circ\neq \emptyset$.\\
Sia $x_0\in X$ e $r_0>0$ tali che 
$$ B(x_0,r_0)\subseteq \bigcup_{n \in \N} F_n$$
Costruiamo induttivamente una successione di Cauchy, il cui limite (che esiste, vista la completezza) porta ad un assurdo.\\
$$F_0^\circ = \emptyset \quad \implica \quad B\left( x_0, \frac{r_0}{3} \right) \not \subset F_0$$
Ora essendo $F_0$ chiuso 
$$ \exists x_1 \in B\left( x_0, \frac{r_0}{3}\right) \sbarra F_0 \quad \exists r_1<\frac{r_0}{3} \text{ tale che } B(x_1, r_1) \cap F_0 = \emptyset$$
Ripetendo il ragionamento con $x_1, r_1, F_1$ 
$$ \exists x_2 \in B\left( x_1, \frac{r_1}{3} \right)\sbarra F_1 \quad \exists r_2 < \frac{r_1}{3} \text{ tale che } B(x_2,r_2) \cap F_1 = \emptyset$$
In questo modo costruisco  successione $\{ x_n \}\subseteq X $ e $\{ r_n\}\subseteq R_+$ con le seguenti propiet\`a
\begin{enumerate}
\item $r_{n+1}< \frac{r_n}{3}\leq \frac{r_0}{3^{n+1}}$
\item $x_{n+1}\in B\left( x_n, \frac{r_n}{3} \right)\sbarra F_n$ dunque $d(x_n, x_{n+1})< \frac{r_n}{3} \leq \frac{r_0}{3^{n+1}}$
\item $B(x_{n+1}, r_{n+1}) \cap F_n = \emptyset$
\end{enumerate}
Mostriamo adesso che $\{ x_n\}$ \`e di Cauchy.\\
Siano $m>n$ allora 
$$ d(x_n,x_m) \leq d\left( x_n, x_{n+1} \right)+ \cdots + d\left( x_{m-1},x_m \right)< \frac{r_n}{3}+ \frac{r_{n+1}}{3} + \cdots + \frac{r_{m-1}}{3} \leq \frac{r_n}{3}\sum_{k=0}^\infty \frac{1}{3^k}=\frac{r_n}{2}\leq \frac{r_0}{2\cdot 3^2}$$
Essendo la successione di Cauchy esiste $x_\infty=\lim x_n$.\\
Mostriamo che $x_\infty\not \in F_n$.\\
Passando $m\to \infty	$ in $d\left( x_{n+1},x_m \right)<\frac{r_{n+1}}{2}$
 si ha $d\left( x_{n+1}, x_\infty \right)\leq \frac{r_{n+1}}{2}$ dunque $x_\infty \in B\left( x_{n+1}, r_{n+1} \right)$ tale palla per 3 non interseca $F_n$.\\
 Mostriamo che $d\left( x_\infty, x_0 \right)<r_0$.\\
 Passando $m \to + \infty$ alla disuguaglianza $d\left( x_0, x_m \right)< \frac{r_0}{2}$ otteniamo $d(x_0, x_\infty) < r_0$ ovvero 
$$ x_\infty \in b\left( x_0, r_0 \right)\subseteq \bigcup_{n \in \N} F_n$$ il che \`e assurdo 
\endproof
\end{thm}
\begin{oss}
\end{oss}Il Teorema \`e detto di Baire, che contiene anche un altro enunciato (non dimostriamo) :
\begin{thm}$X$ localmente compatto e $T2$ $\implica$ $X$ spazio di Baire
\end{thm}
\spazio 
\begin{defn}Sia $X$ topologico, $y \in X$ si dice punto isolato se $\{y\} \subseteq X$ \`e un aperto 
\end{defn}
\begin{defn}$Y \subseteq X$ si dice \textbf{ perfetto } se e chiuso e privo di punti isolati
\end{defn}
\begin{prop}
$Y\subseteq \R$ perfetto, allora $Y$ \`e pi\`u che numerabile
\proof Supponiamo $Y=\ds\{y_n\}_{n \in \N}$.\\
Vogliamo trovare $n \in \N$ per cui $y_n\in Y$ \`e isolato.
$$ Y \subseteq \R \text{ chiuso } \quad \implica \quad Y \text{ spazio metrico completo} \quad \implica \quad Y \text{ spazio di Baire}$$
Ora $Y=\ds \bigcup_{n\in \N} \{y_n\}$ \`e unione numerabile di chiusi e $Y^\circ = Y \neq \emptyset$.\\
Per il teorema esiste un chiuso non raro dunque $\exists n$ tale che $y_n\in Y$ punto isolato (non pu\`o essere $\overline{\{y_n\}}^\circ =\emptyset$ per ogni $n \in \N$
\endproof
\end{prop}
\newpage
\begin{thm}Esiste una funzione $f:\, [0.1]\to \R$ continua non derivabile in nessun punto
\proof A BREVE
\end{thm}
\end{document}