\documentclass[a4paper,12pt]{article}
\usepackage[a4paper, top=2cm,bottom=2cm,right=2cm,left=2cm]{geometry}

\usepackage{bm,xcolor,mathdots,latexsym,amsfonts,amsthm,amsmath,
					mathrsfs,graphicx,cancel,tikz-cd,hyperref,booktabs,caption,amssymb,amssymb,wasysym}
\hypersetup{colorlinks=true,linkcolor=blue}
\usepackage[italian]{babel}
\usepackage[T1]{fontenc}
\usepackage[utf8]{inputenc}
\newcommand{\s}[1]{\left\{ #1 \right\}}
\newcommand{\sbarra}{\backslash} %% \ 
\newcommand{\ds}{\displaystyle} 
\newcommand{\alla}{^}  
\newcommand{\implica}{\Rightarrow}
\newcommand{\iimplica}{\Leftarrow}
\newcommand{\ses}{\Leftrightarrow} %se e solo se
\newcommand{\tc}{\quad \text{ t. c .} \quad } % tale che 
\newcommand{\spazio}{\vspace{0.5 cm}}
\newcommand{\bbianco}{\textcolor{white}{,}}
\newcommand{\bianco}{\textcolor{white}{,} \\}% per andare a capo dopo 																					definizioni teoremi ...


% campi 
\newcommand{\N}{\mathbb{N}} 
\newcommand{\R}{\mathbb{R}}
\newcommand{\Q}{\mathbb{Q}}
\newcommand{\Z}{\mathbb{Z}}
\newcommand{\K}{\mathbb{K}} 
\newcommand{\C}{\mathbb{C}}
\newcommand{\F}{\mathbb{F}}
\newcommand{\p}{\mathbb{P}}

%GEOMETRIA
\newcommand{\B}{\mathfrak{B}} %Base B
\newcommand{\D}{\mathfrak{D}}%Base D
\newcommand{\RR}{\mathfrak{R}}%Base R 
\newcommand{\Can}{\mathfrak{C}}%Base canonica
\newcommand{\Rif}{\mathfrak{R}}%Riferimento affine
\newcommand{\AB}{M_\D ^\B }% matrice applicazione rispetto alla base B e D 
\newcommand{\vett}{\overrightarrow}
\newcommand{\sd}{\sim_{SD}}%relazione sx dx
\newcommand{\nvett}{v_1, \, \dots , \, v_n} % v1 ... vn
\newcommand{\ncomb}{a_1 v_1 + \dots + a_n v_n} %a1 v1 + ... +an vn
\newcommand{\nrif}{P_1, \cdots , P_n} 
\newcommand{\bidu}{\left( V^\star \right)^\star}

\newcommand{\udis}{\amalg}
\newcommand{\ric}{\mathfrak{U}}
\newcommand{\inclu}{\hookrightarrow }
%ALGEBRA

\newcommand{\semidir}{\rtimes}%semidiretto
\newcommand{\W}{\Omega}
\newcommand{\norma}{\vert \vert }
\newcommand{\bignormal}{\left\vert \left\vert}
\newcommand{\bignormar}{\right\vert \right\vert}
\newcommand{\normale}{\triangleleft}
\newcommand{\nnorma}{\vert \vert \, \cdot \, \vert \vert}
\newcommand{\dt}{\, \mathrm{d}t}
\newcommand{\dz}{\, \mathrm{d}z}
\newcommand{\dx}{\, \mathrm{d}x}
\newcommand{\dy}{\, \mathrm{d}y}
\newcommand{\amma}{\gamma}
\newcommand{\inv}[1]{#1^{-1}}
\newcommand{\az}{\centerdot}
\newcommand{\ammasol}[1]{\tilde{\gamma}_{\tilde{#1}}}
\newcommand{\pror}[1]{\mathbb{P}^#1 (\R)}
\newcommand{\proc}[1]{\mathbb{P}^#1(\C)}
\newcommand{\sol}[2]{\widetilde{#1}_{\widetilde{#2}}}
\newcommand{\bsol}[3]{\left(\widetilde{#1}\right)_{\widetilde{#2}_{#3}}}
\newcommand{\norm}[1]{\left\vert\left\vert #1 \right\vert \right\vert}
\newcommand{\abs}[1]{\left\vert #1 \right\vert }
\newcommand{\ris}[2]{#1_{\vert #2}}
\newcommand{\vp}{\varphi}
\newcommand{\vt}{\vartheta}
\newcommand{\wt}[1]{\widetilde{#1}}
\newcommand{\pr}[2]{\frac{\partial \, #1}{\partial\, #2}}%derivata parziale
%per creare teoremi, dimostrazioni ... 
\theoremstyle{plain}
\newtheorem{thm}{Teorema}[section] 
\newtheorem{ese}[thm]{Esempio} 
\newtheorem{ex}[thm]{Esercizio} 
\newtheorem{fatti}[thm]{Fatti}
\newtheorem{fatto}[thm]{Fatto}

\newtheorem{cor}[thm]{Corollario} 
\newtheorem{lem}[thm]{Lemma} 
\newtheorem{al}[thm]{Algoritmo}
\newtheorem{prop}[thm]{Proposizione} 
\theoremstyle{definition} 
\newtheorem{defn}{Definizione}[section] 
\newcommand{\intt}[2]{int_{#1}^{#2}}
\theoremstyle{remark} 
\newtheorem{oss}{Osservazione} 
\newcommand{\di }{\, \mathrm{d}}
\newcommand{\tonde}[1]{\left( #1 \right)}
\newcommand{\quadre}[1]{\left[ #1 \right]}
\newcommand{\w}{\omega}

% diagrammi commutativi tikzcd
% per leggere la documentazione texdoc

\begin{document}
\textbf{Lezione del 19 + parte del 23 marzo}
\begin{cor}Sia $f:\, U \to \C$ olomorfa con $U \subseteq \C$ aperto connesso.\\
Le seguenti affermazioni sono equivalenti
\begin{itemize}
\item[(i)] $f$ \`e costante in $U$
\item[(ii)] $f'$ \`e identicamente nullo 
\item[(iii)] $Re(f)$ \`e costante in $U$
\item[(iv)] $Im(f)$ \`e costante in $U$
\end{itemize}
\proof \bbianco
\begin{itemize}
\item $(i) \implica (iii)$ e $(i) \implica (iv)$ sono ovvie
\item  $(i)\ses (ii)$ Una funzione \`e costante se e solo se ha Jacobiano nullo.\\
Ora essendo $f$ olomorfa il suo Jacobiano \`e rappresentata dalla matrice  
$\begin{pmatrix}
\alpha & -\beta \\ \beta & \alpha
\end{pmatrix}$ 
con $f'(z_0)=\alpha + i \beta$.\\
Ora il jacobiano \`e identicamente nullo se e solo se $\alpha=\beta =0$ da cui se e solo se $f'(z)=0$
\item $(iii)\implica (i)$ Scriviamo  nella base $\{ 1, i\}$ di $\C$ 
$$f(x,y)=u(x,y)+i v(x,y)$$ 
$Re(f)$ costante \`e equivalente a $u$ costante equivalentemente
 $\pr u x = 0 = \pr u y $\\
Per Cauchy-Riemann si ha $\pr v x = \pr u x = 0 $ dunque la tesi 
\item $(iv)\implica (i)$ in modo analogo al punto precedente
\end{itemize}
\endproof
\end{cor}
\newpage
\section{Serie di potenze}
\begin{defn}[Assoluta convergente]\bianco 
Sia $\ds (c_n)_{n \in \N} $ una successione di numeri complessi.\\
Diciamo che $\ds \sum_{n\geq 0} c_n$ \`e assolutamente convergente se la serie $\sum \vert c_n \vert$ \`e convergente
\end{defn}
\begin{ex}La serie $\sum\frac{i}{n!}$ \`e assolutamente convergente?
\end{ex}
\begin{prop}Siano $\sum a_n$ e $\sum b_n$ due serie assolutamente convergenti 
\begin{itemize}
\item $\sum (a_n + b_n) $ \`e assolutamente convergente e la somma di questa serie \`e ottenuta sommando la somma delle serie dati 
\item se $c_n =\sum_{p=0}^ n a_p b_{n-p}$ la serie $\sum c_n$ \`e assolutamente convergente e la sua somma \`e uguale al prodotto della somma delle 2 serie date
\end{itemize}
\end{prop}

\begin{defn}[Raggio di convergenza]\bianco 
Sia $\sum a_n z^n$ una serie di potenza chiamiamo raggio di convergenza la quantit\`a 
$$\rho = \sup \{ r\in \R \, r>0 \, \vert \, \sum \vert a_n \vert r^n  \text{ \`e convergente } \}$$
\end{defn}
\begin{oss} $\rho$ pu\`o essere finito e in questo caso, $\rho\geq 0 $ oppure $\rho$ \`e infinito
\end{oss}
\begin{defn}Chiamiamo disco di convergenza l'insieme $\{ z \in \C \, \vert \,  \, \vert z\vert < \rho\}$
\end{defn}
\begin{oss}Il disco di convergenza \`e aperto.\\
Se $\rho=0$ allora il disco \`e vuoto 
\end{oss}
\begin{prop}Data una serie di potenza $\sum a_n z^n$ esiste $0\leq \rho \leq  \infty$ tale che 
\begin{itemize}
\item $\rho=0$ la serie converge per $z=0$
\item $\rho=\infty$  la serie converge assolutamente per ogni $z$
\item $0<\rho<\infty$ allora se $\vert z\vert > \rho $ la serie converge assolutamente, per $\vert z \vert > \rho $ la serie non converge
\end{itemize}
Inoltre si ha la formula di  Hadamard
$$\frac{1}{\rho}=\limsup \vert a_n \vert^{1/n}$$ 
con la convenzione 
$$\rho = 0 \text{ se  il limite superiore \`e } \infty$$
$$\rho = \infty \text{ se  il limite superiore \`e } 0$$
\end{prop}
\begin{ex}Calcolare il raggio di convergenza delle seguenti serie
\begin{itemize}
\item $\sum n! z^n$
\item $\sum \frac{1}{n!}z^n$
\item $\sum (-1)^n \frac{z^{2n}}{(2n)!}$
\item $\sum (-1)^n \frac{z^{2n+1}}{(2n+1)!}$
\end{itemize}
\end{ex}
\begin{fatto}Siano $\sum a_n z^n$ e $\sum b_n z^n $ serie di potenze con raggio di convergenza $>R$ per un certo $R$ allora
$$S(z)=\sum a_n z^n + \sum b_n z^n$$
$$P(z)=\left(\sum a_n z^n \right) \left( \sum b_n z^n\right)$$
hanno raggio di convergenza minore di $R$.\\
Inoltre $\forall r \in \C $ con $\abs{r} < R$ si ha
$$ S(r) =\sum a_n r^n + \sum b_n r^n$$
$$P(r)=\left(\sum a_n r^n \right) \left( \sum b_n r^n\right)$$
\end{fatto}
\begin{fatto}
Sia $f(z)=\sum a_n z^n$ chiamiamo serie derivata la serie di potenze $\sum n a_n z^{n-1}$ e la denotiamo con $f'(z)$.\\
$f$ e $f'$ hanno lo stesso raggio di convergenze
\end{fatto}
\newpage
\section{Esponenziale e logaritmo complesso }
\begin{defn}Fissato $z\in \C$ chiamiamo esponenziale del numero complesso $z$ la quantit\`a
$$e^z = \sum \frac{1}{n!}z^n$$
\end{defn}
\begin{defn}
Fissato $z\in \C$ chiamiamo coseno del numero complesso $z$ la quantit\`a
$$\cos z  = \sum (-1)^n \frac{z^{2n}}{(2n)!}$$
\end{defn}
\begin{defn}
Fissato $z\in \C$ chiamiamo seno  del numero complesso $z$ la quantit\`a
$$\sin z  = \sum (-1)^n \frac{z^{2n+1}}{(2n+1)!}$$
\end{defn}
\begin{oss}Le definizioni sono ben poste, avendo le serie raggio di convergenza $\infty$
\end{oss}
\begin{ex}Provare che le definizioni date oggi e nella lezione precedente coincidono
\end{ex}
\begin{ese}Dati $z,z'\in \C$, proviamo che $e^{z+z'}=e^z+e^{z'}$
\proof
Siano $a_n =\frac{1}{n}z^n$ e $b_n =\frac{1}{n}(z')^n$ allora 
$$c_n=\sum_{p=0}^n a_p b_{n-p} = \sum_{p=0}^n \left( \frac{z^p}{p!}\right)\tonde{\frac{(z')^{n-p}}{(n-p)!}} = \frac{1}{n!} \sum_{p=0}^n \frac{n!}{p! (n-p)!} z^p (z')^{n-p}  =\frac{1}{n!} \sum_{p=0}^n { n \choose p} z^p (z')^{n-p}$$ 
dunque 
$c_n= \frac{1}{n!}(z+z')^n$.\\
Per quanto abbiamo visto sulle serie di potenze $\sum c_n$ converge assolutamente con 
$$e^{z+z'} = \sum c_n = \tonde{\sum a_n} \tonde{\sum b_n} = e^z\cdot e^{z'}$$
\endproof
\end{ese}
\begin{oss}Siano $z,z' \in \C$ allora 
$e^z =e^{z'} \quad \ses \quad z'=z+i 2\pi k \text{ con } k \in \Z$\\
infatti dall'esercizio precedente $e^{z'} \cdot e^{-z}=e^{z'-z}$.\\
Ora $e^z=e^{Re(z)}( \cos Im (z) + i \sin Im (z)$ dunque se $w=z'-z$ otteniamo 
$$e^z =e^{z'} \quad \ses \quad e^{w}=1 \quad \ses \quad \begin{cases} e^{Re(w)}=1  \\
\ cos (Im(w)) +i\sin (Im(w))=1 \end{cases} 
\quad \ses \quad \begin{cases} Re(w)=0 \\ Im(w)=2\pi k \text{ con } k \in \Z 
\end{cases}$$
\end{oss}
\newpage
\begin{defn} Sia $z\in \C\setminus \{0 \}$ allora definiamo il logaritmo del numero complesso $z$ come
$$\log(z) = \log(\abs z ) + i arg(z)$$
\end{defn}
\begin{oss}Nella definizione c'\`e un'ambiguit\`a derivante dal fatto che $arg(z) \in \frac{\R}{2\pi\Z}$
\end{oss}
Vediamo come sia possibile definire una funzione $z\to \log (z)$ 
\begin{defn}[branca]\bianco
Sia $D$ un insieme aperto e connesso di $\C$ con $0\not \in D$.\\
Diciamo che $f:\, D \to \C$ continua \`e una branca di $\log(z)$ se $e^{f(z)}=z$\\
($f(z)$ \`e uno dei possibili valori di $\log(z)$)\\ \\ 
Sia $D$ come sopra  e $g:\, D \to \C$ continua \`e una branca di $arg(z)$ se 
$z=\abs z e^{i g(z)}$\\
($g(z)$ \`e uno dei possibili valori di $arg(z)$)
\end{defn}
\begin{oss} Se $g$ \`e una branca di $arg(z)$ allora  $f(x)= \log(\abs z ) + i g(x)$ \`e una branca di $\log z $
\end{oss}
\begin{prop}Assumiamo che esista una branca di $\log(z)$ in $D$, allora tutte le altre di $\log z$ in $D$ sono della forma $f(z)+ k (2\pi i)$ per qualche $z\in \Z$.\\
Inoltre $f(z) + k(2\pi i)$ \`e una branca di $\log(z)$ in $D$ per ogni $z$ intero
\proof Siano $f(z)$ e $g(z)$ due branche di $log(z)$ in $D$ consideriamo la funzione 
$$h(z) = \frac{1}{2\pi i } (g(z)-h(z)) :\, D\to \C$$
Osserviamo che $h$ \`e continua e ha immagine contenuta in $\Z$ (essendo $f(z)$ e $g(z)$ due branche di $\log(z)$ allora $e^{f(z)}=z= e^{g(z)}$ due per quanto abbiamo osservato sull'esponenziale $f(z)$ e $g(z)$ differiscono per $k(2\pi i )$)\\
Poich\`e $D$ \`e connesso allora $h$ \`e costante dunque $h(t)=k \, \, \forall t \in D$ dunque 
$$\frac{1}{2\pi i}(g(t)-f(t))=k \quad \implica \quad g(t)=f(t) + 2k \pi i \, \, \forall t\in D $$
\end{prop}
\begin{oss}Per ogni $z\in D=\{ z\in \C \, \vert \, Re(z)>0\}$ esiste unico 
$\phi\in \quadre{ -\frac{\pi}{2},\frac{\pi}{2}}$
 tale che $z=\abs z e^{i \phi}$.\\
Definiamo la funzione $Arg :\, D \to \C$ con $Arg(z)= \phi$.\\
Se mostriamo che tale funzione \`e continua, abbiamo costruito una branca dell'argomento
\end{oss}
\begin{prop}$Arg:\, D \to \C$ \`e continua
\proof Sia $U=\{ z\in D \, \vert \, abs(z)=1 \}$ sia $f:\, D \to U$ con $f(z) = \frac{z}{\abs(z)}$, chiaramente $f$ \`e continua\\
Osserviamo che  per ogni $z\in D$  $arg(z) = arg\tonde{f(z)}$ dunque abbiamo il seguente diagramma  cui abbiamo il seguente diagramma commutativo 
$$\begin{tikzcd}
D \arrow[rr,"f"] \arrow[dr,"Arg"] & & U \arrow[dl,"Arg"] \\
& \C
\end{tikzcd}$$
Essendo $f$ continua basta provare che $Arg:\, U \to \C$ \`e continua.\\
Per costruzione tale mappa \`e l'inversa della mappa 
$g:\, \left] -\frac{\pi}{2},\frac{\pi}{2}\right[ \to U $ dove $g(y)=e^{iy}$.\\
Estendiamo tale mappa a $$\wt g :\quadre{ -\frac{\pi}{2},\frac{\pi}{2}} \to \{ u \in \C \, \vert \, \abs{u} = 1 \text{ e } Re(u) \geq 0 \}$$
$\wt g$ \`e continua e bigettiva da un compatto ad uno spazio di Hausdorff dunque \`e un omeomorfismo, in particolare la sua inversa \`e continua, da cui anche l'inversa di $g$ ($f$) \`e continua
\end{prop}
\begin{defn}Chiamiamo branca principale di $\log z$ la funzione continua
$$\log(\abs z ) + i Arg(z) \text{ per } z \in D =\{  z\in \C\, \vert \, Re(z)>0\}$$
\end{defn} 
\spazio 
\begin{prop}La serie di potenze $\sum (-1)^{n+1} \frac{z^n}{n}$ converge per $\abs z <1$ ed \`e uguale alla branca principale di $\log(z+1)$
\end{prop}
\begin{prop}Se $f(z)$ \`e una branca di $\log z$ in un insieme aperto e connesso la funzione ammette derivata $\frac{1}{z}$
\end{prop}
\begin{defn}$\forall z, \alpha \in \C$ con $z\neq 0 $ poniamo 
$$z^\alpha =e^{\alpha \log(z)}$$
\end{defn}
\end{document}