\documentclass[a4paper,12pt]{article}
\usepackage[a4paper, top=2cm,bottom=2cm,right=2cm,left=2cm]{geometry}

\usepackage{bm,xcolor,mathdots,latexsym,amsfonts,amsthm,amsmath,
					mathrsfs,graphicx,cancel,tikz-cd,hyperref,booktabs,caption,amssymb,amssymb,wasysym}
\hypersetup{colorlinks=true,linkcolor=blue}
\usepackage[italian]{babel}
\usepackage[T1]{fontenc}
\usepackage[utf8]{inputenc}
\newcommand{\s}[1]{\left\{ #1 \right\}}
\newcommand{\sbarra}{\backslash} %% \ 
\newcommand{\ds}{\displaystyle} 
\newcommand{\alla}{^}  
\newcommand{\implica}{\Rightarrow}
\newcommand{\iimplica}{\Leftarrow}
\newcommand{\ses}{\Leftrightarrow} %se e solo se
\newcommand{\tc}{\quad \text{ t. c .} \quad } % tale che 
\newcommand{\spazio}{\vspace{0.5 cm}}
\newcommand{\bbianco}{\textcolor{white}{,}}
\newcommand{\bianco}{\textcolor{white}{,} \\}% per andare a capo dopo 																					definizioni teoremi ...


% campi 
\newcommand{\N}{\mathbb{N}} 
\newcommand{\R}{\mathbb{R}}
\newcommand{\Q}{\mathbb{Q}}
\newcommand{\Z}{\mathbb{Z}}
\newcommand{\K}{\mathbb{K}} 
\newcommand{\C}{\mathbb{C}}
\newcommand{\F}{\mathbb{F}}
\newcommand{\p}{\mathbb{P}}

%GEOMETRIA
\newcommand{\B}{\mathfrak{B}} %Base B
\newcommand{\D}{\mathfrak{D}}%Base D
\newcommand{\RR}{\mathfrak{R}}%Base R 
\newcommand{\Can}{\mathfrak{C}}%Base canonica
\newcommand{\Rif}{\mathfrak{R}}%Riferimento affine
\newcommand{\AB}{M_\D ^\B }% matrice applicazione rispetto alla base B e D 
\newcommand{\vett}{\overrightarrow}
\newcommand{\sd}{\sim_{SD}}%relazione sx dx
\newcommand{\nvett}{v_1, \, \dots , \, v_n} % v1 ... vn
\newcommand{\ncomb}{a_1 v_1 + \dots + a_n v_n} %a1 v1 + ... +an vn
\newcommand{\nrif}{P_1, \cdots , P_n} 
\newcommand{\bidu}{\left( V^\star \right)^\star}

\newcommand{\udis}{\amalg}
\newcommand{\ric}{\mathfrak{U}}
\newcommand{\inclu}{\hookrightarrow }
%ALGEBRA

\newcommand{\semidir}{\rtimes}%semidiretto
\newcommand{\W}{\Omega}
\newcommand{\norma}{\vert \vert }
\newcommand{\bignormal}{\left\vert \left\vert}
\newcommand{\bignormar}{\right\vert \right\vert}
\newcommand{\normale}{\triangleleft}
\newcommand{\nnorma}{\vert \vert \, \cdot \, \vert \vert}
\newcommand{\dt}{\, \mathrm{d}t}
\newcommand{\dz}{\, \mathrm{d}z}
\newcommand{\dx}{\, \mathrm{d}x}
\newcommand{\dy}{\, \mathrm{d}y}
\newcommand{\amma}{\gamma}
\newcommand{\inv}[1]{#1^{-1}}
\newcommand{\az}{\centerdot}
\newcommand{\ammasol}[1]{\tilde{\gamma}_{\tilde{#1}}}
\newcommand{\pror}[1]{\mathbb{P}^#1 (\R)}
\newcommand{\proc}[1]{\mathbb{P}^#1(\C)}
\newcommand{\sol}[2]{\widetilde{#1}_{\widetilde{#2}}}
\newcommand{\bsol}[3]{\left(\widetilde{#1}\right)_{\widetilde{#2}_{#3}}}
\newcommand{\norm}[1]{\left\vert\left\vert #1 \right\vert \right\vert}
\newcommand{\abs}[1]{\left\vert #1 \right\vert }
\newcommand{\ris}[2]{#1_{\vert #2}}
\newcommand{\vp}{\varphi}
\newcommand{\vt}{\vartheta}
\newcommand{\wt}[1]{\widetilde{#1}}
\newcommand{\pr}[2]{\frac{\partial \, #1}{\partial\, #2}}%derivata parziale
%per creare teoremi, dimostrazioni ... 
\theoremstyle{plain}
\newtheorem{thm}{Teorema}[section] 
\newtheorem{ese}[thm]{Esempio} 
\newtheorem{ex}[thm]{Esercizio} 
\newtheorem{fatti}[thm]{Fatti}
\newtheorem{fatto}[thm]{Fatto}

\newtheorem{cor}[thm]{Corollario} 
\newtheorem{lem}[thm]{Lemma} 
\newtheorem{al}[thm]{Algoritmo}
\newtheorem{prop}[thm]{Proposizione} 
\theoremstyle{definition} 
\newtheorem{defn}{Definizione}[section] 
\newcommand{\intt}[2]{int_{#1}^{#2}}
\theoremstyle{remark} 
\newtheorem{oss}{Osservazione} 
\newcommand{\di }{\, \mathrm{d}}
\newcommand{\tonde}[1]{\left( #1 \right)}
\newcommand{\quadre}[1]{\left[ #1 \right]}
\newcommand{\w}{\omega}

% diagrammi commutativi tikzcd
% per leggere la documentazione texdoc

\begin{document}
\textbf{Lezione del 16 ottobre di Gandini}
\section{Assiomi di separabilit\'a}

\begin{defn}
 Uno spazio topologico $X$ si dice $T1$ se
 
$$ \forall x,y\in X \, x \neq y \quad \exists U,V\subseteq X \text{ aperti } \quad x \in U \sbarra V \text{ e } y \in V \sbarra U $$
\end{defn}

\begin{prop}[Definizione alternativa]\bianco
 Sia $X$ uno spazio topologico
 $$ X \text{ \`e } T1 \quad \ses \quad \text{ i punti sono chiusi}$$
 \proof $\overline{\{x\}} = \{ y \in X \, \vert \, \text{ tutti gli intorni che contengono } y \text{ contengono } x \}$
\\ $\implica$ Essendo $X$ $T1$
$$ \forall y \in X \, y \neq x \quad \implica \quad \exists U \subseteq X \text{  aperto che contiene} y \text{ tale che }  x\not \in V  \quad \implica \quad y\not \in \overline{\{ x \}}$$
$\iimplica$ Sia $y\neq x $ allora $V=X\sbarra \{x\}$ e $U=X\sbarra\{y\}$ sono aperti in quanto i punti sono chiusi.\\
Dunque $U$ \`e un aperto che contiene $x$ ma non $Y$ e viceversa $V$ contiene $y$ ma non $x$.
\endproof
\end{prop}
\begin{oss}\label{cofT1}Se $\tau$ \`e una topologia su $X$, allora $\tau$ soddisfa $T1$ se e solo se $\tau$ \`e pi\`u fine della topologia 
\end{oss}
\spazio
\begin{ex}
$$ X \text{ \`e } T1 \quad \ses \quad \{ x \} = \bigcap_{U\in I(x)} U $$
\end{ex}
\spazio
\begin{defn}
 Uno spazio topologico $X$ si dice $T2$ se
$$ \forall x,y\in X \, x\neq y \quad \exists U,V \subseteq X \text{ aperti disgiunti } \quad x \in U \, y \in V $$
 \end{defn}
\begin{defn}
 Uno spazio topologico che verifica l'assioma $T2$ prende il nome di spazio di Hausdorff o spazio separato
\end{defn}
\begin{prop}[Definizione alternativa]\bianco
$$ X\text{\`e } T2 \quad \ses \quad \Delta_X\subseteq X \times X \text{ \`e chiuso con la topologia di sottospazio}$$
dove $\Delta_X=\{ (x,x) \, \vert \, x \in  X \}$ \`e la diagonale di $X$
\proof $\implica$ Sia $x\neq y $ allora $(x,y) \not \in \Delta_X$.\\
Siano $U \in I(x)$ e $V \in I(y)$ disgiunti da cui 
$$(x,y) \in U \times V \subseteq (X \times X ) \sbarra \Delta_X$$
ovvero $U\times V$ \`e un intorno di $(x,y)$ dunque \`e aperto\\
$\iimplica$ Il complementare della diagonale \`e aperto allora se $x\neq y$
$$ \exists U, V \subseteq X \text{ aperti tali che  }  (x,y)\in U \times V \subseteq (X\times X) \sbarra \Delta_X $$ 
inoltre $U$ e $V$ sono disgiunti (se $\exists z\in U \cap V $ allora $(z,z)\in U \times V$ e $(z,z)\in \Delta_X$ ma avevamo supposto che $U\times V$ fosse contenuto nel complementare della diagonale)

\end{prop}
\spazio
\begin{ex}
 $$ X \text{ \`e } T2 \quad \ses \quad \{ x\} =\bigcap_{U\in I(x)} \overline{U}$$
\end{ex}

\begin{prop} Sia $X$ uno spazio topologico allora 
$T2 \implica T1$
\proof 
Se $U,V$ sono aperti disgiunti tali che $x\in U $ e $y\in V$ allora, in particolare, $x\in U\sbarra V $ e $y \in V\sbarra U$

\end{prop}

\spazio
\begin{prop}
Sottospazi e prodotti arbitrati di $T2$ (risp. $T1$) sono ancora $T2$ (risp. $T1$)
\proof Mostriamo che prodotti di $T2$ \`e $T2$\\
Consideriamo  $$X=\prod_{\alpha \in A} X_\alpha  \text{ dove gli } X_\alpha \text{ sono } T2$$
Sia $x\neq y$ allora $\exists a \in A $ tale che $x_a=x(a) \neq y(a)=y_a$.\\
Essendo $X_a$ $T2$ 
$$ \exists U_a \in I(x_a) \quad \exists V_a \in I(y_a) \quad \text{ con } U_a \cap V_a = \emptyset$$
Sia
$$ U= \prod_{\alpha \in A } U_\alpha \text{ dove } U_\alpha = \begin{cases}U_a\text{ se } \alpha=a\\ 
									   X_\alpha \text{ se } \alpha \neq a 
                                                  
                                                              \end{cases}
 \quad 
V= \prod_{\alpha \in A } V_\alpha \text{ dove } V_\alpha = \begin{cases}V_a\text{ se } \alpha=a\\ 
									   X_\alpha \text{ se } \alpha \neq a 
                                                  
                                                              \end{cases}$$
Per come abbiamo definito una base della topologia prodotto $U$ e $V$ sono aperti in $X$ inoltre essi sono disgiunti perch\`e $U_a$ e $V_a$ sono disgiunti

\end{prop}
\begin{oss}
Se ho $2$ topologie su $X$ con $\tau_1 \subseteq \tau_2$
$$ \tau_1 \text{ \`e } T1 \text{(risp. } T2 \text{)} \quad \implica \quad \tau_2 \text{ \`e } T1 \text{(risp. } T2 \text{)}$$

\end{oss}
\spazio
\subsection{Alcune propiet\`a di $T2$}
\begin{prop}
Sia $Y$ uno spazio $T2$ e $f:\, X \to Y$ una funzione continua, allora il grafico di $f$ \`e chiuso ovvero
$$ \Gamma_f=\{ (x,f(x)) \, \vert \, x \in X \} \subseteq X \times X \text{ \`e un chiuso }$$
\proof Consideriamo la funzione
$$ F:\, X \times Y  \to Y \times Y \quad (x,y) \to (f(x),y)$$
essa \`e continua per come \`e stata definita la topologia prodotto e perc\`e $f$ \`e continua.\\
Osserviamo che $F^{-1}(\Delta_X) = \Gamma_f$ dunque essendo la diagonale un chiuso, anche il grafico lo \`e
\endproof
\end{prop}
\begin{prop}
Sia $Y$ uno spazio $T2$ e $f,g:\, X \to Y$ continue allora
$$ C= \{ x \in X \, \vert \, f(x)=g(x) \} \text{ \`e chiuso } $$
\proof Consideriamo la funzione 
$$ F:\, X \to Y \times Y \quad x \to (f(x),g(x)) $$
essa \`e continua dunque $F^{-1}(\Delta_X)$ \`e chiuso essendo chiusa la diagonale ma $F^{-1}(\Delta_X)=C$
\endproof
\end{prop}
\begin{cor}
Sia $X$ uno spazio $T2$ e $f:\, X \to X $ continua, allora $Fix(f)$ \'e chiuso
\proof Basta porre $g=id_X$ e usare la proposizione precedente
\end{cor}
\begin{prop}
 Sia $Y$ uno spazio $T2$ e $f,g:\, X \to Y $ continue.\\
 Sia $Z\subseteq X$ un denso tale che $f(z)=g(z)$ $ \forall z \in Z $.\\
 Allora $f=g$
\proof Per la proposizione precedente $\{ f(x)=g(x)\}$ \`e un chiuso, tale chiuso contiene un denso quindi contine la sua chiusura, ovvero, tutto lo spazio $X$
\end{prop}

\spazio
\begin{prop}
Sia $X$ uno spazio $T2$ e $\ds \{ x_n \}_{n \in \N} $ una successione convergente. Il limite della successione \`e unico
\proof Siano $x,y$ due limiti della successione.\\
Poich\`e la successione converge a $x$ allora
$$ \forall U \in I(x) \exists n_1 \in \N \quad \{x_n\} \subseteq U \quad \forall n\geq n_1$$
inoltre converge anche a $y$ quindi 
$$ \forall V \in I(y) \exists n_2 \in \N \quad \{x_n\} \subseteq V \quad \forall n\geq n_2$$
quindi vale 
$$ \exists n_0=\max \{ n_1,n_2\} \quad \{x_n\} \subseteq V\cap U \quad \forall n\geq n_0$$
dunqe $\forall U\in I(x)\, \forall V \in I(y)$ accade $U\cap V \neq \emptyset$ ma ci\`o viola l'assioma $T2$
\endproof
\end{prop}
\newpage

\begin{defn}[T3]\bianco
Uno spazio topologico $X$ si dice $T3$ se 
$$ \forall C \subseteq X \text{ chisuo} \quad \forall x \in X \sbarra C \quad \exists U,V \text{ aperti disgiunti } \quad x\in U \quad C \subseteq V$$
\end{defn}
\begin{defn}$X$ \`e uno spazio topologico regolare se soddisfa $T1$ e $T3$
\end{defn}
\spazio
\begin{defn}Uno spazio topologico $X$ si dice $T4$ se 
$$ \forall C, D \subseteq X \text{ chiusi disgiunti } \quad \exists U,V \subseteq X \text{ aperti disgiunti } \quad C \subseteq U \quad D\subseteq V $$
\end{defn}
\begin{defn}$X$ \`e uno spazio topologico normale se soddisfa $T1$ e $T4$
\end{defn}
\begin{oss}Possiamo riformulare la condizione in $T4$ come segue:\\
Per ogni coppia di chiusi $C,D$ , $\exists U \in X$ tale che 
$$ C \subseteq U \subseteq \overline{U} \subseteq X \sbarra D$$
Infatti, in queste ipotesi $U$ \`e un aperto che contenente $C$  e $X\sbarra \overline{U}$ \`e un aperto contenente $D$, inoltre i $2$ aperti sono, in modo ovvio, disgiunti.\\
Se $X$ \`e $T4$ allora $ U \subseteq X \sbarra V $ essendo $U,V$ disgiunti.\\
Inoltre poich\`e $X \sbarra V $ \`e un chiuso contenente $U$, contiene anche la sua chiusura dunque 
$$ U \subseteq \overline{ U } \subseteq X\sbarra V $$ 
concludiamo osservando che $D \subseteq V $ implica  $X\sbarra V \subseteq X \sbarra D $\\
Possiamo fare un ragionamento analogo anche con $T3$
\end{oss}

\begin{prop} Sottospazi e prodotti di $T3$ sono $T3$
\proof  Mostriamo che sottospazi di $T3$ sono $T3$ l'altra \`e analoga a quella fatta nel caso di $T2$.\\
Sia $X$ uno spazio vettoriale $T3$.\\
Siano $Z\subseteq X $ (con topologia di sottospazio),  $z\in Z$ e $C\subseteq Z $ chiuso.\\
Dalla definizione di topologia di sottospazio
$$ C\text{ chiuso } \implica \exists D \subseteq X \text{ chiuso } \quad C = Z\cap D $$
Ora essendo $X$ $T3$ 
\end{prop}
\spazio
\begin{prop}Un sottospazio chiuso di $T4$ \`e un $T4$
\proof Sia $X$ uno spazio $T4$ e $Z\subseteq X $ un chiuso.\\
Siano $C,D\subseteq Z \text{ chiusi disgiunti }$.\\
Dalla definizione di topologia di sottospazio
$$ \exists C_1 \subseteq X \text{ chiusi } \quad C= C_1\cap Z $$
$$ \exists D_1 \subseteq X \text{ chiusi } \quad D= D_1\cap Z $$
Ora essendo $Z$ un chiuso di $X$ e poich\`e l'intersezione di $2$ chiusi \`e un chiuso: $C$ e $D$ sono chiusi disgiunti di $X$ dunque 
$$ \exists U_1,V_1\subseteq X \text{ aperti disgiunti } \quad C\subseteq U_1 \quad D \subseteq V_1$$ 
Ora poich\`e $C\subseteq Z$ vale in particolare $C\subseteq U_1 \cap Z$ e similmente $D\subseteq V_1\cap Z$.\\
Ora $U=U_1\cap Z $ e $V=V_1\cap Z$ sono aperti disgiunti di $Z$ otteniamo che $Z$ \`e $T4$
\endproof

\end{prop}
\begin{oss}In generale,  sottospazi e prodotti di $T4$ non sono $T4$.\\
\end{oss}
\newpage


\begin{prop} $\text{ metrizzabile } \implica \text{ normale }$
\proof Sia $(X,d)$ uno spazio metrico, per provare che $X$ \`e normale basta dimostrare che \`e $T4$ infatti $X$ \`e $T2$ quindi $T1$.\\
Ricordiamo come avevamo definito la funzione distanza da un sottoinsieme $Z$ (27 Settembre Esercizio 0.2) 
$$ d_Z(x)=\inf_{z\in Z } d(x,z)$$ 
 essa \`e continua e $ \overline{Z}=d_Z^{-1}(\{0\}) $ (27 Settembre Esercizio 0.5).\\
Siano $C,D$ chiusi disgiunti.\\
Definiamo  $$ f :\, X \to [0,3] \quad x \to \frac{3d_C(x)}{d_C(x)+d_D(x)} $$
tale funzione, essendo composizione di funzione continue \`e continua ed inoltre \`e ben definita: $C\cap D =\emptyset $ implica $d_C(x) \neq d_D(x)$.\\

$$U=f^{-1}([0,1)) \quad C=\overline{C}=f^{-1}(0) \subseteq U $$
$$V=f^{-1}((2,3]) \quad D=\overline{D}=f^{-1}(3) \subseteq V $$
Ora poich\`e $C$ e $D$ sono disgiunti anche $U$ e $V$.\\
Essendo gli intervalli $[0,1)$ e $(2,3]$ aperti in $[0,3]$ (con la topologia di sottospazio) $U$ e $V$ sono aperti 
\endproof
\end{prop}
\spazio
\begin{oss}
$$ \text{ metrizzabile} \implica ( T4 + T1 ) \implica (T3+T1) \implica T2 \implica T1 $$
in modo equivalente
$$ \text{ metrizzabile} \implica \text{ normale } \implica \text{ regolare } \implica \text{ Hussdorff } \implica T1$$
Infatti se $X$ \`e $T1$ allora i punti sono chiusi.
\end{oss}
Mostriamo che, in generale, le implicazioni opposte non sono vere.\\
(quelle che mancano nella prossima lezione)
\begin{itemize}

\item Normale $\not \implica$ metrizzabile . Sia $\R_S$ la retta di Sorgenfray.\\
$\R_S$,come sappiamo,  non \`e metrizzabile).\\
$\R_S$ \`e $T1$ in quanto ha una topologia meno fine di quella euclidea e $T2\implica T1$.\\
Resta da provare che \`e $T4$, siano $C,D$ chiusi disgiunti.
$$ \forall c \in C \quad c \in \R\sbarra D \text{ che \`e aperto quindi } \exists c' \in \R \quad [c,c') \subseteq \R \sbarra D $$
Dunque 
$$ C \subseteq U = \bigcup_{c\in C} [c,c') \text{ con } U \text{ aperto }$$
Similmente possiamo fare con $D$ ottenendo
$$ D\subseteq V = \bigcup_{d\in D} [d,d') \text{ con } V \text{ aperto }$$
Mostriamo che $U \cap V = \emptyset$ ovvero che $[c,c')\cap [d,d') = \emptyset$ $\forall c \in C \, d\in D$\\
Essendo $C$ e $ D$ disgiunti in particolare $c\neq d$, assumiamo senza perdere di generalit\`a che $c<d$.
$$ c \in \R \sbarra D \quad \implica \quad [c,c') \cap D = \emptyset \quad \implica \quad c ' \leq d $$ 

\item Regolare $\not \implica$ normale
\item $T2$  $\not \implica$ regolare
\item $T1$   $\not \implica$ $T2$. Sia $X$ uno spazio infinito con la topologia cofinita.\\
$X$ \`e $T1$ per l'osservazione~\ref{cofT1}, osserviamo che in $X$ non esistono aperti disgiunti.\\
Siano $U,V$ aperti disgiunti allora $U\subseteq X\sbarra V$ dunque $U$ \`e finito essendo $X\sbarra V$ finito.\\
 $X\sbarra U$ \`e un chiuso (complementare di un aperto) ma \`e infinito, ci\`o \`e assurdo.\\
Nella topologia cofinita i chiusi sono finiti 

\end{itemize}
\end{document}



