 \documentclass[a4paper,12pt]{article}
\usepackage[a4paper, top=2cm,bottom=2cm,right=2cm,left=2cm]{geometry}

\usepackage{bm,xcolor,mathdots,latexsym,amsfonts,amsthm,amsmath,
					mathrsfs,graphicx,cancel,tikz-cd,hyperref,booktabs,caption,amssymb,amssymb,wasysym}
\hypersetup{colorlinks=true,linkcolor=blue}
\usepackage[italian]{babel}
\usepackage[T1]{fontenc}
\usepackage[utf8]{inputenc}
\newcommand{\s}[1]{\left\{ #1 \right\}}
\newcommand{\sbarra}{\backslash} %% \ 
\newcommand{\ds}{\displaystyle} 
\newcommand{\alla}{^}  
\newcommand{\implica}{\Rightarrow}
\newcommand{\iimplica}{\Leftarrow}
\newcommand{\ses}{\Leftrightarrow} %se e solo se
\newcommand{\tc}{\quad \text{ t. c .} \quad } % tale che 
\newcommand{\spazio}{\vspace{0.5 cm}}
\newcommand{\bbianco}{\textcolor{white}{,}}
\newcommand{\bianco}{\textcolor{white}{,} \\}% per andare a capo dopo 																					definizioni teoremi ...


% campi 
\newcommand{\N}{\mathbb{N}} 
\newcommand{\R}{\mathbb{R}}
\newcommand{\Q}{\mathbb{Q}}
\newcommand{\Z}{\mathbb{Z}}
\newcommand{\K}{\mathbb{K}} 
\newcommand{\C}{\mathbb{C}}
\newcommand{\F}{\mathbb{F}}
\newcommand{\p}{\mathbb{P}}

%GEOMETRIA
\newcommand{\B}{\mathfrak{B}} %Base B
\newcommand{\D}{\mathfrak{D}}%Base D
\newcommand{\RR}{\mathfrak{R}}%Base R 
\newcommand{\Can}{\mathfrak{C}}%Base canonica
\newcommand{\Rif}{\mathfrak{R}}%Riferimento affine
\newcommand{\AB}{M_\D ^\B }% matrice applicazione rispetto alla base B e D 
\newcommand{\vett}{\overrightarrow}
\newcommand{\sd}{\sim_{SD}}%relazione sx dx
\newcommand{\nvett}{v_1, \, \dots , \, v_n} % v1 ... vn
\newcommand{\ncomb}{a_1 v_1 + \dots + a_n v_n} %a1 v1 + ... +an vn
\newcommand{\nrif}{P_1, \cdots , P_n} 
\newcommand{\bidu}{\left( V^\star \right)^\star}

\newcommand{\udis}{\amalg}
\newcommand{\ric}{\mathfrak{U}}
\newcommand{\inclu}{\hookrightarrow }
%ALGEBRA

\newcommand{\semidir}{\rtimes}%semidiretto
\newcommand{\W}{\Omega}
\newcommand{\norma}{\vert \vert }
\newcommand{\bignormal}{\left\vert \left\vert}
\newcommand{\bignormar}{\right\vert \right\vert}
\newcommand{\normale}{\triangleleft}
\newcommand{\nnorma}{\vert \vert \, \cdot \, \vert \vert}
\newcommand{\dt}{\, \mathrm{d}t}
\newcommand{\dz}{\, \mathrm{d}z}
\newcommand{\dx}{\, \mathrm{d}x}
\newcommand{\dy}{\, \mathrm{d}y}
\newcommand{\amma}{\gamma}
\newcommand{\inv}[1]{#1^{-1}}
\newcommand{\az}{\centerdot}
\newcommand{\ammasol}[1]{\tilde{\gamma}_{\tilde{#1}}}
\newcommand{\pror}[1]{\mathbb{P}^#1 (\R)}
\newcommand{\proc}[1]{\mathbb{P}^#1(\C)}
\newcommand{\sol}[2]{\widetilde{#1}_{\widetilde{#2}}}
\newcommand{\bsol}[3]{\left(\widetilde{#1}\right)_{\widetilde{#2}_{#3}}}
\newcommand{\norm}[1]{\left\vert\left\vert #1 \right\vert \right\vert}
\newcommand{\abs}[1]{\left\vert #1 \right\vert }
\newcommand{\ris}[2]{#1_{\vert #2}}
\newcommand{\vp}{\varphi}
\newcommand{\vt}{\vartheta}
\newcommand{\wt}[1]{\widetilde{#1}}
\newcommand{\pr}[2]{\frac{\partial \, #1}{\partial\, #2}}%derivata parziale
%per creare teoremi, dimostrazioni ... 
\theoremstyle{plain}
\newtheorem{thm}{Teorema}[section] 
\newtheorem{ese}[thm]{Esempio} 
\newtheorem{ex}[thm]{Esercizio} 
\newtheorem{fatti}[thm]{Fatti}
\newtheorem{fatto}[thm]{Fatto}

\newtheorem{cor}[thm]{Corollario} 
\newtheorem{lem}[thm]{Lemma} 
\newtheorem{al}[thm]{Algoritmo}
\newtheorem{prop}[thm]{Proposizione} 
\theoremstyle{definition} 
\newtheorem{defn}{Definizione}[section] 
\newcommand{\intt}[2]{int_{#1}^{#2}}
\theoremstyle{remark} 
\newtheorem{oss}{Osservazione} 
\newcommand{\di }{\, \mathrm{d}}
\newcommand{\tonde}[1]{\left( #1 \right)}
\newcommand{\quadre}[1]{\left[ #1 \right]}
\newcommand{\w}{\omega}

% diagrammi commutativi tikzcd
% per leggere la documentazione texdoc

\begin{document}
\textbf{Lezioni del 27  Febbraio del prof. Frigerio}
\begin{oss}Siano $p\, E \to X$ rivestimento, $F=\inv p(x)$ e $x_0 \in X$.\\
La monodomia $F \times \pi_1(X,x_0) \to F$ \`e transitiva $\ses$ E connesso.\\
\proof $\iimplica$ Dati $\tilde{x_0},\tilde{x_1} \in F$ se $E$ \`e connesso per archi 
$$ \exists \tilde{\gamma} :\, [0,1]\to E\text{ con }\tilde{\gamma}(0)=\tilde{x_0}\text{ e } \tilde{\amma}(1)=\tilde{x_1}$$
Posto $\amma = p \circ \tilde{\amma}$ per come \`e definita l'azione si ha
$$ \tilde{x_0}\az [ \gamma ] = \ammasol {x_0} (1)=\tilde{\gamma} (1) =\tilde{x_1}$$
$\implica$ Poich\`e l'azione \`e transitiva presi $2$ punti in $F$  allora esiste un cammino che collega questi 2 punti.\\
Sia $\tilde{x_0}, \tilde{x_1}\in E$, pongo $x_0=p(\tilde{x_0})$ e $x_1=p(\tilde{x_1})$.\\
Poich\`e $x_0, x_1 \in X$ che \`e connesso per archi, esiste un arco 
$\amma$ che li connette, sollevando $\gamma$  
a partire da $\tilde{x_0}$otteno un arco che collega $\tilde{x_0}$ a $\tilde{y}$.\\
Per definizione di rivestimento $\tilde{y}$ e $\tilde{x_1}$ sono nella stessa fibra e dunque sono connessi da un arco.\\
\end{oss}

\spazio
\begin{defn}$p:E \to X$ rivestimento, si dice universale se $E$ \`e semplicemente connesso.
\end{defn}
\begin{prop}$p:E \to X$ rivestimento universale.\\
Siano $x_0\in X$ e $\tilde{x_0}\in F = \inv{p}(x_0)$ allora 
$$ \psi:\, \pi_1(X,x_0) \to F \quad \psi([\gamma])=x_0 \az [\gamma]$$
\`e una bigezione
\proof
Poich\`e $E$ \`e connesso per archi, la surgettivit\`a discenda dal fatto che l'azione \`e transitiva.\\
Mostriamo che \`e iniettiva
$$ \psi ([ \gamma_1]) = \psi ([\gamma_2]) \quad \ses \quad \tilde{x_0}\az[\gamma_1] = \tilde{x_0}\az [\gamma_2] \quad \ses \quad \ssol{\gamma_1}{x_0} (1)= \ssol {\amma_2}{x_0}(1)$$
Poich\`e $E$ \`e semplicemente connesso 
$\ds \ssol{\gamma_1}{x_0} \sim \ssol {\amma_2}{x_0}$ (come cammini) 
dunque $\gamma_1=p \circ \ssol{\gamma_1}{x_0}$ e $\gamma_2=p \circ \ssol{\gamma_2}{x_0}$ sono omotopi come cammini da cui $[\gamma_1]=[\amma_2]$
\endproof
\end{prop}
\spazio
\begin{thm}$\pi_1\left( S^1 \right) \cong \Z$
\proof Come abbiamo osservato $p:\, \R \to S^1 $ tale che $p(t)=(\cos (2\pi t ), \sin (2\pi t))$ \`e un rivestimento .\\
Poniamo $F=\inv{p}((1,0))$ dunque $F=\Z$.\\
Poniamo 
$$ \psi:\, \pi_1\left( S^1, (1,0) \right) \to \Z \quad \psi([\gamma])= 0 \az [\gamma]$$
Poich\`e $\R$ \`e contraibile, \`e semplicemente connesso dunque $\psi$ \`e una bigezione.\\
Mostriamo che \`e anche un omomorfismo di gruppi.\\
Dati $[\alpha], [\beta]\in \pi_1\left( S^1,(1,0) \right)$ ho che 
$$ \psi ([\alpha]\cdot [\beta])=\psi([\alpha\star \beta])= \left(\tilde{\alpha\star \beta}\right)_0 (1)= \tilde{\alpha}_0 \star \tilde{\beta}_{\tilde{\alpha_0}(1)}(1)=\tilde{\beta}_{\tilde{\alpha}_0(1)}(1)$$
Ora $\tilde{\beta}_{{\tilde{\alpha}}_0(1)}$ e $\tilde{\alpha}_0(1)+\tilde{\beta}_0$ sono entrambi sollevamenti di $\beta$ a partire dallo stesso punto iniziale ($\tilde{\alpha}_0(1)$ \`e un numero \`e indica di quanto occorre "traslare") infatti
$$ p\left( \tilde{\alpha}_0(1)+\tilde{\beta}_0(t) \right)= p\left( k + \tilde{\beta}_0(t) \right)= p\left( \tilde{\beta}_0(t) \right)=\beta(t)$$
dove abbiamo sfruttato il fatto che la funzione $p$ \`e intero-periodica.\\
Concludiamo, osservando, 
$$ \psi([\alpha\star \beta])=\left(\tilde{\alpha_0}(1)+\tilde{\beta}_0\right)(1)=\tilde{\alpha_0}(1)+\tilde{\beta}_0(1) =\psi([\alpha])+\psi([\beta])$$
\end{thm}
\begin{oss}Tramite $\pi_1\left( S^1 \right) \cong \Z$ l'elemento $n\in \Z$ \`e rappresentato da $\amma(t)=( \cos (2n\pi t ) , \sin (2 n \pi t))$ in quanto $\tilde{\gamma}_0(t)=nt$ e $\tilde{\gamma}_0(1)=n$.\\
$\amma$ \`e un laccio  che fa $n$ giri di $S^1$
\end{oss}
\begin{prop}$R\subseteq X$ retratto. Allora $\forall x_0\in R$ si ha $i_\star$ iniettiva e $r_\star$ surgettiva
\proof Poich\`e $r\circ i = Id_R $ si ha $(r \circ i )_\star=Id_{\pi_1(R,x_0)}= r_\star \circ i_\star$.
\end{prop}
\begin{cor}$S^1=\partial D^2$ non \`e un retratto di $D^2$
\proof Essendo $D^2$ convesso \`e contraibile, dunque semplicemente connesso.\\
\end{cor}
\begin{thm}[del punto fisso di Brower]\bianco
Sia $f:\, D^2 \to D^2 \text{ continua}$ allora $f$ ha un punto fisso
\proof Supponiamo, per assurdo, $f(x) \neq x \, \, \forall x\in D^2$.\\
Mostriamo come costruire una retrazione $r:D^2\to S^1$.\\
Cerco $t\geq 0$ tale che $\left\vert \left \vert  f(x)+t(x-f(x))  \right \vert \right\vert ^2=1$.\\
Pongo dunque $r(x)=f(x)+t(x-f(x))$.\\
Mostriamo che $t$ dipende in modo continuo da $X$.\\
$t$ si ottiene risolvendo 
$$ 1=\norm{f(x)}^2+2t\langle f(x), x-f(x)\rangle + t^2 \norm{x-f(x)}^2$$
che \`e un'equazione di secondo grado, i cui coefficienti dipendono in modo continuo da $x$ .\\
La soluzione che ci interesse \`e quella della forma $\ds \frac{-b+\sqrt{\Delta}}{2a}$ perch\`e voglio $t\geq 0 $ e so che $ a>0$ \\
Per costruzione si ha che $r(x)\in S^1 \, \, \forall x \in D^2$ e $r(x)=x \,\, \forall x\in S^1$ dunque \`e una retrazione, il che \`e assurdo per il corollario precedente
\end{thm}
\begin{oss}La funzione costruita nel teorema \`e la funzione che associa ad $x$ il punto d'intersezione tra la semiretta uscente da $f(x)$ e passante da $x$ con $S^1$
\end{oss}
\begin{ex}La funzione $f:\, \C \sbarra \{ 0\}\to \C \sbarra\{ 0 \} $ data da $f(z)=z^n$ \`e un rivestimento di grado $n$
\end{ex}
\spazio 
\begin{thm}Dati $X,Y $ allora si ha $\pi_1(X\times Y, (x_0, y_0) ) \cong \pi_1(X,x_0) \times \pi_1(Y,y_0)$
\proof Siano $\pi_X$ e $\pi_Y$ le proiezioni canoniche, $i:\, X \to X\times Y $ e $j:\, Y \to X\times Y$ date da $i(x)=(x,1)$ e $j(y)=(1,y)$.\\
Pongo 
$$ \psi: \pi_1 ( X\times Y, (x_0, y_0)) \to \pi_1 (X,x_0) \times \pi_1(Y,y_0) \quad \psi([\alpha])=\left( \left( \pi_X \right)_\star([\alpha]), \left( \pi_Y \right)_\star ([\alpha]) \right)$$
$\psi$ \`e un ben definito omomorfismo di gruppi.\\ 
Mostriamo che \`e suriettivo.\\
 Dati $\beta\in \p1_(X,x_0)$ e $\gamma\in \pi_1(Y,y_0)$, poch\`e 
 $$\pi_X \circ i =Id_X \text{ e } \pi_Y \circ i = \text{costante}_{y_0}$$
si ha 
$$ \left( \pi_X \right)_\star \left( i_\star(\beta) \right)=\beta \text{ e } \left( \pi_Y \right)_\star \left( i_\star(\beta) \right)=1 $$ 
similmente si prova che 
$$ \left( \pi_X \right)_\star \left( j_\star(\gamma) \right)=1 \text{ e } \left( \pi_Y \right)_\star \left( j_\star(\gamma) \right)=\gamma $$ 
da cui $\psi\left( i_\star(\beta) \cdot j_\star(\gamma) \right)= (\beta, \gamma)$\\
Mostriamo l'iniettivit\`a .\\
Sia $\psi(\alpha)=1 $ con $\alpha=\left[ \left( \amma_1, \amma_2 \right)\right]$ dove 
$$ \gamma_1:\, [0,1]\to X \quad  \gamma_2:\, [0,1]\to Y$$
Se $H_1$ \`e un omotopia tra $\amma_1 $ e $c_{x_0}$ (a valori in $X$) e $H_2$ \`e un omotopia tra $\amma_2$ e $c_{y_0}$ allora la mappa
$$ H :\, [0,1]\times [0,1]\to X \times Y \quad H(t,s)=\left( H_1(t,s), H_2(t,s) \right)$$ 
mostra che $\alpha=1$\\
\endproof
\end{thm}
\begin{cor}
\end{cor} $$ \pi_1\left( S^1 \times S^1 \right) = \Z \oplus \Z$$
\end{document}