\documentclass[a4paper,12pt]{article}
\usepackage[a4paper, top=2cm,bottom=2cm,right=2cm,left=2cm]{geometry}

\usepackage{bm,xcolor,mathdots,latexsym,amsfonts,amsthm,amsmath,
					mathrsfs,graphicx,cancel,tikz-cd,hyperref,booktabs,caption,amssymb,amssymb,wasysym}
\hypersetup{colorlinks=true,linkcolor=blue}
\usepackage[italian]{babel}
\usepackage[T1]{fontenc}
\usepackage[utf8]{inputenc}
\newcommand{\s}[1]{\left\{ #1 \right\}}
\newcommand{\sbarra}{\backslash} %% \ 
\newcommand{\ds}{\displaystyle} 
\newcommand{\alla}{^}  
\newcommand{\implica}{\Rightarrow}
\newcommand{\iimplica}{\Leftarrow}
\newcommand{\ses}{\Leftrightarrow} %se e solo se
\newcommand{\tc}{\quad \text{ t. c .} \quad } % tale che 
\newcommand{\spazio}{\vspace{0.5 cm}}
\newcommand{\bbianco}{\textcolor{white}{,}}
\newcommand{\bianco}{\textcolor{white}{,} \\}% per andare a capo dopo 																					definizioni teoremi ...


% campi 
\newcommand{\N}{\mathbb{N}} 
\newcommand{\R}{\mathbb{R}}
\newcommand{\Q}{\mathbb{Q}}
\newcommand{\Z}{\mathbb{Z}}
\newcommand{\K}{\mathbb{K}} 
\newcommand{\C}{\mathbb{C}}
\newcommand{\F}{\mathbb{F}}
\newcommand{\p}{\mathbb{P}}

%GEOMETRIA
\newcommand{\B}{\mathfrak{B}} %Base B
\newcommand{\D}{\mathfrak{D}}%Base D
\newcommand{\RR}{\mathfrak{R}}%Base R 
\newcommand{\Can}{\mathfrak{C}}%Base canonica
\newcommand{\Rif}{\mathfrak{R}}%Riferimento affine
\newcommand{\AB}{M_\D ^\B }% matrice applicazione rispetto alla base B e D 
\newcommand{\vett}{\overrightarrow}
\newcommand{\sd}{\sim_{SD}}%relazione sx dx
\newcommand{\nvett}{v_1, \, \dots , \, v_n} % v1 ... vn
\newcommand{\ncomb}{a_1 v_1 + \dots + a_n v_n} %a1 v1 + ... +an vn
\newcommand{\nrif}{P_1, \cdots , P_n} 
\newcommand{\bidu}{\left( V^\star \right)^\star}

\newcommand{\udis}{\amalg}
\newcommand{\ric}{\mathfrak{U}}
\newcommand{\inclu}{\hookrightarrow }
%ALGEBRA

\newcommand{\semidir}{\rtimes}%semidiretto
\newcommand{\W}{\Omega}
\newcommand{\norma}{\vert \vert }
\newcommand{\bignormal}{\left\vert \left\vert}
\newcommand{\bignormar}{\right\vert \right\vert}
\newcommand{\normale}{\triangleleft}
\newcommand{\nnorma}{\vert \vert \, \cdot \, \vert \vert}
\newcommand{\dt}{\, \mathrm{d}t}
\newcommand{\dz}{\, \mathrm{d}z}
\newcommand{\dx}{\, \mathrm{d}x}
\newcommand{\dy}{\, \mathrm{d}y}
\newcommand{\amma}{\gamma}
\newcommand{\inv}[1]{#1^{-1}}
\newcommand{\az}{\centerdot}
\newcommand{\ammasol}[1]{\tilde{\gamma}_{\tilde{#1}}}
\newcommand{\pror}[1]{\mathbb{P}^#1 (\R)}
\newcommand{\proc}[1]{\mathbb{P}^#1(\C)}
\newcommand{\sol}[2]{\widetilde{#1}_{\widetilde{#2}}}
\newcommand{\bsol}[3]{\left(\widetilde{#1}\right)_{\widetilde{#2}_{#3}}}
\newcommand{\norm}[1]{\left\vert\left\vert #1 \right\vert \right\vert}
\newcommand{\abs}[1]{\left\vert #1 \right\vert }
\newcommand{\ris}[2]{#1_{\vert #2}}
\newcommand{\vp}{\varphi}
\newcommand{\vt}{\vartheta}
\newcommand{\wt}[1]{\widetilde{#1}}
\newcommand{\pr}[2]{\frac{\partial \, #1}{\partial\, #2}}%derivata parziale
%per creare teoremi, dimostrazioni ... 
\theoremstyle{plain}
\newtheorem{thm}{Teorema}[section] 
\newtheorem{ese}[thm]{Esempio} 
\newtheorem{ex}[thm]{Esercizio} 
\newtheorem{fatti}[thm]{Fatti}
\newtheorem{fatto}[thm]{Fatto}

\newtheorem{cor}[thm]{Corollario} 
\newtheorem{lem}[thm]{Lemma} 
\newtheorem{al}[thm]{Algoritmo}
\newtheorem{prop}[thm]{Proposizione} 
\theoremstyle{definition} 
\newtheorem{defn}{Definizione}[section] 
\newcommand{\intt}[2]{int_{#1}^{#2}}
\theoremstyle{remark} 
\newtheorem{oss}{Osservazione} 
\newcommand{\di }{\, \mathrm{d}}
\newcommand{\tonde}[1]{\left( #1 \right)}
\newcommand{\quadre}[1]{\left[ #1 \right]}
\newcommand{\w}{\omega}

% diagrammi commutativi tikzcd
% per leggere la documentazione texdoc

\begin{document}
\section{Indice di avvolgmento}
Fissato $a\in \C$, sappiamo che $\C\setminus \{a  \}$ si ritrae per deformazione su un cerchio di raggio $1$ e centro $a$, preso $x_0$ sul bordo del cerchio ($x_0\in \partial B(a,1)$), fissiamo un isomorfismo canonico
$$ f:\, \pi_1(\C\setminus \{a \}, x_0) \to \Z \quad [\gamma] \to 1 $$
dove $\gamma$ \`e una parametrizzazione di $\partial B(a,1)$ che percorre la circonferenza in senso antiorario.\\
Come gi\`a osservato esiste  una bigezione naturale
$$ \Omega(x_0,x_0) \to \W(S^1, x_0) \quad \gamma \to \hat{\gamma}$$
Questa bigezione induce un omomorfismo 
$$\partial:\, \pi_1(\C\setminus\{ a \}, x_0) \to [ S^1, \C \setminus\{ a \} \quad [\gamma]\to [\hat{\gamma}]$$
dove con $[S', \C \setminus \{a \}]$ intendiamo le classi di omotopie di mappe continue $S^1\to \C\setminus\{a\}$
Tale mappa risulta suriettiva ed inoltre $\partial([\gamma]) = \partial([\gamma'])$ se e solo se $[\gamma]$ e $[\gamma']$ sono coniugati.\\
Ora $\pi_1(\C\setminus\{a\}, x_0)$ \`e abeliano, dunque la mappa \`e iniettiva.\\
Definiamo per composizione la mappa 
$$ \psi:\,  \partial^{-1}\circ f:\,[S^1, \C\setminus\{a\}] \to \Z$$
\begin{defn}Sia $\gamam:\, [0,1]\to \C\setminus \{a\}$ un cammino chiuso in $x_0$.\\
Denotiamo indice di $\gamma$ rispetto ad $a$ l'intero $\psi([\hat{\gamma}])$ e lo denotiamo con $I(\gamma, a)$
\end{defn}
\begin{thm}Sia $\gamma:\, [0,1]\to \C\setminus\{a\}$ un cammino chiuso. Allora 
$$ I(\gamma, a) = \frac{1}{2\pi 1 }\int_\gamam \frac{\dz}{z-a}$$
\proof Come sappiamo $\frac{1}{z-a}$ \`e olomorfa in $\C\setminus \{a\}$ dunque la forma $\w=\frac{\dz}{z-a}$ \`e chiusa.\\
Possiamo integrare $\w$ lungo curve $\gamma$ continue, e il valore dell'integrale non dipende dal rappresentante nella classe di omotopia, dunque \`e ben definita la funzione 
$$ \varphi:\, [S^1, \C\setminus\{a \}] \to \C$$ 
$$ [\hat{\gamma}]\to \frac{1}{2\pi i }\int_\gamma \w $$
Mostriamo che $\varphi = \psi$ il che conclude la dimostrazione.\\
Notiamo che 
$$ [S^1, \C\setminus\{ a\}]  =\{ [\hat{\gamma_n}]\, : \, \gamma_n:\, [0,1]\to \C\setminus \{a\} \text{ dove } t \to a + e^{2\pi i n t}\}$$
dunque $\psi([\hat{\gamma_n}])  =n$ (avvolto $n$ volte).\\
Ora abbiamo visto che 
$$ \int_{\gamma_n} \frac{\dz}{z-a} =2\pi i n $$ 
da cui la tesi ]
\end{thm}
\end{document}