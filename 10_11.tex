\documentclass[a4paper,12pt]{article}
\usepackage[a4paper, top=2cm,bottom=2cm,right=2cm,left=2cm]{geometry}

\usepackage{bm,xcolor,mathdots,latexsym,amsfonts,amsthm,amsmath,
					mathrsfs,graphicx,cancel,tikz-cd,hyperref,booktabs,caption,amssymb,amssymb,wasysym}
\hypersetup{colorlinks=true,linkcolor=blue}
\usepackage[italian]{babel}
\usepackage[T1]{fontenc}
\usepackage[utf8]{inputenc}
\newcommand{\s}[1]{\left\{ #1 \right\}}
\newcommand{\sbarra}{\backslash} %% \ 
\newcommand{\ds}{\displaystyle} 
\newcommand{\alla}{^}  
\newcommand{\implica}{\Rightarrow}
\newcommand{\iimplica}{\Leftarrow}
\newcommand{\ses}{\Leftrightarrow} %se e solo se
\newcommand{\tc}{\quad \text{ t. c .} \quad } % tale che 
\newcommand{\spazio}{\vspace{0.5 cm}}
\newcommand{\bbianco}{\textcolor{white}{,}}
\newcommand{\bianco}{\textcolor{white}{,} \\}% per andare a capo dopo 																					definizioni teoremi ...


% campi 
\newcommand{\N}{\mathbb{N}} 
\newcommand{\R}{\mathbb{R}}
\newcommand{\Q}{\mathbb{Q}}
\newcommand{\Z}{\mathbb{Z}}
\newcommand{\K}{\mathbb{K}} 
\newcommand{\C}{\mathbb{C}}
\newcommand{\F}{\mathbb{F}}
\newcommand{\p}{\mathbb{P}}

%GEOMETRIA
\newcommand{\B}{\mathfrak{B}} %Base B
\newcommand{\D}{\mathfrak{D}}%Base D
\newcommand{\RR}{\mathfrak{R}}%Base R 
\newcommand{\Can}{\mathfrak{C}}%Base canonica
\newcommand{\Rif}{\mathfrak{R}}%Riferimento affine
\newcommand{\AB}{M_\D ^\B }% matrice applicazione rispetto alla base B e D 
\newcommand{\vett}{\overrightarrow}
\newcommand{\sd}{\sim_{SD}}%relazione sx dx
\newcommand{\nvett}{v_1, \, \dots , \, v_n} % v1 ... vn
\newcommand{\ncomb}{a_1 v_1 + \dots + a_n v_n} %a1 v1 + ... +an vn
\newcommand{\nrif}{P_1, \cdots , P_n} 
\newcommand{\bidu}{\left( V^\star \right)^\star}

\newcommand{\udis}{\amalg}
\newcommand{\ric}{\mathfrak{U}}
\newcommand{\inclu}{\hookrightarrow }
%ALGEBRA

\newcommand{\semidir}{\rtimes}%semidiretto
\newcommand{\W}{\Omega}
\newcommand{\norma}{\vert \vert }
\newcommand{\bignormal}{\left\vert \left\vert}
\newcommand{\bignormar}{\right\vert \right\vert}
\newcommand{\normale}{\triangleleft}
\newcommand{\nnorma}{\vert \vert \, \cdot \, \vert \vert}
\newcommand{\dt}{\, \mathrm{d}t}
\newcommand{\dz}{\, \mathrm{d}z}
\newcommand{\dx}{\, \mathrm{d}x}
\newcommand{\dy}{\, \mathrm{d}y}
\newcommand{\amma}{\gamma}
\newcommand{\inv}[1]{#1^{-1}}
\newcommand{\az}{\centerdot}
\newcommand{\ammasol}[1]{\tilde{\gamma}_{\tilde{#1}}}
\newcommand{\pror}[1]{\mathbb{P}^#1 (\R)}
\newcommand{\proc}[1]{\mathbb{P}^#1(\C)}
\newcommand{\sol}[2]{\widetilde{#1}_{\widetilde{#2}}}
\newcommand{\bsol}[3]{\left(\widetilde{#1}\right)_{\widetilde{#2}_{#3}}}
\newcommand{\norm}[1]{\left\vert\left\vert #1 \right\vert \right\vert}
\newcommand{\abs}[1]{\left\vert #1 \right\vert }
\newcommand{\ris}[2]{#1_{\vert #2}}
\newcommand{\vp}{\varphi}
\newcommand{\vt}{\vartheta}
\newcommand{\wt}[1]{\widetilde{#1}}
\newcommand{\pr}[2]{\frac{\partial \, #1}{\partial\, #2}}%derivata parziale
%per creare teoremi, dimostrazioni ... 
\theoremstyle{plain}
\newtheorem{thm}{Teorema}[section] 
\newtheorem{ese}[thm]{Esempio} 
\newtheorem{ex}[thm]{Esercizio} 
\newtheorem{fatti}[thm]{Fatti}
\newtheorem{fatto}[thm]{Fatto}

\newtheorem{cor}[thm]{Corollario} 
\newtheorem{lem}[thm]{Lemma} 
\newtheorem{al}[thm]{Algoritmo}
\newtheorem{prop}[thm]{Proposizione} 
\theoremstyle{definition} 
\newtheorem{defn}{Definizione}[section] 
\newcommand{\intt}[2]{int_{#1}^{#2}}
\theoremstyle{remark} 
\newtheorem{oss}{Osservazione} 
\newcommand{\di }{\, \mathrm{d}}
\newcommand{\tonde}[1]{\left( #1 \right)}
\newcommand{\quadre}[1]{\left[ #1 \right]}
\newcommand{\w}{\omega}

% diagrammi commutativi tikzcd
% per leggere la documentazione texdoc

\begin{document}
\section{Sottospazio prodotto}
\textbf{Lezione del 11 ottobre del prof. Frigerio}

\begin{prop}Sia $(X,d)$ metrico.\\
Allora $\overline{d}:\, X \times X \to \R $ definita come $\overline{d}(x,y)=\min\{ d(x,y), 1\}$ \`e una distanza topologicamente equivalente a $d$ dunque la topologia di uno spazio metrizzabile \`e indotta da una distanza $\leq 1$
\proof  \bbianco
\begin{itemize}
\item

Mostriamo che $\overline{d}$ \`e una distanza.\\
La non negativit\`a e la simmetria seguono in maniera diretta dalle analoghe propiet\`a su $d$.\\
Siano $x,y,z$ e proviamo che $\overline{d}(x,z)\leq \overline{d}(x,y) +\overline{d}(y,z)$\\
Se almeno uno tra $\overline{d}(x,y)$ e $\overline{d}(xyz)$ \`e uguale a 1 ho concluso infatti $\overline{d}(x,z)\leq 1$.\\
Altrimenti
$$ \overline{d}(x,z) \leq d(x,z)\leq d(x,y)+ d(y,z)$$
ma $\overline{d}(x,z) \neq 1 $ dunque $\overline{d}(x,y) = d(x,y)$ in modo analogo $\overline{d}(y,z)=d(y,z)$.
\item Mostriamo che le $2$ topologie indotte sono topologicamente equivalenti\\
Come base della topologia associata ad una distanza si prendono le palle di raggio $R$ al variare di $R<1$.\\
Ora $\forall x \in X $ e $\forall R<1 \quad B_d(x,R)= B_{\overline{d}}(x,R)$ dunque le 2 topologie coincidono
\end{itemize}
\endproof
\end{prop}
\spazio
\begin{defn}[Funzione lipschitziana]\bianco
Sia $f:\, (X,d)\to (Y, d') $ diciamo che $f$ \`e $k$-lipschitz se $k>0$ e
$$ \forall x_1, x_2 \in X \quad d'(f(x_1),f(x_2))\leq k \cdot d(x_1,x_2)$$
\end{defn}
\begin{oss}Sia $f:\, X \to Y $ una funzione $k$-lipschitz.\\
Poich\`e $f\left( B\left( x, \frac{\varepsilon}{k} \right) \right) \subseteq B(f(x),\varepsilon)$ una funzione lipschitziana \`e continua 
\end{oss}
\newpage
\begin{thm}Siano $(X_i,d_i)$ spazi metrici con $i\in \N$, allora
$$ X =\prod_{i \in \N} X_i \text{ \`e metrizzabile}$$
\proof Devo costruire $d:X\times X \to \R$ che induce la topologia prodotto.\\
$\forall i \in \N$ posso supporre che $d_i \leq 1$.\\
Denotiamo con $\ds (x_i)_{i \in \N} $ gli elementi  di $X$, dove $x_i \in X_i$.\\
Pongo
$$ d(x,y)= \sum_{i=0}^\infty 2^{-i} d_i(x_i,y_i)$$ 
tale serie converge poich\`e $d_i<1$ e la serie $\ds \sum 2^{-i}$ converge.\\
\`E di facile verifica che $d$ cos\`i definita \`e una distanza, sia $\tau_d$ la topologia che induce.\\
Sia $P_i:\, X\to X_i$ la proiezione su $X_i$ 
$$ d_i(\pi_i(x),\pi_i(y))= d_i(x_i,y_i) = 2^i(2^{-i} d_i(x_i,y_i) \leq 2^i d(x,y)$$
dunque la funzione $P_i$ \`e $2^i$-lipschitz dunque continua quindi $\tau_d > \tau_{prod}$
infatti $\tau_{prod}$ \`e la meno fine proiezione che rende continue le proiezioni.\\
Resta da vedere che $\tau_d> \tau_{prod}$.\\
Sia $B=B_d(x,\varepsilon)\subseteq X$ un aperto, dunque  per  $y \in B$  $\exists \delta>0 $ tale che $B_d(x,\delta)\subseteq B$.\\ Resta da dimostrare che $\exists U $ aperto di $\tau_{prod}$ con $y\in U \subseteq B_d(x,\delta)$.\\
Sia $n_0\in \N$ tale che $\ds \sum_{i=n_0+1}^\infty 2^{-i} <\frac{\delta}{2}$ che esiste essendo la serie convergente.
Pongo
$$ U = B_{d_0}\left( y_0, \frac{\delta}{4} \right)\times \dots \times 
 B_{d_{n_0}}\left( y_{n_0}, \frac{\delta}{4}\right) \times X_{n_0+1}\times \dots \times X_n \times \dots$$
 Se $z\in U $ allora $d_i(y_i,z_i)< \frac{\delta}{4} \quad  \forall i \leq n_0 $
Allora
$$ d(y,z) = \sum_{i=0}^\infty 2^{-1} d_i(y_i,z_i) = \sum_{i=0}^{n_0} 2^{-1}d(y_i,z_i)+ \sum_{i=n_0+1}^\infty 2^{-1} d_i(y_i,z_i)<$$ 
$$< \frac{\delta}{4} \sum_{i=0}^{n_0} 2^{-i}+ \sum_{i=n_0+1}^\infty < 2 \frac{\delta}{4} + \frac{\delta}{2}=\delta$$
Dunque $U\subseteq B_d(y, \delta)\subseteq B $
\endproof
 \end{thm}
 \begin{oss}Genericamente, il prodotto pi\`u che numerabile di spazi metrici non \`e primo-numerabile dunque non \`e metrizzabile
 \end{oss}
 \begin{fatto}
 Se $(X,d)$ e $(Y,d')$ sono metrici, $X \times Y $ \`e metrizzabile ed ha topologia indotta da una delle seguenti metriche
 \begin{enumerate}
 \item $d_\infty(x_1,y_1),(x_2,y_2)) = \max \{ d(x_1, x_2) ,d'(y_1,y_2)\}$
 \item $d_2(x_1,y_1),(x_2,y_2))= \sqrt{d(x_1,x_2)^2+ d'(y_1,y_2)^2}$
 \item $d_1(x_1,y_1),(x_2,y_2))= d(x_1,x_2)+ d'(y_1,y_2)$
 \end{enumerate}
 L'equivalenza si dimostra come abbiamo provato l'equivalenza su $\R^n$.\\
Per vedere che inducono la topologia prodotto usiamo il teorema precedente utilizzando $\frac{1}{2}d_1$  che \`e topologicamente equivalente a $d_1$ e ponendo $x=(x_1,x_2, 0, \dots )$ e $y=(y_1,y_2,0, \dots)$ 
 \end{fatto}
 \newpage
 \begin{ex}Sia $X$ topologico, $A,B \subseteq X$ 
 \begin{enumerate}
 \item $\overline{A\cup B}= \overline{A} \cup \overline{B}$
 \item L'uguaglianza \`e falsa per unioni infinite
 \item Enunciati "duali" per la parte interna
 \item In generale  $\overline{A\cap B} \neq \overline{A} \cap \overline{B}$
 \item $ \overline{\overline{A}^\circ}^\circ = \overline{A}^\circ$
 \item Trovare $A\subseteq X$ per cui $A$, $\overline{A}$, $\overline{A}^\circ$ , $ \overline{\overline{A}^\circ}$ sia tutte diverse
 \item Sia $A\subseteq Y \subseteq X$ allora la chiusura di $A$ in $Y$ (con topologia di sottospazio) \`e $\overline{A}\cap Y$
 \end{enumerate}
 \end{ex}
\end{document}