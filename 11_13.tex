\documentclass[a4paper,12pt]{article}
\usepackage[a4paper, top=2cm,bottom=2cm,right=2cm,left=2cm]{geometry}

\usepackage{bm,xcolor,mathdots,latexsym,amsfonts,amsthm,amsmath,
					mathrsfs,graphicx,cancel,tikz-cd,hyperref,booktabs,caption,amssymb,amssymb,wasysym}
\hypersetup{colorlinks=true,linkcolor=blue}
\usepackage[italian]{babel}
\usepackage[T1]{fontenc}
\usepackage[utf8]{inputenc}
\newcommand{\s}[1]{\left\{ #1 \right\}}
\newcommand{\sbarra}{\backslash} %% \ 
\newcommand{\ds}{\displaystyle} 
\newcommand{\alla}{^}  
\newcommand{\implica}{\Rightarrow}
\newcommand{\iimplica}{\Leftarrow}
\newcommand{\ses}{\Leftrightarrow} %se e solo se
\newcommand{\tc}{\quad \text{ t. c .} \quad } % tale che 
\newcommand{\spazio}{\vspace{0.5 cm}}
\newcommand{\bbianco}{\textcolor{white}{,}}
\newcommand{\bianco}{\textcolor{white}{,} \\}% per andare a capo dopo 																					definizioni teoremi ...


% campi 
\newcommand{\N}{\mathbb{N}} 
\newcommand{\R}{\mathbb{R}}
\newcommand{\Q}{\mathbb{Q}}
\newcommand{\Z}{\mathbb{Z}}
\newcommand{\K}{\mathbb{K}} 
\newcommand{\C}{\mathbb{C}}
\newcommand{\F}{\mathbb{F}}
\newcommand{\p}{\mathbb{P}}

%GEOMETRIA
\newcommand{\B}{\mathfrak{B}} %Base B
\newcommand{\D}{\mathfrak{D}}%Base D
\newcommand{\RR}{\mathfrak{R}}%Base R 
\newcommand{\Can}{\mathfrak{C}}%Base canonica
\newcommand{\Rif}{\mathfrak{R}}%Riferimento affine
\newcommand{\AB}{M_\D ^\B }% matrice applicazione rispetto alla base B e D 
\newcommand{\vett}{\overrightarrow}
\newcommand{\sd}{\sim_{SD}}%relazione sx dx
\newcommand{\nvett}{v_1, \, \dots , \, v_n} % v1 ... vn
\newcommand{\ncomb}{a_1 v_1 + \dots + a_n v_n} %a1 v1 + ... +an vn
\newcommand{\nrif}{P_1, \cdots , P_n} 
\newcommand{\bidu}{\left( V^\star \right)^\star}

\newcommand{\udis}{\amalg}
\newcommand{\ric}{\mathfrak{U}}
\newcommand{\inclu}{\hookrightarrow }
%ALGEBRA

\newcommand{\semidir}{\rtimes}%semidiretto
\newcommand{\W}{\Omega}
\newcommand{\norma}{\vert \vert }
\newcommand{\bignormal}{\left\vert \left\vert}
\newcommand{\bignormar}{\right\vert \right\vert}
\newcommand{\normale}{\triangleleft}
\newcommand{\nnorma}{\vert \vert \, \cdot \, \vert \vert}
\newcommand{\dt}{\, \mathrm{d}t}
\newcommand{\dz}{\, \mathrm{d}z}
\newcommand{\dx}{\, \mathrm{d}x}
\newcommand{\dy}{\, \mathrm{d}y}
\newcommand{\amma}{\gamma}
\newcommand{\inv}[1]{#1^{-1}}
\newcommand{\az}{\centerdot}
\newcommand{\ammasol}[1]{\tilde{\gamma}_{\tilde{#1}}}
\newcommand{\pror}[1]{\mathbb{P}^#1 (\R)}
\newcommand{\proc}[1]{\mathbb{P}^#1(\C)}
\newcommand{\sol}[2]{\widetilde{#1}_{\widetilde{#2}}}
\newcommand{\bsol}[3]{\left(\widetilde{#1}\right)_{\widetilde{#2}_{#3}}}
\newcommand{\norm}[1]{\left\vert\left\vert #1 \right\vert \right\vert}
\newcommand{\abs}[1]{\left\vert #1 \right\vert }
\newcommand{\ris}[2]{#1_{\vert #2}}
\newcommand{\vp}{\varphi}
\newcommand{\vt}{\vartheta}
\newcommand{\wt}[1]{\widetilde{#1}}
\newcommand{\pr}[2]{\frac{\partial \, #1}{\partial\, #2}}%derivata parziale
%per creare teoremi, dimostrazioni ... 
\theoremstyle{plain}
\newtheorem{thm}{Teorema}[section] 
\newtheorem{ese}[thm]{Esempio} 
\newtheorem{ex}[thm]{Esercizio} 
\newtheorem{fatti}[thm]{Fatti}
\newtheorem{fatto}[thm]{Fatto}

\newtheorem{cor}[thm]{Corollario} 
\newtheorem{lem}[thm]{Lemma} 
\newtheorem{al}[thm]{Algoritmo}
\newtheorem{prop}[thm]{Proposizione} 
\theoremstyle{definition} 
\newtheorem{defn}{Definizione}[section] 
\newcommand{\intt}[2]{int_{#1}^{#2}}
\theoremstyle{remark} 
\newtheorem{oss}{Osservazione} 
\newcommand{\di }{\, \mathrm{d}}
\newcommand{\tonde}[1]{\left( #1 \right)}
\newcommand{\quadre}[1]{\left[ #1 \right]}
\newcommand{\w}{\omega}

% diagrammi commutativi tikzcd
% per leggere la documentazione texdoc

\begin{document}
\textbf{Lezione del 13 Novembre del Prof. Frigerio}
\begin{defn}[Sottosuccessione]\bianco
Sia $\{ a_n\}$ una successione a valori in uno spazio topologico $X$.\\
Una sottosuccessione $\ds \{ a_{n_i}\}$ \`e una sottofamiglia di $\{a_n\}$ dove $\{ n_i\}$ \`e una successione strettamente crescente in $\N$
\end{defn}
\spazio
\begin{defn}Sia $X$ topologico.\\
$X$ si dice compatto per successioni se ogni successione in $X$ ammette una sottosuccessione convergente
\end{defn}
\begin{prop}Sia $X$ primo-numerabile
$$ X \text{ compatto } \quad \implica \quad X \text{ compatto per successioni}$$
\proof Sia $\{x_n\}\subseteq X$ una successione
$$\forall m \in \N \text{ sia } C_m = \overline{\{ x_n \, \vert \, n \geq m }$$ vale che $C_m$ chiuso e $C_{m+1}\subseteq C_m$.\\
Poich\`e $X$ compatto , deduciamo $\ds \bigcap_{m \in \N} C_m \neq \emptyset$, da cui sia $ \ds \overline{x}\in \bigcap_{m \in \N} C_m$\\
Costruiamo una successione che tende a $\overline{x}$.\\
Sia $\ds \{ U_i\}$ il sistema di fondamentale numerabile di  intorni di $\overline{x}$; a meno di sostituire $U_i$ con $ U_1 \cap \dots \cap U_i$ posso supporre $U_{i+1 } \subseteq U_i$ $\forall i\in \N$.\\
Costruisco induttivamente $\{x_{n_1}\}$ come segue:\\
Poich\`e $\overline{x}\in C_0$ allora $\exists n_0$ tale che $x_{n_0}\in U_0$ (se  $x\in \overline{C}$ allora tutti gli intorni di $\overline{x}$ intersecano $C$\\
So che $\overline{x}\in C_{n_0+1} $ allora $\exists n_1 \geq n_0+1 > n_0$ per cui $x_{n_1}\in U_1$\\
Procedo in questo modo costruendo una successione strettamente crescente $\{ n_i\}$ con  $x_{n_i} \in U_i$.\\
Proviamo che $x_{n_i}\to \overline{x}$.\\
Sia $U$ un generico intorno di $\overline{x}$, allora dalla definizione di sistema fondamentale di intorni si ha $$\exists i_0 \in \N \text{ tale che }U_{i_0} \subseteq U$$
Ora $\forall i\geq i_0 $ si ha $x_{n_i} \in U_i \subseteq U_{i_0} \subseteq U $ da cui la tesi
\endproof
\end{prop}
\spazio
\begin{thm}Sia $X$ secondo-numerabile
$$ X \text{ compatto } \quad \ses \quad X \text{ compatto per successioni}$$
\proof $\implica$ secondo-numerabile $\implica$ primo-numerabile da cui la tesi.\\
$\iimplica$ In modo contronominale.\\
Poich\`e $X$ non \`e compatto, $\exists \ric =\{ U_i\}$ ricoprimento con aperti di base senza sottoricoprimenti finiti .\\
Costruisco una successione, chiedendo che 
$$ x_i \in X \sbarra ( U_0 \cup \dots \cup U_{i})$$
la successione \`e ben definita poich\`e per definizione  di $\ric$ $U_0 \cup \dots \cup U_i \neq X $ $\forall i \in \N$ .\\
Supponiamo, per assurdo, che $x_n$ ammetta una sottosuccessione convergente e sia $\overline{x}$ questo limite.\\
Poich\`e $\ric$ \`e un ricoprimento di $X$ , $ x \in U_{i_0}$ per un certo $i_0 \in \N$, essendo $U_{i_0}$ un intorno di $\overline{x}$ allora per definizione di limite\\
$ \vert \{ i \in \N \, \vert \, x_i \in U_{i_0} \}\vert  = \infty$ ma per costruzione $x_i \not \in U_{i_0}$  se $i>i_0$\\
Ci\`o \`e assurdo, dunque, $x_n$ non ha sottosuccessioni convergenti \endproof
\end{thm}
\begin{lem}Sia $\{f_n\}$ una successione di funzioni con $f_n:\, X \to A $.
$$f_n \to f \text{ in } A^X \quad \ses \quad f_n \to f \text{ puntualmente} $$
\proof $\implica$ nella lezione dell 8 Novembre\\
$\iimplica$ Fisso $V$ intorno di $f$ in $A^x$, dunque, dalla definizione di topologia prodotto $f\in U \subseteq V $ dove 
$$ U = \bigcap_{i=1}^n \pi^{-1}(W_i) \text{ con } W_i \text{ intorno di } f(x_i) \text{ in } A $$
ovvero $W_i= ( f(x_i) -\varepsilon_i , f(x_i)+\varepsilon_i)$ dunque fissando $x_1, \dots, x_n$ e $\varepsilon_1, dots, \varepsilon_n$ ottengo
$$ U =\{ g:\, X \to A \, \vert \, \vert g(x_i)-f(x_i) \vert < \varepsilon_i \, \forall i=1,\dots, n \}$$
Poich\`e $f_n \to f $ puntualmente 
$$ \forall i=1, \dots, n \quad \exists n_1 \quad f_n(x_i)-f(x_i)\vert \leq \varepsilon_i$$
da cui
$$ \forall i=1, 
\dots, n \quad \exists n_0 =\max_{i=1,\dots, n } \{ n_i \} \quad \vert f_n (x_i) - f(x_i)\vert \leq \varepsilon_i \, \forall n \geq n_0 \quad  \implica \quad f_n \in U \subseteq V \, \forall n \geq n_0$$
Da cui $f_n \to f$ in $A^X$
\endproof
\end{lem}
\spazio
\begin{oss}In generale
$$ X \text{ compatto } \not \implica X \text{ compatto per successioni}$$
Prendiamo come esempio $X=[0,1]^{[0,1]}$.\\
Per Tychnoff $X$ \`e compatto.\\
Cerco una successione $f_n:\, [0,1]\to [0,1]$ senza sottosuccessioni puntualmente convergenti.\\
Pongo
$$ f^n(x)= 10^n \cdot x - \lfloor 10^n \cdot x \rfloor \text{ la parte decimale di } 10^n \cdot x $$
$f_n(x)$ \`e l'allineamento proprio decimale $0,a_1\, a_2 \dots \, a_n \dots $ dove $a_i$ \`e la $(n+i)$-esima cifra dopo la virgola dell'allineamento decimale proprio che rappresenta $x$.\\
Data una qualsiasi successione crescente $\{ n_i\}$ di indici, prendo $x$ avente $( i \mod 10)$ come $n_i+1$ cifra decimale.\\
Per costruzione $f_{n_i}(x)$ ha come prima cifra dopo la virgola $(i \mod 10)$ dunque $f_{n_i}$ non converge
\end{oss}
\end{document}