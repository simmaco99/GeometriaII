\documentclass[a4paper,12pt]{article}
\usepackage[a4paper, top=2cm,bottom=2cm,right=2cm,left=2cm]{geometry}

\usepackage{bm,xcolor,mathdots,latexsym,amsfonts,amsthm,amsmath,
					mathrsfs,graphicx,cancel,tikz-cd,hyperref,booktabs,caption,amssymb,amssymb,wasysym}
\hypersetup{colorlinks=true,linkcolor=blue}
\usepackage[italian]{babel}
\usepackage[T1]{fontenc}
\usepackage[utf8]{inputenc}
\newcommand{\s}[1]{\left\{ #1 \right\}}
\newcommand{\sbarra}{\backslash} %% \ 
\newcommand{\ds}{\displaystyle} 
\newcommand{\alla}{^}  
\newcommand{\implica}{\Rightarrow}
\newcommand{\iimplica}{\Leftarrow}
\newcommand{\ses}{\Leftrightarrow} %se e solo se
\newcommand{\tc}{\quad \text{ t. c .} \quad } % tale che 
\newcommand{\spazio}{\vspace{0.5 cm}}
\newcommand{\bbianco}{\textcolor{white}{,}}
\newcommand{\bianco}{\textcolor{white}{,} \\}% per andare a capo dopo 																					definizioni teoremi ...


% campi 
\newcommand{\N}{\mathbb{N}} 
\newcommand{\R}{\mathbb{R}}
\newcommand{\Q}{\mathbb{Q}}
\newcommand{\Z}{\mathbb{Z}}
\newcommand{\K}{\mathbb{K}} 
\newcommand{\C}{\mathbb{C}}
\newcommand{\F}{\mathbb{F}}
\newcommand{\p}{\mathbb{P}}

%GEOMETRIA
\newcommand{\B}{\mathfrak{B}} %Base B
\newcommand{\D}{\mathfrak{D}}%Base D
\newcommand{\RR}{\mathfrak{R}}%Base R 
\newcommand{\Can}{\mathfrak{C}}%Base canonica
\newcommand{\Rif}{\mathfrak{R}}%Riferimento affine
\newcommand{\AB}{M_\D ^\B }% matrice applicazione rispetto alla base B e D 
\newcommand{\vett}{\overrightarrow}
\newcommand{\sd}{\sim_{SD}}%relazione sx dx
\newcommand{\nvett}{v_1, \, \dots , \, v_n} % v1 ... vn
\newcommand{\ncomb}{a_1 v_1 + \dots + a_n v_n} %a1 v1 + ... +an vn
\newcommand{\nrif}{P_1, \cdots , P_n} 
\newcommand{\bidu}{\left( V^\star \right)^\star}

\newcommand{\udis}{\amalg}
\newcommand{\ric}{\mathfrak{U}}
\newcommand{\inclu}{\hookrightarrow }
%ALGEBRA

\newcommand{\semidir}{\rtimes}%semidiretto
\newcommand{\W}{\Omega}
\newcommand{\norma}{\vert \vert }
\newcommand{\bignormal}{\left\vert \left\vert}
\newcommand{\bignormar}{\right\vert \right\vert}
\newcommand{\normale}{\triangleleft}
\newcommand{\nnorma}{\vert \vert \, \cdot \, \vert \vert}
\newcommand{\dt}{\, \mathrm{d}t}
\newcommand{\dz}{\, \mathrm{d}z}
\newcommand{\dx}{\, \mathrm{d}x}
\newcommand{\dy}{\, \mathrm{d}y}
\newcommand{\amma}{\gamma}
\newcommand{\inv}[1]{#1^{-1}}
\newcommand{\az}{\centerdot}
\newcommand{\ammasol}[1]{\tilde{\gamma}_{\tilde{#1}}}
\newcommand{\pror}[1]{\mathbb{P}^#1 (\R)}
\newcommand{\proc}[1]{\mathbb{P}^#1(\C)}
\newcommand{\sol}[2]{\widetilde{#1}_{\widetilde{#2}}}
\newcommand{\bsol}[3]{\left(\widetilde{#1}\right)_{\widetilde{#2}_{#3}}}
\newcommand{\norm}[1]{\left\vert\left\vert #1 \right\vert \right\vert}
\newcommand{\abs}[1]{\left\vert #1 \right\vert }
\newcommand{\ris}[2]{#1_{\vert #2}}
\newcommand{\vp}{\varphi}
\newcommand{\vt}{\vartheta}
\newcommand{\wt}[1]{\widetilde{#1}}
\newcommand{\pr}[2]{\frac{\partial \, #1}{\partial\, #2}}%derivata parziale
%per creare teoremi, dimostrazioni ... 
\theoremstyle{plain}
\newtheorem{thm}{Teorema}[section] 
\newtheorem{ese}[thm]{Esempio} 
\newtheorem{ex}[thm]{Esercizio} 
\newtheorem{fatti}[thm]{Fatti}
\newtheorem{fatto}[thm]{Fatto}

\newtheorem{cor}[thm]{Corollario} 
\newtheorem{lem}[thm]{Lemma} 
\newtheorem{al}[thm]{Algoritmo}
\newtheorem{prop}[thm]{Proposizione} 
\theoremstyle{definition} 
\newtheorem{defn}{Definizione}[section] 
\newcommand{\intt}[2]{int_{#1}^{#2}}
\theoremstyle{remark} 
\newtheorem{oss}{Osservazione} 
\newcommand{\di }{\, \mathrm{d}}
\newcommand{\tonde}[1]{\left( #1 \right)}
\newcommand{\quadre}[1]{\left[ #1 \right]}
\newcommand{\w}{\omega}

% diagrammi commutativi tikzcd
% per leggere la documentazione texdoc

\begin{document}
\textbf{Lezione del 26 marzo}
\begin{defn}Sia $\w$ una 1-forma differenziale esatta su $D$ 
\begin{itemize}
\item $\w$ \`e esatta se $\exists F:\, D\to \C$ con $\w =\di F$
\item $\w$ \`e chiusa se \`e localmente esatta, ovvero 
$$\forall p\in D\, \, \exists U \subset D \text{ con } p\in U \text{ e } F_U:\, U \to \C \text{ con } \di F_U =\ris \w  U $$
\end{itemize}
In tal caso $F$ si chiama primitiva di $\w$, mentre $F_U$ primitiva locale di $\w$
\end{defn}
\begin{oss}
Ricordando che 
$\di F = \pr F x \di x + \pr F y \di y $ da cui $\w=P\di x + Q \di y $ \`e esatta se esiste $F$ tale che $P =\pr F x $ e $Q = \pr F y $
\end{oss}
\section{Integrali curvilinei}
\begin{defn}[Integrali complessi]\bianco 
Sia $f:\, [a,b]\to \C$ continua, poniamo 
$$\int_a^b f(t) \dt = \int_a^b Re(f(t))\dt + i \int_a^b Im(f(t))\dt$$
\end{defn}
\begin{defn}Sia $D$ un dominio aperto connesso di $\C$ e fissiamo  $\w$ una 1-forma differenziale complessa su $D$.\\
Se $\gamma:\, [a,b]\to D$ \`e un cammino $C^1$ definiamo 
$$\int_\gamma \w = \int_a^b \w_{\amma(t)} (\amma'(t))\dt$$
dove  se $\amma(t)=(x(t),y(t))=x(t) + iy(t)$ allora $\amma'(t) = (x'(t),y'(t))=x'(t) + iy'(t)$ mentre $\w_{\amma(t)} =\w(\gamma(t))$ 
\end{defn}
\begin{oss}Se $\w=P\di x + Q \di y $ allora 
$$\w_{\amma(t)}(\amma'(t)) = P(\amma(t))\di x (\amma'(t))+ Q(\amma(t))\di y (\amma'(t))=P(\amma(t))x'(t) + Q(\amma(t))y'(t)$$
\end{oss}
\spazio 
\begin{ese}Sia $D=\C^\star = \C \setminus\{ 0 \}$,  $\w = \frac{1}{z}\di z $ e $\gamma(t) =e^{2\pi i t } = \cos (2\pi t) + \sin (2\pi t)$\\
Calcolare $\int_\amma \w$\\ \\
Ricordando che $\di z $ \`e l'identit\`a di $\C$ e poich\`e $\amma'(t) = 2\pi i \gamma(t)$ ottengo 
$$\w_{\amma(t)}(\amma'(t)) = \frac{\di z}{\amma(t) }\cdot 2\pi i\gamma(t) = 2\pi i $$
da cui $\int_\amma \w=\int_{0}^1 2\pi i \dt = 2\pi i $.\\ \\
Svolgiamo lo stesso conto in coordinate (usando $\di x $ e $\di y$)
$$\frac{\di z}{z}= \frac{1}{x+iy }(\di x + i \di y ) = \frac{x-iy}{x^2+y^2}(\di x + i \di y) = \frac{x\di x + y\di y }{x^2+y^2} +i \frac{x\di y -  y \di x }{x^2+y^2}$$
Ora $\amma(t) = x(t) + i y(t)$ dove $x(t) = \cos (2\pi t) $ e $y(t) =\sin (2\pi t)$
$$\di x( \amma'(t))=x'(t) =-2\pi \sin(2\pi t)$$
$$\di y( \amma'(t))=y'(t) =2\pi \cos(2\pi t)$$
da cui 
$$(x\di x + y \di y)_{\amma(t)}(\amma'(t)) = \cos (2\pi t ) (-2\pi \sin (2\pi t))+\sin(2\pi t ) (2\pi \cos(2\pi t))=0$$

$$\tonde{i\frac{x\di y - y \di x}{x^2+y^2}}_{\amma(t)}(\amma'(t)) = i\frac{2pi (\cos^2 (2\pi t) + \sin^2(2\pi t))}{\cos^2 (2\pi t) + \sin^2(2\pi t)}=2\pi$$
Abbiamo ritrovato 
$$\int_\amma \frac{\di z }{x} =\int_\amma \frac{x\di x + y \di y}{x^2+y^2}+i \int_\amma \frac{x\di y - y \di y}{x^2 + y^2}=2\pi i $$

\end{ese}

\begin{prop}Propriet\`a elementari dell'integrale curvileneo
\begin{enumerate}
\item Sia $\amma:\, [a, b]\to D$, se $\psi:[c,d]\to [a,b]$ \`e $C^1$ con $\psi(c)=a $ e $\psi(d)=b$ allora
$$\int_\amma \w = \int_{\amma \circ \psi} \w $$
ovvero l'integrale non dipende dalle riparametrizzazioni che preservano il verso 
\item Sia $\amma:\, [a, b]\to D$, se $\psi:[c,d]\to [a,b]$ \`e $C^1$ con $\psi(c)=b $ e $\psi(d)=a$ allora
$$\int_\amma \w = -\int_{\amma \circ \psi} \w $$
\item Se $\amma = \amma_1\star \amma_2$ giunzione di cammini $C^1$ allora
$$\int_\amma\w = \int_{\amma_1}\w + \int_{\amma_2}\w$$
\item Se $\amma:\, [a,b]\to D$ \`e un cammino $C^1$ a tratti (ovvero continua e tale che esistono $a=t_0<t_1<\dots<t_n=b$ tali che $\ris{\amma} {[t_i,t_{i+1}]}$ sia $C^1$ allora
$$\int_\amma \w =\sum_{i=0}^n \int_{\ris \amma {[t_i,t_{i+1}]}} \w $$
\end{enumerate}
\proof\bbianco
\begin{enumerate}
\item $$\int_{\amma \circ \psi} \w = \int_c^d \w_{\amma(\psi(t))} ((\amma\circ \psi)'(t))\di t = 
\int_c^d \w_{\amma(\psi(t))} (\amma'(\psi(t))\psi'(t))\di t=\int_c^d \psi'(t) \cdot \w_{\amma(\psi(t))} (\amma'(\psi(t)))\di t$$
in quanto $\psi'(t)\in \R$ da cui 
$$\int_{\amma \circ \psi} \w = \int_{\psi(c)}^{\psi(t)} \w(\gamma(s))\gamma'(s)\di s = \int_{a}^{b} \w(\gamma(s))\gamma'(s)\di s= \int_\amma \w$$
\item Stessa dimostrazione del punto precedente ma usando il fatto che $\int_a^b f(t)\dt=\int_b^a f(t)\dt$
\end{enumerate}
\end{prop}

\begin{lem}Sia $D\subseteq \C$ aperto connesso. Allora $D$ \`e connesso per archi (in particolare per archi $C^1$ a tratti)
\proof Quasi identica a localmente connessa per archi implica connesso per archi.\\
Basta osservare che ogni punto di $D$ ha un intorno connesso per archi $C^1$: preso $x_0\in D$ si mostra che l'insieme dei punti di $D$ connessi a $x_0$ da un arco $C^1$ a tratti \`e aperto e chiuso
\end{lem}

\begin{lem}Sia $\w$ una 1-forma differenziale esatta su $D$ con $\w = \di F$.\\
Allora $\forall \amma:\, [a,b]\to D $ che \`e $C^1$ a tratti vale
$$\int_\amma \w = F(\amma(b))-F(\amma(a))$$
\proof Sia $a=t_0<t_1<\dots<t_n=b$ una partizione tale che $\ris \amma {[t_i,t_{i+1}]}$ sia $C^1$.\\
Basta provare che $\forall i$ vale
$$\int_{\ris \amma {[t_i,t_{i+1}]}}  \w =F(\amma(t_{i+1}))-F(\amma(t_i))$$ 
(in quanto si ha 
$$\int_\amma \w = \sum_{i=0}^{n-1} \int_{\ris \amma {[t_i,t_{i+1}]} } \w= F(\amma(t_n))-F(\amma(t_0))=F(\amma(b))-F(\amma(a))$$
i termini centrali si cancellano)\\
Ma 
$$\int_{\ris \amma {[t_i,t_{i+1}]}}\w = \int_{\ris \amma {[t_i,t_{i+1}]}}\di F =\int_{t_i}^{t_{i+1}} \di F_{\amma(t)} (\amma'(t))\di t =\int_{t_1}^{t_{i+1}} (F\circ \amma)'(t)\dt = F(\amma(t_{i+1}))-F(\amma(t_i)) $$
\end{lem}
\begin{cor}Sia $D\subseteq \C$ aperto connesso, $F:\, D\to \C$.
$$\di F = 0\quad \implica \quad F \text{ costante}$$
\proof $\forall a,b\in D$ esiste $\amma$ che connette $a$ a $b$ che \`e $C^1$ a tratti da cui 
$$F(b)-F(a) =\int_\amma \di F = 0$$
\end{cor}

\begin{cor}Sia $D\subseteq \C$ aperto connesso. Se $F$ \`e una primitiva di $\w$, tutte e sole le primitive di $\w$ si ottengono sommando una costante a $F$
\proof Se $G$ \`e un'altra primitiva, $\di G = \di F$ allora $\di (G-F) =0$ dunque $G=F + cost$.\\
Il viceversa\`e ovvio in quanto $\di (F+cost)=\di F$
\end{cor}

\begin{thm} Sia $D\subseteq \C$ aperto connesso, e $\w$ una 1-forma differenziale su $D$
$$\w \text{ esatta } \quad \ses \quad \int_\amma \w = 0 \text{ per ogni } \amma \text{ loop } C^1 \text{ a tratti a valori in } D$$
\proof $\implica$ Se $\amma:\, [a,b]\to D$ \`e un loop e $\w$ \`e esatta, $\w=\di F$ da cui 
$$\int_\amma \w =F(\amma(b))-F(\amma(a)) = 0 $$
se $\amma$ \`e un loop allora $\amma(a) =\amma(b)$\\
$\iimplica$ Costruiamo una primitiva di $\w$ come segue.\\
Fissato $x_0\in D$, allora $\forall p \in D$ scelgo un cammino $\amma_p:\,[0,1]\to D\, \, C^1$ a tratti con $\amma(0)=x_0$ e $\amma(1) =p$ e pongo $$F(p)=\int_{\amma_p}\w$$
Mostriamo che $F(p)$ \`e ben definita, cio\`e non dipende dalla scelta di $\amma_p$.\\
Sia $\alpha_p$ un altro cammino che congiunge $x_0$ a $p$, ora $\amma_p \star \overline{\alpha_p}$ \`e un loop, dunque,  per ipotesi 
$$0 = \int_{\amma_p\star \overline{\alpha}_p}\w = \int_{\amma_p} \w + \int_{\overline{\alpha}_p} \w = \int_{\amma_p} \w -\int_{\alpha_p}\w$$ 
cio\`e $\int_{\amma_p} \w = \int_{\alpha_p} \w$\\
Mostriamo che $F$ \`e differenziabile con $\di F = \w$.\\
Se $\w=P\di x+ Q\di y$ basta vedere che $\pr F x = P $  e $\pr F y = Q$ (in quanto $P$ e $Q$ sono continue per ipotesi da cui per il teorema del differenziale totale, $F$ ammetterebbe derivati parziali continue dunque \`e differenziabile con $\di F = \pr F x \di x + \pr F y \di y$).\\
Mostriamo che $\pr F x = P $ (l'altra si fa in modo analogo).\\
Sia $\amma_{z_0}$ un cammino che connette $x_0$ a $z_0\,\, C^1$ a tratti, sia $\gamma_h$ un cammino che congiunge $z_0$ a $z_0+h$ dunque
$$F(z_0+h)-F(z_0) =\int_{\gamma_{z_0}\circ \amma_h} \w - \int_{\amma_{z_0}}\w= \int_{\amma_h}\w$$
Sia $\amma_h (t) = z_0+t$ (cammino lungo $x$ e costante lungo $y$) per cui 
$$\w_{\amma_{h(t)}}(\amma'_h(t))=\w_{h(t)}(1) = P(z_0+t)\di x (1) + Q(z_0+t)\di y (1)= P(z_0+t)$$
dunque 
$$F(z_0+h) - F(z_0) =\int_0^h P(z_0+t) \dt $$
dividendo per $h$ otteniamo 
$$\frac{F(z_0+h)-F(z_0)}{h}=\frac{1}{h}\int_{0}^h P(z_0+t)\dt = P(z_0+\xi_h)$$
dove l'ultima uguaglianza deriva dal teorema della media integrale con $0\leq \xi_h\leq h$\\
Passando al limite per $h\to 0$ e usando la continuit\`a di $P$ otteniamo 
$\pr F x (z_0) = P(z_0)$
\endproof

\end{thm}




















\end{document}