\documentclass[a4paper,12pt]{article}
\usepackage[a4paper, top=2cm,bottom=2cm,right=2cm,left=2cm]{geometry}

\usepackage{bm,xcolor,mathdots,latexsym,amsfonts,amsthm,amsmath,
					mathrsfs,graphicx,cancel,tikz-cd,hyperref,booktabs,caption,amssymb,amssymb,wasysym}
\hypersetup{colorlinks=true,linkcolor=blue}
\usepackage[italian]{babel}
\usepackage[T1]{fontenc}
\usepackage[utf8]{inputenc}
\newcommand{\s}[1]{\left\{ #1 \right\}}
\newcommand{\sbarra}{\backslash} %% \ 
\newcommand{\ds}{\displaystyle} 
\newcommand{\alla}{^}  
\newcommand{\implica}{\Rightarrow}
\newcommand{\iimplica}{\Leftarrow}
\newcommand{\ses}{\Leftrightarrow} %se e solo se
\newcommand{\tc}{\quad \text{ t. c .} \quad } % tale che 
\newcommand{\spazio}{\vspace{0.5 cm}}
\newcommand{\bbianco}{\textcolor{white}{,}}
\newcommand{\bianco}{\textcolor{white}{,} \\}% per andare a capo dopo 																					definizioni teoremi ...


% campi 
\newcommand{\N}{\mathbb{N}} 
\newcommand{\R}{\mathbb{R}}
\newcommand{\Q}{\mathbb{Q}}
\newcommand{\Z}{\mathbb{Z}}
\newcommand{\K}{\mathbb{K}} 
\newcommand{\C}{\mathbb{C}}
\newcommand{\F}{\mathbb{F}}
\newcommand{\p}{\mathbb{P}}

%GEOMETRIA
\newcommand{\B}{\mathfrak{B}} %Base B
\newcommand{\D}{\mathfrak{D}}%Base D
\newcommand{\RR}{\mathfrak{R}}%Base R 
\newcommand{\Can}{\mathfrak{C}}%Base canonica
\newcommand{\Rif}{\mathfrak{R}}%Riferimento affine
\newcommand{\AB}{M_\D ^\B }% matrice applicazione rispetto alla base B e D 
\newcommand{\vett}{\overrightarrow}
\newcommand{\sd}{\sim_{SD}}%relazione sx dx
\newcommand{\nvett}{v_1, \, \dots , \, v_n} % v1 ... vn
\newcommand{\ncomb}{a_1 v_1 + \dots + a_n v_n} %a1 v1 + ... +an vn
\newcommand{\nrif}{P_1, \cdots , P_n} 
\newcommand{\bidu}{\left( V^\star \right)^\star}

\newcommand{\udis}{\amalg}
\newcommand{\ric}{\mathfrak{U}}
\newcommand{\inclu}{\hookrightarrow }
%ALGEBRA

\newcommand{\semidir}{\rtimes}%semidiretto
\newcommand{\W}{\Omega}
\newcommand{\norma}{\vert \vert }
\newcommand{\bignormal}{\left\vert \left\vert}
\newcommand{\bignormar}{\right\vert \right\vert}
\newcommand{\normale}{\triangleleft}
\newcommand{\nnorma}{\vert \vert \, \cdot \, \vert \vert}
\newcommand{\dt}{\, \mathrm{d}t}
\newcommand{\dz}{\, \mathrm{d}z}
\newcommand{\dx}{\, \mathrm{d}x}
\newcommand{\dy}{\, \mathrm{d}y}
\newcommand{\amma}{\gamma}
\newcommand{\inv}[1]{#1^{-1}}
\newcommand{\az}{\centerdot}
\newcommand{\ammasol}[1]{\tilde{\gamma}_{\tilde{#1}}}
\newcommand{\pror}[1]{\mathbb{P}^#1 (\R)}
\newcommand{\proc}[1]{\mathbb{P}^#1(\C)}
\newcommand{\sol}[2]{\widetilde{#1}_{\widetilde{#2}}}
\newcommand{\bsol}[3]{\left(\widetilde{#1}\right)_{\widetilde{#2}_{#3}}}
\newcommand{\norm}[1]{\left\vert\left\vert #1 \right\vert \right\vert}
\newcommand{\abs}[1]{\left\vert #1 \right\vert }
\newcommand{\ris}[2]{#1_{\vert #2}}
\newcommand{\vp}{\varphi}
\newcommand{\vt}{\vartheta}
\newcommand{\wt}[1]{\widetilde{#1}}
\newcommand{\pr}[2]{\frac{\partial \, #1}{\partial\, #2}}%derivata parziale
%per creare teoremi, dimostrazioni ... 
\theoremstyle{plain}
\newtheorem{thm}{Teorema}[section] 
\newtheorem{ese}[thm]{Esempio} 
\newtheorem{ex}[thm]{Esercizio} 
\newtheorem{fatti}[thm]{Fatti}
\newtheorem{fatto}[thm]{Fatto}

\newtheorem{cor}[thm]{Corollario} 
\newtheorem{lem}[thm]{Lemma} 
\newtheorem{al}[thm]{Algoritmo}
\newtheorem{prop}[thm]{Proposizione} 
\theoremstyle{definition} 
\newtheorem{defn}{Definizione}[section] 
\newcommand{\intt}[2]{int_{#1}^{#2}}
\theoremstyle{remark} 
\newtheorem{oss}{Osservazione} 
\newcommand{\di }{\, \mathrm{d}}
\newcommand{\tonde}[1]{\left( #1 \right)}
\newcommand{\quadre}[1]{\left[ #1 \right]}
\newcommand{\w}{\omega}

% diagrammi commutativi tikzcd
% per leggere la documentazione texdoc

\begin{document}
\textbf{Lezione del 9 ottobre di Gandini}
\section{Sottospazi topologici}

\begin{defn}[Topologia di sottospazio]\bianco
Sia $X$ uno spazio topologico e $Y \subseteq X$.\\
La topologia indotta su $Y$ detta topologia di sottospazio \`e la topologia meno fine che rende continua l'inclusione $i :X \inclu Y $
\begin{oss}La definizione \`e ben posta perch\`e l'intersezione di topologie che rendono $i$ continua \`e una topologia (meno fine) che rende $i$ continua
\end{oss}
\end{defn}
\begin{oss}Sia $\tau_Y$ una topologia su $Y$
$$ i \text{ continua } \ses U\cap Y = i^{-1}(U) \in \tau_Y \quad \forall U\in \tau_X$$
\end{oss}
\begin{prop}Sia $Y \subseteq X $ sottospazio topologico.
$$ A \subseteq Y \text{ aperto } \ses A=U \cap Y \text{ con } U \subseteq X \text{ aperto} $$
\proof $\implica$  $i^{-1}(U)=U \cap Y =A $ e dalla continuit\`a di $i$ segue che $A$ \`e aperto.\\
$\implica$ In modo ovvio si verifica che $\tau=\{ U \cap Y \, \vert \, U \subseteq X \text{ aperto }\}$ \`e una topologia, tale topologia rende continua $i$.\\
Dalla definizione di topologia di sottospazio si ha $\tau_{ssp}< \tau$ dunque $\forall A \in \tau_{ssp}$ si ha $A \in \tau$ ovvero $A=U\cap Y$ con $U$ aperto di $X$
\endproof
\end{prop}
\begin{oss}\label{base_sott}Supponiamo $\B$ base per la topologia su $X$.\\
Allora $\B'=\{ B \cap Y \, \vert \, B \in \B \}$ \`e una base per la topologia di sottospazio su $Y$
\end{oss}
\begin{prop}Sia $(X,d)$ metrico e $Y \subseteq X$.\\
Allora $\left( Y, d_{Y \times Y } \right)$ \'e uno spazio metrico.\\
La topologia di sottospazio su $Y$ coincide con la topologia indotta dalla restrizione della distanza.
\end{prop}
\spazio
\begin{defn}Sia $Y\subseteq X$ sottospazio metrico.\\
Allora diciamo che $Y$ \`e discreto se la topologia di sottospazio di $Y$ \`e quella discreta.\\
In modo equivalente: $\forall y \in Y \quad \exists U \subseteq X \text{ aperto}$ tale che $U\cap Y =\{y\}$
\end{defn}
\spazio
\begin{prop}[Propriet\`a universale delle immersioni]\bianco
Sia $Y\subseteq X$ un sottospazio topologico, $Z$ uno spazio topologico, $f:\, Z \to Y$ allora
$$ f\text{ continua } \ses i \circ f \text{ continua } $$
\proof $\implica$  la funzione $i$ \`e continua per definizione, inoltre composizione di funzioni continue \`e continua da cui la tesi.\\
$\iimplica$ Sia $i\circ f$ continua.\\
Sia $A\subseteq Y$ un aperto allora $\exists U \subseteq X $ aperto tale che $A=U \cap Y$
$$ f^{-1}(A)=f^{-1}(U \cap Y)= f^{-1}(i^{-1}(U)) = (i \circ f)^{-1}(U)$$ che \`e aperto per ipotesi
\endproof
\end{prop}
Il prossimo teorema ci fornisce il motivo per cui la propiet\`a \`e detta universale
\begin{thm}La propriet\`a universale caratterizza in modo unico la topologia di sottospazio di $Y \subseteq X$.\\
Vale a dire:\\
La topologia di sottospazio \`e l'unica topologia si $Y$ con la propriet\`a:
$$ \forall Z \text{ spazio topologico} \quad \forall f:\, Z \to Y $$
$$ f \text{ continua} \quad \ses \quad i \circ f \text{ continua}$$
\proof 
Abbiamo dimostrato che la topologia di sottospazio verifica la propriet\`a universale dobbiamo provare che \`e unica.\\
Indichiamo con $\tau_X$ la topologia su $X$ e con  $\tau_{ssp}$ la topologia di sottospazio su $Y$.\\
Sia $\tau_Y$ una topologia su $Y$ che verifica la propriet\`a universale, abbiamo dunque il seguente diagramma 
$$\begin{tikzcd} & (X, \tau_X) \\
(Z,\tau_Z) \arrow[bend left]{ur}{i \circ f} \arrow[r,"f"] &(Y,\tau_Y) \arrow[u, "i",hook]
\end{tikzcd}$$
\begin{itemize}
\item 
Prendiamo $(Z, \tau_Z)=(Y,\tau_{ssp})$ e $f=Id_Y$
ottenendo 
$$\begin{tikzcd} & (X, \tau_X) \\
(Y,\tau_{ssp}) \arrow[bend left]{ur}{i } \arrow[hook]{r}{id_Y} &(Y,\tau_Y) \arrow[u, "i",hook]
\end{tikzcd}$$
Ora per definizione di topologia di sottospazio $i:\, (Y,\tau_{ssp}) \inclu (X,\tau_X)$ \`e continua dunque $id_Y:\, (Y,\tau_{ssp}) \to (Y,\tau_Y)$ \`e continua dunque $\tau_{Y}<\tau_{ssp}$
\item Prendiamo $(Z,\tau_Z)= (Y,\tau_Y) $ e $f=Id_Y$ ottenendo
$$\begin{tikzcd} & (X, \tau_X) \\
(Y,\tau_{Y}) \arrow[bend left]{ur}{i } \arrow{r}{id_Y} &(Y,\tau_Y) \arrow[u, "i",hook]
\end{tikzcd}$$
Ora $id_Y:\, (Y,\tau_Y) \to (Y, \tau_Y)$ \`e continua quindi per la propriet\`a universale risulta continua anche $i:(Y,\tau_Y)\inclu (X,\tau_X)$ dunque poich\'e $\tau_{ssp}$ \`e la meno fine topologia che rende continua l'inclusione sicuramente $\tau_{ssp}<\tau_Y$
\end{itemize}
Valgono entrambe le inclusione dunque $\tau_Y=\tau_{ssp}$
\endproof
\end{thm}
Possiamo dare una nuova definizione, equivalente alla precedente
\begin{defn}La topologia di sottospazio \`e l'unica topologia su $Y$ con la prooiet\`a universale
\end{defn}
\newpage
\section{Applicazioni aperte e chiuse}
\begin{defn}Sia $f:\, X \to Y$ continua, allora
\begin{itemize}
\item $f$ \`e detta mappa aperta se $f(A)$ \`e un aperto $ \forall A$ aperto 
\item $f$ \`e detta mappa chiusa se $f(C)$ \`e un chiuso $\forall C$ chiuso 
\end{itemize}
\end{defn}
\begin{ese} Sia $X=(a,b)$.\\
$(a,b)\inclu \R$ non \`e chiusa infatti $(a,b)$ \`e un chiuso in $X$ ma $(a,b)$ (tutto l'insieme \`e sempre un chiuso) non \`e un chiuso in $\R$. La funzione \`e invece aperta \\
$[a,b]\inclu \R$ in modo analogo \`e chiusa ma non aperta\\
$[a,b)\inclu \R$ non \`e aperta e nemmeno chiusa
\end{ese} 
\begin{oss} $f$ continua e bigettiva $\not \implica$ $f$ omeomorfismo 
\end{oss}
\begin{oss}
$f:X\to Y$ continua e bigettiva, allora
$$ f \text{ omeomorfismo } \quad \ses \quad f \text{ aperta} \quad \ses \quad f\text{ chiusa}$$
Dove il secondo $\ses$ deriva dal fatto che gli assiomi di aperto e di chiuso sono tra loro equivalenti
\end{oss}
\spazio
\section{Immersioni}
\begin{defn}[Immersione]\bianco
Sia $f:\, X \to Y$ continua e iniettiva.\\
$f$ \`e detta immersione (topologica) se 
$$ A \subseteq X \text{ aperto} \quad \ses \quad \exists U \subseteq Y \text{ aperto } \quad A=f^{-1}(U)$$
in modo equivalente se 
$$ C \subseteq X \text{ chiuso } \quad \ses \quad \exists Z\subseteq Y \text{ chiuso } \quad C=f^{-1}(Z)$$
\end{defn}
\begin{ese}Supponiamo $X\subseteq Y$ allora 
$$i:\, X \inclu Y \text{ immersione} \quad \ses \quad \tau_X = \tau_{ssp}$$
\end{ese}
\begin{oss}$f:\, X \to Y $ \`e un immersione se e solo se l'applicazione indotta $\tilde{f}:\, X \to f(X)$ \`e un omeomorfismo ($f(X)$ \`e dotato della topologia di sottospazio)
\end{oss}
\begin{prop}Sia $f:\,X \to Y $ continua. Allora:
\begin{enumerate}

\item$f$ chiusa e iniettiva $\ses$ $f$ immersione chiusa $\ses$ $f$ immersione con $f(X)$ chiuso
\item $f$ aperta e iniettiva $\ses$ $f$ immersione aperta $\ses$ $f$ immersione con $f(X)$ aperto 
\end{enumerate}
\proof Dimostriamo la proposizione $1$ l'altra \`e analoga
\begin{itemize}
\item 
Chiaramente 
$$ f \text{ chiusa e iniettiva} \iimplica f \text{ immersione chiusa} \implica f \text{ immersione con } f(X)  \text{ chiuso}$$
\item $f$ \`e iniettiva e chiusa $\implica$ $f$ immersione chiusa.\\
Sia $C \subseteq X $ chiuso.\\ 
Essendo $f$ chiusa  $f(C)$ \`e un chiuso di $Y$.\\
Inoltre essendo $f$ iniettiva $f^{-1}(f(A))=A $ per $A\subseteq X$ dunque:
$$  C \subseteq X \text{ chiuso }  \quad \implica \quad \exists Z=f(C)\subseteq   Y \text{ chiuso} \quad f^{-1}(Z)=C$$
Inoltre $f^{-1}(Z) $ con $Z$ chiuso in $Y$ \`e un chiuso di $X$ essendo la funzione $f$ continua
\item $f$ immersione con $f(X)$ chiuso $\implica$ $f$ immersione chiusa.\\
Sia $C\subseteq X$ un chiuso, allora $f(C)= Z \cap f(X) $ con $Z\subseteq Y$ chiuso, infatti, dall'osservazione precedente
$$ \tilde{f}:\, X \to f(X) \text{ \`e un omeomorfismo dove } f(X) \text{ ha la topologia di sottospazio} $$
Ora $f(X)$, per ipotesi, \`e chiuso dunque anche $C=Z\cap f(X)$ \`e chiuso e quindi $f$ \`e chiusa
\end{itemize}
\endproof
\end{prop}
\newpage
\section{Sottospazi e assiomi di numerabilit\`a}

\begin{oss}Sia $X$ uno spazio topologico e $Y\subseteq X$ con la topologia di sottospazio
\begin{itemize}
\item $X$ secondo-numerabile $\implica$ $Y$ secondo-numerabile.\\
Segue dalla forma della base della topologia di sottospazio (Osservazione~\ref{base_sott})
\item $X$ primo-numerabile $\implica$ $Y$ primo-numerabile.
\item $X$ metrico separabile $\implica$ Y metrico separabile.\\
Segue dal fatto che 
$X$ metrico separabile $\ses$ $X$ secondo-numerabile e \\
$X$ secondo numerabile $\implica$  $Y$ secondo-numerabile 
\item $X$ separabile $\not \implica$ $Y$ separabile.\\
Consideriamo $\R^2$ con la topologia di Sorgenfrey: generata dagli insiemi  $[a,b)\times [c,d)$.\\
$\R^2$ con questa topologia \`e separabile infatti $\Q^2$ \`e numerabile e denso.\\
Consideriamo l'insieme $Z=\{ (x,-x) \, \vert \, x\in \R \} \subseteq \R^2$.\\
Osserviamo che ogni punto di $Z$ \`e intersezione di $Z$ con un aperto quindi $Z$ con la topologia di sottospazio \`e omeomorfo a $\R$ con la topologia discreta, per l'osservazione precedente $Z$ non pu\`o essere separabile.
\end{itemize}
\end{oss}
\end{document}