\documentclass[a4paper,12pt]{article}
\usepackage[a4paper, top=2cm,bottom=2cm,right=2cm,left=2cm]{geometry}

\usepackage{bm,xcolor,mathdots,latexsym,amsfonts,amsthm,amsmath,
					mathrsfs,graphicx,cancel,tikz-cd,hyperref,booktabs,caption,amssymb,amssymb,wasysym}
\hypersetup{colorlinks=true,linkcolor=blue}
\usepackage[italian]{babel}
\usepackage[T1]{fontenc}
\usepackage[utf8]{inputenc}
\newcommand{\s}[1]{\left\{ #1 \right\}}
\newcommand{\sbarra}{\backslash} %% \ 
\newcommand{\ds}{\displaystyle} 
\newcommand{\alla}{^}  
\newcommand{\implica}{\Rightarrow}
\newcommand{\iimplica}{\Leftarrow}
\newcommand{\ses}{\Leftrightarrow} %se e solo se
\newcommand{\tc}{\quad \text{ t. c .} \quad } % tale che 
\newcommand{\spazio}{\vspace{0.5 cm}}
\newcommand{\bbianco}{\textcolor{white}{,}}
\newcommand{\bianco}{\textcolor{white}{,} \\}% per andare a capo dopo 																					definizioni teoremi ...


% campi 
\newcommand{\N}{\mathbb{N}} 
\newcommand{\R}{\mathbb{R}}
\newcommand{\Q}{\mathbb{Q}}
\newcommand{\Z}{\mathbb{Z}}
\newcommand{\K}{\mathbb{K}} 
\newcommand{\C}{\mathbb{C}}
\newcommand{\F}{\mathbb{F}}
\newcommand{\p}{\mathbb{P}}

%GEOMETRIA
\newcommand{\B}{\mathfrak{B}} %Base B
\newcommand{\D}{\mathfrak{D}}%Base D
\newcommand{\RR}{\mathfrak{R}}%Base R 
\newcommand{\Can}{\mathfrak{C}}%Base canonica
\newcommand{\Rif}{\mathfrak{R}}%Riferimento affine
\newcommand{\AB}{M_\D ^\B }% matrice applicazione rispetto alla base B e D 
\newcommand{\vett}{\overrightarrow}
\newcommand{\sd}{\sim_{SD}}%relazione sx dx
\newcommand{\nvett}{v_1, \, \dots , \, v_n} % v1 ... vn
\newcommand{\ncomb}{a_1 v_1 + \dots + a_n v_n} %a1 v1 + ... +an vn
\newcommand{\nrif}{P_1, \cdots , P_n} 
\newcommand{\bidu}{\left( V^\star \right)^\star}

\newcommand{\udis}{\amalg}
\newcommand{\ric}{\mathfrak{U}}
\newcommand{\inclu}{\hookrightarrow }
%ALGEBRA

\newcommand{\semidir}{\rtimes}%semidiretto
\newcommand{\W}{\Omega}
\newcommand{\norma}{\vert \vert }
\newcommand{\bignormal}{\left\vert \left\vert}
\newcommand{\bignormar}{\right\vert \right\vert}
\newcommand{\normale}{\triangleleft}
\newcommand{\nnorma}{\vert \vert \, \cdot \, \vert \vert}
\newcommand{\dt}{\, \mathrm{d}t}
\newcommand{\dz}{\, \mathrm{d}z}
\newcommand{\dx}{\, \mathrm{d}x}
\newcommand{\dy}{\, \mathrm{d}y}
\newcommand{\amma}{\gamma}
\newcommand{\inv}[1]{#1^{-1}}
\newcommand{\az}{\centerdot}
\newcommand{\ammasol}[1]{\tilde{\gamma}_{\tilde{#1}}}
\newcommand{\pror}[1]{\mathbb{P}^#1 (\R)}
\newcommand{\proc}[1]{\mathbb{P}^#1(\C)}
\newcommand{\sol}[2]{\widetilde{#1}_{\widetilde{#2}}}
\newcommand{\bsol}[3]{\left(\widetilde{#1}\right)_{\widetilde{#2}_{#3}}}
\newcommand{\norm}[1]{\left\vert\left\vert #1 \right\vert \right\vert}
\newcommand{\abs}[1]{\left\vert #1 \right\vert }
\newcommand{\ris}[2]{#1_{\vert #2}}
\newcommand{\vp}{\varphi}
\newcommand{\vt}{\vartheta}
\newcommand{\wt}[1]{\widetilde{#1}}
\newcommand{\pr}[2]{\frac{\partial \, #1}{\partial\, #2}}%derivata parziale
%per creare teoremi, dimostrazioni ... 
\theoremstyle{plain}
\newtheorem{thm}{Teorema}[section] 
\newtheorem{ese}[thm]{Esempio} 
\newtheorem{ex}[thm]{Esercizio} 
\newtheorem{fatti}[thm]{Fatti}
\newtheorem{fatto}[thm]{Fatto}

\newtheorem{cor}[thm]{Corollario} 
\newtheorem{lem}[thm]{Lemma} 
\newtheorem{al}[thm]{Algoritmo}
\newtheorem{prop}[thm]{Proposizione} 
\theoremstyle{definition} 
\newtheorem{defn}{Definizione}[section] 
\newcommand{\intt}[2]{int_{#1}^{#2}}
\theoremstyle{remark} 
\newtheorem{oss}{Osservazione} 
\newcommand{\di }{\, \mathrm{d}}
\newcommand{\tonde}[1]{\left( #1 \right)}
\newcommand{\quadre}[1]{\left[ #1 \right]}
\newcommand{\w}{\omega}

% diagrammi commutativi tikzcd
% per leggere la documentazione texdoc


\begin{document}
\textbf{Lezione dei 25 Ottobre di Gandini }
\begin{defn}[Convesso]\bianco
$A\subseteq \R^n$ si dice convesso se 
$$ \forall x,y \in A \quad tx-(1-t)y \in A \quad \forall t \in [0,1]$$
Ovvero il segmento che congiunge $2$ punti dell'insieme \`e tutto contenuto nell'insieme
\end{defn}
\begin{oss}$A\subseteq \R^n $ convesso $\implica $ $A$ connesso per archi $\implica$ $A$ connesso.\\
Come cammino scelgo il segmento infatti, dalla definizione, \`e tutto contenuto nell'insieme $A$
\end{oss}
\spazio

\begin{prop}Sia $I \subseteq \R$.\\
I seguenti fatti sono equivalenti:
\begin{enumerate}
\item[(i)]$I$ \`e convesso ovvero \`e un intervallo
\item[(ii)] $I$ \`e connesso per archi
\item[(iii)] $I$ \`e connesso
\end{enumerate}
\proof (i)$\implica$(ii)$\implica$(iii).\\
Mostriamo che (iii)$\implica$(i) in modo contronominale.\\
Supponiamo che $I$ non sia convesso dunque 
$$ \exists a<b<c \quad a,c\in I \quad b \not \in I $$
dunque ottengo
$$ I = ( (-\infty, b) \cap I ) \cup ( (b,+\infty)\cap I)$$
ovvero $I$ si scrive come unione di $2$ aperti disgiunti, $I$ \`e sconnesso.
\endproof
\end{prop}
\begin{ese}$(0,1)$ non \`e omeomorfo a $[0,1]$ \\
Supponiamo, per assurdo che $f:\, [0,1) \to (0,1)$ sia un omeomorfismo.\\
Ora $[0,1) \sbarra \{ 0 \} $ \`e connesso mentre $(0,1) \sbarra \{ f(0)\}$ non lo \`e.\\
In modo analogo si prova che $(0,1), [0,1]$ e $[0,1), [0,1]$ non sono omeomorfi
\end{ese}
\spazio
\begin{ese} Connesso $\not \implica$ connesso per archi \proof
Sia 
$$ Y = \left\{ \left. \left( x, \sin \frac{1}{x}\right) \, \right\vert \, x >0 \right\} \subseteq \R^2 $$
$Y$ in modo ovvio \`e connesso per archi (il grafico della funzione $f(x)= \left( x , \sin \frac{1}{x}\right)$ \`e il cammino cercato)\\
Sia 
$$ X =\overline{Y}=Y \cup \{ (0,t) \, \vert \, \vert t\vert \leq 1 \}$$
$X$ \`e connesso in quanto chiusura di un connesso, mostriamo che non \`e connesso per archi.\\
Supponiamo, per assurdo, che esista un cammino 
$$ \alpha:\, [ 0, 1]\to X \quad \alpha(0)= (0,0) \text{ e } \alpha(1)\in Y$$
dove indicheremo $\alpha(t) = (x(t), y(y))$ 
$$ x:\, [ 0,1]\to [0, + \infty) \text{ continue}$$
$$y:\, [0,1]\to [-1,-1] \text{continue}$$
Sia $$\Omega=\{ t \in (0,1) \, \vert \, x(t)=0\}=x^{-1}(\{0\})$$
Tale insieme \`e chiuso (controimmagine di un chiuso) e limitato dunque ammette un massimo $t_0=\max \Omega$, per ipotesi $t_0<1$ infatti $\alpha(1)\in y$.\\
Supponiamo che $y(t_0)\geq 0$ (altra analoga) dunque per continuit\`a di $y$ 
\begin{equation}
\label{assu1}
 \exists \delta >0 \quad y(t) \geq - \frac{1}{2} \quad \forall t \in [t_0,t_0+\delta]
 \end{equation}
Ora $x ( [ t_0 ,t_0 + \delta] ) $ \`e un connesso che contiene $0=x(t_0)$ dunque
$$ \exists \varepsilon \quad [0,\varepsilon) \subseteq x ( [t_0, t_0 +\delta])$$
Per avere un assurdo, basta trovare $(\lambda, \mu) \in \alpha([t_0,t_0+\delta])$ con $\mu< -\frac{1}{2}$, ci\`o \`e assurdo per \ref{assu1}.\\
Cerchiamo $$\lambda\in (0,\varepsilon] \text{ con } \sin \frac{1}{\lambda}=-1 \quad \lambda =\frac{1}{2k\pi-\frac{\pi}{2}}$$
Dunque se prendiamo un $k>>0$ allora $\lambda\in (0, \varepsilon)$ da cui abbiamo un assurdo
\end{ese}

\newpage
\begin{defn}[Giunzione di cammini]\bianco
Siano $x,y,z\in X $ e  $\alpha,beta:\, [0,1]\to X $ cammini tali che 
$$ \begin{cases} \alpha(0)=x\\
\alpha(1)=y 
\end{cases}\text{ e } \begin{cases} \beta(0)=y\\
\beta(1)=z
\end{cases}$$
Allora la giunzione di $\alpha$ e $\beta$ \`e il cammino 
$$ \gamma= ( \alpha \star \beta )(t) = \begin{cases} 
\alpha(2t) \text{ se } t \in \left[ 0, \frac{1}{2}\right]\\
\beta(2t-1) \text{ se } t \in \left[\frac{1}{2},1\right]
\end{cases}$$
\end{defn}
\begin{oss}La giunzione \`e ben definita ed \`e un cammino che congiunge $x$ a $z$ infatti
$$ \gamma(0)=x \quad \gamma(1)=z \quad \gamma\left(\frac{1}{2}\right) = \alpha(1) =\beta(0)=y$$
Inoltre \`e continua poich\`e $[0,1]=\left[ 0, \frac{1}{2}\right] \cup \left[ \frac{1}{2}, 1 \right]$ \`e un ricoprimento chiuso finito dunque fondamentale, inoltre,$\gamma_{\vert \left[0, \frac{1}{2}\right]} $ e $\gamma_{\vert \left[ \frac{1}{2}, 1 \right]}$ sono continue dunque anche $\gamma$ lo \`e
\end{oss}
\begin{defn}[Componenti connesse per archi]\bianco
Dato $x\in X$ uno spazio topologico, la componente connessa per archi di $x$ \`e il massimo connesso per archi che contiene $x$.\\
In modo equivalente:\\
Dato $x\in X$ la sua componente connessa per archi \`e
$$ C_a(x) =\{ y \in X \, \vert \, \exists \alpha:\, [0,1] \to X \text{ con } \alpha(0)=x \text{ e } \alpha(1)=y \}$$
\end{defn}
Mostriamo che la definizione \`e ben posta
\begin{prop}
La relazione $\sim $ su $X$ definita come 
$$ x\sim y \quad \ses \quad \exists:\, \alpha:\, [0,1]\to X \text{ con } \alpha(0)=x \text{ e } \alpha(1)=y$$
\`e una relazione di equivalenza
\proof
\bbianco
\begin{itemize}
\item $\sim$ \`e riflessiva, prendo il cammino $\alpha(t)=x$ dunque $x\sim x$
\item $\sim$ \`e simmetrica\\
Sia $\alpha$ un cammino che congiunge $x$ con $y$ allora il cammino $\beta(t)=\alpha(1-t)$ congiunge $y$ con $x$
\item $\sim$ \`e transitiva.\\
SIa $\alpha$ cammino che congiunge $x$ a $y$ e $\beta$ cammino che congiunge $y$ a $z$ allora la loro giunzione congiunge $x$ a $z$
\end{itemize}
\endproof
\end{prop}
\begin{oss}Le componenti connesse per archi non sono n\`e aperte n\`e chiuse 
\end{oss}
\spazio
\begin{prop}Supponiamo che ogni punto abbia un intorno connesso.\\
Allora le componenti connesse sono aperte
\proof Sia $x\in X$ e $C(x)$ la sua componente connessa.\\
$$ \forall y \in C(x)  \text{ sia } U \in I(y) \text{ un intorno connesso} \quad \implica \quad U \subseteq C(y) \text{ per massimalit\`a}$$
Ora, poich\`e  le componenti connesse formano una partizione $C(x)=C(y)$.\\
$$ \forall y \in C(x) \quad \exists U \in I(y) \quad U \subseteq C(x)$$
\endproof
\end{prop}
\begin{prop}Supponiamo che ogni punto abbia un intorno connesso per archi.\\
Allora le componenti connesse per archi sono aperte e coincidono con le componenti connesse
\proof
Sia $x\in X$ e $C_a(x)$ la sua componente connessa per archi
$$\forall y \in C_a(x) \text{ sia } U\in I(y) \text{ intorno connesso per archi}$$
Ora $C_a(x) \cap U $ \`e connesso per archi dunque $U\subseteq C_a(x)$.\\
Vediamo che $C_a(x)=C(x)$.\\
Chiaramente $C_a(x) \subseteq C(x)$, mostriamo l'altra inclusione.\\
$$C(x) = \coprod_{y \in C(x)} C_a(y) \quad \implica \quad C(x) \cap C_a(x) \text{ aperto poich\`e unione di aperti } \quad \implica \quad C(x) \text{ connesso}$$
dunque $C_a(x)=C(x)$
\endproof
\end{prop}
\begin{defn}$X$ \`e detto localmente connesso se ogni punto ammette un sistema fondamentale di intorni connessi.\\
$X$ \`e detto localmente connesso per archi se ogni punto ammette un sistema fondamentale di intorni connessi per archi
\end{defn}
\spazio
\begin{ese}Spazio connesso  per archi non localmente connesso.
\proof Consideriamo $(\Q \times \R) \cup ( \R \times \{ 0\}$ unione delle rette verticale a distanza razionale unito ad una retta orizzantale.\\
Tale insieme prende il nome di pettine infinito.\\
Tale insieme \`e connesso per archi, ma non connesso.\\
Consideriamo una palla centrata in $(y,x)$ e raggio $<x$.\\
Allora la palla si pu\`o dividere con un irrazionale, dunque si scrive come unione di due aperti disgiunti
\end{ese}
\newpage
\begin{ex}Prodotto di connessi per archi \`e connesso per archi
\end{ex}
\begin{ex}Prodotto di 2 connessi \`e connesso 
\proof Supponiamo $X,Y$ connessi.\\
Per assurdo
$$ X \times Y = A \cup B\text{ decomposizione in aperti disgiunti}$$
Ora anche $X = \pi_X(A) \cup \pi_X(B)$ \`e una decomposizione in aperti, essendo $X$ connesso $x_0 \in \pi_X(A) \cap \pi_X(B)$.\\
Considero $\{x_0\} \times Y$ esse \`e omeomorfo a $Y$ dunque connesso.\\
Sia
$$ A_1= A \cap (\{ x_0 \times Y\}) \quad B_1= B \cap (\{ x_0 \times Y\})$$
dunque 
$$ \{ x_0 \} \times Y = A_1 \cup B_1 \text{ decomposizione in aperti non vuoti}$$
dunque $$A_1 \cap A_2 \neq \emptyset \quad \implica \quad A\cap B \neq \emptyset$$
\end{ex}
\begin{ex}Il prodotto arbitrario di connessi \`e connesso
\end{ex}
\begin{ex} Sia $n>1$ allora
$$\R^n \sbarra\{ \text{insieme numerabile} \} \text{ \`e connesso per archi} $$
\end{ex}
\end{document}