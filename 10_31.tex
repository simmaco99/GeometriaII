 \documentclass[a4paper,12pt]{article}
\usepackage[a4paper, top=2cm,bottom=2cm,right=2cm,left=2cm]{geometry}

\usepackage{bm,xcolor,mathdots,latexsym,amsfonts,amsthm,amsmath,
					mathrsfs,graphicx,cancel,tikz-cd,hyperref,booktabs,caption,amssymb,amssymb,wasysym}
\hypersetup{colorlinks=true,linkcolor=blue}
\usepackage[italian]{babel}
\usepackage[T1]{fontenc}
\usepackage[utf8]{inputenc}
\newcommand{\s}[1]{\left\{ #1 \right\}}
\newcommand{\sbarra}{\backslash} %% \ 
\newcommand{\ds}{\displaystyle} 
\newcommand{\alla}{^}  
\newcommand{\implica}{\Rightarrow}
\newcommand{\iimplica}{\Leftarrow}
\newcommand{\ses}{\Leftrightarrow} %se e solo se
\newcommand{\tc}{\quad \text{ t. c .} \quad } % tale che 
\newcommand{\spazio}{\vspace{0.5 cm}}
\newcommand{\bbianco}{\textcolor{white}{,}}
\newcommand{\bianco}{\textcolor{white}{,} \\}% per andare a capo dopo 																					definizioni teoremi ...


% campi 
\newcommand{\N}{\mathbb{N}} 
\newcommand{\R}{\mathbb{R}}
\newcommand{\Q}{\mathbb{Q}}
\newcommand{\Z}{\mathbb{Z}}
\newcommand{\K}{\mathbb{K}} 
\newcommand{\C}{\mathbb{C}}
\newcommand{\F}{\mathbb{F}}
\newcommand{\p}{\mathbb{P}}

%GEOMETRIA
\newcommand{\B}{\mathfrak{B}} %Base B
\newcommand{\D}{\mathfrak{D}}%Base D
\newcommand{\RR}{\mathfrak{R}}%Base R 
\newcommand{\Can}{\mathfrak{C}}%Base canonica
\newcommand{\Rif}{\mathfrak{R}}%Riferimento affine
\newcommand{\AB}{M_\D ^\B }% matrice applicazione rispetto alla base B e D 
\newcommand{\vett}{\overrightarrow}
\newcommand{\sd}{\sim_{SD}}%relazione sx dx
\newcommand{\nvett}{v_1, \, \dots , \, v_n} % v1 ... vn
\newcommand{\ncomb}{a_1 v_1 + \dots + a_n v_n} %a1 v1 + ... +an vn
\newcommand{\nrif}{P_1, \cdots , P_n} 
\newcommand{\bidu}{\left( V^\star \right)^\star}

\newcommand{\udis}{\amalg}
\newcommand{\ric}{\mathfrak{U}}
\newcommand{\inclu}{\hookrightarrow }
%ALGEBRA

\newcommand{\semidir}{\rtimes}%semidiretto
\newcommand{\W}{\Omega}
\newcommand{\norma}{\vert \vert }
\newcommand{\bignormal}{\left\vert \left\vert}
\newcommand{\bignormar}{\right\vert \right\vert}
\newcommand{\normale}{\triangleleft}
\newcommand{\nnorma}{\vert \vert \, \cdot \, \vert \vert}
\newcommand{\dt}{\, \mathrm{d}t}
\newcommand{\dz}{\, \mathrm{d}z}
\newcommand{\dx}{\, \mathrm{d}x}
\newcommand{\dy}{\, \mathrm{d}y}
\newcommand{\amma}{\gamma}
\newcommand{\inv}[1]{#1^{-1}}
\newcommand{\az}{\centerdot}
\newcommand{\ammasol}[1]{\tilde{\gamma}_{\tilde{#1}}}
\newcommand{\pror}[1]{\mathbb{P}^#1 (\R)}
\newcommand{\proc}[1]{\mathbb{P}^#1(\C)}
\newcommand{\sol}[2]{\widetilde{#1}_{\widetilde{#2}}}
\newcommand{\bsol}[3]{\left(\widetilde{#1}\right)_{\widetilde{#2}_{#3}}}
\newcommand{\norm}[1]{\left\vert\left\vert #1 \right\vert \right\vert}
\newcommand{\abs}[1]{\left\vert #1 \right\vert }
\newcommand{\ris}[2]{#1_{\vert #2}}
\newcommand{\vp}{\varphi}
\newcommand{\vt}{\vartheta}
\newcommand{\wt}[1]{\widetilde{#1}}
\newcommand{\pr}[2]{\frac{\partial \, #1}{\partial\, #2}}%derivata parziale
%per creare teoremi, dimostrazioni ... 
\theoremstyle{plain}
\newtheorem{thm}{Teorema}[section] 
\newtheorem{ese}[thm]{Esempio} 
\newtheorem{ex}[thm]{Esercizio} 
\newtheorem{fatti}[thm]{Fatti}
\newtheorem{fatto}[thm]{Fatto}

\newtheorem{cor}[thm]{Corollario} 
\newtheorem{lem}[thm]{Lemma} 
\newtheorem{al}[thm]{Algoritmo}
\newtheorem{prop}[thm]{Proposizione} 
\theoremstyle{definition} 
\newtheorem{defn}{Definizione}[section] 
\newcommand{\intt}[2]{int_{#1}^{#2}}
\theoremstyle{remark} 
\newtheorem{oss}{Osservazione} 
\newcommand{\di }{\, \mathrm{d}}
\newcommand{\tonde}[1]{\left( #1 \right)}
\newcommand{\quadre}[1]{\left[ #1 \right]}
\newcommand{\w}{\omega}

% diagrammi commutativi tikzcd
% per leggere la documentazione texdoc

\begin{document}
\textbf{Lezioni del 31  Ottobre del prof. Frigerio}
\begin{thm}Sia $X$ uno spazio topologico e $Y_i$ una famiglia di sottospazi chiusi con $Y_{i_0}$ compatto per qualche $i_0 \in I$.
$$\forall J \subseteq I \text{ finito } \quad  \bigcap_{i \in J } Y_i \neq \emptyset \quad \implica \quad \bigcap_{i \in I } Y_i \neq \emptyset$$
\proof $\forall i \in I$ poniamo $Z_i = Y_{i_0} \cap Y_i$ tale insieme \`e chiuso di $Y_{i_0}$.\\
Sia  $W_i = Y_{i_0} \sbarra Z_i  = Y_{i_0}\sbarra Y_i$ tale insieme \`e un aperto di $Y_{i_0}$.\\
Supponiamo, per assurdo, che $\bigcap_{i \in I }Y_i=\emptyset$.\\
La famiglia $\{W_i\}_{i \in I}$ \`e un ricoprimento aperto di $Y_{i_0}$ infatti: 
$$ \exists p \not \in \bigcup W_i \quad \implica \quad p\in Y_{i_0} \quad \implica \quad p \in \bigcap Y_i $$
Dalla compattezza di $Y_{i_0}$ segue che $$Y_{i_0}= W_{i_1}\cap \dots \cap W_{i_n} = \left( Y_{i_0} \sbarra Y_{i_1} \right) \cup \dots \cup \left( Y_{i_0} \sbarra Y_{i_n} \right) = Y_{i_0}\sbarra \left( Y_{i_1} \cap \dots \cap Y_{i_n} \right)$$
Ora se $A = A\sbarra B $ allora $A\cap B = \emptyset $ da cui
$$ Y_{i_0} \cap \left( Y_{i_1} \cap \dots \cap Y_{i_n} \right)=\emptyset$$
Ma ci\`o \`e assurdo in quanto ogni famiglia finita di $Y_i$ si deve intersecare in modo non banale.\\
\endproof
\end{thm}

\begin{cor}Sia $\{ Y_n \}_{n \in \N}$ \`e una famiglia di sottoinsiemi non vuoti  e chiusi di $X$.\\
Se $Y_0$  \`e compatto e $Y_{n+1} \subseteq Y_n$ $\forall n \in N $ allora $\ds \bigcap_{n\in \N} Y_n \neq \emptyset$
\end{cor}
\begin{oss} Il teorema \`e falso se non si richiede $Y_{i_0}$ compatto, prendiamo come controesempio  $X=\R$ e $Y_n=[n,+\infty)$ con $n \in \N$
\end{oss}
\spazio
\begin{thm}Sia $X$ uno spazio topologico e $\B$ una sua base.\\
Se ogni ricoprimento di $X$ con aperti di $\B$ ammette un sottoricoprimento finito, allora $X$ \`e compatto
\proof Sia $\ric =\ds \{ U_i\}_{i\in I}$ un generico ricoprimento aperto di $X$.\\
Per definizione di ricoprimento 
$$ \forall x \in X \quad \exists i(x) \in I \quad x \in U_{i(x)}$$
e dalla definizione di base
$$\exists B_x\in \B \quad x \in B_x \subseteq U_{i(x)}$$
Per costruzione $\{ B_x\}_{x\in X}$ \`e un ricoprimento aperto di $X$ con aperti di $\B$ (si dice che $\{ B_x\}$ \`e un raffinamento di $\ric$).\\
Per ipotesi 
$$ X= B_{x_i} \cup \dots \cup B_{x_n} \subseteq U_{i(x_1)} \cup \dots \cup U_{i(x_n)}$$
\endproof
\end{thm}
\spazio
\begin{thm}$X,Y$ compatto $\implica$ $X \times Y$ compatto
\proof Per il lemma posso partire da un ricoprimento di $X\times Y$ della forma
$$ \ric =\{ U_i \times V_i \}_{i \in I} \text{ dove } U_i \text{ aperto di }  X \text{ e } V_i \text{ aperto di } Y$$
$\forall x \in X$ il sottoinsieme $\{ x \} \times Y $ \`e compatto in quanto omeomorfo a $Y$, per cui
$$ \exists J_x \subseteq I \text{ finito } \quad \{ x \} \times Y \subseteq \bigcup_{ i \in J_x} (U_i \times V_i)$$
Pongo $U_x = \ds \bigcap_{i \in J_x} U_i$ tale insieme \`e aperto in quanto intersezione di finiti aperti.\\
Per costruzione $U_x \times Y \subseteq \ds \bigcup_{i \in J_x} (U_i \times V_i)$\\
Poich\`e $\{ U_x\}_{x\in X} $ \`e un ricoprimento aperto di $X$ compatto si ha $X= U_{x_1} \cup \dots \cup U_{x_n}$ allora 
$$ X \times Y = \subseteq \bigcup_{k=1}^n \left( U_{x_k} \times Y \right) \subseteq \bigcup_{k=1}^n \bigcup_{i \in J_{x_k}} ( U_i \times V_i)$$
ho ricoperto $X\times Y$ con finiti elementi di $\ric$\\
\endproof
\end{thm}
\begin{oss}Siano $A\subseteq X $ e $B\subseteq Y$.\\
La topologia prodotto di $A\times B$ (entrambi muniti della topologia di sottospazio) coincide con quella di sottospazio che $A\times B$ eredita da $X\times Y$.\\
Per il teorema precedente se $A,B$ sono sottospazio compatti allora $A\times B$ \`e un sottospazio compatto di $X\times Y$
\end{oss} 
\begin{prop} $$C\subseteq \R^n \text{ compatto } \quad \ses \quad C \text{ chiuso e limitato}$$
\proof $\implica$  Essendo $C$ compatto allora \`e limitato.\\
Essendo $\R^n$ metrico allora \`e di Hausdorff dunque i compatti sono chiusi\\
$\iimplica$ Se $C$ \`e limitato allora $\exists R>0$ per cui $C\subseteq [-R,R]^n \subseteq \R^n$.\\
Ora $[-R,R]$ \`e compatto in quanto prodotto finito di copie del compatto $[-R,R]\cong [0,1]$.\\
Se $C$ \`e chiuso, \`e perci\`o chiuso in un compatto dunque compatto.\\
\endproof
\end{prop}
\newpage
\begin{thm}$f:\, X \to Y$ continua con $X$ compatto e $Y$ di Housdorff allora $f$ \`e chiusa
\proof Se $C\subseteq X $ \`e chiuso, allora $C$ \`e compatto (chiuso in compatto) dunque $f(C)$ \`e compatto e perci\`o chiuso in quanto $Y$ \`e di Housdorff.\\
\endproof
\end{thm}
\spazio
\begin{defn}$X$ topologico si dice \textbf{compattamente generato} se i compatti di $X$ formano un ricoprimento fondamentale
\end{defn}
\begin{lem}Se ogni $x\in X$ ha un intorno compatto allora $X$ \`e compattamente generato
\proof Dalla definizione di ricoprimento fondamentale, basta vedere che 
$$ A\subseteq X  \text{ con } A \cap K \text{ aperto in } K \, \forall K \text{ compatto } \quad \implica \quad A \text{ aperto in } X $$
Sia $p\in A$ mostriamo che $p \in A^\circ$\\
Per ipotesi $\exists U$ intorno compatto di $p$ cio\`e $p\in U^\circ \subseteq U$.\\
Ora, per ipotesi, $A\cap U$ aperto in $U$ i dunque anche $A \cap U^\circ =(A \cap U )\cap U^\circ $ aperto in $U^\circ$.\\
Ora $ A \cap U^\circ$ \`e aperto in $X$ essendo aperto di aperto dunque 
$$ p \in  A\cap U^\circ \subseteq A \quad \implica \quad  p\in A^\circ \text{ essendo } A \cap U^\circ \text{ aperto }$$
\end{lem}
\begin{oss}$\R^n$ \`e compattamente generato in quanto ogni punto ammette un intorno compatto
\end{oss}
\begin{ex}Nessun punto di $\Q$ ammette un intorno compatto
\proof Supponiamo che $U \subseteq \Q $ sia un intorno di $p\in \Q$.\\
Essendo $U$ intorno $\exists V\subseteq \Q$ aperto di $\Q$ con $p\in V \subseteq U $ da cui 
$$ \exists \varepsilon>0 \quad p \in \left( \, (p-\varepsilon, p+\varepsilon) \cap Q \right) \subseteq U$$
Se $U$ fosse compatto allora $U$ chiuso in $\R$ da cui 
$$\overline{(p-\varepsilon, p+\varepsilon) \cap Q } = [p-\varepsilon,p+\varepsilon] \subseteq \Q$$
tale inclusione \`e assurda infatti $U \subseteq \Q $ ma $[p-\varepsilon , p+\varepsilon ] \not \subset \Q$
\end{ex}
\spazio
\begin{defn}$f:\, X\to Y$ \`e \text{propia} se $f^{-1}(K)$ \`e compatto di $X$ per ogni $K$ compatto di $Y$
\end{defn}
\begin{thm}$f:\, X \to Y$ continua e propia.\\
Se $Y$ \`e di Housdorff e compattamente generato allora $f$ \`e chiusa
\proof Sia $C\subseteq X $ chiuso.\\
Poich\`e $Y$ \`e compattamente generato basta vedere che $f(C)\cap K $ \`e chiuso in $K$ $\forall K\subseteq Y $ compatto.\\
Ora
$$ f(C) \cap K = f\left(C\cap f^{-1}(K) \right)$$
ed essendo $f$ propria allora $f^{-1}(K)$ \`e compatto.\\
Per cui $C \cap f^{-1}(K)$ \`e compatto (chiuso in un compatto).\\
Ora $f \left( C \cap f^{-1}(K) \right) $ \`e compatto ed essendo $T2$ \`e chiuso
\endproof
\end{thm}
\end{document}	