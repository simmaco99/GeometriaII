\documentclass[a4paper,12pt]{article}
\usepackage[a4paper, top=2cm,bottom=2cm,right=2cm,left=2cm]{geometry}

\usepackage{bm,xcolor,mathdots,latexsym,amsfonts,amsthm,amsmath,
					mathrsfs,graphicx,cancel,tikz-cd,hyperref,booktabs,caption,amssymb,amssymb,wasysym}
\hypersetup{colorlinks=true,linkcolor=blue}
\usepackage[italian]{babel}
\usepackage[T1]{fontenc}
\usepackage[utf8]{inputenc}
\newcommand{\s}[1]{\left\{ #1 \right\}}
\newcommand{\sbarra}{\backslash} %% \ 
\newcommand{\ds}{\displaystyle} 
\newcommand{\alla}{^}  
\newcommand{\implica}{\Rightarrow}
\newcommand{\iimplica}{\Leftarrow}
\newcommand{\ses}{\Leftrightarrow} %se e solo se
\newcommand{\tc}{\quad \text{ t. c .} \quad } % tale che 
\newcommand{\spazio}{\vspace{0.5 cm}}
\newcommand{\bbianco}{\textcolor{white}{,}}
\newcommand{\bianco}{\textcolor{white}{,} \\}% per andare a capo dopo 																					definizioni teoremi ...


% campi 
\newcommand{\N}{\mathbb{N}} 
\newcommand{\R}{\mathbb{R}}
\newcommand{\Q}{\mathbb{Q}}
\newcommand{\Z}{\mathbb{Z}}
\newcommand{\K}{\mathbb{K}} 
\newcommand{\C}{\mathbb{C}}
\newcommand{\F}{\mathbb{F}}
\newcommand{\p}{\mathbb{P}}

%GEOMETRIA
\newcommand{\B}{\mathfrak{B}} %Base B
\newcommand{\D}{\mathfrak{D}}%Base D
\newcommand{\RR}{\mathfrak{R}}%Base R 
\newcommand{\Can}{\mathfrak{C}}%Base canonica
\newcommand{\Rif}{\mathfrak{R}}%Riferimento affine
\newcommand{\AB}{M_\D ^\B }% matrice applicazione rispetto alla base B e D 
\newcommand{\vett}{\overrightarrow}
\newcommand{\sd}{\sim_{SD}}%relazione sx dx
\newcommand{\nvett}{v_1, \, \dots , \, v_n} % v1 ... vn
\newcommand{\ncomb}{a_1 v_1 + \dots + a_n v_n} %a1 v1 + ... +an vn
\newcommand{\nrif}{P_1, \cdots , P_n} 
\newcommand{\bidu}{\left( V^\star \right)^\star}

\newcommand{\udis}{\amalg}
\newcommand{\ric}{\mathfrak{U}}
\newcommand{\inclu}{\hookrightarrow }
%ALGEBRA

\newcommand{\semidir}{\rtimes}%semidiretto
\newcommand{\W}{\Omega}
\newcommand{\norma}{\vert \vert }
\newcommand{\bignormal}{\left\vert \left\vert}
\newcommand{\bignormar}{\right\vert \right\vert}
\newcommand{\normale}{\triangleleft}
\newcommand{\nnorma}{\vert \vert \, \cdot \, \vert \vert}
\newcommand{\dt}{\, \mathrm{d}t}
\newcommand{\dz}{\, \mathrm{d}z}
\newcommand{\dx}{\, \mathrm{d}x}
\newcommand{\dy}{\, \mathrm{d}y}
\newcommand{\amma}{\gamma}
\newcommand{\inv}[1]{#1^{-1}}
\newcommand{\az}{\centerdot}
\newcommand{\ammasol}[1]{\tilde{\gamma}_{\tilde{#1}}}
\newcommand{\pror}[1]{\mathbb{P}^#1 (\R)}
\newcommand{\proc}[1]{\mathbb{P}^#1(\C)}
\newcommand{\sol}[2]{\widetilde{#1}_{\widetilde{#2}}}
\newcommand{\bsol}[3]{\left(\widetilde{#1}\right)_{\widetilde{#2}_{#3}}}
\newcommand{\norm}[1]{\left\vert\left\vert #1 \right\vert \right\vert}
\newcommand{\abs}[1]{\left\vert #1 \right\vert }
\newcommand{\ris}[2]{#1_{\vert #2}}
\newcommand{\vp}{\varphi}
\newcommand{\vt}{\vartheta}
\newcommand{\wt}[1]{\widetilde{#1}}
\newcommand{\pr}[2]{\frac{\partial \, #1}{\partial\, #2}}%derivata parziale
%per creare teoremi, dimostrazioni ... 
\theoremstyle{plain}
\newtheorem{thm}{Teorema}[section] 
\newtheorem{ese}[thm]{Esempio} 
\newtheorem{ex}[thm]{Esercizio} 
\newtheorem{fatti}[thm]{Fatti}
\newtheorem{fatto}[thm]{Fatto}

\newtheorem{cor}[thm]{Corollario} 
\newtheorem{lem}[thm]{Lemma} 
\newtheorem{al}[thm]{Algoritmo}
\newtheorem{prop}[thm]{Proposizione} 
\theoremstyle{definition} 
\newtheorem{defn}{Definizione}[section] 
\newcommand{\intt}[2]{int_{#1}^{#2}}
\theoremstyle{remark} 
\newtheorem{oss}{Osservazione} 
\newcommand{\di }{\, \mathrm{d}}
\newcommand{\tonde}[1]{\left( #1 \right)}
\newcommand{\quadre}[1]{\left[ #1 \right]}
\newcommand{\w}{\omega}

% diagrammi commutativi tikzcd
% per leggere la documentazione texdoc

\begin{document}
\textbf{Lezione del 14 Novembre del Prof. Frigerio}
\begin{defn}[Successione di Cauchy]\bianco
Sia $(X,d)$ uno spazio metrico.\\
Una successione $\{ a_n\}\subseteq X$ \`e detta di Cauchy se 
$$ \forall \varepsilon>0 \quad \exists n_0 \quad d(x_n, x_m)\leq \varepsilon \quad \forall n,m \geq n_0$$
\end{defn}
\begin{lem}Se $\{ x_n\}$ \`e convergente allora \`e di Cauchy
\proof Sia $\overline{x}$ il limite di $x_n$ dunque 
$$ \forall \varepsilon>0 \quad \exists n_0 \quad d(x_n,\overline{x}) \leq \frac{\varepsilon}{2} \quad \forall n \geq n_0$$
dunque $$\forall n, m \geq n_0 \quad d(x_n,x_m) \leq d(x_n , \overline{x}) + d(x_m, \overline{x}) = \frac{\varepsilon}{2}+ \frac{\varepsilon}{2}=\varepsilon$$
\endproof
\end{lem}
\spazio
\begin{defn}[Spazio completo]\bianco
$X$ spazio topologico si dice completo se ogni successione di Cauchy in $X$ \`e convergente
\end{defn}
\begin{oss}$\Q$ non \`e completo, sia $\{ x_n \} \subseteq \Q $ convergente a $\sqrt{2}$ tale successione \`e di Cauchy, essendo convergente, ma non converge in $\Q$
\end{oss}
\spazio
\begin{lem}Sia $\{ x_n\}$ di Cauchy. Se $\{ x_n \} $ ammette una sottosuccessione convergente allora $\{ x_n\}$ \`e essa stessa convergente.
\proof Poich\`e $\{ x_n\}$ \`e di Cauchy
$$ \forall \varepsilon>0 \quad \exists n_0 \quad d(x_n, x_m)\leq \frac{\varepsilon}{2} \quad \forall n, m \geq n_0$$
Ora sia $\overline{x}=\ds \lim_{i \to + \infty} x_{n_i}$ dunque per definizione di limite
$$ \forall \varepsilon>0 \quad \exists j \in \N  \quad d (\overline{x}, x_{n})\leq \frac{\varepsilon}{2} \quad \forall n\geq n_j$$
Sia $n_{\overline{i}} \geq n_j \geq n_0$ dunque 
$$ \forall n \geq n_{\overline{i}} \quad d(\overline{x},x_n) \leq d(\overline{x},x_{n_{\overline{i}}})+ d(x_{n_{\overline{i}}},x_n)\leq \frac{\varepsilon}{2}+\frac{\varepsilon}{2}= \varepsilon$$
dunque $\ds \lim_{n\to +\infty} x_n =\overline{x} $
\endproof \end{lem}
\begin{cor}Sia $X$ uno spazio metrico
$$ X \text{ compatto per successioni} \quad \implica \quad X \text{ completo}$$
\end{cor}
\spazio
\begin{lem}Sia $X$ \`e completo  e $Y \subseteq X $ con la metrica indotta
$$ Y \text{ completo } \quad \ses \quad Y \text{ chiiuso in } X $$
\proof In modo contronominale.\\
Se $Y$ non \`e chiuso allora $\exists \{ y_n\} \subseteq Y $ con $\ds \lim_{n\to \infty} y_n =\overline{y}\not \in Y $,infatti essendo $Y$  primo numerabile, 
 $\overline{Y}= \{ \text{ punti limiti delle successioni } \subseteq Y  \} $.\\
Ora $\{ y_n\}$ converge, dunque \`e di Cauchy ma non converge in $Y$, $Y$ non \`e completo.\\
$\iimplica$ Sia $\{ y_n\}\subseteq Y$ di Cauchy.\\
Per completezza di $X$ $\ds\exists \lim_{n\to \infty} y_n =\overline{x}$, ora essendo $Y$ chiuso $\overline{x}\in \overline{Y}=Y $ dunque $Y$ completo 
\endproof
\end{lem}
\spazio
\begin{defn} Sia $X$ uno spazio metrico.\\
$X$ \`e totalmente limitato se $\forall\varepsilon>0$ esiste un ricoprimento finito di $X$ fatto con palle di raggio $\varepsilon$
\end{defn}
\begin{lem}$$ X \text{ totalmente limitato } \quad \implica \quad X \text{ limitato}$$
\proof Dalla totale  limitatezza  ponendo $\varepsilon=1$ si ottiene 
$$ X = \bigcup_{i=1}^n  B(x_i, 1)$$
Posto $M=\max_{i,j=1,\dots, n } \{ d(x_i,x_j)$ 
$$ \forall x, y \in X \quad \exists x_i, x_j \quad x \in B(x_i,1) \text{ e } y \in B(x_j,1)$$
da cui
$$ d(x,y)\leq d(x,x_i)+ d(x_i,x_j)+ d(x_j,y)\leq 1 + M +1 +M = M +2 $$
\endproof
\begin{oss}Non vale il viceversa.\\
Se su $\R$ poniamo $d(x,y)=\min \{ \vert x-y \vert, 1\}$ allora $(\R,d)$ \`e limitato ma non totalmente limitato, non esistono ricoprimenti finiti con palle di raggio $1/2$
\end{oss}
\end{lem}
\begin{prop} $(X,d)$ totalmente limitato $\implica$ a base numerabile
\proof Per un teorema gi\`a visto basta vedere che \`e separabile (metrico e separabile implica a base numerabile)
$$ \forall n \quad \exists F_n \subseteq X \text{ finito con } X = \bigcup_{p \in F_n} B\left( p, \frac{1}{n} \right)$$
Allora $\bigcup_{n \geq 1} F_n $ \`e numerabile, mostriamo che \`e denso.\\
Sia $U\subseteq X$ un aperto non vuoto, dunque $\exists n_0 \in \N $ $\exists z \in X$ tale che $B \left( z , \frac{1}{n_0}\right) \subseteq U $.\\
Ora le palle di raggio $\frac{1}{n_0}$ sono un ricoprimento da cui  $\exists p \in F_{n_0}$ tale che  $z \in B\left( p, \frac{1}{n_0} \right)$.\\
Dunque $p \in \bigcup F_n $ e $ p  \in B \left( z, \frac{1}{n_0}\right) \subseteq U $ da cui $\bigcup F_n$ interseca ogni aperto non vuoto da cui denso.
\endproof
\end{prop}

\begin{thm}
 Sia $(X,d)$ metrico. I seguenti fatti sono equivalenti
 \begin{itemize}
  \item[(i)] $X$ compatto
  \item[(ii)] $X$ compatto per successioni
  \item[(iii)] $X$ totalmente limitato e completo
 \end{itemize}
 Inoltre in questi casi $X$ ammette una base numerabile
 \proof \bbianco
\begin{itemize}
 \item (i) $\implica$ (ii) Deriva dal fatto che metrico implica primo-numerabile
 \item (ii) $\implica$ (iii) Abbiamo gi\`a visto che compatto per successione implica completo.\\
 Mostriamo che $X$ \`e totalmente limitato in modo contronominale.\\
 Se $X$ non fosse totalmente limitato allora $\exists \varepsilon>0$ tale che $X$ non sia ricoperto da un numero finito di palle di raggio $\varepsilon$.\\
 Costruiamo induttivamente una successione $\{x_n\}$.\\
 Prendiamo $x_0\in X$ (qualunque).\\
 Poniamo $x_{n+1}\not\in B(x_0,\varepsilon) \cup \dots \cup B(x_n, \varepsilon)$.\\
 Tale successione non ammette sottosuccessione convergenti infatti $d(x_n, x_m) \geq \varepsilon $ per tutti gli $n \neq m $
\item (iii) $\implica$ (i) Poich\`e totalmente limita implica a base numerabile, sotto le ipotesi (iii) vale (i) $\ses$ (ii).\\
Mostriamo, dunque, che $X$ \`e compatto per successioni.\\
Sia $\{ x_n\}$ una successione in $X$, essendo $X$ completo, troviamo una sottosuccessione di Cauchy (dunque convergente).\\
Per totale limitatezza, $\forall n \in \N$ esiste un ricoprimento finito $\ric_n$ di X fatto con palle di raggio $2^{-n}$.
Essendo $\ric_0$ finito
$$ \exists W_0 \in \ric_0 \quad I_0=\{ n \in \N \, \vert \, \vert \, x_n \in W_0 \} \text{ infinito}$$
(la successione si deve ripartire in finite palle, dunque deve esistere una palla che contiene infinti termini della successione)\\
Pongo $n_0=\min I_0$.\\
Essendo $\ric_1$ finito
$$ \exists W_1 \in \ric_1 \quad I_1=\{ n \in \N \, \vert \, \vert \, n> n_0 \text{ e } x_n \in W_1 \} \text{ infinito}$$

Pongo $n_1=\min I_1$.\\
Proseguo induttivamente ottenendo una sottosuccessione $\{x_{n_i}\}$.\\
Mostriamo che tale successione \`e di Cauchy; per ogni $i,j \geq j_0$ segue che $x_{n_i}, x_{n_j} \in W_{j_0}$\\
Ora $W_{j_0}$ \`e una palla di raggio $2^{-j_0}$ da cui $d(x_{n_i}, x_{n_j}) \leq 2^{-j_0+1} < \varepsilon$
\end{itemize}
\endproof
\end{thm}


\end{document}