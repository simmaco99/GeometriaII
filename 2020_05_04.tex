\documentclass[a4paper,12pt]{article}
\usepackage[a4paper, top=2cm,bottom=2cm,right=2cm,left=2cm]{geometry}

\usepackage{bm,xcolor,mathdots,latexsym,amsfonts,amsthm,amsmath,
					mathrsfs,graphicx,cancel,tikz-cd,hyperref,booktabs,caption,amssymb,amssymb,wasysym}
\hypersetup{colorlinks=true,linkcolor=blue}
\usepackage[italian]{babel}
\usepackage[T1]{fontenc}
\usepackage[utf8]{inputenc}
\newcommand{\s}[1]{\left\{ #1 \right\}}
\newcommand{\sbarra}{\backslash} %% \ 
\newcommand{\ds}{\displaystyle} 
\newcommand{\alla}{^}  
\newcommand{\implica}{\Rightarrow}
\newcommand{\iimplica}{\Leftarrow}
\newcommand{\ses}{\Leftrightarrow} %se e solo se
\newcommand{\tc}{\quad \text{ t. c .} \quad } % tale che 
\newcommand{\spazio}{\vspace{0.5 cm}}
\newcommand{\bbianco}{\textcolor{white}{,}}
\newcommand{\bianco}{\textcolor{white}{,} \\}% per andare a capo dopo 																					definizioni teoremi ...


% campi 
\newcommand{\N}{\mathbb{N}} 
\newcommand{\R}{\mathbb{R}}
\newcommand{\Q}{\mathbb{Q}}
\newcommand{\Z}{\mathbb{Z}}
\newcommand{\K}{\mathbb{K}} 
\newcommand{\C}{\mathbb{C}}
\newcommand{\F}{\mathbb{F}}
\newcommand{\p}{\mathbb{P}}

%GEOMETRIA
\newcommand{\B}{\mathfrak{B}} %Base B
\newcommand{\D}{\mathfrak{D}}%Base D
\newcommand{\RR}{\mathfrak{R}}%Base R 
\newcommand{\Can}{\mathfrak{C}}%Base canonica
\newcommand{\Rif}{\mathfrak{R}}%Riferimento affine
\newcommand{\AB}{M_\D ^\B }% matrice applicazione rispetto alla base B e D 
\newcommand{\vett}{\overrightarrow}
\newcommand{\sd}{\sim_{SD}}%relazione sx dx
\newcommand{\nvett}{v_1, \, \dots , \, v_n} % v1 ... vn
\newcommand{\ncomb}{a_1 v_1 + \dots + a_n v_n} %a1 v1 + ... +an vn
\newcommand{\nrif}{P_1, \cdots , P_n} 
\newcommand{\bidu}{\left( V^\star \right)^\star}

\newcommand{\udis}{\amalg}
\newcommand{\ric}{\mathfrak{U}}
\newcommand{\inclu}{\hookrightarrow }
%ALGEBRA

\newcommand{\semidir}{\rtimes}%semidiretto
\newcommand{\W}{\Omega}
\newcommand{\norma}{\vert \vert }
\newcommand{\bignormal}{\left\vert \left\vert}
\newcommand{\bignormar}{\right\vert \right\vert}
\newcommand{\normale}{\triangleleft}
\newcommand{\nnorma}{\vert \vert \, \cdot \, \vert \vert}
\newcommand{\dt}{\, \mathrm{d}t}
\newcommand{\dz}{\, \mathrm{d}z}
\newcommand{\dx}{\, \mathrm{d}x}
\newcommand{\dy}{\, \mathrm{d}y}
\newcommand{\amma}{\gamma}
\newcommand{\inv}[1]{#1^{-1}}
\newcommand{\az}{\centerdot}
\newcommand{\ammasol}[1]{\tilde{\gamma}_{\tilde{#1}}}
\newcommand{\pror}[1]{\mathbb{P}^#1 (\R)}
\newcommand{\proc}[1]{\mathbb{P}^#1(\C)}
\newcommand{\sol}[2]{\widetilde{#1}_{\widetilde{#2}}}
\newcommand{\bsol}[3]{\left(\widetilde{#1}\right)_{\widetilde{#2}_{#3}}}
\newcommand{\norm}[1]{\left\vert\left\vert #1 \right\vert \right\vert}
\newcommand{\abs}[1]{\left\vert #1 \right\vert }
\newcommand{\ris}[2]{#1_{\vert #2}}
\newcommand{\vp}{\varphi}
\newcommand{\vt}{\vartheta}
\newcommand{\wt}[1]{\widetilde{#1}}
\newcommand{\pr}[2]{\frac{\partial \, #1}{\partial\, #2}}%derivata parziale
%per creare teoremi, dimostrazioni ... 
\theoremstyle{plain}
\newtheorem{thm}{Teorema}[section] 
\newtheorem{ese}[thm]{Esempio} 
\newtheorem{ex}[thm]{Esercizio} 
\newtheorem{fatti}[thm]{Fatti}
\newtheorem{fatto}[thm]{Fatto}

\newtheorem{cor}[thm]{Corollario} 
\newtheorem{lem}[thm]{Lemma} 
\newtheorem{al}[thm]{Algoritmo}
\newtheorem{prop}[thm]{Proposizione} 
\theoremstyle{definition} 
\newtheorem{defn}{Definizione}[section] 
\newcommand{\intt}[2]{int_{#1}^{#2}}
\theoremstyle{remark} 
\newtheorem{oss}{Osservazione} 
\newcommand{\di }{\, \mathrm{d}}
\newcommand{\tonde}[1]{\left( #1 \right)}
\newcommand{\quadre}[1]{\left[ #1 \right]}
\newcommand{\w}{\omega}

% diagrammi commutativi tikzcd
% per leggere la documentazione texdoc

\begin{document}
\textbf{Lezione del 4 maggio}
\begin{defn}Sia $J$ un sottoinsieme non vuoto di $\p(V)$ allora definiamo il sottospazio proiettivo associato a $J$ come 
$$ L(J) =\bigcap_{{\p(W)\subseteq \p(V)} \atop{J \subseteq \p(W)}}\p(W)$$
in modo analogo, possiamo definirlo come il pi\`u piccolo sottospazio proiettivo di $\p(V)$ che contiene $J$
\end{defn}
\begin{oss}Sia $S\subseteq \p(V)$ 
$$ L(S) = S \quad \ses \quad S=\p(T) \text{ per qualche } T \text{ sottospazio vettoriale di } V $$
\end{oss}
\begin{defn}Nel caso in cui $J =\s{ P_1, \dots , P_t}$ \`e un insieme finito di punti, per notazione poniamo 
$$ L(J) =L(P_1, \dots , P_t)$$
\end{defn}
\begin{oss}Sia $P_i = [v_i]$ per $i=1, \dots, t $ allora
$$ L(P_1, \dots, P_t) = \p ( Span(v_1, \dots, v_t))$$ 
dunque in particolare
$$ \dim (P_1, \dots, P_t) \leq t-1$$ 
\end{oss}
\spazio
\begin{defn}Siano $P_1, \dots, P_k$ punti dove $P_i=[v_i]$ per $i=1, \dots, k$.\\
Diciamo che tali punti sono linearmente indipendenti se lo sono i vettori $v_1, \dots, v_k$.\\
In caso contrario, diciamo che i punti sono linearmente dipendenti
\end{defn}
\begin{oss}Siano $P=[v]$ e $Q=[w]$ due punti di $\p(V)$
$$ P, Q\text{ sono linearmente indipendenti } \quad \ses \quad P\text{ e } Q \text{ sono distinti}$$
infatti se $[v]\neq [w]$ allora $\forall \lambda\in \K^\star$ si ha $v\neq \lambda w$ ovvero $v,w$ sono linearmente indipendenti
\end{oss}
\begin{oss}Se $P\neq Q$ denotiamo con $L(P,Q)$ l'unica retta proiettiva che passa per $P$ e per $Q$.\\
L'unicit\`a deriva dalla minimalit\`a di $L(P,Q)$
\end{oss}
\begin{oss}Siano $P,Q,R\in \p(V)$ punti distinti.\\
$P,Q,R$ sono linearmente indipendenti se e solo se non sono allineati.\\
In questo piano $L(P,Q,R)$ \`e un piano proiettivo, ed \`e l'unico  che passa per questi $3$ punti
\end{oss}
\begin{ese}[Continuo dell'esercizio della volta precedente] I punti
$$  P=\left[\frac{1}{2},1,1\right]\quad Q  =\quadre{1,\frac{1}{3}, \frac{4}{3}} \quad R = [2,-1, 2]$$ 
sono linearmente dipendenti in quanto la matrice
$ \begin{pmatrix}
 1&3 & 2\\
 2&1 & -1\\ 
 2&4 & 2
\end{pmatrix}$ ha determinante nullo
\end{ese}
\spazio
\begin{oss}Siano $P_1, \dots,P_t\in \p(V)$
$$ P_1,\dots, P_t\text{ linearmente indipendenti} \quad \implica\quad t\leq \dim V = n+1$$
\end{oss}
\spazio
\begin{defn}Siano $P_1, \dots, P_t\in \p(V)$\\
Diremo che tali punti sono in posizione generale se 
\begin{itemize}
\item $t\leq n+1$ e sono linearmente indipendenti
\item $t>n+1$ e per ogni scelta di $n+1$ punti tra essi, otteniamo un sottoinsieme costituito da punti linearmente indipendenti 
\end{itemize}
\end{defn}
\begin{oss}Se $P_1, \dots, P_t$ sono in posizione generale con $t\geq n+1$ allora $L(P_1, \dots, P_t) = \p(V)$
\end{oss}
\begin{oss}Sia $\s{e_0, \dots, e_n}$ un riferimento proiettivo di $\p(V)$ allora i punti fondamentali e il punto unit\`a sono in posizione generale 
\end{oss}
Mostriamo che vale una sorta di viceversa
\begin{lem}Sia $V$ uno spazio vettoriale di dimensione $n+1$.\\
Siano $P_0, \dots, P_{n+1}\in \p(V)$ $n+2$ punti in posizione generale.\\
Allora $\s{P_0, \dots, P_{n+1}}$ definisce un riferimento proiettivo per cui
\begin{itemize}
\item $P_0, \dots, P_n$ sono i punti fondamentali
\item $P_{n+1}$ \`e il punto unit\`a 
\end{itemize} 
\proof Assumiamo $P_i = [v_i]$ per $i=0, \dots, n+1$.\\
Per ipotesi $v_0, \dots, v_n$ sono linearmente indipendenti, di conseguenza
$$ v_{n+1} = a_0 v_0+\dots + a_n v_n$$
Mostriamo che $a_i \neq 0 $ per $i=0, \dots n$.\\
Supponiamo, per assurdo $a_0=0$ allora $v_{n+1}$ si esprime come combinazione lineare di $v_1, \dots, v_n$ da cui l'insieme $\s{v_{n+1}, v_1, \dots, v_n}$ \`e un insieme costituito da vettori linearmente dipendenti, da  cui $P_{n+1},P_1, \dots, P_n $ sarebbero dipendenti, il che \`e assurdo per l'ipotesi sulla loro posizione generale.\\
Definiamo un riferimento proiettivo ponendo $e_i= a_i v_i$ per $i=0, \dots, n$ di conseguenza $v_{n+1} = e_0+\dots+ e_n$ da cui la tesi 
\end{lem}
\spazio
\begin{ese}Consideriamo i seguenti punti in $\pror 3$
$$ P_1=[ 1,0,1,2]\quad P_2=[0,1,1,1]\quad P_3=[2,1,1,2]\quad P_4=[1,1,2,3]$$
Mostrare se tali punti sono in posizione generale e calcolare $\dim(L(P_1, \dots, P_4))$\\
Essendo i punti $4$ come la dimensione dello spazio, tali punti sono in posizione generale se e solo se sono linearmente indipendenti
\end{ese}
\newpage
\section{Equazioni di sottospazi proiettivi}
Sia $\dim V = n+1$.\\
Fissiamo un sottospazio vettoriale $W$ di $V$ di dimensione $k$, allora $W$ possiede una base $\s{ v_0, \dots , v_k}$ dunque $p(W) = L(P_0, \dots, P_k)$ con $P_i=[v_i]$.\\
Ora $\forall P=[v]\in \p(W)$ si ha 
\begin{equation}
\label{eq_par}
 v= \lambda_0 v_0+ \dots + \lambda_k v_k
 \end{equation}
Fissando un riferimento proiettivo $e_0, \dots, e_n$ di $\p(V)$ allora
$$ P=[x_0, \dots , x_n]\quad P_i=[y_{i,0}, \dots, y_{i,n}]$$
possiamo riscrivere~\ref{eq_par} nella seguente forma 
\begin{equation}
\label{eq_par2}
\begin{cases} x_0 = y_0 \lambda_{0,0} +\dots + \lambda+k y_{k,0} \\
\vdots \\
 x_n = y_0 \lambda_{0,n} +\dots + \lambda+k y_{k,n} 
 \end{cases}
 \end{equation}
 detta equazione parametrica di $L(P_1, \dots,P_k)$
\begin{ese}[Equazione parametrica di una retta] Consideriamo il caso $\dim \p(W)=1$ allora~\ref{eq_par2} diventa
$$\begin{cases} x_0 = \lambda_0 y_{0,0} + \lambda_1 y_{1,0} \\
\vdots \\
x_n =\lambda_0 y_{0,n} + \lambda_1 y_{1,n}
\end{cases}$$
ed \`e l'equazione parametrica della retta $L(P_0, P_1)$ con $P_0=[y_{0,0}, \dots, y_{0,n}]$ e $P_1 =[ y_{1,0}, \dots, y_{1,n}]$
\end{ese}
\spazio
Andiamo ora a studiare l'equazione cartesiana della retta per 2 punti in $\p^2(\K)$.\\
Siano $P=[p_0,p_1,p_2]$ e $Q=[q_0, q_1,q_2]$ con $P\neq 0$.\\
Allora l'equazione cartesiana della retta per $P,Q$ \`e della forma 
$$\det \begin{pmatrix}
x_0 & p_0& q_0 \\
x_1 & p_1& q_1 \\
x_2 & p_2& q_2 
\end{pmatrix}=0$$
infatti se $[x_0,x_1,x_2]\in L(P,Q)$ allora  i vettori $$\begin{pmatrix} p_0 \\ p_1\\ p_2  
\end{pmatrix}, \begin{pmatrix}
q_0 \\ q_1 \\ q_2
\end{pmatrix} , \begin{pmatrix}
 x_0\\x_1 \\x_2
\end{pmatrix}$$
sono linearmente dipendenti.\\
In modo analogo se   $ P=[p_0, p_1,p_2,p_3],Q=[q_0, q_1,q_2,q_3]$ e $ R=[r_0, r_1,r_2,r_3] \in \p^3(\K)$ allora il piano per $P,Q,R$ ha equazioni cartesiane della forma 
$$\det \begin{pmatrix}
x_0 & p_0& q_0 & r_0 \\
x_1 & p_1& q_1 & r_1\\
x_2 & p_2& q_2  & r_2 \\
x_3 & p_3& q_3  & r_3
\end{pmatrix}=0$$

\newpage
\begin{defn}Siano $S_1,S_2$ due sottospazi proiettivi di $\p(V)$ allora chiamiamo sottospazio somma di $S_1,S_2$ il sottospazio proiettivo
$$ L(S_1,S_2) = L(S_1\cup S_2)$$
ovvero il sottospazio proiettivo generato dall'unione
\end{defn}
\begin{lem}Se $S_1=\p(W_1)$ e $S_2 = \p(W_2)$ allora 
$$ L(S_1,S_2) = \p(W_1+W_2)$$
\proof Essendo $L(S_1, S_2)$ un sottospazio proiettivo, si ha $L(S_1, S_2) = \p(W)$ per qualche $W$ sottospazio vettoriale di $V$.\\
Osserviamo che 
$$S_1\subseteq L(S_1,S_2) \quad \implica \quad W_1 \subseteq W $$
$$S_2\subseteq L(S_1,S_2) \quad \implica \quad W_2 \subseteq W $$
da cui $W_1+W_2 \subseteq W$ ovvero $\p(W_1+W_2) \subseteq \p(W)$.\\
Andiamo a mostrare l'altra inclusione.\\
$$W_1 \subseteq W_1 + W_2  \quad\implica\quad S_1 =\p(W_1)\subseteq \p(W_1+W_2)$$
$$W_2 \subseteq W_1 + W_2  \quad\implica\quad S_2 =\p(W_2)\subseteq \p(W_1+W_2)$$
Ora $L(S_1, S_2)$ \`e il pi\`u piccolo sottospazio che contiene $S_1\cup S_2$ da cui $L(S_1, S_2) \subseteq \p(W_1+W_2)$

\end{lem}

\begin{prop}[Formula di Grassman proiettiva]\bianco 
Siano $S_1,S_2$ sottospazi proiettivi allora
$$ \dim L(S_1,S_2) = \dim L(S_1) + \dim L(S_2) -\dim (S_1\cap S_2)$$
\proof Dalla formula di Grassman per sottospazi vettoriali otteniamo
$$ \dim L(S_1,S_2) = \dim \p(W_1+W_2) =\dim (W_1+W_2) -1 =\dim W_1 +\dim W_2 -\dim (W_1\cap W_2) =$$
$$=(\dim W_1 -1) +(\dim W_2 -1) -(\dim (W_1 \cap W_2)-1)=\dim L(S_1) + \dim L(S_2) -\dim (S_1\cap S_2)$$
\end{prop}
\begin{oss}Dalla formula di Grassman proiettiva otteniamo una stima sulla dimensione dell'intersezione infatti
$$ \dim (S_1\cap S_2) \geq \dim S_1 + \dim S_2 - \dim \p(V)$$
\end{oss}
\begin{prop}\bbianco
\begin{itemize}
\item In un piano proiettivo due rette si incontrano 
\item In uno spazio proiettivo di dimensione 3 una retta ed un piano si incontrano 
\item In uno spazio proiettivo di dimensione 3 due piani distinti hanno in comune una retta
\end{itemize}
\proof \bbianco
\begin{itemize}
\item Siano $S_1,S_2$ due rette proiettive in un piano proiettivo dunque 
$$\dim (S_1\cap S_2) \geq \dim S_1 + \dim S_2  -\dim \p(V) = 1+1-2=0$$
dunque la loro intersezione \`e almeno un punto 
\item Siano $S_1$ un piano e $S_2$ una retta allora
$$ \dim (S_1\cap S_2) \geq \dim S_1 + \dim S_2 -\dim \p(V) = 2+1-3 = 0$$]
dunque la loro intersezione \`e almeno un punto
\item Siano $S_1, S_2$ due piani proiettivi distinti.\\
Essendo distinti si ha $\dim (S_1\cap S_2)<2$ inoltre dalla formula di Grassman
$$ \dim (S_	1\cap S_2)\geq \dim S_1 +\dim S_2 -\dim \p(V) = 2 +2 -3 = 1$$
dunque $1\leq \dim(S_1\cap S_2) <2$ da cui la tesi

\end{itemize}
\end{prop}
\end{document}