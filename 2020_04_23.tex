\documentclass[a4paper,12pt]{article}
\usepackage[a4paper, top=2cm,bottom=2cm,right=2cm,left=2cm]{geometry}

\usepackage{bm,xcolor,mathdots,latexsym,amsfonts,amsthm,amsmath,
					mathrsfs,graphicx,cancel,tikz-cd,hyperref,booktabs,caption,amssymb,amssymb,wasysym}
\hypersetup{colorlinks=true,linkcolor=blue}
\usepackage[italian]{babel}
\usepackage[T1]{fontenc}
\usepackage[utf8]{inputenc}
\newcommand{\s}[1]{\left\{ #1 \right\}}
\newcommand{\sbarra}{\backslash} %% \ 
\newcommand{\ds}{\displaystyle} 
\newcommand{\alla}{^}  
\newcommand{\implica}{\Rightarrow}
\newcommand{\iimplica}{\Leftarrow}
\newcommand{\ses}{\Leftrightarrow} %se e solo se
\newcommand{\tc}{\quad \text{ t. c .} \quad } % tale che 
\newcommand{\spazio}{\vspace{0.5 cm}}
\newcommand{\bbianco}{\textcolor{white}{,}}
\newcommand{\bianco}{\textcolor{white}{,} \\}% per andare a capo dopo 																					definizioni teoremi ...


% campi 
\newcommand{\N}{\mathbb{N}} 
\newcommand{\R}{\mathbb{R}}
\newcommand{\Q}{\mathbb{Q}}
\newcommand{\Z}{\mathbb{Z}}
\newcommand{\K}{\mathbb{K}} 
\newcommand{\C}{\mathbb{C}}
\newcommand{\F}{\mathbb{F}}
\newcommand{\p}{\mathbb{P}}

%GEOMETRIA
\newcommand{\B}{\mathfrak{B}} %Base B
\newcommand{\D}{\mathfrak{D}}%Base D
\newcommand{\RR}{\mathfrak{R}}%Base R 
\newcommand{\Can}{\mathfrak{C}}%Base canonica
\newcommand{\Rif}{\mathfrak{R}}%Riferimento affine
\newcommand{\AB}{M_\D ^\B }% matrice applicazione rispetto alla base B e D 
\newcommand{\vett}{\overrightarrow}
\newcommand{\sd}{\sim_{SD}}%relazione sx dx
\newcommand{\nvett}{v_1, \, \dots , \, v_n} % v1 ... vn
\newcommand{\ncomb}{a_1 v_1 + \dots + a_n v_n} %a1 v1 + ... +an vn
\newcommand{\nrif}{P_1, \cdots , P_n} 
\newcommand{\bidu}{\left( V^\star \right)^\star}

\newcommand{\udis}{\amalg}
\newcommand{\ric}{\mathfrak{U}}
\newcommand{\inclu}{\hookrightarrow }
%ALGEBRA

\newcommand{\semidir}{\rtimes}%semidiretto
\newcommand{\W}{\Omega}
\newcommand{\norma}{\vert \vert }
\newcommand{\bignormal}{\left\vert \left\vert}
\newcommand{\bignormar}{\right\vert \right\vert}
\newcommand{\normale}{\triangleleft}
\newcommand{\nnorma}{\vert \vert \, \cdot \, \vert \vert}
\newcommand{\dt}{\, \mathrm{d}t}
\newcommand{\dz}{\, \mathrm{d}z}
\newcommand{\dx}{\, \mathrm{d}x}
\newcommand{\dy}{\, \mathrm{d}y}
\newcommand{\amma}{\gamma}
\newcommand{\inv}[1]{#1^{-1}}
\newcommand{\az}{\centerdot}
\newcommand{\ammasol}[1]{\tilde{\gamma}_{\tilde{#1}}}
\newcommand{\pror}[1]{\mathbb{P}^#1 (\R)}
\newcommand{\proc}[1]{\mathbb{P}^#1(\C)}
\newcommand{\sol}[2]{\widetilde{#1}_{\widetilde{#2}}}
\newcommand{\bsol}[3]{\left(\widetilde{#1}\right)_{\widetilde{#2}_{#3}}}
\newcommand{\norm}[1]{\left\vert\left\vert #1 \right\vert \right\vert}
\newcommand{\abs}[1]{\left\vert #1 \right\vert }
\newcommand{\ris}[2]{#1_{\vert #2}}
\newcommand{\vp}{\varphi}
\newcommand{\vt}{\vartheta}
\newcommand{\wt}[1]{\widetilde{#1}}
\newcommand{\pr}[2]{\frac{\partial \, #1}{\partial\, #2}}%derivata parziale
%per creare teoremi, dimostrazioni ... 
\theoremstyle{plain}
\newtheorem{thm}{Teorema}[section] 
\newtheorem{ese}[thm]{Esempio} 
\newtheorem{ex}[thm]{Esercizio} 
\newtheorem{fatti}[thm]{Fatti}
\newtheorem{fatto}[thm]{Fatto}

\newtheorem{cor}[thm]{Corollario} 
\newtheorem{lem}[thm]{Lemma} 
\newtheorem{al}[thm]{Algoritmo}
\newtheorem{prop}[thm]{Proposizione} 
\theoremstyle{definition} 
\newtheorem{defn}{Definizione}[section] 
\newcommand{\intt}[2]{int_{#1}^{#2}}
\theoremstyle{remark} 
\newtheorem{oss}{Osservazione} 
\newcommand{\di }{\, \mathrm{d}}
\newcommand{\tonde}[1]{\left( #1 \right)}
\newcommand{\quadre}[1]{\left[ #1 \right]}
\newcommand{\w}{\omega}

% diagrammi commutativi tikzcd
% per leggere la documentazione texdoc

\begin{document}
\textbf{Lezione del 23 aprile}
\begin{thm}Unicit\`a dello sviluppo in serie di Laurent di una funzione $f$ olomorfa su una corona circolare
\proof Senza perdit\`a di generalit\`a supponiamo $z_0=0$ e sia $f:\, D \to \C$ olomorfa con 
$$ D=\s{z\in \C \, : \, \rho_2<\abs z<\rho_1}$$ 
con sviluppo
$$ f(z) =\sum_{n \in \Z} a_n z^n$$ 
Sia $k\in \Z$ fissato e sia $\gamma:\, [0,1]\to D$ con $I(\gamma,0) =1$.\\
Calcoliamo 
$$\int_\gamma \frac{f(z)}{z^{k+1}} \dz  = \int_{\gamma} \tonde{\frac{\sum_{n \in \Z} a_n z^n }{z^{k+1}}} \dz = \int_\gamma \tonde{\sum_{n\in \Z} a_nz^{n-k-1}}\dz = \sum_{n\in \Z} a_n \int_{\gamma} z^{n-k-1} \dz $$
Ora $z^i\dz $ \`e esatta su $C\setminus \s 0$ dunque su $D$ per $i\neq -1$, dunque se $n-k-1\neq -1$ ovveri $n\neq k$ si ha 
$$\int_\gamma z^{n-k-1} \dz =0$$
dunque nella somma di sopra otteniamo 
$$ \int_{\gamma} \frac{f(z)}{z^{k+1}} = \sum_{n \in \Z} \int_{\gamma}z^{n-k-1} \dz = a_k \int_{\gamma} \frac{1}{z} \dz =2\pi i a_k $$
dunque $a_k$ \`e univocamente determinato da $f$ 
\end{thm}
\begin{oss}Osserviamo che per $k=-1$ otteniamo 
$$ a_{-1}  =\frac{1}{2\pi i } \int_{\gamma} f(z) \dz $$
tale coefficiente prende il nome di residuo di $f$ in $z_0$
\end{oss}
\spazio
\begin{defn}Sia $f:\, B \setminus \s {z_0}\to \C$ olomorfa.\\
Chiamiamo residuo di $f$ in $z_0$
$$Res(f,z_0) = a_{-1}$$ 
dove 
$$\sum_{n \in \Z} a_n (z-z_0)^n$$
\`e lo sviluppo di Lorent di $f$ centrato in $z_0$
\end{defn}
\newpage
\begin{prop}[Caratterizzazione dei poli]\bianco
Sia $D$ aperto e $z_0\in D$, supponiamo $f:\, D\setminus \s{z_0} \to \C$  olomorfa.\\
Sia $z_0$ una singolarit\`a isolata di $f$.\\
Sono fatti equivalenti
\begin{enumerate}
\item $z_0$ \`e un polo di ordine $n_0$
\item $f(z) = \frac{g(z)}{(z-z_0)^{n_0}} $ dove $g$ \`e olomorfa in $D$ con $g(z_0)\neq 0$
\item $\frac{1}{f(z)}$ si estende ad una funzione olomorfa $k:\, U \to \C$ in un intorno di $z_0$ con $z_0$ zero di ordine $n_0$ per $k$ 
\end{enumerate}
\proof\bbianco
\begin{itemize}
\item $1\implica 2$.\\
Dalla definizione di polo si ha 
$$ f(z) =\sum_{n\in \Z}a_n (z-z_0)^n \text{ con } a_{-n_0} \neq 0$$
dunque 
$$ f(z) = (z-z_0)^{-n_0} \sum_{n \geq 0 } a_{n-n_0} (z-z_0)^n =\frac{1}{ (z-z_0)^{n_0} } g(z)$$
ora $g(z)$ \`e olomorfa (\`e analitica) ed inoltre $g(z_0) =a_{-n_0} \neq 0$
\item $2\implica 3$\\
Supponiamo
$$ f(z) = \frac{g(z)}{(z-z_0)^{n_0}} \text{ con } g(z_0)\neq 0$$
ora in un intorno di $z_0$, la funzione $g$ non si annulla per cui \`e ben definita e olomorfa la funzione 
$$k(z) = \frac{1}{f(z)} = \frac{(z-z_0)^{n_0}}{g(z)}$$
tale funzione ha uno zero di ordine $n_0$ in $z_0$
\item $3\implica 1 $\\
Se $$k(z) = h(z) ( z-z_0)^{n_0} \text{ con } h(z_0)\neq 0 \text{ e } k = \frac{1}{f}$$
allora in $B \setminus\s{z_0}$ ($B$ palla che contiene $z_0$) abbiamo 
$$ f(z) =  \frac{1}{h(z) (z-z_0)^{n_0}}$$ 
ora $\frac{1}{h(z)}$ \`e olomorfa dunque analitica da cui 
$$ f(z) = \tonde{\sum_{n \geq 0} a_n (z-z_0)^n} \frac{1}{(z-z_0)^{n_0}} \sum_{n\geq -n_0} a_{n+n_0} (z-z_0)^n $$ 
ora $z_0$ \`e un polo di $f$ di ordine $n_0$ in quanto $a_0\neq 0$ infatti $\frac{1}{h(z_0)}\neq 0$
\end{itemize}
\end{prop}
\newpage
\begin{defn}Una funzione meromorfa su $D$ \`e una funzione olomorfa 
$$f:\, D\setminus S \to C$$
dove $S$ \`e un insieme discreto di punti (chiuso in $D$) e 
$$ \forall z_0\in S \quad f \text{ ha un polo in } z_0$$
\end{defn}
\begin{oss}Grazie alla caratterizzazione dei poli, se $f,g:\, D \to \C$ sono olomorfe e $g$ non \`e costantemente nulla, allora $\frac{f}{g}$ \`e meromorfa.\\
Infatti gli zeri di una funzione olomorfa sono discreti.\\
L'unica cosa non completamente ovvia \`e capire cosa succede in $z_0$ e tale che $f(z_0) =g(z_0)$.\\
Se $n$ \`e l'ordine di $z_0$ come zero di $f$ e $m$ \`e l'ordine di $z_0$ come zero di $g$.
\begin{itemize}
\item  Se $n\geq m$ allora $\frac{f}{g}$ si estende ad una funzione olomorfa in $D$ 
\item Se $n<m$ allora $\frac{f}{g}$ ha un polo di ordine $\abs{n-m}$ 
\end{itemize}
\end{oss}
Andiamo a studiare i comportamenti della funzioni "vicino" alle singolarit\`a isolate
\begin{cor} Se $f$ ha un polo in $z_0$ allora $\ds \lim_{z\to z_0} \abs{f(z)}=+\infty$
\proof Dalla caratterizzazione dei poli sappiamo che 
$$ f(z) =\frac{g(z)}{(z-z_0)^n} \text{ con } g(z_0) \neq 0 \text{ e } n>0$$
dunque 
$$\abs{f(z)} = \abs{ \frac{g(z)}{(z-z_0)^n} }\to +\infty$$
\end{cor}
Al contrario, vicino a singolarit\`a essenziali abbiamo 
\begin{thm}[di Weistrass]Sia $z_0$ una singolarit\`a essenziale di $f:\, D \setminus \s{z_0}\to \C$.\\
Allora per ogni intorno contenuto in $D$ $U$ di $z_0$ $f\tonde{U \setminus\s{z_0}}$ \`e denso in $\C$
\proof Supponiamo per assurdo che esista un intorno $U\subseteq D$ di $x_0$ tale che $f\tonde{U\setminus \s{z_0}}$ non sia denso dunque 
$$ \exists a \in \C \, \, \exists R>0 \text{ con } B(a,R)\cap f
\tonde{U\setminus \s{z_0}} =\emptyset $$
Sia $$g:\, U \setminus \s{z_0} \to \C \text{ con } g(z) =\frac{1}{f(z)-a}$$
Dunque abbiamo che 
$$\forall v \in U \quad \abs{g(z)} =\frac{1}{\abs{f(z) -a}}\leq \frac{1}{R}$$
Abbiamo dunque $g(z)$ limitata in un intorno di $z_0$ da cui $z_0$ \`e una singolarit\`a eliminabile per $z_0$.\\
Abbiamo che $g$ si estende ad una funzione olomorfa, che denotiamo ancora $g$,  su tutto $U$.\\
Ma allora $f(z) = \frac{1}{g(z)} +a$ ha un polo in $z_0$ il che contraddice che $z_0$ sia una singolarit\`a essenziale 
\end{thm}
\begin{ese}La funzione $f(z) =e^{1/z}$ ha una singolarit\`a essenziale in $z_0=0$
\end{ese}
\begin{cor}Se $z_0$ \`e una singolarit\`a essenziale per $f$ allora $\ds \lim_{z\to z_0} f(z) $ non esiste 
\proof Dal teorema di Weistrass, segue che per ogni  $a\in \C$ allora posso costruire una successione $z_n\to z_0$ tale che $\lim_{n\to +\infty} f(z_n)=a$.
\end{cor}
\newpage
\begin{defn}Sia $D\subseteq\C$ chiuso con $D^\circ\neq \emptyset$  e sia $\Gamma=\partial D $\\
Diciamo che $D$ ha bordo $C^1$ a tratti se pet ogni $ \Gamma_i$  componente connessa di  $ \Gamma $
$$\exists \gamma_i:\, J_ i\to \Gamma_i  \, \, C^1 \text{ a tratti  dove } J_i \subseteq \R \text{ intervallo e } \gamma_i \text{ surgettiva  iniettiva eccetto sugli estremi } $$
\end{defn}
\begin{ese}\bbianco
\begin{itemize}
\item $D=\R\cup \overline{B(0,1)}$ ha come bordo $(-\infty,-1]\cup S^1 \cup[1,+\infty)$ dunque non ha bordo $C^1$ a tratti
\item $D=\overline{B(0,1)}$ ha $\partial D =S^1$ ed ha perci\`o bordo $C^1$  a tratti
\item $D_R=\s{ z\in \C\, : \, \abs{z} \leq R \, \, Im(z)\geq 0 }$ con $R>0$ ha bordo $C^1$ a tratti
\end{itemize}
\end{ese}
Le componenti di bordo di un sottoinsieme $D$ con bordo $C^1$ a tratti si possono orientare canonicamente come segue
\begin{defn}Diciamo che 
$ \gamma:\, J \to \Gamma' $ \`e positiva se $ \forall t_0\in J $ tale che $\gamma'(t_0)$ sia definita, il vettore $-i\gamma'(t_0)$ punta all'esterno di $D$ in $\gamma(t_0)$
\end{defn}
\begin{defn}Un vettore $v$ ``punta all'esterno di $D$ " in $z_0\in \partial D$ se $$\exists\varepsilon>0\quad  \s{t\in (-\varepsilon, \varepsilon) \, \vert \, z_0 + tv\in D } =[-\varepsilon, 0]$$
\end{defn}
\begin{defn}Sia $R\subseteq D$ \`e un dominio con bordo $C^1$ di componenti connesse $\Gamma_1, \dots, \Gamma_k$ con parametrizzazioni positive $\gamma_1, \dots, \gamma_k$.\\
Se $\w$ \`e una $1$-forma su $D$ allora poniamo 
$$ \int_{\partial R } \w = \sum_{i=1}^ k \int_{\gamma_i} \w $$
\end{defn}
\end{document}