 \documentclass[a4paper,12pt]{article}
\usepackage[a4paper, top=2cm,bottom=2cm,right=2cm,left=2cm]{geometry}

\usepackage{bm,xcolor,mathdots,latexsym,amsfonts,amsthm,amsmath,
					mathrsfs,graphicx,cancel,tikz-cd,hyperref,booktabs,caption,amssymb,amssymb,wasysym}
\hypersetup{colorlinks=true,linkcolor=blue}
\usepackage[italian]{babel}
\usepackage[T1]{fontenc}
\usepackage[utf8]{inputenc}
\newcommand{\s}[1]{\left\{ #1 \right\}}
\newcommand{\sbarra}{\backslash} %% \ 
\newcommand{\ds}{\displaystyle} 
\newcommand{\alla}{^}  
\newcommand{\implica}{\Rightarrow}
\newcommand{\iimplica}{\Leftarrow}
\newcommand{\ses}{\Leftrightarrow} %se e solo se
\newcommand{\tc}{\quad \text{ t. c .} \quad } % tale che 
\newcommand{\spazio}{\vspace{0.5 cm}}
\newcommand{\bbianco}{\textcolor{white}{,}}
\newcommand{\bianco}{\textcolor{white}{,} \\}% per andare a capo dopo 																					definizioni teoremi ...


% campi 
\newcommand{\N}{\mathbb{N}} 
\newcommand{\R}{\mathbb{R}}
\newcommand{\Q}{\mathbb{Q}}
\newcommand{\Z}{\mathbb{Z}}
\newcommand{\K}{\mathbb{K}} 
\newcommand{\C}{\mathbb{C}}
\newcommand{\F}{\mathbb{F}}
\newcommand{\p}{\mathbb{P}}

%GEOMETRIA
\newcommand{\B}{\mathfrak{B}} %Base B
\newcommand{\D}{\mathfrak{D}}%Base D
\newcommand{\RR}{\mathfrak{R}}%Base R 
\newcommand{\Can}{\mathfrak{C}}%Base canonica
\newcommand{\Rif}{\mathfrak{R}}%Riferimento affine
\newcommand{\AB}{M_\D ^\B }% matrice applicazione rispetto alla base B e D 
\newcommand{\vett}{\overrightarrow}
\newcommand{\sd}{\sim_{SD}}%relazione sx dx
\newcommand{\nvett}{v_1, \, \dots , \, v_n} % v1 ... vn
\newcommand{\ncomb}{a_1 v_1 + \dots + a_n v_n} %a1 v1 + ... +an vn
\newcommand{\nrif}{P_1, \cdots , P_n} 
\newcommand{\bidu}{\left( V^\star \right)^\star}

\newcommand{\udis}{\amalg}
\newcommand{\ric}{\mathfrak{U}}
\newcommand{\inclu}{\hookrightarrow }
%ALGEBRA

\newcommand{\semidir}{\rtimes}%semidiretto
\newcommand{\W}{\Omega}
\newcommand{\norma}{\vert \vert }
\newcommand{\bignormal}{\left\vert \left\vert}
\newcommand{\bignormar}{\right\vert \right\vert}
\newcommand{\normale}{\triangleleft}
\newcommand{\nnorma}{\vert \vert \, \cdot \, \vert \vert}
\newcommand{\dt}{\, \mathrm{d}t}
\newcommand{\dz}{\, \mathrm{d}z}
\newcommand{\dx}{\, \mathrm{d}x}
\newcommand{\dy}{\, \mathrm{d}y}
\newcommand{\amma}{\gamma}
\newcommand{\inv}[1]{#1^{-1}}
\newcommand{\az}{\centerdot}
\newcommand{\ammasol}[1]{\tilde{\gamma}_{\tilde{#1}}}
\newcommand{\pror}[1]{\mathbb{P}^#1 (\R)}
\newcommand{\proc}[1]{\mathbb{P}^#1(\C)}
\newcommand{\sol}[2]{\widetilde{#1}_{\widetilde{#2}}}
\newcommand{\bsol}[3]{\left(\widetilde{#1}\right)_{\widetilde{#2}_{#3}}}
\newcommand{\norm}[1]{\left\vert\left\vert #1 \right\vert \right\vert}
\newcommand{\abs}[1]{\left\vert #1 \right\vert }
\newcommand{\ris}[2]{#1_{\vert #2}}
\newcommand{\vp}{\varphi}
\newcommand{\vt}{\vartheta}
\newcommand{\wt}[1]{\widetilde{#1}}
\newcommand{\pr}[2]{\frac{\partial \, #1}{\partial\, #2}}%derivata parziale
%per creare teoremi, dimostrazioni ... 
\theoremstyle{plain}
\newtheorem{thm}{Teorema}[section] 
\newtheorem{ese}[thm]{Esempio} 
\newtheorem{ex}[thm]{Esercizio} 
\newtheorem{fatti}[thm]{Fatti}
\newtheorem{fatto}[thm]{Fatto}

\newtheorem{cor}[thm]{Corollario} 
\newtheorem{lem}[thm]{Lemma} 
\newtheorem{al}[thm]{Algoritmo}
\newtheorem{prop}[thm]{Proposizione} 
\theoremstyle{definition} 
\newtheorem{defn}{Definizione}[section] 
\newcommand{\intt}[2]{int_{#1}^{#2}}
\theoremstyle{remark} 
\newtheorem{oss}{Osservazione} 
\newcommand{\di }{\, \mathrm{d}}
\newcommand{\tonde}[1]{\left( #1 \right)}
\newcommand{\quadre}[1]{\left[ #1 \right]}
\newcommand{\w}{\omega}

% diagrammi commutativi tikzcd
% per leggere la documentazione texdoc

\begin{document}
\textbf{Lezioni del 25  Settembre del prof. Frigerio}
\section{Spazi metrici}

\begin{defn}[Spazio metrico]\bianco
Uno spazio metrico \`e una coppia $(X,d)$
$X$ \`e un insieme e $d:\, X \times X \to \R$ con le seguenti propiet\'a:
\begin{itemize}
	\item[(i)] Assioma di non negativit\'a
	$$\forall x,y \in X \quad d(x,y) \geq 0 \text{ inoltre  }  d(x,y)=0 \ses x=y$$
	\item[(ii)] Assioma di simmetria
	$$ \forall x,y \in X \quad d(x,y)=d(y,x)$$
	\item[(iii)] Disuguaglianza triangolare
	$$ \forall x,y,z\in X \quad d(x,z)\leq d(x,y)+d(y,z)$$
	
\end{itemize}	
\end{defn}

\begin{enumerate}
	\item $X=\R$ con $d(x,y)=\vert x - y \vert $
	\item $X=\R^n$ sia $x=(x_1, \dots, x_n ) $ e $y=(y_1, \dots, y_n)$ allora consideriamo
	 $$ d_E(x,y)=\sqrt{ \sum_{i=1}^n (x_i-y_i)^2} \text{ distanza euclidea }$$
	 $$ d_1(x,y)= \sum_{i=1}^n \vert x_1 -y_1 \vert \text{ distanza l1 } $$
	 $$ d_\infty(x,y) = \max\{ \vert x_i - y_i \vert  \, i=1,\dots, n \} 	\text{ distanza l}\infty$$
	 \item $X$ un insieme generico, definiamo la distanza discreta
	 $$ d(x,y)=\begin{cases}
	 	1 \quad \text{ se } x\neq y \\
	 	0 \quad \text{ se } x=y 
	 \end{cases}$$
	 \item $X=\{ f:\, [0,1] \to \R \, : \, f \text{ continua} \}$
	 $$ d_1(f,g) = \int_0^1 \vert f(t)- g(t)\vert \dt $$
	 $$ d_2(f,g)=  \sqrt{ \int_0^1 \vert f(t)-g(t)\vert^2 } \dt$$
	 $$ d_\infty(f,g)= \sup_{t\in[0,1]} \{ \vert f(t)-g(t)\}$$
	 
	\end{enumerate}	

\begin{prop}$d_\infty$ \`e  una distanza
\begin{itemize}
	\item[(i)] Essendo $d_\infty$ il $\max$ di numeri non negativi essa \`e necessariamente non negativa, inoltre se $d_\infty(x,y)=0$ allora necessariamente $\vert x_i - y_i\vert =0 $ ovvero $x_i=y_1 \, \forall i$ dunque $x=y$
	\item[(ii)] La simmetria segue dal fatto che $\vert x_i -y_i \vert = \vert y_i - x_i\vert$
	\item[(iii)] Dalla definizione di $max$ 
	$$ \exists j \, d_\infty (x,z) = \vert x_j - z_j \vert \leq \vert x_j - y_j \vert   + \vert y_j - z_j \vert $$
	Dove abbiamo usato la disuguaglianza triangolare del valore assoluto su $\R$.\\
	Ora
	$$ \vert x_j - y_j\vert \leq \max \{ x_i - y_i \, i=1, \dots, n \} = d_\infty(x,y)$$
	$$ \vert y_j - z_j\vert \leq \max \{ y_i - z_i \, i=1, \dots, n \} = d_\infty(y,z)$$
Dunque otteniamo la disuguaglianza triangolare voluta

\end{itemize}
\end{prop}

\spazio 
\begin{defn}[Embedding isometrico]\bianco 
Sia $f:\, (X,d) \to (Y,d')$ allora $f$ \`e un embedding isometrico se 
$$ d'(f(x), f(y))= d(x,y) \quad \forall x,y \in X$$	
\end{defn}
\begin{fatti}\bbianco
\begin{enumerate}
	\item $id:\, (X,d) \to (X,d)$ \`e un embedding isometrico
	\item Composizione di embedding isometrici \`e un embedding isometrico
	\item Se un embedding isometrico $f$ \`e bigettivo, $f^{-1}$ \`e un embedding isometrico.\\ In tal caso $f$ \`e una isometria
	\item Un embedding isometrico \`e iniettivo (da cui il nome)
	$$ f(x_1)=f(x_2) \quad \implica \quad  0= d'(f(x_1), f(x_2))=d(x_1,x_2) \quad \implica \quad x_1=x_2$$
	\item Un embedding isometrico \`e un isometria se e solo se \`e surgettivo 
	\item Se $(X,d)$ \`e fissato. L'insieme delle isometrie da $X$ in se stesso \`e un gruppo con la composizione, tale gruppo si chiama $Isom(X,d)$
\end{enumerate}
\end{fatti}
\newpage
\subsection{Continuit\'a}

\begin{defn}[Palla]Sia $(X,d)$ uno spazio metrico, $r\in \R$ e $P\in X$
$$ B(P,r) =\{ x \in X \, : \, d(P,x)<r \}$$	
\end{defn}

\begin{defn}[Continuit\'a locale]\bianco
$ f:\, (X,d) \to (Y,d')$ si dice continua in $x_0 \in X$ se 
$$ \forall \varepsilon >0 \, \exists \delta >0 \quad f(B(x_0,\delta))\subseteq B(f(x_0), \varepsilon)$$
o in modo equivalente
	$$ \forall \varepsilon >0 \, \exists \delta >0 \quad B(x_0,\delta)\subseteq f^{-1}(B(f(x_0), \varepsilon))$$
\end{defn}
\begin{defn}[Continuit\'a globale]\bianco
$f:\, (X,d) \to (Y,d')$ \`e continua se \`e continua in ogni $x_0\in X$
\end{defn}
\begin{oss}\bbianco
\begin{enumerate}
	\item Gli embedding isometrici sono continui (basta porre $\delta=\varepsilon$)
	\item Una funzione costante \`e continua ($\delta=1$), ma le funzioni costanti non sono embedding isometrici dunque le funzioni continue sono maggiori degli \`e isometrici
\end{enumerate}	
\end{oss}
Svincoliamo formalmente la continuit\'a dalla distanza
\begin{defn}[Aperto]\label{aperti_d}\bianco
Sia $X$ uno spazio metrico, Un insieme  $A\subseteq X $ \`e aperto se 
$$ \forall x\in A \quad \exists r>0 \tc B(x,r)\subseteq A $$
 \end{defn}
 \begin{ex}
 Le palle aperte sono aperte	
 \end{ex}
 
\begin{thm}
$$ f:\, (X,d) \to (Y,d') \text{ \`e continua } \quad \ses \quad \forall A \text{ aperto di } Y \quad f^{-1}(A) \text{\`e aperto in } X $$
\proof $\implica$ Sia $A\subseteq Y$ un aperto.\\
Sia $x_0 \in f^{-1}(A)$ allora  $f(x_0)\in A$ ed essendo $A$ un aperto 
$$ \exists \varepsilon >0 \tc B(f(x_0),\varepsilon) \subseteq A $$
Ora sfruttando la continuit\'a di $f$ 
$$ \exists \delta>0 \quad B(x_0,\delta)\subseteq f^{-1}(B(f(x_0), \varepsilon))$$
Ora poich\`e 
$B(f(x_0), \varepsilon)) \subseteq A $ allora $B(x_0,\delta) \subseteq f^{-1}(A)$\\
Per arbitrariet\'a di $x_0$ $f^{-1}(A)$ \`e aperto\\
$\Leftarrow$
Sia $x_0\in X$ e $\varepsilon>0$.\\
Ora $B(f(x_0), \varepsilon)$ \`e un aperto di $Y$, dunque $ f^{-1}(B(f(x_0),\varepsilon)) $ \`e un aperto di $X$ per ipotesi.\\
Dalla definizione di aperto  e poich\`e  $x_0 \in f^{-1}(B(f(x_0),\varepsilon))$  segue che
$$ \exists \delta >0 \quad B(x_0, \delta) \subseteq f^{-1}(B(f(x_0),\varepsilon))$$
dunque $f$ \`e continua in $x_0$, da cui la tesi per arbitariet\`a di $x_0$
\endproof
\begin{oss}
La continuit\`a di $f$ dipende solo dalla famiglia degli aperti di $X$ e di $Y$, solo indirettamente da $d$ e $d'$
\end{oss}
\end{thm}
\end{document}