\documentclass[a4paper,12pt]{article}
\usepackage[a4paper, top=2cm,bottom=2cm,right=2cm,left=2cm]{geometry}

\usepackage{bm,xcolor,mathdots,latexsym,amsfonts,amsthm,amsmath,
					mathrsfs,graphicx,cancel,tikz-cd,hyperref,booktabs,caption,amssymb,amssymb,wasysym}
\hypersetup{colorlinks=true,linkcolor=blue}
\usepackage[italian]{babel}
\usepackage[T1]{fontenc}
\usepackage[utf8]{inputenc}
\newcommand{\s}[1]{\left\{ #1 \right\}}
\newcommand{\sbarra}{\backslash} %% \ 
\newcommand{\ds}{\displaystyle} 
\newcommand{\alla}{^}  
\newcommand{\implica}{\Rightarrow}
\newcommand{\iimplica}{\Leftarrow}
\newcommand{\ses}{\Leftrightarrow} %se e solo se
\newcommand{\tc}{\quad \text{ t. c .} \quad } % tale che 
\newcommand{\spazio}{\vspace{0.5 cm}}
\newcommand{\bbianco}{\textcolor{white}{,}}
\newcommand{\bianco}{\textcolor{white}{,} \\}% per andare a capo dopo 																					definizioni teoremi ...


% campi 
\newcommand{\N}{\mathbb{N}} 
\newcommand{\R}{\mathbb{R}}
\newcommand{\Q}{\mathbb{Q}}
\newcommand{\Z}{\mathbb{Z}}
\newcommand{\K}{\mathbb{K}} 
\newcommand{\C}{\mathbb{C}}
\newcommand{\F}{\mathbb{F}}
\newcommand{\p}{\mathbb{P}}

%GEOMETRIA
\newcommand{\B}{\mathfrak{B}} %Base B
\newcommand{\D}{\mathfrak{D}}%Base D
\newcommand{\RR}{\mathfrak{R}}%Base R 
\newcommand{\Can}{\mathfrak{C}}%Base canonica
\newcommand{\Rif}{\mathfrak{R}}%Riferimento affine
\newcommand{\AB}{M_\D ^\B }% matrice applicazione rispetto alla base B e D 
\newcommand{\vett}{\overrightarrow}
\newcommand{\sd}{\sim_{SD}}%relazione sx dx
\newcommand{\nvett}{v_1, \, \dots , \, v_n} % v1 ... vn
\newcommand{\ncomb}{a_1 v_1 + \dots + a_n v_n} %a1 v1 + ... +an vn
\newcommand{\nrif}{P_1, \cdots , P_n} 
\newcommand{\bidu}{\left( V^\star \right)^\star}

\newcommand{\udis}{\amalg}
\newcommand{\ric}{\mathfrak{U}}
\newcommand{\inclu}{\hookrightarrow }
%ALGEBRA

\newcommand{\semidir}{\rtimes}%semidiretto
\newcommand{\W}{\Omega}
\newcommand{\norma}{\vert \vert }
\newcommand{\bignormal}{\left\vert \left\vert}
\newcommand{\bignormar}{\right\vert \right\vert}
\newcommand{\normale}{\triangleleft}
\newcommand{\nnorma}{\vert \vert \, \cdot \, \vert \vert}
\newcommand{\dt}{\, \mathrm{d}t}
\newcommand{\dz}{\, \mathrm{d}z}
\newcommand{\dx}{\, \mathrm{d}x}
\newcommand{\dy}{\, \mathrm{d}y}
\newcommand{\amma}{\gamma}
\newcommand{\inv}[1]{#1^{-1}}
\newcommand{\az}{\centerdot}
\newcommand{\ammasol}[1]{\tilde{\gamma}_{\tilde{#1}}}
\newcommand{\pror}[1]{\mathbb{P}^#1 (\R)}
\newcommand{\proc}[1]{\mathbb{P}^#1(\C)}
\newcommand{\sol}[2]{\widetilde{#1}_{\widetilde{#2}}}
\newcommand{\bsol}[3]{\left(\widetilde{#1}\right)_{\widetilde{#2}_{#3}}}
\newcommand{\norm}[1]{\left\vert\left\vert #1 \right\vert \right\vert}
\newcommand{\abs}[1]{\left\vert #1 \right\vert }
\newcommand{\ris}[2]{#1_{\vert #2}}
\newcommand{\vp}{\varphi}
\newcommand{\vt}{\vartheta}
\newcommand{\wt}[1]{\widetilde{#1}}
\newcommand{\pr}[2]{\frac{\partial \, #1}{\partial\, #2}}%derivata parziale
%per creare teoremi, dimostrazioni ... 
\theoremstyle{plain}
\newtheorem{thm}{Teorema}[section] 
\newtheorem{ese}[thm]{Esempio} 
\newtheorem{ex}[thm]{Esercizio} 
\newtheorem{fatti}[thm]{Fatti}
\newtheorem{fatto}[thm]{Fatto}

\newtheorem{cor}[thm]{Corollario} 
\newtheorem{lem}[thm]{Lemma} 
\newtheorem{al}[thm]{Algoritmo}
\newtheorem{prop}[thm]{Proposizione} 
\theoremstyle{definition} 
\newtheorem{defn}{Definizione}[section] 
\newcommand{\intt}[2]{int_{#1}^{#2}}
\theoremstyle{remark} 
\newtheorem{oss}{Osservazione} 
\newcommand{\di }{\, \mathrm{d}}
\newcommand{\tonde}[1]{\left( #1 \right)}
\newcommand{\quadre}[1]{\left[ #1 \right]}
\newcommand{\w}{\omega}

% diagrammi commutativi tikzcd
% per leggere la documentazione texdoc


\begin{document}
\textbf{Lezione del 20 Novembre di Gandini}
\begin{thm}[del numero di Lebegue]\bianco
Sia $(X,d)$ uno spazio metrico compatto.\\
Sia $\ric$ un ricoprimento aperto di $X$
$$ \exists \varepsilon >0 \quad \forall x \in X \quad \exists U \in \ric \quad B(x,\varepsilon) \subseteq U $$
\proof Sia 
$$ X_n = \{  x \in X \, \vert \, \exists U \in \ric \quad B\left(  x, 2^{-n}\right)\subseteq U \}$$
la tesi \`e dunque equivalente a dire che $X_n =X $ per un certo $n \in \N$
Osserviamo che $U = \ds \bigcup_{n \in \N } X_n$ infatti $\ric $ \`e un ricoprimento aperto.\\
Osserviamo,  inoltre, $X_n \subseteq X_{n+1}$, in realt\`a vale che 
$$ X_n \subseteq X_{n+1}^ \circ $$ 
infatti preso $x\in X_n$ si ha $B\left( x, 2^{-n} \right) \subseteq U $ con $U\in \ric $.\\
Mostriamo adesso che $B \left( x, 2^{-(n+1)}\right) \subseteq X_{n+1}$ da cui $x \in X_{n+1}^\circ$.\\
Sia $y \in B \left( x, 2^{-(n+1)} \right)$ dunque $d(y,x) < {2^{-(n+1)}}$.\\
Sia $z \in B\left( y, 2^{-(n+1)} \right)$ da cui  $d(z,y)< 2^{-(n+1)}$.\\
 $d(x,z)<{2^{-n}}$ ovvero $z\in B \left( x, 2^{-n}\right)$.\\
Abbiamo provato che 
$$ y \in B\left( x , 2^{-(n+1)}\right) \quad \implica \quad y \in X_{n+1} \quad  \implica \quad X_n \subseteq X_{n+1}^\circ$$
Dunque $\{ X_n^\circ\}$ \`e un ricoprimento aperto e data la compattezza di $X$ posso estrarre un sottoricoprimento finito, dunque posso estrarre il massimo (gli aperti sono inscatolati), da cui $X=X_n^\circ$ da cui la tesi
\endproof
\end{thm}
\begin{defn}[Uniforme continuit\`a]\bianco
Sia $f:\,(X,d_X) \to (Y,d_Y)$ una funzione tra spazi metrici, $f$ \`e uniformemente continua se 
$$ \forall \varepsilon>0 \quad \exists \delta>0 \quad d_Y(f(x),
f(x')) \leq \varepsilon  \text{ se } d_X(x,,x') \leq \delta$$
\end{defn}
\begin{thm}[Heine-Cantor]\bianco
Siano $(X,d_X)$ e $(Y,d_Y)$ spazi metrici con $X$ compatto.
$$f:\, X \to Y \text{ continua } \quad \implica \quad f \text{ uniformemente continua}$$
\proof Sia $\varepsilon>0$ cerchiamo un $\delta>0$ tale che 
$$f(B(x, \delta))\subseteq B\left( f(x), \frac{\varepsilon}{2}\right) \,  \forall x \in X \text{  ossia per cui } B(x,\delta)\subseteq f^{-1}\left( B \left( f(x), \frac{\varepsilon}{2}\right)\right)$$
Applichiamo il teorema del numero di Lebague al ricoprimento aperto 
$$ \ric = \left\{ f^{-1} \left( B \left( f(x) , \frac{\varepsilon}{2}\right)\right)\right\}_{x \in X }$$
da cui $\exists \delta>0$ tale che $$\forall x \in X \quad \exists U \in \ric \quad B(x,\delta)\subseteq U$$ in modo equivalente $$\forall x \in X\quad  \exists x' \in X   \quad B(x,\delta) \subseteq f^{-1}\left( B \left( f(x'),\frac{\varepsilon}{2}\right)\right)$$
Sia $y \in B(x,\delta)$ allora 
$$ f(y) \in B \left( f(x'), \frac{\varepsilon}{2}\right) \text{ e } f(x) \in B \left( f(x'), \frac{\varepsilon}{2}\right) $$ 
da cui $d_y(f(x), f(y) ) \leq \varepsilon$\endproof
\end{thm}\newpage

Sia $X$ un insieme arbitrario e $Y$ spazio metrico completo
$$ B(X,Y)=\{ f:\, X \to Y \text{ limitate}\}$$
Mostriamo che $B(X,Y)$ \`e completo con la distanza $$d_\infty( f, g) =\sup_{x\in X } d_y (f(x),g(x))$$
Sia $\{ f_n\} \subseteq B(X,Y)$ una successione di Cauchy allora proviamo che $f_n \to f $.\\
Essendo la successione di Cauchy $\forall x \in X $ anche $\{f_n(x)\}\subseteq Y$ \`e di Cauchy e per completezza di $Y$ ammette un limite in $Y$, sia $f(x)=\lim f_n(x)$.\\
Proviamo che $f$ cos\`i definita \`e limitata
$$ d_Y(f(x),f(x'))  \leq d_Y(f(x), f_n(x))+ d_Y(f_n(x),f_n(x'))+ d_Y(f_n(x')+f(x')) $$
Tale distanza \`e limitata infatti  poich\`e $f_n$ converge a $f$ puntualmente $d_Y(f_n(x),f(x))$ e $d_Y(f_n(x'),f(x'))$ sono limitate ed essendo $f_n$ limitata anche $d_Y(f_n(x),f_n(x'))$ lo \`e.\\
Mostriamo che $f_n \to f$ in $d_\infty$
$$ d_\infty(f,f_n) = \sup_{x\in X} d_y(f(x), f_n(x))$$
ora 
$$ d_Y(f(x), f_n(x))= d_Y \left( \lim_{m \to \infty} f_m (x),f_n(x) \right)$$
Essendo $d_Y(\cdot. f_n(x))$ continua si ha 
$$ d_Y(f(x),f_n(x)) = \lim_{n \to \infty} d_Y(f_n(x),f_m(x)) \leq\varepsilon \quad \forall \varepsilon>0 \quad n \geq n_0$$
infatti la successione \`e di Cauchy dunque $\forall \varepsilon>0 $ $\exists n_0$ tale che $d_\infty(f,f_n)\leq \varepsilon$ per $n \geq n_0$.\\
Abbiamo provato che $B(X,Y)$ \`e completo in $dd_\infty$
\begin{ex}
Supponiamo, adesso, $X$ topologico e poniamo 
$$ C_B(X,Y)=\{ f \in B(X,Y) \text{ continue}\}$$
Allora 
$C_B(X,Y)$ \`e chiuso dunque $C_B$ con $d_\infty$ completo.\\
Inoltre se $\{ f_n\} \subseteq C_B(X,Y)$ di Cauchy con limite $f \in B(X,Y)$ allora $f$ \`e continua
\end{ex}
\end{document}